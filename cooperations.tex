% -*- root: main.tex -*-

\chapter{Unstable cooperations}

\todo{Write an introduction for me.}





\todo[inline]{This section can probably be introduced as trying to enrich the notion of a context for cohomology theories swallowing \emph{spaces}.  The stable cooperations are totally captured by the notion of a context, but there are two approaches to (distinct sorts of) unstable operations: Hopf rings and power operations.  We plan on talking about both (and they both have something to say about $MU\<6\>$), so this can be a common thread.}




\section{Unstable contexts}

\todo[inline]{I would like to construct a framework that these things live in, like simplicial schemes gave a framework for stable contexts.  At first glance, though, I'm not sure what to do / whether this is possible.  My best guess for now is some arboreal thing, where $E_* \OS{F}{*}$ plays the role of an operadic right-module and $E_* \OS{E}{*}$ governs the internal structure of the tree.  This doesn't seem quite right, though, and I'm already getting sick of thinking about it.}

Let $E$ be a ring spectrum with constituent spaces $\OS{E}{*}$ in its $\Omega$--spectrum.  Then for a space $X$ we define the space \[E(X) = \colim_{j \to \infty} \Loops^j (\OS{E}{j} \sm X) = \Loops^\infty (E \sm \Susp^\infty X)\] which has the property $\pi_* E(X) = \widetilde E_* X$.  This construction defines a functor with two evident natural transformations:\todo{Sort out this question mark.}
\begin{align*}
\eta \co X & \simeq S^0 \sm X \\
& \to \OS{E}{0} \sm X \\
& \to \colim_{j \to \infty} \Loops^j(\OS{E}{j} \sm X) \simeq E(X), \\
\\
\mu \co E(E(X)) & = \colim_{j \to \infty} \Loops^j \left( \OS{E}{j} \sm \colim_{k \to \infty} \Loops^k (\OS{E}{k} \sm X) \right) \\
& \xrightarrow{\simeq?} \colim_{\substack{j \to \infty \\ k \to \infty}} \Loops^{j+k} (\OS{E}{j} \sm \OS{E}{k} \sm X) \\
& \xrightarrow{\mu} \colim_{\substack{j \to \infty \\ k \to \infty}} \Loops^{j+k} (\OS{E}{j+k} \sm X) \xleftarrow{\simeq} E(X).
\end{align*}

\begin{definition}
Using these maps, we can construct the following cosimplicial space, the unstable analogue of the descent object from \Cref{StableContextLecture}:
\[\sheaf{UD}_E(X) := \left\{
\begin{tikzcd}
\begin{array}{c} E \\ \circ \\ X \end{array} \arrow[leftarrow, shift left=\baselineskip]{r}{\mu} \arrow[shift left=(2*\baselineskip)]{r}{\eta_L} \arrow{r}{\eta_R} &
\begin{array}{c} E \\ \circ \\ E \\ \circ \\ X \end{array} \arrow[shift left=(3*\baselineskip)]{r} \arrow[leftarrow, shift left=(2*\baselineskip)]{r} \arrow[shift left=\baselineskip]{r}{\Delta} \arrow[leftarrow]{r} \arrow[shift right=\baselineskip]{r} &
\begin{array}{c} E \\ \circ \\ E \\ \circ \\ E \\ \circ \\ X \end{array} \arrow[shift left=(4*\baselineskip)]{r} \arrow[leftarrow, shift left=(3*\baselineskip)]{r} \arrow[shift left=(2*\baselineskip)]{r} \arrow[leftarrow, shift left=\baselineskip]{r} \arrow{r} \arrow[leftarrow, shift right=\baselineskip]{r} \arrow[shift right=(2*\baselineskip)]{r} &
\cdots
\end{tikzcd}
\right\}.\]
Its totalization gives the \textit{unstable $E$--completion of $X$}.
\end{definition}

\begin{definition}
Under suitable hypotheses, we extract from this an \textit{unstable context for $E$}:
\begin{align*}
\mathcal{UM}_E & := \Spec \pi_* \sheaf{UD}_E(S^0) \\
& = \left\{
\begin{tikzcd}[ampersand replacement=\&]
\Spec \pi_* E \arrow{r} \arrow[leftarrow,shift left=\baselineskip]{r}
\arrow[leftarrow,shift right=\baselineskip]{r}
\&
\Spec \pi_* \left( \begin{array}{c} E \\ \circ \\ E \end{array} \right)
\arrow[leftarrow, shift left=(2*\baselineskip)]{r} \arrow[shift left=\baselineskip]{r} \arrow[leftarrow]{r} \arrow[shift right=\baselineskip]{r} \arrow[leftarrow, shift right=(2*\baselineskip)]{r}
\&
\cdots
\end{tikzcd}
\right\}
\end{align*}
as well as a quasicoherent sheaf over it associated to $X$:
\begin{align*}
\sheaf{UM}_E(X) & := \left(\pi_* \sheaf{UD}_E(X)\right){\widetilde{}} \\
& = \left\{
\begin{tikzcd}[ampersand replacement=\&]
\pi_* \left( \begin{array}{c} E \\ \circ \\ X \end{array} \right)\widetilde{} \arrow[leftarrow, shift left=\baselineskip]{r} \arrow[shift left=(2*\baselineskip)]{r} \arrow{r} \&
\pi_* \left( \begin{array}{c} E \\ \circ \\ E \\ \circ \\ X \end{array} \right)\widetilde{} \arrow[shift left=(3*\baselineskip)]{r} \arrow[leftarrow, shift left=(2*\baselineskip)]{r} \arrow[shift left=\baselineskip]{r} \arrow[leftarrow]{r} \arrow[shift right=\baselineskip]{r} \&
\pi_* \left( \begin{array}{c} E \\ \circ \\ E \\ \circ \\ E \\ \circ \\ X \end{array} \right)\widetilde{} \arrow[shift left=(4*\baselineskip)]{r} \arrow[leftarrow, shift left=(3*\baselineskip)]{r} \arrow[shift left=(2*\baselineskip)]{r} \arrow[leftarrow, shift left=\baselineskip]{r} \arrow{r} \arrow[leftarrow, shift right=\baselineskip]{r} \arrow[shift right=(2*\baselineskip)]{r} \&
\cdots
\end{tikzcd}
\right\}.
\end{align*}
\end{definition}

say when this thing has good convergence properties to $\pi_* X$ (cf. Section 3 of BCM)

say when this thing has an identifiable $E_2$ term (cf. the end of Section 6 of BCM in particular)

give a description of the first two stages of the context in terms of Hopf rings

mention how to select the \emph{additive} unstable cooperations

------------




Hopf rings: algebraic and topological

\begin{definition}\todo{Can this definition be made without specifying the grading as such and instead using a $\G_m$--action?}
A Hopf ring $A_{*, *}$ over a graded ring $R_*$ is itself a graded ring object in the category $\CatOf{Coalgebras}_{R_*}$, sometimes called an $R_*$--coalgebraic graded ring object.  It has the following structure maps:
\begin{align*}
+ & \co A_{s, t} \times A_{s, t} \to A_{s, t} & \text{($A_{s, t}$ is an abelian group)} \\
\cdot & \co R_{s'} \otimes_{R_*} A_{s, t} \to A_{s+s', t} & \text{($A_{*, t}$ is a $R_*$--module)} \\
\Delta & \co A_{s, t} \to \bigoplus_{s' + s'' = s} A_{s', t} \otimes_{R_*} A_{s'', t} & \text{($A_{*, t}$ is a $R_*$--coalgebra)} \\
\ast & \co A_{s, t} \otimes_{R_*} A_{s', t} \to A_{s + s', t} & \text{(addition for the ring in $R_*$--coalgebras)} \\
\eta_\ast & \co R_* \to A_{*, 0} & \text{(null element for ring addition)} \\
\chi & \co A_{s, t} \to A_{s, t} & \text{(negation for the ring in $R_*$--coalgebras)} \\
\circ & \co A_{s, t} \otimes_{R_*} A_{s', t'} \to A_{s + s', t + t'} & \text{(multiplication map for the ring in $R_*$--coalgebras)} \\
\eta_\circ & \co R_* \to A_{*, 0} & \text{(null element for ring multiplication)}.
\end{align*}
These are required to satisfy various commutative diagrams. The least obvious is displayed in \Cref{DistributivityDiagram}, encoding the distributivity of $\circ$--``multiplication'' over $\ast$--``addition''.
\end{definition}

\begin{figure}
\begin{center}
\begin{tikzcd}
A_{s, t} \otimes_{R_*} (A_{s', t'} \otimes_{R_*} A_{s'', t'}) \arrow{r}{1 \otimes \ast} \arrow{d}{\Delta \otimes (1 \otimes 1)} & A_{s, t} \otimes_{R_*} A_{s' + s'', t'} \arrow{dddd}{\circ} \\
\left(\bigoplus_{s_1 + s_2 = s} A_{s_1, t} \otimes_{R_*} A_{s_2, t} \right) \otimes_{R_*} (A_{s', t'} \otimes_{R_*} A_{s'', t'}) \arrow{d}{\simeq} \\
\bigoplus_{s_1 + s_2 = s} \left(A_{s_1, t} \otimes_{R_*} A_{s_2, t}  \otimes_{R_*} A_{s', t'} \otimes_{R_*} A_{s'', t'} \right) \arrow{d}{1 \otimes \tau \otimes 1} \\
\bigoplus_{s_1 + s_2 = s} \left(A_{s_1, t} \otimes_{R_*} A_{s', t'} \otimes_{R_*} A_{s_2, t} \otimes_{R_*} A_{s'', t'} \right) \arrow{d}{\circ \otimes \circ} \\
\bigoplus_{s_1 + s_2 = s} \left(A_{s_1 + s', t + t'} \otimes_{R_*} A_{s_2 + s'', t + t'}\right) \arrow{r}{\ast} & A_{s + s' + s'', t + t'}, \\
\end{tikzcd}
\end{center}
\caption{The distributivity axiom for $\ast$ over $\circ$ in a Hopf algebra.}\label{DistributivityDiagram}
\end{figure}

The context of additive unstable cooperations: make the module of $\ast$--indecomposables $QE_* \Susp^{-*} \OS{E}{*}$ into an algebra using the $\circ$--product, noting that it descends to a map \[QE_* \Susp^{-n} \OS{E}{n} \otimes QE_* \Susp^{-m} \OS{E}{m} \to QE_* \Susp^{-(n+m)} \OS{E}{n+m}.\]  The cooperations that survive to the $\ast$--indecomposables are exactly the additive ones.

The context of all unstable cooperations: don't pass to the indecomposable quotient.  This is pretty complicated.

Unbalanced unstable cooperations and Morita functors: I feel that given an unstable $E^*$--comodule $M$, one can produce an unstable $F^*$--comodule by a tensoring operation: $M \otimes_? F_* \OS{E}{*}$ (or something?).  If $M$ is actually $M = E^* X$ for a space $X$, then there should be a comparison map \[F^* X \to E^* X \otimes_? F_* \OS{E}{*}.\]  This will rarely be an isomorphism, but it should fit into the sheaf-over-a-simplicial-scheme picture as something like a face map.

Associated Adams spectral sequences and completions?

Comparison with past scholia on ``piles''?

\citeme{Bendersky Curtis Miller's~\cite{BCM} \textit{The unstable Adams spectral sequence for generalized homology}}
\citeme{Boardman Johnson Wilson's \textit{Unstable operations in generalized cohomology}}

Goerss mentions inverting the suspension element as being the same as stabilizing. This is a nice picture. Cf.\ Example 10.5 of Goerss's \textit{Hopf rings, Dieudonn\'e modules, and $E_* \Loops^2 S^3$}.

Lukas's main suggestion was to consider how the shriek functors were computed in excellent cases, so that maybe the push-pull formula could be directly interpreted on the level of spectra.  I think this is a good thing to think about.  Maybe it will tell us how ``handed'' Hopf rings appear in isolation, without their chirally dual mate.

Talk about how $\S \to E$ is being treated like a cover. If you have two covers, you can take their joint cover $\S \to E \vee F$, whose $1$--simplices pick up a mixed term: \[(E \vee F) \sm (E \vee F) \simeq E^{\sm 2} \vee F^{\sm 2} \vee (E \sm F)^{\vee 2}.\]  This mixed term prevents double-counting, and it makes it sort of reasonable that it should involve isomorphisms between mixed formal group types.  (This mixed cooperations perspective might have to wait until the chapter on unstable contexts.)

I'm fairly certain that the monad associated to the unstable composite adjunction is monoidal.  It show immediately follow that the unstable $E$--homology descent object gives a quasicoherent sheaf over the unstable $E$--homology context formed by taking the unstable $E$--homologies of all of the spheres\ldots

Does the adjective ``Cartesian'' make sense here?  Is it still the case that the topological object is always ``Cartesian'', and $\pi_*$ preserves Cartesianness in the presence of loads of flatness hypotheses on $E_* \OS{E}{*}$?





\section{Unstable cooperations in additive homology}\label{UnstableSteenrodCoops}

The bar spectral sequence: connectivity assumption, it's a spectral sequence of Hopf algebras, interaction with the $\circ$--product structure

Restricting to additive cooperations: the indecomposable quotient

The homology suspension element, the relation $e \circ e = b_{(0)}$, and passing to the stabilization

\todo{Neil's MO answer about $H_* K(\Z, 3)$}


-----

\begin{example}
Because $H\Z/2$ has field coefficients, it always has K\"unneth isomorphisms, and in particular we can study the Hopf ring \[A_{s, t} = (H\Z/2)_s \OS{H\Z/2}{t} = H_s(K(\Z/2, t); \F_2)\] for $E = H\Z/2$ and $F = H\Z/2$.  We compute this object in three steps.

First, recall that there is a canonical isomorphism $H^*(\RP^\infty; \F_2) \cong \F_2[x]$ for $|x| = 1$.  In turn, the homology groups $H_*(\RP^\infty; \F_2)$ can be written as \[H_*(\RP^\infty; \F_2) \cong \F_2\{a_0, a_1, a_2, \ldots, a_n, \ldots\}\] for generators $a_n$ of degree $n$ with Kronecker pairing $\<x^m, a_n\> = \delta{}^m{}_n$.  For degree reasons, the diagonal on cohomology takes the form $\Delta x = x \otimes 1 + 1 \otimes x$ (cf.\ \Cref{HF2StackExample}), from which it follows that the $\ast$--algebra structure on homology takes the form of a divided power algebra on a single class: \[H_*(\RP^\infty; \F_2) \cong \Gamma[a_\emptyset] \cong \bigotimes_{j=0}^\infty \F_2[a_{(j)}] / (a_{(j)})^2.\]  Mysteriously, we have labeled the generating class ``$a_\emptyset$'' and defined $a_\emptyset^{[2^j]} =: a_{(j)}$. This notation will align with complications later in the example.

Second, using the fact that $\RP^\infty \simeq K(\Z/2, 1) \simeq B\Z/2$ has a bar filtration, we study the associated bar spectral sequence, which has the signature \[E^2_{*, *} = \Tor^{H_*(\Z/2; \F_2)}_{*, *}(\F_2, \F_2) \Rightarrow H_*(K(\Z/2, 1); \F_2).\]  The homology of the discrete group $\Z/2$ can be expressed as an algebra as
\[
H_*(\Z/2; \F_2) \cong \left. \F_2[\underline 0, \underline 1] \middle/ \left(\begin{array}{c} \underline 0 = 1 \\ \underline 1^{\ast 2} = \underline 0\end{array}\right) \right. \cong \F_2[\underline 1 - 1] / (\underline 1 - 1)^{\ast 2} =: \F_2[a_\emptyset] / a_\emptyset^{\ast 2},
\]
where $\underline n$ denotes the degree--zero class representing the point $n$ in $\Z/2$ and we have additionally defined $a_\emptyset := \underline 1 - 1$.  The $\Tor$ groups of this truncated polynomial algebra form a divided power algebra~\cite{TateResolutions}: \[\Tor^{H_*(\Z/2; \F_2)}_{*, *} \cong \Gamma[a_\emptyset].\]  It follows that the bar spectral sequence collapses at $E^2 = E^\infty$, and there are no differentials.

Third, we use the second step as the base of an inductive argument, powered by Ravenel and Wilson's key lemma~\cite[Theorem 2.2]{RavenelWilson}, to analyze the other bar spectral sequences: \[\Tor^{H_*(K(\Z/2, t); \F_2)}_{*, *}(\F_2; \F_2) \Rightarrow H_*(K(\Z/2, t+1); \F_2).\]  Inductively assume that the $t$\th level $A_{*, t}$ of the Hopf ring is an exterior $\ast$--algebra on classes which are $t$--fold $\circ$--products of the classes $a_{(j)}$.  It follows that the $\Tor$ groups of the bar spectral sequence computing $A_{t+1, *}$ form a divided power algebra generated by the same $t$--fold $\circ$--products.  An analogue of another Ravenel--Wilson lemma~\cite[Lemma 9.5]{RavenelWilson} gives a congruence\footnote{It's conceivable that this congruence can be repaired to an equality, since the $2$--series for $\G_a$ is so abbreviated.  I have not worked this out.} \[(a_{(j_1)} \circ \cdots \circ a_{(j_t)})^{[2^{j_{t+1}}]} \equiv a_{(j_1)} \circ \cdots \circ a_{(j_t)} \circ a_{(j_{t+1})} \pmod{\text{decomposables}}.\]  It follows from the key lemma~\cite[Theorem 2.2]{RavenelWilson}, which lets us transport differentials from earlier bar spectral sequences to the current one by applying the $\circ$--product, that the differentials vanish:
\begin{align*}
d((a_{(j_1)} \circ \cdots \circ a_{(j_t)})^{[2^{j_{t+1}}]}) & \equiv d(a_{(j_1)} \circ \cdots \circ a_{(j_t)} \circ a_{(j_{t+1})}) \pmod{\text{decomposables}} \\
& = a_{(j_1)} \circ d(a_{(j_2)} \circ \cdots \circ a_{(j_{t+1})}) = 0.
\end{align*}
Hence, the spectral sequence collapses and the induction holds.  It follows that \[A_{*, *} \stackrel\cong\leftarrow \bigoplus_{t=0}^\infty (H_*(\RP^\infty; \F_2))^{\wedge t},\] where $(-)^{\wedge t}$ denotes the $t$\textsuperscript{th} exterior power in the category of Hopf algebras.  The leftward direction of this isomorphism is realized by the $\circ$--product.
\end{example}

\begin{definition}
We spell out some of the Hopf algebra constructions named above.  For a cocommutative $R$--coalgebra $C$, we define its free commutative and cocommutative Hopf $R$--algebra~\cite{Takeuchi} to have underlying algebra \[\frac{\operatorname{SymmetricAlgebra} \left(C \otimes_R (\chi C)\right)}{\left( \begin{array}{c} c \otimes \chi c = 1 \end{array} \right)}\] with diagonal \[\Delta(c_1 \otimes \cdots \otimes c_k \otimes \chi c'_1 \otimes \cdots \otimes \chi c'_{k'}) = \Delta c_1 \otimes \cdots \otimes \Delta c_k \otimes \chi (\Delta c'_1) \otimes \cdots \otimes \chi(\Delta c'_{k'})\] and antipode \[\chi(c_1 \otimes \cdots \otimes c_k \otimes \chi c'_1 \otimes \cdots \otimes \chi c'_{k'}) = \chi c_1 \otimes \cdots \otimes \chi c_k \otimes c'_1 \otimes \cdots \otimes c'_{k'}.\]  Then, given a Hopf $R$--algebra $A$, we define the free Hopf ring~\cite[Definition 4.2, Proposition 2.16]{HuntonTurner} to be \[\left. \bigoplus_{k=0}^\infty A^{\wedge_R k} \middle/ \left(x \wedge y = \sum_i (x'_i \ast y') \wedge (x''_i \ast y'') \middle| \begin{array}{c} y = y' \ast y'', \\ \Delta x = \sum_i x'_i \otimes x''_i \end{array} \right) \right.\] with $\circ$--product given by the natural maps $A^{\wedge_R n} \otimes_R A^{\wedge_R m} \to A^{\wedge_R (n+m)}$.
\end{definition}

\begin{remark}
The odd--primary analogue of this result appears in Wilson's book~\cite[Theorem 8.5]{Wilson}.  In that situation, the bar spectral sequences do not degenerate but rather have a single family of differentials, and the result imposes a single relation on the free Hopf ring.
\end{remark}

\todo[inline]{Pass to the additive unstable cooperations and note that you recover the \emph{monoid of endomorphisms} of $\G_a$.  In algebraic terms, this looks like the dual Steenrod algebra where $\xi_0$ has not been inverted.}











\section{Complex-orientable cooperations}

Our goal for today is to study the mixed unstable cooperations $E_* \OS{MU}{2*}$ for a complex-orientable cohomology theory $E$.

----

\needproof{Coalgebraic formal schemes}

Let $R$ and $S$ be graded rings.  We can form a Hopf ring over $R$ by forming the ``ring--ring'' $R[S]$: as an $R$--module, this is free and generated by symbols $[s]$ for $s \in S$.  The two Hopf ring products $\ast$ and $\circ$ are determined by the formulas
\begin{align*}
[s] \ast [s'] & = [s + s'], &
[s] \circ [s'] & = [s \cdot s'],
\end{align*}
and distributivity over $\circ$ over $\ast$ is enforced by an $R$--module quotient.

Given an $R$--coalgebra $C$, we can form the free commutative Hopf algebra on $C$ by taking its associated symmetric algebra.  This is a degenerate case of a free Hopf ring construction: taking $S$ to be an auxiliary ring, we can form a free Hopf ring under $R[S]$ spanned by $R[S]$ and free $\ast$-- and $\circ$--products of elements of $C$.

\begin{definition}\todo{I don't like the upper-$R$ notation.  Having a scheme theoretic description of this object should let us pick a better name.}
Let $E_*^R \OS{G}{*}$ be the quotient of the free Hopf ring under $E_*[G^*]$ generated by $C = E_* \CP^\infty$ by the relation \[b(s) +_{[G]} b(t) = b(s +_E t),\] where $b(s) = \sum_i b_i x^i$ and hence
\begin{align*}
b(s +_E t) & = \sum_n b_n \left(\sum_{i, j} a_{ij}^E s^i t^j \right)^n, \\
b(s) +_{[G]} b(t) & = \bigast_{i, j} [a_{ij}^G] \circ \left( \sum_k b_k s^k \right)^{\circ i} \circ \left( \sum_\ell b_\ell t^\ell \right)^{\circ j}.
\end{align*}
\end{definition}

\begin{lemma}
There is a natural map $E_*^R \OS{G}{*} \to E_* \OS{G}{*}$ for complex-orientable ring spectra $E$ and $G$.
\end{lemma}
\begin{proof}
\citeme{Theorem 3.8 of RW, an easy pullback argument}
\end{proof}

\begin{theorem}
The coalgebraic formal scheme $\Sch Q^* E_*^R \OS{G}{*}$ gives a model \[\Sch Q^* E_*^R \OS{G}{*} \cong \InternalHom{FormalGroups}(\CP^\infty_E, \CP^\infty_G).\]
\end{theorem}
\begin{proof}
\todo{I don't know how to prove this. I assume it's true, though.}
\end{proof}

----
\citeme{Pages 266--270 of Ravenel--Wilson, especially the bottom of 268.}

We begin by calculating $H\F_{p*} \OS{BP}{2*}$.  One might think that this is a first guess at an accomplishable task, but we will quickly show that it is, in a certain sense, the universal example of this calculation.  We will show the following theorem:
\begin{theorem}\label{HFpBPCooperationsTheorem}
The natural homomorphism \[H\F_{p*}^R \OS{BP}{2*} \to H\F_{p*} \OS{BP}{2*}\] is an isomorphism.  In particular, $H\F_{p*} \OS{BP}{2*}$ is even--concentrated.
\end{theorem}
\noindent We will prove this by a fairly elaborate counting argument, and as such our first move will be to produce an upper bound for the size of the source Hopf ring.  To abbreviate notation, we will write $H$ for $H\F_p$ for the remainder of the lecture.

\begin{lemma}
As a $\circ$--algebra, \[Q^* H_0^R \OS{BP}{2*} \cong \F_p[[v_n] - [0_{-|v_n|}] \mid n \ge 1],\] where $0_{-|v_n|}$ denotes the null element of $BP^{|v_n|}(*)$.
\end{lemma}
\begin{proof}
By construction, $H_0^R \OS{BP}{2*}$ is the Hopf ring-ring $\F_p[BP^*]$.  Calculating modulo $\ast$--decomposables,
\begin{align*}
0 & = ([x] - [0]) \ast ([y] - [0]) \\
& = [x + y] - [x] - [y] + [0] \\
& = ([x + y] - [0]) - ([x] - [0]) - ([y] - [0]).
\end{align*}
It follows that a $\Z_{(p)}$--basis of $BP^*$ gives an $\F_p$--basis of $Q^* \F_p[BP^*]$ under the map $x \mapsto [x] - [0]$.  Finally, since \[([x] - [0]) \circ ([y] - [0]) = ([xy] - [0]),\] we see that the image of the polynomial generators $v_n$ freely generate the $\circ$--product structure, proving the lemma.
\end{proof}

Directly from the definition of $H_*^R \OS{MU}{2*}$, we now know that $Q^* H_*^R \OS{MU}{2*}$ is generated by $[v_n] - [0_{-|v_n|}]$ for $n \ge 1$ and $b_j$, $j \ge 0$.  In fact, it suffices to consider $b_{p^i} = b_{(i)}$, $i \ge 0$.\citeme{Lemma 4.14 of Ravenel Wilson}

\textbf{Insert this somehow. Probably it belongs back in an earlier discussion:}
\begin{lemma}
In $Q^* H_* \OS{BP}{2} / I^{\circ 2} \circ Q^* H_* \OS{BP}{2}$, for any $n$ we have \[\sum_{i=1}^n [v_i] \circ b_{(n-i)}^{\circ p^i} = 0.\]
\end{lemma}
\begin{proof}
\citeme{Lemma 3.14 of Ravenel Wilson}
\end{proof}

\begin{corollary}
Let $r_n$, the $n${\th} relation, denote the same sum taken in $Q^* H_*^R \OS{BP}{2*}$ instead.  Then $r_n$ is in the ideal generated by $I^{\circ 2}$. \qed \todo{Ravenel and Wilson are working with $\OS{MU}{2*}$ here, so they also include $[x_{2i}]$ for $i \ne p^j - 1$.  I don't think we need to.}
\end{corollary}

\begin{lemma}\citeme{Lemma 4.15.b of Ravenel Wilson. Very tedious.}
The sequence $(r_1, r_2, \ldots)$ is regular in the polynomial algebra \[A = \F_p[[x_{2i}], b_{(j)} \mid i > 0, j \ge 0]. \qed\] 
\end{lemma}

\begin{lemma}
Set
\begin{align*}
c_{i,j} & = \dim_{\F_p} Q^* H_i^R \OS{BP}{2j}, &
d_{i,j} & = \dim_{\F_p} \F_p[[v_n], b_{(0)}]_{i,j}.
\end{align*}
Then $c_{i,j} \le d_{i,j}$ and $d_{i,j} = d_{i+2,j+1}$.
\end{lemma}
\begin{proof}
We have seen that $c_{i,j}$ is bounded by the $\F_p$--dimension of $\F_p[[v_n], b_{(j)}]_{i,j}$ modulo the ideal $(r_1, r_2, \ldots)$.  But, since this ideal is regular and $|r_j| = |b_{(j)}|$, this is the same count as $d_{i,j}$.  The other relation among the $d_{i,j}$ follows from multiplication by $b_{(0)}$, with $|b_{(0)}| = (2, 1)$.\todo{We also need that one of the bidegrees of $[v_n]$ is zero, right?}
\end{proof}

This completes the estimation portion of the argument.  We now turn to showing that this estimation is \emph{sharp} and that the natural map is \emph{onto}, and hence an isomorphism, using the bar spectral sequence.  Recalling that the bar spectral sequence converges to a the homology of the \emph{connective} delooping, let $\OS{BP}{2*}'$ denote the connected component of $\OS{BP}{2*}$ containing $[0_{2*}]$.  We will then demonstrate the following theorem inductively:
\begin{theorem}\label{HFpBPCooperationsInduction}
Take $k$ to be the induction index.
\begin{enumerate}
\item $Q^* H_{\le 2(k-1)} \OS{BP}{2j}'$ is generated by $\circ$--products of the $[v_n]$ and $b_{(j)}$.
\item $H_{\le 2(k-1)} \OS{BP}{2*}'$ is isomorphic to a polynomial algebra in this range.
\item For $0 < i \le 2(k-1)$, we have $d_{i,j} = \dim_{\F_p} Q^* H_i \OS{BP}{2j}$.
\end{enumerate}
\end{theorem}

Before addressing the theorem, we show that this finishes our calculation:
\begin{proof}[{Proof of \Cref{HFpBPCooperationsTheorem}, assuming \Cref{HFpBPCooperationsInduction} for all $k$}]
Recall that we are considering the natural map \[H_*^R \OS{BP}{2*} \to H_* \OS{BP}{2*}.\]  The first part of \Cref{HFpBPCooperationsInduction} shows that this map is a surjection.  The third part of \Cref{HFpBPCooperationsInduction} together with our counting estimate shows that the induced map \[Q^* H_*^R \OS{BP}{2*} \to Q^* H_* \OS{BP}{2*}\] is an isomorphism.  Finally, the second part of \Cref{HFpBPCooperationsInduction} says that the original map, before passing to $\ast$--indecomposables, must be an isomorphism as well.
\end{proof}

\begin{proof}[{Proof of \Cref{HFpBPCooperationsInduction}}]
The infinite loopspaces in $\OS{BP}{2*}$ are related by $\Loops^2 \OS{BP}{2(*+1)}' = \OS{BP}{2*}$, so we will use two bar spectral sequences to extract information about $\OS{BP}{2(*+1)}'$ from $\OS{BP}{2*}$.  Since we have assumed that $H_{\le 2(k-1)} \OS{BP}{2*}$ is polynomial in the indicated range, we know that in the first spectral sequence \[E^2_{*, *} = \Tor^{H_* \OS{BP}{2*}}_{*, *}(\F_p, \F_p) \Rightarrow H_* \OS{BP}{2*+1}\] the $E^2$--page is, in the same range, exterior on generators in $\Tor$--degree $1$ and topological degree one higher than the generators in the polynomial algebra.  Since differentials lower $\Tor$--degree, the spectral sequence is multiplicative, and there are no classes on the $0$--line, it collapses in the range $[0, 2k-1]$.  Additionally, since all the classes are in odd topological degree, there are no algebra extension problems, and we conclude that $H_* \OS{BP}{2(k-1)+1}$ is indeed exterior up through degree $(2k-1)$.

We now consider the second bar spectral sequence \[E^2_{*, *} = \Tor^{H_* \OS{BP}{2*+1}}_{*, *}(\F_p, \F_p) \Rightarrow H_* \OS{BP}{2(*+1)}.\]  The $\Tor$ algebra of an exterior algebra is divided power on a class of topological dimension one higher.  Since these classes are now all in even degrees, the spectral sequence collapses in the range $[0, 2k]$.  Additionally, these primitive classes are related to the original generating classes by double suspension, i.e., by circling with $b_{(0)}$.  This shows the first inductive claim on the primitive classes through degree $2k$, and we must work to show that this is enough.\todo{This isn't done: see the bottom of page 269 in Ravenel--Wilson.  It's an easy argument: the differences between $\gamma_{p^i}(x)$ and the classes captured thus far all lie in lower filtration.}

\todo{Then, the top of page 270 shows claims 2 and 3.  The point is that we have already established surjectivity, so we can now use our size bounds.  The algebra is too big if there are any leftover indecomposables, so everything is decomposable in terms of the generators we've found, and we're furthermore exactly of the right size to be polynomial.}
\end{proof}

\todo{This argument is distinct from the argument on the next day, which induces on $j$ rather than on $i$.  That's worth making a point of.}

------

\subsection*{Now walk down to any complex-orientable $E$}

The main theorem: The Ravenel--Wilson calculation of $E_* \OS{MU}{*}$ for complex orientable $E$

This follows from: Since $H\Z_* \OS{MU}{2*}$ is $\Z$--free and even, $MU_* \OS{MU}{2*}$ is $MU_*$--free and even.  It follows that since the free Hopf ring map to $H\Z_* \OS{MU}{2*}$ is an isomorphism, so is the one tensored up to $MU_*$.

This follows from: Calculation of $H\Z_* \OS{MU}{2*}$: no torsion, since $H\F_{p*} \OS{MU}{2*}$ is even, and since the free map is an isomorphism for all primes it's an isomorphism integrally

This follows from: Since $MU_{(p)}$ splits multiplicatively as a wedge of $BP$s, it suffices to show the freeness and evenness of $H\F_{p*} \OS{BP}{2*}$.














\section{Cooperations between theories at geometric points}

\todo{This isn't a good title. You specifically mean from the additive point to a finite height point.}

Throughout today, we will write $K$ for a Morava $K$--theory $K_\Gamma$ (which, if you like, you can take to be $K(d)$) and $A$ for a finitely generated abelian group, and $H$ for the associated Eilenberg--Mac Lane spectrum.  Our goal is to study the unstable mixed cooperations $K_* \OS{H}{*}$.  Of course, this fits into our broader program of understanding various forms of unstable cooperations, especially if we were to pursue a ``stalkwise analysis'' of the sort in \Cref{ChapterFiniteSpectra}.  However, this calculation is especially interesting because of the appearance of the Eilenberg--Mac Lane spaces $\OS{H}{*}$ in other settings.  For instance, if we want to analyze the $K$--homology of a Postnikov tower (as we will in \Cref{ChapterSigmaOrientation}), we will naturally encounter pieces of $K_* \OS{H}{*}$, and we would be wise to have a firm handle on these objects.  It is another tribute to the power of structure that the successful way to approach this computation is not one-at-a-time, as one coming from the Postnikov perspective might attempt, but all-at-once, as suggested by the unstable cooperations picture.

Our calculation will eventually turn into an induction, so we will pursue a simple example first: the $K$--theory of just the classifying space $BA$, rather than a general Eilenberg--Mac Lane space.  Since $K$--theory has K\"unneth isomorphisms and $B(A_1 \times A_2) \simeq BA_1 \times BA_2$, it suffices to do the computation just for $A = C_{p^j}$.

\begin{theorem}\citeme{Theorem 5.7 of RW, or Prop 2.4.4 of HL}
There is an isomorphism \[BS^1[p^j]_K \cong BS^1_K[p^j].\]
\end{theorem}
\begin{proof}
Consider the diagram of spherical fibrations:\todo{Put in a pullback corner here.}
\begin{center}
\begin{tikzcd}
S^1 \arrow{r} \arrow[-,double]{d} & B(S^1[p^j]) \arrow{r} \arrow{d} & BS^1 \arrow{d}{p^j} \\
S^1 \arrow{r} & ES^1 \arrow{r} & BS^1.
\end{tikzcd}
\end{center}
The induced long exact sequence (known as the Gysin sequence, or as the couple in the Serre spectral sequence for the first fibration) takes the form
\begin{center}
\begin{tikzcd}
& K_* BS^1 \arrow{rd}{- \frown [p^j](x)} \\
K_*(BS^1[p^j]) \arrow{ru} & & K_* BS^1 \arrow{ll}{\partial}
\end{tikzcd}
\end{center}
where $x$ is a coordinate on $BS^1_K$.  Because $BS^1_K$ is of finite height, the right diagonal map is surjective.  It follows that $\partial = 0$, and so this gives a short exact sequence of Hopf algebras, which we can reinterpret as a short exact sequence of group schemes \[B(S^1[p^j])_K \to BS^1_K \xrightarrow{p^j} BS^1_K. \qedhere\]
\end{proof}

There are a couple of approaches to the rest of this calculation, i.e., $K_* \OS{H}{q}$ for $q > 1$.  The original, due to Ravenel and Wilson, is to complete the calculation for the smallest abelian group $C_p$ and then induct upward toward more complicated groups like $C_{p^j}$ and $C_{p^\infty}$.  More recently, there is also a preprint of Hopkins and Lurie that begins with $A = C_{p^\infty}$ and then works downward.  We will do the \emph{easy} parts of both calculations, to give a feel for their relative strengths and deficiencies.

The Ravenel--Wilson version of the calculation proceeds much along the same lines as \Cref{UnstableSteenrodCoops}.  We will study the bar spectral sequences \[\Tor^{K_* \OS{H}{q}}_{*, *}(K_*, K_*) \Rightarrow K_* \OS{H}{q+1}\] for different indices $q$ and use the $\circ$--product to push differentials around among them.  We begin by rephrasing the calculation above in terms of the case $q = 0$.\citeme{Theorem 8.1 of RW}  In that setting, the ground algebra is given by \[K_* \OS{H\Z/p^j}{0} = K_*[[1]] / \<[1]^{p^j} - 1\> = K_*[[1] - [0]] / \<[1] - [0]\>^{p^j}.\]  Then, the $\Tor$--algebra for the truncated polynomial algebra $K_*[a_\emptyset] / a_\emptyset^{p^j}$ is given by the formula \[\Tor^{K_*[a_\emptyset] / a_\emptyset^{p^j}}_{*, *}(K_*, K_*) = \Lambda[\sigma a_\emptyset] \otimes \Gamma[\phi a_\emptyset],\] the combination of an exterior algebra and a divided power algebra.  We know which classes are supposed to survive this spectral sequence, and hence we know where the differentials must be:
\begin{align*}
d(\phi a_\emptyset)^{[p^{dj}]} & = \sigma a_\emptyset, \\
\Rightarrow d(\phi a_\emptyset)^{[i + p^{dj}]} & = \sigma a_\emptyset \cdot (\phi a_\emptyset)^{[i]}.
\end{align*}
After this differential the spectral sequence collapses, but there are some multiplicative extensions to sort out when $j > 1$.  Of course, these are all determined by already knowing the multiplicative structure on $K_* \OS{H\Z/p^j}{1}$.

We now turn to the general finite case:
\begin{theorem}\citeme{Theorems 9.2 and 11.1 of RW}
Using the $\circ$--product, \[K_* \OS{H\Z/p^j}{q} = \Alt^q \OS{H\Z/p^j}{1}.\]
\end{theorem}
\begin{proof}[Proof sketch]
The inductive step turns out to be extremely index-rich, so I won't be so explicit or complete, but I'll point out the major landmarks.  It will be useful to use the shorthand $a_{(i)} = a_\emptyset^{[p^i]}$, where $(i)$ is thought of as a multi-index with one entry.

We proceed by induction, assuming that $K_* \OS{H\Z/p^j}{q} = \Alt^q \OS{H\Z/p^j}{1}$ for a fixed $q$.  Computing the algebraic homology of $K_* \OS{H\Z/p^j}{q}$ yields a tensor of divided power and exterior classes, a pair for each algebra generator of $K_* \OS{H\Z/p^j}{q}$.  There is then a wonderful rewriting formula:\citeme{Ravenel Wilson, somewhere} \[(\phi a_{(i_1, \ldots, i_q)})^{[p^j]} \equiv  (\phi a_{(i_1, \ldots, i_{q-1})})^{[p^j]} \circ a_{(i_q + j)} \mod *\text{--decomposables}.\]  Since every class can be so decomposed, all the differentials and extensions are determined by the previous spectral sequence.  In particular, classes are hit by differentials exactly when $i_q + j$ is large enough.  It follows that the inductive assumption that $K_* \OS{H\Z/p^j}{q+1}$ is an exterior power holds, and the class $(\phi a_{(i_1, \ldots, i_q)})^{[p^j]}$ represents $a_{(j, i_1 + j, \ldots, i_q + j)}$.\todo{Indicate exactly where the intertwining between different values of $j$ happens.}
\end{proof}

\todo{Mike has said something about the pairing $C_{p^j} \times C_{p^j}^* \to \Q/\Z$ not being functorial in $j$ (so as to pass to the direct limit) which gave me pause.  I should make sure I'm not messing something up here.}

The case $j = 1$ of this proof is messy enough, and the case of a general $j$ requires interrelating the cases using the restriction map $C_{p^j} \to C_{p^{j+1}}$ and the projection map $C_{p^{j+1}} \to C_{p^j}$.  Then, these tools are revisited to give a computation in the limiting case $A = C_{p^\infty}$, where there's a $p$--adic equivalence $HC_{p^\infty} \simeq\widehat{{}_p} \Susp H\Z$.\citeme{Theorem 12.4 of RW}  The calculation in this setting is the most interesting one of all --- after all, it contains the case $BS^1_K$, which is of special interest to us.  Remarkable, it is easier to access directly than passing through all of this intermediate work.  To begin, we need the following algebraic calculation:

\begin{theorem}\citeme{Theorem 2.2.10 of Hopkins--Lurie}\todo{Use the same cohomology notation you have been using for the cohomology of formal groups on previous days.}
Suppose we have an exact sequence of Hopf $k$--algebras \[k \to A' \to A \xrightarrow{u} A'' \to k\] such that $A$ is connected and $F$--divisible, $A'$ is finite--dimensional, and the map $u$ factors through the relative Frobenius ${A''}^{(p)} \to A''$.  Write $\partial: \Ext^1 A' \to \Ext^2 A''$ for the going--around map.  The following are true:
\begin{itemize}
\item The Hopf algebra $A''$ is connected and $F$--divisible.
\item The map $\partial$ induces an isomorphism \[\Sym^* \Ext^1 A' \to \Ext^* A''.\]
\item Let $y_1, \ldots, y_n$ form a basis for $\Ext^1 A'$.  Then $\Ext^* A'$ is freely generated by the elements $y_*$ as a module over $\Ext^* A$. \qed
\end{itemize}
\end{theorem}

\begin{corollary}\citeme{Example 2.2.12 of Hopkins--Lurie}
If $A$ is a connected $p$--divisible Hopf $k$--algebra, then \[k \to A[p^j] \to A \xrightarrow{p^j} A \to k\] is such an exact sequence.  Hence, $\Ext^* A$ is isomorphic to the symmetric algebra on $\Ext^1 A[p^j]$.  \qed
\end{corollary}


\todo[inline]{The point is supposed to be that there are no differentials in the limiting $j \to \infty$ case of the above spectral sequences.  So, all that's left is to check that the pairing gives the right isomorphism, which is an exercise in counting dimensions.  This is Prop 2.4.11 in Hopkins--Lurie.}





\begin{remark}
You'll notice that in $K_* \OS{H}{q+1}$ if we let the $q$--index tend to $\infty$, we get the $K$--homology of a point.  This is another way to see that the stable cooperations $K_* H$ vanish, meaning that the \emph{only} information present comes from unstable cooperations.\todo{We could even provide a quick proof of the stable calculation?  Cf.\ http://mathoverflow.net/questions/220952/localization-at-the-johnson-wilson-spectrum-and-rationalization.}
\end{remark}

\begin{remark}
Since everything in sight is even, you also get a calculation of the $E$--theory for free.  In fact, as Hopf algebras, \[E_\Gamma \OS{H(\Q/\Z)}{q} \simeq \Alt^q E_\Gamma \OS{H(\Q/\Z)}{1}.\]
\end{remark}

\begin{remark}
Also mention the base--change type formula: as an abelian group, a positive characteristic field $k$ splits as a wedge of $C_p$s, so \[K_* \OS{Hk}{*} = K_* \OS{H\F_p}{*} \otimes_{\F_p} k.\]
\end{remark}





Maybe talk about some consequences: the Hopkins--Ravenel--Wilson results on finite Postnikov towers and so on?








\section{Dieudonn\'e modules}

\todo[inline]{This section is out of logical order with the B--G one.  Goerss's proof that $\Loops^2 S^3$ has anything to do with anything relies on knowing that the Snaith splitting of $\Loops^2 S^3$ into Brown--Gitler spectra is directly relevant to Dieudonn\'e theory.  I think this section really should be an exposition of the algebraic ideas involved (perhaps closing with the Gross--Hopkins period map?) and that all topological applications should be postponed until the next lecture.}

Three approaches: projective generators, the de Rham crystal, and Cartier's curves functor

$\CatOf{HopfAlgebras}^{> 0}_{\F_p/}$

Goerss's result on $H_* \OS{E}{*}$ for Landweber flat $E$

\todo{Dieudonn\'e theory is also about taking primitives in some sort of cohomology.  Can this be connected to the additivity condition on unstable operations?}


Goerss also talks about ``Hopf ring hom'', and how, since many of the Hopf rings appearing in algebraic topology are ``free'', Hopf ring hom off of them agrees with just Hopf algebra hom (or Dieudonne module hom) off of their generating object.  That's probably worth pointing out, since nonadditive unstable operations seem so unwieldy.


You can write down a tensor product for Hopf algebras.  Equation 7.6 has a description of the box tensor functor for Dieudonn\'e modules.  The Dieudonn\'e module functor is monoidal for these two products.


As a running example, you should rephrase a bunch of the results from the previous lecture (on $K(d)_* \OS{Hk}*$) in terms of Dieudonn\'e modules.


There's also Chapter 3 of Hopkins--Lurie, which constructs an alternating power group scheme without passing through Dieudonn\'e modules.  It should at least be mentioned.







\section{Brown--Gitler spectra}

Classification of finite Hopf algebras over $\F_p$ with $\Gm$--action

There's a collection of Hopf algebras called $H(n)$ which are the projective covers of the standard Hopf algebras $S(n)$, themselves described as symmetric Hopf algebras on a single class in dimension $n$.  The $H(n)$ form a series of projective \emph{generators} of the category.  So, we can give a Tannakian description of the category as modules over the ``endomorphism ring'' $\mathsf{Hom}(H(*), H(*))$, which is calculable.  You end up seeing the same data as a covariant Dieudonn\'e module, which is cool, and generally given a hopf algebra $H$ you can build a graded Dieudonn\'e module functor by \[D_n(H) = \CatOf{HopfAlgebras}_{/\F_p}(H(n), H).\]

Take $X \to Y \to Z$ to be a cofiber sequence of spectra.  Goerss's main theorem is that for $n \not\equiv \pm 1 \pmod{2p}$, there is a short exact sequence of Hopf algebras \[D_n H_* \Loops^\infty X \to D_n H_* \Loops^\infty Y \to D_n H_* \Loops^\infty Z.\]  (This really requires the congruence condition on $n$ and some series study of fibrations of infinite loopspaces due to Moore and Smith.)  It follows from Brown representability that, subject to this congruence condition, there is a spectrum $B_n$ such that \[B_{n*}(X) = D_n H_* \Loops^\infty X.\]

Then he shows that these recover Brown--Gitler spectra: $H^* B_n$ looks like a chunk of the mod--$p$ Steenrod algebra, up through degree dependent on $n$, and also the natural map $(B_n)_n Z \to H_n Z$ is surjective for finite CW--complexes $Z$.  The second part doesn't seem to be very hard, cf.\ Lemma 3.3.  The cohomology statement is pretty cool: computing homology instead, you get \[H_k B_n = (B_n)_k H\Z = D_n H_* \Loops^\infty K(\Z, n - k) = [Q H_* \Loops^\infty K(\Z, n-k)]_n,\] where the last equality relies on the congruence condition.

A lot of this requires knowing things about the behavior of Cotor spectral sequences and unstable modules over the Steenrod algebra.




\todo[inline]{Ask Mike (and Jacob?) if there are analogues of these results for $kO$ which explain Mahowald's generalized $K$--theoretic Brown--Gitler spectra.}







\subsection*{Things that belong in this chapter}

Theorem 6.1 of R--W \textit{The Hopf ring for complex bordism} sounds like something related to Quillen's elementary proof.

Section III.11 of Wilson's \textit{Primer} has a synopsis of how additive unstable operations should be treated.  (In particular, he remarks on pp.\ 62-3 that primitives in unstable operations are the \emph{additive} unstable operations, which seems important.)  Possibly this is enough to understand how additive unstable cooperations should be treated, or maybe unstable cooperations generally.

There's also a document by Boardman, Johnson, and Wilson (Chapter 2 of the \textit{Handbook of Algebraic Topology}) that discusses an equivalence between Steve's approach and ``unstable comodules''.  Please read this.

Snowball a discussion of coalgebraic formal schemes into one of these sections.





