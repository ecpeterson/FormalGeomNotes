% -*- root: main.tex -*-

\chapter{Unstable cooperations}

\todo{Write an introduction for me.}





\todo[inline]{This section can probably be introduced as trying to enrich the notion of a context for cohomology theories swallowing \emph{spaces}.  The stable cooperations are totally captured by the notion of a context, but there are two approaches to (distinct sorts of) unstable operations: Hopf rings and power operations.  We plan on talking about both (and they both have something to say about $MU\<6\>$), so this can be a common thread.}




\section{Unstable contexts}

Hopf rings: algebraic and topological

\begin{definition}\todo{Can this definition be made without specifying the grading as such and instead using a $\G_m$--action?}
A Hopf ring $A_{*, *}$ over a graded ring $R_*$ is itself a graded ring object in the category $\CatOf{Coalgebras}_{R_*}$, sometimes called an $R_*$--coalgebraic graded ring object.  It has the following structure maps:
\begin{align*}
+ & \co A_{s, t} \times A_{s, t} \to A_{s, t} & \text{($A_{s, t}$ is an abelian group)} \\
\cdot & \co R_{s'} \otimes_{R_*} A_{s, t} \to A_{s+s', t} & \text{($A_{*, t}$ is a $R_*$--module)} \\
\Delta & \co A_{s, t} \to \bigoplus_{s' + s'' = s} A_{s', t} \otimes_{R_*} A_{s'', t} & \text{($A_{*, t}$ is a $R_*$--coalgebra)} \\
\ast & \co A_{s, t} \otimes_{R_*} A_{s', t} \to A_{s + s', t} & \text{(addition for the ring in $R_*$--coalgebras)} \\
\eta_\ast & \co R_* \to A_{*, 0} & \text{(null element for ring addition)} \\
\chi & \co A_{s, t} \to A_{s, t} & \text{(negation for the ring in $R_*$--coalgebras)} \\
\circ & \co A_{s, t} \otimes_{R_*} A_{s', t'} \to A_{s + s', t + t'} & \text{(multiplication map for the ring in $R_*$--coalgebras)} \\
\eta_\circ & \co R_* \to A_{*, 0} & \text{(null element for ring multiplication)}.
\end{align*}
These are required to satisfy various commutative diagrams. The least obvious is displayed in \Cref{DistributivityDiagram}, encoding the distributivity of $\circ$--``multiplication'' over $\ast$--``addition''.
\end{definition}

\begin{figure}
\begin{center}
\begin{tikzcd}
A_{s, t} \otimes_{R_*} (A_{s', t'} \otimes_{R_*} A_{s'', t'}) \arrow{r}{1 \otimes \ast} \arrow{d}{\Delta \otimes (1 \otimes 1)} & A_{s, t} \otimes_{R_*} A_{s' + s'', t'} \arrow{dddd}{\circ} \\
\left(\bigoplus_{s_1 + s_2 = s} A_{s_1, t} \otimes_{R_*} A_{s_2, t} \right) \otimes_{R_*} (A_{s', t'} \otimes_{R_*} A_{s'', t'}) \arrow{d}{\simeq} \\
\bigoplus_{s_1 + s_2 = s} \left(A_{s_1, t} \otimes_{R_*} A_{s_2, t}  \otimes_{R_*} A_{s', t'} \otimes_{R_*} A_{s'', t'} \right) \arrow{d}{1 \otimes \tau \otimes 1} \\
\bigoplus_{s_1 + s_2 = s} \left(A_{s_1, t} \otimes_{R_*} A_{s', t'} \otimes_{R_*} A_{s_2, t} \otimes_{R_*} A_{s'', t'} \right) \arrow{d}{\circ \otimes \circ} \\
\bigoplus_{s_1 + s_2 = s} \left(A_{s_1 + s', t + t'} \otimes_{R_*} A_{s_2 + s'', t + t'}\right) \arrow{r}{\ast} & A_{s + s' + s'', t + t'}, \\
\end{tikzcd}
\end{center}
\caption{The distributivity axiom for $\ast$ over $\circ$ in a Hopf algebra.}\label{DistributivityDiagram}
\end{figure}

The context of additive unstable cooperations: make the module of $\ast$--indecomposables $QE_* \Susp^{-*} \OS{E}{*}$ into an algebra using the $\circ$--product, noting that it descends to a map \[QE_* \Susp^{-n} \OS{E}{n} \otimes QE_* \Susp^{-m} \OS{E}{m} \to QE_* \Susp^{-(n+m)} \OS{E}{n+m}.\]  The cooperations that survive to the $\ast$--indecomposables are exactly the additive ones.

The context of all unstable cooperations: don't pass to the indecomposable quotient.  This is pretty complicated.

Unbalanced unstable cooperations and Morita functors: I feel that given an unstable $E^*$--comodule $M$, one can produce an unstable $F^*$--comodule by a tensoring operation: $M \otimes_? F_* \OS{E}{*}$ (or something?).  If $M$ is actually $M = E^* X$ for a space $X$, then there should be a comparison map \[F^* X \to E^* X \otimes_? F_* \OS{E}{*}.\]  This will rarely be an isomorphism, but it should fit into the sheaf-over-a-simplicial-scheme picture as something like a face map.

Associated Adams spectral sequences and completions?

Comparison with past scholia on ``piles''?






\section{Coalgebraic formal schemes}

Cartier duality

Free formal groups and symmetric algebras

The strong colimit lemma






\section{Global cooperations}

The bar spectral sequence

The Ravenel--Wilson calculation of $E_* \OS{MU}{*}$ for complex orientable $E$





\section{Cooperations between theories at geometric points}

$BS^1[p^j]_{K(n)} = BS^1_{K(n)}[p^j]$

The Ravenel--Wilson calculation of $K(n)_* \OS{Hk}{*} = \Alt^* BS^1_{K(n)}[p] \otimes_{\F_p} k$



\section{Dieudonn\'e modules}

The practical approach: projective generators of finite Hopf algebras

The de Rham crystal of a formal group

Goerss's result on $H_* \OS{E}{*}$ for Landweber flat $E$

\todo{Dieudonn\'e theory is also about taking primitives in some sort of cohomology.  Can this be connected to the additivity condition on unstable operations?}




\section*{Things that belong in this section}

Theorem 6.1 of R--W \textit{The Hopf ring for complex bordism} sounds like something related to Quillen's elementary proof.

Section III.11 of Wilson's \textit{Primer} has a synopsis of how additive unstable operations should be treated.  (In particular, he remarks on pp.\ 62-3 that primitives in unstable operations are the \emph{additive} unstable operations, which seems important.)  Possibly this is enough to understand how additive unstable cooperations should be treated, or maybe unstable cooperations generally.

There's also a document by Boardman, Johnson, and Wilson (Chapter 2 of the \textit{Handbook of Algebraic Topology}) that discusses an equivalence between Steve's approach and ``unstable comodules''.  Please read this.





