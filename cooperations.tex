% -*- root: main.tex -*-

\chapter{Unstable cooperations}

\todo{Write an introduction for me.}





\todo[inline]{This section can probably be introduced as trying to enrich the notion of a context for cohomology theories swallowing \emph{spaces}.  The stable cooperations are totally captured by the notion of a context, but there are two approaches to (distinct sorts of) unstable operations: Hopf rings and power operations.  We plan on talking about both (and they both have something to say about $MU\<6\>$), so this can be a common thread.}



\section{Coalgebraic formal schemes}

Hopf rings

The bar spectral sequence

Cartier duality



\section{Global cooperations}

The Ravenel--Wilson calculation of $E_* \OS{MU}{*}$ for complex orientable $E$





\section{Cooperations between theories at geometric points}

$BS^1[p^j]_{K(n)} = BS^1_{K(n)}[p^j]$

The Ravenel--Wilson calculation of $K(n)_* \OS{Hk}{*} = \Alt^* BS^1_{K(n)}[p] \otimes_{\F_p} k$



\section{Dieudonn\'e modules}

The practical approach: projective generators of finite Hopf algebras

The de Rham crystal of a formal group

Goerss's result on $H_* \OS{E}{*}$ for Landweber flat $E$




\section*{Things that belong in this section}

Theorem 6.1 of R--W \textit{The Hopf ring for complex bordism} sounds like something related to Quillen's elementary proof.

Section III.11 of Wilson's \textit{Primer} has a synopsis of how additive unstable operations should be treated.  (In particular, he remarks on pp.\ 62-3 that primitives in unstable operations are the \emph{additive} unstable operations, which seems important.)  Possibly this is enough to understand how additive unstable cooperations should be treated, or maybe unstable cooperations generally.

There's also a document by Boardman, Johnson, and Wilson (Chapter 2 of the \textit{Handbook of Algebraic Topology}) that discusses an equivalence between Steve's approach and ``unstable comodules''.  Please read this.






