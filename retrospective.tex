% -*- root: main.tex -*-

I've been asked to describe what I remember about the genesis of the
$\sigma$--orientation and related matters.  I'll do my best, with the
caveat that my collaborators may have different memories, and if they do,
theirs are just as right as mine.

The story starts around the fall of 1989.  I had spent a good deal of
time in the late 1980s thinking about elliptic cohomology and not
really getting anywhere, and I wanted to just think about homotopy
theory for a while.  At the time Haynes Miller and I, inspired by work
of Andy Baker and Jim McClure, had shown that the space of
$A_{\infty}$ automorphisms of the Morava $E$--theory spectrum
associated with the universal defomormation of a height $d$ formal
group law over $\overline{\F}_{p}$ was homotopy discrete and equivalent to
the semidirect product of the Morava stabilizer group and the Galois
group of $\overline{\F}_{p}/\F_{p}$.  In fact Haynes and I worked out this
proof at a reception for new faculty, in the very first conversation
we had when I arrived at MIT in September 1989.  The reason for
wanting this action was to construct the spectrum $EO_{d}=E_{d}^{hH}$
associated to a finite subgroup $H$ of the Morava stabilizer.  At the
time that just seemed like a cool thing to do, but in retrospect the
possibility of doing so must have been inspired by Ravenel's
paper~\cite{RavenelNonexistenceArfInvariantElts}.  This led to the project of computing as many of
the homotopy groups of $EO_{d}$ as one could.  The first interesting
examples were for $d=(p-1)$, which is the smallest value of $d$ for
which there is an element of order $p$ in the stabilizer group.  This
was the case Ravenel pointed to in~\cite{RavenelNonexistenceArfInvariantElts} and it had immediate
applications to the non-existence of Smith-Toda complexes.  The next
case after that was the binary tetrahedral group in the height two
stabilizer group at $p=2$.

Back in those days I drove an Alfa Romeo Spyder, and at some point it
needed a new head gasket.  I had dropped the car off at small local
shop, and when I went to pick it up the next day the mechanic told me
he hadn't been able to get to it.  I asked how long it would be, and he
told me four hours.  Conveniently, I was in that wonderful state of being
obsessed with a math problem, so I just shrugged my shoulders and said
that was cool, I'd wait.  (Also, conveniently, this was before cell
phones and the distractions of the internet.)  I pulled out a pencil
and paper, and I managed in that time to get a formula for the action of
the group on the ring of functions and compute the cohomology.  I also
had a method for getting the first differential, and after writing it
down I realized it formally implied all the other differentials.  When
I got home I wrote---by hand---a postscript file for a picture of the
whole spectral sequence.

It happened that Mahowald made a visit to MIT shortly after that and I
showed him the computation.  He immediately recognized it and told me
he had published a paper with Don Davis proving that such a spectrum
could not exist.  He also said he had never fully believed the proof
but never could find anything wrong with it.  I still don't know how
he recognized the computation.  What I had drawn was what we now think
of as the Adams--Novikov spectral sequence for the $K(2)$--localization
of $\tmf$, and what Mahowald had related it to was a spectrum whose
cohomology is $\mathcal A^* \mmod \mathcal A(2)^*$.
Technically there wasn't quite a contradiction.  However
we soon convinced ourselves this spectrum $EO_{2}$ probably did imply
the existence of a spectrum whose cohomology is $\mathcal A^* \mmod \mathcal A(2)^*$,
and that the
Davis--Mahowald argument probably applied to $EO_{2}$ as well.  We both
worked pretty hard trying to find the resolution.  I was worried about
something foundational in the theory of $A_{\infty}$ ring spectra and
devoted a lot of time to that, and Mahowald perused his argument with
Davis.  Mark and I went through the Davis--Mahowald argument very, very
carefully.  It involved an long series of incredibly dexterous moves,
and I think I learned more about how homotopy groups work in
checking that argument than from any other experience---but we
couldn't find a mistake.  On April Fool's day Mark found the
error: his paper with Davis was completely fine, and the error was in
the accepted computations of the homotopy groups of spheres.  Though I
don't recall if this was 1990 or 1991, the day has stayed with me.
Adams hadn't been gone long and it was his tradition to give a lecture
every April Fool's day proving two contradictory statements, and
challenge the audience to find a mistake.  Mark and I felt he had
given us one last private April Fool's lecture.

Davis and Mahowald had really wanted to have a spectrum whose cohomology
is $\mathcal A^* \mmod \mathcal A(2)^*$, and without it they had made do
with the Thom spectrum $MO[8, \infty)$ associated to the $7$--connected cover
$BO[8, \infty)$.  Bahri and Mahowald~\cite{BahriMahowald} had shown that
there was an isomorphism of $\mathcal A$--modules
\[
H\F_2^{\ast}(MO[8, \infty)) \approx \mathcal A^* \mmod \mathcal A(2)^* \oplus M
\]
in which $M$ is $15$--connected.  This situation was meant to be an
analogue of the situation with $\Spin$ cobordism (with cohomology
$\mathcal A^* \mmod \mathcal A(1)^* \oplus N$ for some $7$--connected $N$) and connected
$K$--theory $kO$ (with cohomology $\mathcal A^* \mmod \mathcal A(1)^*$).  This made it
natural to construct a non-periodic version $eo_{2}$ of $EO_{2}$ with
cohomology $\mathcal A^* \mmod \mathcal A(2)^*$.  Mahowald and I succeeded in doing so at
the Mittag-Lefler institute in the fall of 1993.  It also made it
natural to look for an ``orientation'' \[MO[8, \infty) \to eo_{2}\] analogous
to the Atiyah--Bott--Shapiro orientation $M\Spin \to kO$.  At the time
there was no known method of construction of the Atiyah--Bott--Shapiro
orientation that did not rely on the interpretation of $KO$--theory in
terms of vector bundles, so this seemed to be a tricky problem.  I
conceived of a two-stage program for doing this: the first step was
to produce a map $MO[8, \infty) \to E_{2}$ invariant up to homotopy under
the action of the binary tetrahedral group, and the second step was to
rigidify everything in sight by requiring all of the maps to be
$A_{\infty}$ or $E_{\infty}$.

When I got back to MIT in the winter of 1994, Matt Ando and Neil
Strickland were around and we got to thinking pretty hard about trying
to understand the $E_{\infty}$ or $A_{\infty}$ maps from $MO[8, \infty)$ to
$E_{2}$.  We weren't getting anywhere when Mark Hovey told us that
computations he and Ravenel and done seemed to indicate the that there
couldn't be a map of spectra $MO[8, \infty) \to E_{2}$ invariant up to
homotopy under the action of the binary tetrahedral group.  This
didn't look good for the first step of the program, so Matt and Neil
and I started thinking about how one might understand the cohomology
of $MO[8, \infty)$ in terms of formal groups.  We found an answer in terms
of cubical structures on formal groups and realized that there was a
canonical map from $MO[8, \infty)$ to any complex oriented cohomology
theory $E$ whose formal group was the formal completion of an elliptic
curve~\cite{AHSTheoremOfTheCube}.  This led to the picture
conjectured in my 1994 ICM talk~\cite{HopkinsICMZurich} of the
$\tmf$ sheaf and the relationship between the conjectured $MO[8, \infty)$
orientation and the Witten genus.  It took a while but eventually the
$\tmf$ sheaf was constructed, we found the right way to think about
$E_{\infty}$ orientations of Thom spectra, and with Charles Rezk were
able to produce the $\sigma$--orientation as well as the
Atiyah-Bott-Shapiro orientation using homotopy theory.  I announced
those results at the 2002 ICM~\cite{HopkinsICMBeijing}.

In the end the homotopy fixed point spectra $E_{d}^{hH}$ turned out to
have many applications, from the non-existence of Smith-Toda
complexes, computations in chromatic homotopy theory at low primes, and
even to the Kervaire invariant problem.  However, generalizing the whole
package with the orientation (which after all, was the problem that
led to $\tmf$) is still quite a mystery.  In the early
1990s Hovey found an ingenious argument showing that for height $d>2$
there can't be an orientation $MO[d, \infty) \to E^{hH}_{d}$ if $H$ contains
a non-trivial element of order $p$.  What should play the role of an
orientation in those cases is pretty much up for grabs.
