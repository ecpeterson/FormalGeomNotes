\documentclass{amsart}

%\usepackage{amsmath,amssymb,amsthm}

\usepackage{fullpage}
\usepackage[urw-garamond]{mathdesign}

\usepackage{tikz-cd}

\usepackage{todonotes}

\newcommand{\Z}{\mathbb Z}
\renewcommand{\S}{\mathbb S}
\newcommand{\F}{\mathbb F}
\newcommand{\G}{\mathbb G}
\newcommand{\R}{\mathbb R}
\newcommand{\RP}{\R\mathrm P}
\newcommand{\C}{\mathbb{C}}
\newcommand{\CP}{\C\P}
\newcommand{\A}{\widehat{\mathbb{A}}}
\newcommand{\Q}{\mathbb{Q}}
\newcommand{\M}{\mathcal{M}}
\newcommand{\FH}{\textbf{FH}}
\newcommand{\CH}{\textbf{CH}}
\renewcommand{\L}{\mathcal{L}}
\renewcommand{\H}{\mathcal{H}}
\renewcommand{\P}{\mathbb{P}}

\newcommand{\<}{\langle}
\renewcommand{\>}{\rangle}
\newcommand{\sm}{\wedge}
\newcommand{\Susp}{\Sigma}
\renewcommand{\phi}{\varphi}
\newcommand{\mmod}{/\!\!/}

\newcommand{\context}[1]{\mathcal{M}_{#1}}
\newcommand{\CatOf}[1]{\mathsf{#1}}
\newcommand{\ps}[1]{\llbracket{#1}\rrbracket}
\newcommand{\moduli}[1]{\mathcal{M}_{\mathbf{#1}}}
\newcommand{\OS}[2]{\smash{\underline{#1}}_{#2}}

\newcommand{\Spin}{\mathit{Spin}}
\newcommand{\String}{\mathit{String}}
\newcommand{\TMF}{\mathit{TMF}}
\newcommand{\tmf}{\mathit{tmf}}
\newcommand{\BP}{\mathit{BP}}
\newcommand{\MU}{\mathit{MU}}
\newcommand{\Tate}{\mathrm{Tate}}
\newcommand{\gl}{\mathit{gl}}
\newcommand{\GL}{\mathit{GL}}
\newcommand{\perf}{\mathrm{perf}}

\DeclareMathOperator{\Spec}{Spec}
\DeclareMathOperator{\Spf}{Spf}
\DeclareMathOperator{\colim}{colim}
\DeclareMathOperator{\End}{End}
\DeclareMathOperator{\Div}{Div}
\DeclareMathOperator{\SDiv}{SDiv}
\DeclareMathOperator{\Sq}{Sq}
\DeclareMathOperator{\Sym}{Sym}
\DeclareMathOperator{\Aut}{Aut}

% \numberwithin{equation}{section}

\theoremstyle{plain}
\newtheorem*{theorem}{Theorem}
\newtheorem*{proposition}{Proposition}
\newtheorem*{lemma}{Lemma}
\newtheorem*{corollary}{Corollary}
\newtheorem*{conjecture}{Conjecture}
\theoremstyle{definition}
\newtheorem*{definition}{Definition}
\newtheorem*{construction}{Construction}
\newtheorem*{warning}{Important Warning}
\theoremstyle{remark}
\newtheorem*{remark}{Remark}
\newtheorem*{example}{Example}

\title{Formal Geometry in Algebraic Topology}
\author{Eric Peterson}

\begin{document}

\maketitle

\textbf{Class information}

\vspace{2\baselineskip} \noindent \textit{Meeting times: }
Spring 2016, MWF 12pm--1pm.

\vspace{2\baselineskip} \noindent \textit{Goals: }
The primary goal of this class is to teach students to view results in algebraic topology through the lens of (formal) algebraic geometry.

\vspace{2\baselineskip} \noindent \textit{Grading: }
This class won't have any official assignments. I'll give references as readings for those who would like a deeper understanding, though I'll do my best to ensure that no extra reading is required to follow the arc of the class.

I do want to assemble course notes from this class, but it's unlikely that I will have time to type \emph{all} of them up. Instead, I would like to ``crowdsource'' this somewhat: I'll type up skeletal notes for each lecture, and then we as a class will try to flesh them out as the semester progresses. As incentive to help, those who contribute to the document will have their name included in the acknowledgements, and those who contribute \emph{substantially} will have their name added as a coauthor. Everyone could use more CV items.


\newpage

\section{Jan 25}

\section{Jan 27}

\section{Jan 29}

\section{Feb 1}

\section{Feb 3}

\section{Feb 5}

\section{Feb 8}

\section{Feb 10}

\section{Feb 12}

\section{Feb 17}

\section{Feb 19}

\section{Feb 21}

\section{Feb 24}

\section{Feb 26}

\section{Feb 28}

\section{Mar 2}

\section{Mar 4}

\section{Mar 7}

\section{Mar 9}

\section{Mar 11}

\section{Mar 21}

\section{Mar 23}

\section{Mar 25}

\section{Mar 28}

\section{Mar 30}

\section{Apr 1}

\section{Apr 4}

\section{Apr 6}

\section{Apr 8}

\section{Apr 11}

\section{Apr 13}

\section{Apr 15}

\section{Apr 18}

\section{Apr 20}

\section{Apr 22}

\section{Apr 25}

\section{Apr 27}

\section{Apr 29}

\section{May 2}

\section{May 4}




\newpage

\section{Ideas}
\begin{enumerate}
\item Overview of the class. (Orientations and theories of integration. Statement of the $\sigma$--orientation.)

\textsc{Case study: mod--$2$ homology}
\item Sheaves and formal schemes. The Steenrod algebra and $\context{H\F_2}$.
\item The mod--$2$ Adams spectral sequence. Sheaf cohomology. 
\item The sheaf $\context{H\F_2}(MO)$ and $\pi_* MO$.

\textsc{Introduction to the chromatic program}

\item Neil's $X_E$ construction for a general $E$. Formal schemes and formal groups. Basic theorems on formal varieties.
\item Simplicial presheaves, definition of the context. Homological and cohomological versions. Thom isomorphisms, and Quillen's theorem on $\context{\MU}$.
\item Structure theorems on $\moduli{fg}$. The picture. The definition of $K$-- and $E$--theories.
\item Group schemes and Hopf algebras. Finite dimensional Hopf algebras form an abelian category. Dieudonn\'e theory.
\item The periodicity and thick subcategory theorems. Bousfield localization, chromatic localizations and their properties, chromatic convergence.
\item $E(1)$--local homotopy of the sphere.

\textsc{The $\sigma$--orientation}

\item Thom spectra, line bundles, and divisors
\item The nonrigid, complex $\sigma$--orientation
\item Cohomological versions of AHS: $BU[2k, \infty)_E$.
\item The real version of the $\sigma$--orientation: $B\String_E$
\item Singer--Stong calculation of $H^* BU[2k, \infty)$.

\textsc{Power operations}

\item Ando, Hopkins, Strickland on $H_\infty$--orientations and the norm condition
\item The rigid, real $\sigma$--orientation: AHR. Its effect in homology.
\item The Rezk logarithm and the Bousfield--Kuhn functor
\item Statement of Lurie's characterization of $\TMF$, using this to determine a map from $M\String$ by AHR
\item Dylan's paper on String orientations
\item Matt's calculation of $E_\infty$--orientations of $K(1)$--local spectra using the short free resolution of $MU$ in the $K(1)$--local category

---------------------
\item Cartier duality
\item Subschemes and divisors
\item Coalgebraic formal schemes
\item \textit{Forms of $K$--theory}, Elliptic spectra, Tate $K$--theory, $\TMF$
\item The Ravenel--Wilson calculation, Weil pairings, Neil's MO answer about $H_* K(\Z, 3)$
\item $\sigma$ restricted to $K_{\Tate}$
\item What are $\Theta$--structures for geometers studying abelian varieties?
\item What are Weil pairings for geometers?
\item The Atiyah--Bott--Shapiro orientation (Is there a complex version of this? I understand it as a splitting of $M\Spin$...)
\item The HLP conjecture
\item Sinkinson's calculation and $M\BP\<m\>$--orientations
\item Hovey--Ravenel on nonorientations of $E_n$ by $MO[k, \infty)$. Other things in H--R?
\item Wood's cofiber sequence and $KO_{(p \ge 3)}$
\item The Serre--Tate theorem
\item The thick subcategory theorem.  Nilpotence and periodicity.
\item The chromatic spectral sequence, computations of $\pi_* L_{E(n)} \S$ for low $n$.
\item The fundamental domain of $\pi_{GH}$
\item Orientations and the functor $\gl_1$.
\end{enumerate}

\section{------------------------}



\section{Resources}

Ando, Hopkins, Strickland (Theorem of the Cube)

Ando, Hopkins, Strickland ($H_\infty$ map)

Ando, Strickland

Ando, Hopkins, Rezk

Barry Walker's thesis

Bill Singer's thesis, Bob Stong's \textit{Determination}

Hughes, Lau, Peterson

Morava's \textit{Forms of $K$--theory}

Neil's Functorial Philosophy for Formal Phenomena

Ravenel, Wilson

Kitchloo, Laures, Wilson

\newpage
\newpage
\newpage

\vspace{20\baselineskip}

\begin{center}
What follows are notes from other talks I've given about quasi-relevant material which can probably be cannibalized for this class.
\end{center}

\newpage


\section{$\context{H\F_2}(MO)$}

Hood made the following nice observation. $MO^{H\F_2}$ is the scheme of coordinates on $\RP^\infty_{H\F_2}$, with coordinate ring $\F_2[x_1, x_2, \ldots]$ and corresponding series $f(t) = \sum_{n=1}^\infty x_{n-1} t^n$ and $x_0 = 1$ implicit. This identification is equivalent to Adams's observation that $MO$ is the ``free homotopy ring spectrum'' on $MO(1)$ his sense. Then, $\Spec \mathcal{A}_* = \underline{\operatorname{Aut}}(\widehat{\G}_a)$ acts on this by coordinate changes (and we can pick a left-- or right--action as we see fit). If we pick an action by postcomposition, then we can do the following nice thing: set $f(t) = f_2(t) + g(t)$, where $f_2(t)$ contains just the terms in degrees perfect powers of $2$. Then $f_2^{-1}(f(t))$ is another coordinate with no terms in degrees perfect powers of $2$, and any nontrivial automorphism applied to this ``reduced'' series will re-introduce terms in degrees perfect powers of $2$.  So, this is a canonical form for the series under the $\underline{\operatorname{Aut}}(\widehat{\mathbb G}_a)$--action which admits no further automorphisms. It should follow that $H^*(\context{H\F_2}; \context{H\F_2}(MO))$ has amplitude $0$ and takes the form $\F_2[x_j \mid j \ne 2^n - 1]$, i.e., whose generating function is arbitrary other than having no terms in degrees perfect powers of $2$.

This means that the Adams spectral sequence degenerates and this computes $\pi_* MO$.  (It would be nice to interpret this in terms of a logarithm on $\RP^\infty_{H\F_2}$.)  It also means that the Hurewicz map is injective, hence that $MO$ is a retract of $H\F_2 \sm MO$, hence that $MO$ is an $H\F_2$--module, hence that $MO$ is a wedge of shifts of $H\F_2$.



\newpage



These are notes for a sequence of three lectures delivered at the University of Pittsburgh in June 2015 as part of the workshop \textit{Flavors of Cohomology}.  The goal of the lectures is to advertise a family of cohomology theories called Morava $E$--theories.  Though these spectra do not appear on the first page of any textbook in algebraic topology, they arise naturally in a few different contexts.  Our initial goal will be to show how they arise from the theory of complex-oriented spectra, which will take us on an extended tour of the role of algebraic geometry in the study of homology theories.  Secondly, we will investigate applications suggested by this construction, including the appearance of $E$--theory in the study of finite spectra and in the classification of homology theories with K\"unneth isomorphisms.  Finally, we will talk about the behavior of the $E$--local categories and their role in understanding behaviors in the finite stable category.

The notes are meant to be read by a graduate student with a mild background in algebraic topology: someone with some familiarity with the stable category, with extraordinary cohomology theories, and with simplicial methods.  We also expect some comfortability with basic constructions in algebraic geometry, but by and large we will only encounter the most polite affine schemes and we won't manipulate them in any serious way.

This document was last compiled on \today.









\newpage
\section{Day 1: Quillen's theorem}

\begin{abstract}
For certain ring spectra $E$, we describe a construction of a very rich algebro-geometric category in which $E$--homology is valued, called the \textit{context} for $E$.  We also give a tour of the theory of Thom spectra and announce Quillen's description of the context for the Thom spectrum of the complex $J$--homomorphism.
\end{abstract}


\subsection{Homology cooperations and their structure}

Let's get right to the task advertised in the abstract: for a ring spectrum $E$, we're looking to use algebraic geometry to capture as much of the structure of the output of $E_*$ and $E^*$.  Consider first the case $E = H\F_2$ of ordinary mod--$2$ cohomology, where $H\F_2^*$ is naturally valued in modules for the ``Steenrod algebra'': \[\mathcal A^* \otimes H\F_2^*(X) \to H\F_2^*(X).\]  This action is very useful, but $\mathcal A^*$ has the unfortunate feature of being a highly \emph{noncommutative} ring, which makes it a clumsy object from the perspective of algebraic geometry.  However, the Steenrod algebra is actually a Hopf algebra, and its linear dual $\mathcal A_*$ is \emph{commutative} and it \emph{coacts} on homology: \[(H\F_2)_* X \to (H\F_2)_* X \otimes \mathcal A_*.\]  A theorem of Milnor gives a concise description of this dual Hopf algebra:
\begin{theorem}[Milnor]
There is an isomorphism of rings \[\mathcal A_* \cong \F_2[\xi_1, \xi_2, \ldots, \xi_n, \ldots]\] with diagonal \[\Delta \xi_n = \sum_{j=0}^n \xi_j \otimes \xi_{n-j}^{2^j}.\]
\end{theorem}
\noindent This is a very reasonable commutative ring, so that we might hope to leverage algebraic geometry, and $\Delta$ is expressed by a very reasonable formula, so we might also hope to express arguments with it slickly.

Stable homotopy theorists are also interested in many other ring spectra $E$, but to generalize this story away from $H\F_2$ we will need to more carefully identify its cast of characters by names internal to topology.  After all, taking $E_*$--linear duals is unlikely to be well--behaved in general.  The dual Steenrod algebra arises as the homotopy of $H\F_2 \sm H\F_2$ and the diagonal map has the signature
\begin{center}
\begin{tikzcd}
\mathcal A_* \arrow{r}{\Delta} & \mathcal A_* \otimes_{\F_2} \mathcal A_* \\
\pi_* (H\F_2 \sm H\F_2) \arrow{r} \arrow[-,double]{u} & \pi_* (H\F_2 \sm H\F_2 \sm H\F_2) \arrow[-,double]{u}.
\end{tikzcd}
\end{center}
Together with the ring structure and a healthy obsession with simplicial objects, this is clue enough as to what we should be investigating for general $E$:
\[\mathcal{D}_E(X) := \left\{
\begin{tikzcd}
\begin{array}{c} E \\ \sm \\ X \end{array} \arrow[leftarrow, shift left=\baselineskip]{r}{\mu} \arrow[shift left=(2*\baselineskip)]{r}{\eta_L} \arrow{r}{\eta_R} &
\begin{array}{c} E \\ \sm \\ E \\ \sm \\ X \end{array} \arrow[shift left=(3*\baselineskip)]{r} \arrow[leftarrow, shift left=(2*\baselineskip)]{r} \arrow[shift left=\baselineskip]{r}{\Delta} \arrow[leftarrow]{r} \arrow[shift right=\baselineskip]{r} &
\begin{array}{c} E \\ \sm \\ E \\ \sm \\ E \\ \sm \\ X \end{array} \arrow[shift left=(4*\baselineskip)]{r} \arrow[leftarrow, shift left=(3*\baselineskip)]{r} \arrow[shift left=(2*\baselineskip)]{r} \arrow[leftarrow, shift left=\baselineskip]{r} \arrow{r} \arrow[leftarrow, shift right=\baselineskip]{r} \arrow[shift right=(2*\baselineskip)]{r} &
\cdots
\end{tikzcd}
\right\}.\]
The leftward arrows come from $E$--multiplication and the rightward arrows come from the unit $\S \to E$.\footnote{Incidentally, this cosimplicial ring spectrum has a name: the descent coring for the map $\S \to H\F_2$.  In terms of descent theory, if the map $\S \to E$ is ``of effective descent'', meaning the homotopy limit of this diagram exists and agrees with $\S \sm X$, then the coskeletal spectral sequence gives a way to compute the homotopy of $X$, starting from its homology.  This is the \textit{$E$--Adams spectral sequence}.}

This object is interesting because of its layers.  The homotopy of the $0$\textsuperscript{th}\, level recovers the homology groups $E_* X$.  The maps $\eta_L$ and $\eta_R$ from the $0$\textsuperscript{th}\, level to the $1$\textsuperscript{st}\, level give maps \[E_* X \xrightarrow{E_* \eta_L, E_* \eta_R} (E \sm E)_* X \xleftarrow{\bigstar} E_* E \otimes_{E_*} E_* X,\] but in general $\bigstar$ will not be an isomorphism, inhibiting our discovery of a ``coaction map''.  In good cases, however, this can be repaired:

\begin{definition}
Take $E_* E$ to be an $E_*$--module using the left-unit map.  We will say that $E$ satisfies \FH, the \textbf Flatness \textbf Hypothesis, when the right-unit map $E_* \to E_* E$ is a flat map of $E_*$--modules.\todo{Explain {\FH} in terms of a Kunneth spectral sequence.}
\end{definition}

\noindent If $E$ satisfies \FH, then $\bigstar$ becomes an isomorphism!  In fact, iterating this gives an isomorphism \[\pi_* \mathcal D_E(X)[j] = \pi_* (E^{\sm (j + 1)} \sm X) \xleftarrow{\bigstar} (E_* E)^{\otimes_{E_*} j} \otimes_{E_*} E_* X \cong \pi_* \mathcal D_E[1]^{\otimes_{\pi_* \mathcal D_E[0]} j} \otimes_{\pi_* \mathcal D_E[0]} \pi_* \mathcal D_E(X)[0],\] i.e., the cosimplicial ring $\pi_* \mathcal D_E$ is $1$--truncated and the module $\pi_* \mathcal D_E(X)$ is determined by its $0$\textsuperscript{th}\, level.\footnote{We should further emphasize that even when $X = \S$ for a general $E$ the left- and right-units $E_* \to E_* E$ may differ, making \FH\, have real content.  In the case of $E = H\F_2$, this was not the case, simply because there can't be many maps $\F_2 \to \mathcal A_*$ (and so $H\F_2$ automatically satisfies \FH).  For more complicated rings than $\F_2$, all sorts of behavior can arise.}

Now that I've subjected you to a flurry of ``co-''s, I'd like to take some of them back by finally appealing to algebraic geometry.

\begin{definition}
$E$ satisfies \CH, the \textbf Commutativity \textbf Hypothesis, when $\pi_* E^{\sm j}$ is commutative for all $j \ge 1$.
\end{definition}
\noindent In the case that $E$ satisfies \CH, we can study the simplicial scheme \[\M_E := \Spec \pi_* \mathcal D_E,\] and the cosimplicial object $\pi_* \mathcal D_E(X)$ determines a quasicoherent sheaf $\M_E(X)$ over $\M_E$.

\begin{definition}
The object $\M_E$ is called the \textit{context} of $E$.  The construction $\M_E(X)$ describes $E$--homology as a functor \[E_*: \CatOf{Spaces} \to \CatOf{QCoh}(\M_E).\]  If $E$ satisfies \FH, $\M_E$ takes values in groupoids.\todo{Discriminate between the usefulness of $1$--simplices in $\M_E$ vs in $\M_E(X)$. It's not like $\mathcal A_*$ acts interestingly on $\Spec \F_2$.}
\end{definition}

This is a lot of fancy words for some simple cooperations, but I claim that the conceptual payoff is worth the hassle.  For instance, return to the example $E = H\F_2$, so that $\M_E[0] = \Spec \F_2$ is a point and $\M_E[1] = \Spec \mathcal A_*$ is the spectrum of the infinite polynomial algebra from before.  In order to justify the utility of this language, we should give a geometric description of $\Spec \mathcal A_*$.  Consider the generating function \[F(t) = \sum_{j=0}^\infty \xi_j x^{2^j}.\]  The composition of two such series $F'$ and $F''$ in $\mathcal A_* \otimes \mathcal A_*$ takes the form \[F'(F''(t)) = \sum_{j=0}^\infty \xi'_j \left(\sum_{k=0}^\infty \xi''_k t^{2^k} \right)^{2^j} = \sum_{n=0}^\infty \left( \sum_{j+k=n} \xi'_j (\xi''_k)^{2^j} \right) x^{2^n},\] and so power series composition exactly captures the Milnor diagonal.  The power series $F$ can be identified as the generic mod--$2$ power series satisfying the homomorphism property $F(x' + x'') = F(x') + F(x'')$, and so we identify $\Spec \mathcal A_*$ with $\underline{\operatorname{Aut}}(\G_a)$.\footnote{If this notation makes you uncomfortable, check the end of the talk for an explanation of ``formal group laws''.}  Finally, because $H\F_2$ satisfies \FH, we learn that \[\M_{H\F_2} \simeq \Spec \F_2 \mmod \underline{\operatorname{Aut}}(\G_a).\]  This last line embodies the utility of contexts: starting with this isomorphism, you can unpack that $H\F_2$--homology is valued in $\F_2$--modules with a coaction by a Hopf algebra whose formulas you can write out from memory alone.




\subsection{A general Thom isomorphism}

Today's punchline theorem is about the context $\M_{T(J)}$ of a certain ring spectrum $T(J)$ coming from the theory of Thom spectra.  Once I explain the notation, some of you might recognize this as the complex bordism spectrum, but I don't think I can count on that to quickly supply us with the background we need to recognize $\M_{T(J)}$.  Instead, I'll construct $T(J)$ from scratch in a way that gives us the statements we need for free.  Additionally, this takes us through some interesting tools available to a ``modern'' homotopy theorist --- where ``modern'' primarily means ``geometrically uninclined''.

Given an $S^n$--bundle over a space $X$ \[S^n \to E \xrightarrow\xi X\] its Thom spectrum\footnote{One might prefer the name ``reduced Thom spectrum'', because of the dimension shift in the definition.} $T(\xi)$ is the stable cofiber\todo{Draw a picture of this.}\todo{Does $T(0) \simeq \Susp^\infty_+ X$ follow from the cofiber definition? You need this for the Thom isomorphism, and it seems like it doesn't have the ${}_+$.} \[\Susp^{-n-1} \Susp^\infty_+ E \xrightarrow{\Susp^{-n-1} \Susp^\infty_+ \xi} \Susp^{-n-1} \Susp^\infty_+ X \xrightarrow{\text{cofiber}} T(\xi).\]  Though simple to define, this construction has a number of pleasant properties that indicate it's worth studying:
\begin{enumerate}
\item If $\xi$ is the trivial bundle, then $T(\xi)$ recovers the suspension spectrum $\Susp^\infty_+ X$ of $X$.  In general, then, a twisted bundle $\xi$ should be thought of as giving a \emph{twisted suspension} $T(\xi)$ of $X$.
\item A map of spherical bundles gives rise to a map of Thom spectra, i.e., $T$ is a \emph{functor} \[T: \CatOf{SphericalBundles} \to \CatOf{Spectra}.\]  In particular, this gives rise to a definition of the Thom spectrum for a stable spherical bundle, by taking the colimit over the maps among the stages.
\item Given a vector bundle $V$, we can restrict to the spherical subbundle of unit--length vectors $J(V)$.
\item Finally, $J$ and $T$ are both \emph{monoidal}.  The spherical subbundle $J(V \oplus W)$ is the fiberwise join $J(V) \hat\ast J(W)$ of the individual spherical subbundles, and there is an equivalence $T(\xi \hat\ast \zeta) \simeq T(\xi) \sm T(\zeta)$.\footnote{Incidentally, naturality and monoidality mean that Thom spectra associated to group maps like $J$ have the induced structure of ring spectra.}
\end{enumerate}

We will now deduce the Thom isomorphism theorem from these properties.  The first foothold is that classifying spaces abound: stable spherical bundles are classified by a space $BF$ and stable vector bundles are classified by $BU$.  The fiberwise join and the direct sum constructions imbue $BF$ and $BU$ with the structure of $H$--spaces (in fact, $E_\infty$--spaces), compatible with the induced map \[J: BU \to BF.\]  The second foothold is that the shearing\footnote{This is closely related to a categorical definition of $G$--torsors: a $G$--set $X$ is a $G$--torsor when $(g, x) \mapsto (x, gx)$ is an equivalence.} map $\sigma$ is an equivalence for any group $G$: \[\sigma: (x, y) \mapsto (x y^{-1}, y).\]

Now, we put these two things next to each other.  That $J$ respects product structures is summarized by the commutative diagram
\begin{center}
\begin{tikzcd}
BU \times BU \arrow{r}{\sigma, \simeq} \arrow[bend right]{rrd} & BU \times BU \arrow[crossing over]{d}{J \times J} \arrow{r}{\mu_{BU}} \arrow{rd} & BU \arrow{d}{J} \\
& BF \times BF \arrow[crossing over]{r}{\mu_{BF}} & BF,
\end{tikzcd}
\end{center}
in which we've also drawn the shearing map $\sigma$.  The long composite takes the form \[J \circ \mu_{BU} \circ \sigma (x, y) = J \circ \mu_{BU} (x y^{-1}, y) = J(x y^{-1} y) = J(x).\]  It follows that the second coordinate plays no role, and that the Thom spectrum of the long composite agrees with the Thom spectrum of the map $0 \times J$.\footnote{This is to say that $\mu \circ (0 \times J)$ is homotopic to the long composite, but $(0 \times J)$ is \emph{not} homotopic to $(J \times J) \circ \sigma$.}  Stringing together the properties above, we get: \[T(J) \sm T(J) \simeq T(J \times J) \stackrel{\sigma}{\simeq} T(J \times 0) \simeq T(J) \sm T(0) \simeq T(J) \sm \Susp^\infty_+ BU.\]  It's then easy to extract a more general statement from the one at hand:
\begin{theorem}[Thom, proof by Mahowald]
If $f: G \to BF$ is a group map, $T(f) \to E$ is a ring map, and $\xi: X \to G$ classifies a spherical bundle factoring through $f$, then there is an equivalence \[E \sm T(\xi) \simeq E \sm \Susp^\infty_+ X.\]
\end{theorem}

This is called ``the Thom isomorphism'', and we should take a moment to ponder its significance.  The role of the smash product in stable homotopy theory is that it's used to form homology: \[E_*(X) := \pi_*(E \sm \Susp^\infty_+ X).\]  So, this equivalence is a homotopical form of the assertion that $T(\xi)$ and $\Susp^\infty_+ X$ have the same $E$--homology.  Additionally, because we have this topological statement, we can extract a slightly stronger moral: the twisted suspension embodied by the spherical bundle $\xi$ is \emph{invisible} to the homology theory $E$.



\subsection{Statement of Quillen's theorem}

We've gone far too long without giving an example.  Let $\CP^\infty \simeq BU(1)$ be the classifying space for line bundles, and using $U(1) \simeq S^1$ pass to its  circle--bundle to get
\begin{center}
\begin{tikzcd}
U(1) \arrow{r} \arrow[-,double]{d} & EU(1) \arrow{r}{\mathcal L} \arrow[-,double]{d} & BU(1) \arrow[-,double]{d} \\
S^1 \arrow{r} & * \arrow{r}{J(\mathcal L)} & \CP^\infty.
\end{tikzcd}
\end{center}
Since $EU(1)$ is contractible, we see $T(J(\mathcal L)) \simeq \Susp^{-2} \Susp^\infty \CP^\infty$.  Given a $J$--oriented spectrum $\phi: T(J) \to E$, the Thom isomorphism machinery above furnishes us with isomorphisms
\begin{align*}
E^* \CP^\infty & \cong \tilde E^{*+2} \CP^\infty, &
E^* \CP^n & \cong \tilde E^{*+2} \CP^{n+1}.
\end{align*}
Pushing the canonical class $1 \in E^0 \CP^0$ across this isomorphism, we can inductively deduce\footnote{More miraculously, a piece of vector bundle geometry called the ``splitting principle'' shows that the converse holds: if $E$ is a ring spectrum with a $x$ so that $\S \to \Susp^{-2} \Susp^\infty \CP^\infty \xrightarrow{x} E$ factors the unit map $\S \to E$, then it can be shown that $E$ has a unique $J$--orientation selecting that class.} \[E^* \CP^\infty \cong_\phi E^*\ps{x}.\]

\todo{This day is strangely paced and very hodge-podge. Hm.}As a responsible homotopy theorist, I should admit that spectra are generally very nasty objects, and successfully computing some cohomology ring is actually a pretty big deal.  If we're in a situation where we can \emph{reliably} compute something, it's very important to get all we can from it.  To address this, I'm now going to take off my homotopy theorist hat and put my algebraic geometer hat back on.

As the classifying space for line bundles, $BU(1)$ has a product structure induced by tensoring.  This begets a map
\begin{center}
\begin{tikzcd}
E^* BU(1) \arrow{r} \arrow[-,double]{d}{\cong_\phi} & E^* BU(1) \otimes_{E^*} E^* BU(1) \arrow[-,double]{d}{\cong_\phi} \\
E^*\ps{t} \arrow{r} & E^*\ps{x, y}
\end{tikzcd}
\end{center}
which is determined by the image of $t$, some bivariate power series $x +_\phi y$.  This notation for this series is useful because it helps us remember what axioms it satisfies:
\begin{enumerate}
\item Unitality: $x +_\phi 0 = x$ and $0 +_\phi y = y$.  (Consider tensoring with the trivial line bundle.)
\item Symmetry: $x +_\phi y = y +_\phi x$.  (Tensoring is commutative.)
\item Associativity: $(x +_\phi y) +_\phi z = x +_\phi (y +_\phi z)$.  (Tensoring is associative.)
\end{enumerate}
Such a power series is called a \textit{formal group law}.\footnote{All the formal group laws we'll consider will implicitly be commutative and $1$--dimensional.}  The universal such power series is represented by an affine scheme $\moduli{fgl}$, and the identity orientation of $T(J)$\todo{For the love of Christ, just call this $MU$.} gives a map $\M_{T(J)}[0] \to \moduli{fgl}$.  Moreover, $T(J) \sm T(J)$ is the universal ring spectrum with two $J$--orientations (coming from the left- and right-units) and a transposition relating them: \[T(J) \sm T(J) \xrightarrow{\text{twist}} T(J) \sm T(J).\]  It follows that the induced formal group laws $x +_{\eta_L} y$ and $x +_{\eta_R} y$ must be related by some ``formal group law isomorphism'' $f(t) \in (T(J) \sm T(J))_*\ps{t}$, i.e., a power series $f$ satisfying \[f(x +_{\eta_L} y) = f(x) +_{\eta_R} f(y).\]

\begin{theorem}[Quillen's theorem]
The spectrum $T(J)$ satisfies {\FH} and \CH.  Moreover, the maps
\begin{align*}
\Spec T(J)_* & \to \moduli{fgl}, \\
\Spec T(J)_* T(J) & \to \moduli{fgl} \times \moduli{ps}^{\mathrm{gpd}}, \\
\M_{T(J)} & \to \moduli{fgl} \mmod \moduli{ps}^{\mathrm{gpd}} =: \moduli{fg}
\end{align*}
described above are all equivalences.\todo{Explain what ``gpd'' refers to.}
\end{theorem}

This is a pretty powerful theorem.\footnote{This situation has a strange feature worth remarking on: the ring maps $T(J) \sm T(J) \to E$ act transitively on the set of ring maps $T(J) \to E$, i.e., the ``(decoordinatized) formal group'' associated to $E$ is determined totally by $E$.  This is very different from the algebraic case, where a given ring can support many non-isomorphic formal group laws.}  In our discussion of $T(J)$, we've been so hands off that we've had essentially no control over its behavior.  Nonetheless, this theorem puts $T(J)$ on almost even footing with $H\F_2$: just as the compact description of $\M_{H\F_2}$ given above lets you totally unpack the category in which $H\F_2$--homology is valued, Quillen's description of $\M_{T(J)}$ gives you complete access to the structure theorems governing the category in which $T(J)$--homology is valued.  We will do our best to leverage this tomorrow.













\newpage
\section{Day 2: $E$--theory and periodic self-maps}

\begin{abstract}
We outline a program for studying the functor $\M_{T(J)}(X)$ by first studying the local structure of $\moduli{fg}$.  After a brief tour of the arithmetic literature on formal group laws, we deduce the existence of certain homology theories: the Morava $E$-- and $K$--theories.  We then give examples of local-to-global methods in algebraic topology: for instance, a condition for detecting non-nilpotent self-maps.
\end{abstract}


\subsection{Some philosophy on flat maps}

Yesterday, we developed a rich target for $T(J)$--homology: sheaves over an algebro-geometric object $\M_{T(J)}$.  Furthermore, Quillen's theorem gave an identification $\M_{T(J)} \simeq \moduli{fg}$.  Our initial goal for today is to outline a program by which we can leverage this to study $T(J)$.  Abstractly, one can hope to study any sheaf, including $\M_{T(J)}(X)$, by analyzing its stalks.  The main utility of Quillen's theorem is that it gives us access to a concrete model of $\M_{T(J)}$, so that we can determine where to even look for those stalks.

With this in mind, given a map \[\Spec R \xrightarrow{f} \moduli{fg},\] life would be easiest if the $R$--module determined by $f^* \M_{T(J)}(X)$ were itself the value of a homology theory $R_*(X) = T(J)_* X \otimes_{T(J)_*} R$.  After all, the pullback of some arbitrary sheaf along some arbitrary map has no special behavior, but homology functors do have familiar special behaviors which we could hope to exploit.  Generally, this is unreasonable to expect: homology theories are functors which convert cofiber sequences of spectra to long exact sequences of groups, but base--change from $\moduli{fg}$ to $\Spec R$ preserves exact sequences exactly when $f$ is \textit{flat}.  In that case, this gives the following theorem:

\begin{theorem}[Landweber, part 1]
For any diagram
\begin{center}
\begin{tikzcd}
\Spec R \arrow{r}{i} & \moduli{fgl} \arrow[-,double]{r} \arrow{d} & \M_{T(J)}[0] \arrow{d} \arrow[-,double]{r} & \Spec T(J)_* \\
& \moduli{fg} \arrow[-,double]{r} \arrow[leftarrow]{lu}{\mathrm{flat}} & \M_{T(J)}
\end{tikzcd}
\end{center}
such that the diagonal arrow is flat, the functor \[R_*(X) := T(J)_*(X) \otimes_{T(J)_*} R\] determines a homology theory. 
\end{theorem}

\noindent In the course of proving this theorem, Landweber devised a method to recognize flat maps.  Recall that a map $f$ is flat exactly when for any closed substack $i: A \to \moduli{fg}$ with ideal sheaf $\mathcal I$ there is an exact sequence \[0 \to f^* \mathcal I \to f^* \mathcal O_{\moduli{fg}} \to f^* i_* \mathcal O_A \to 0.\]  Landweber classified the closed substacks of $\moduli{fg}$, thereby giving a method to check maps for flatness.

This appears to be a moot point, however, as it is unreasonable to expect this idea to apply to computing stalks: the inclusion of a closed substack (and so, in particular, a closed point $\Gamma$) is flat only in highly degenerate cases.  This can be repaired: the inclusion of the formal completion of a closed substack of a Noetherian\footnote{$\moduli{fg}$ is not Noetherian, but we will find that each closed point except $\G_a$ lives in an open substack that happens to be Noetherian.} stack is flat, and so we naturally become interested in the infinitesimal deformation spaces of the closed points $\Gamma$ on $\moduli{fg}$.  If we can analyze those, then Landweber's theorem will produce homology theories called $E_\Gamma$.  Moreover, if we find that these deformation spaces are \emph{smooth}, it will follow that their deformation rings support regular sequences.  In this excellent case, by taking the regular quotient we will be able to recover a \emph{homology theory} $K_\Gamma$ which plays the role of computing the stalk of $\M_{T(J)}(X)$ at $\Gamma$.\footnote{Incidentally, this program has no content when applied to $\M_{H\F_2}$, as $\Spec \F_2$ is simply too small.}


\subsection{Local structure of $\moduli{fg}$}

Motivated by the program above, we now set out to describe the local structure of $\moduli{fg}$.  Noting that formal group laws arise as analytic germs of multiplication laws on Lie groups, we will first take a cue from Lie theory and attempt to define exponential and logarithm functions for a given formal group law $F$ over a ring $R$.  In Lie theory, this is accomplished by studying left--invariant differentials: a $1$--form $f(x) dx$ is said to be left--invariant under $F$ when \[f(x) dx = f(y +_F x) d(y +_F x) = f(y +_F x) \frac{\partial(y +_F x)}{\partial x} dx.\]  Restricting to the origin by setting $y = 0$, we deduce the condition \[f(0) = f(x) \cdot \left. \frac{\partial(y +_F x)}{\partial x} \right|_{y=0}.\]  If $R$ is a $\Q$--algebra, then setting the boundary condition $f(0) = 1$ and integrating against $x$ yields \[\log_F(x) = \int \left( \left. \frac{\partial(y +_F x)}{\partial x} \right|_{y=0} \right)^{-1} dx.\]  To see that the series $\log_F$ has the claimed homomorphism property, note that \[\frac{\partial \log_F(y +_F x)}{\partial x} = f(y +_F x) d(y +_F x) = f(x) dx = \frac{\partial \log_F(x)}{\partial x},\] so $\log_F(y +_F x)$ and $\log_F(x)$ differ by a constant.  Checking at $x = 0$ shows that the constant is $\log_F(y)$, hence \[\log_F(x +_F y) = \log_F(x) + \log_F(y).\]  We thus deduce that $\moduli{fg} \times \Spec \Q$ is contractible: every formal group law is uniquely isomorphic to $\G_a$.\todo{What about rescaling? Should you be honest and call this $\moduli{fg}^{(1)}$?}

However, if $R$ is not a $\Q$--algebra, then we may not be able to perform power series integration.  Nonetheless, thinking of the $\Q$--algebra restriction as localization at $(0)$, this inspires us to work arithmetically locally at a prime $p$ and consider $\moduli{fg} \times \Spec \Z_{(p)}$.  This task is eased considerably by the following fundamental theorem of Lazard:

\begin{theorem}[Lazard, part 1]
The ring of functions on $\moduli{fgl}$ is polynomial in infinitely many variables.\footnote{His proof does not give a canonical presentation.  Rationally, these are the coordinate functions selecting the logarithm coefficients.}
\end{theorem}

\noindent As a direct consequence, if $f: S \to R$ is a surjective map of rings and $F_R$ is any formal group law on $R$, then there exists a formal group law $F_S$ on $S$ with $f^* F_S = F_R$.  We can thus reduce to the case where $R$ is a torsion--free (or $\Z$--flat) ring for most of our theorems.

\begin{theorem}[Hazewinkel]
Every formal group law $F$ over a $\Z_{(p)}$--algebra is isomorphic to some $F'$ whose rational logarithm has the form \[\log_{F'}(x) = \sum_{n=0}^\infty \ell_n x^{p^n}.\]  It follows that the radius of convergence of $\log_{F'}$ must be $p^d$ for some $d$.\footnote{If $F$ is additive, then $d$ can be infinite.}  The integer $d$ is called the \emph{height} of $F'$.  It is an isomorphism invariant and it is insensitive to lifts along surjective maps from torsion--free $\Z_{(p)}$--algebras.
\end{theorem}

\begin{theorem}[Lazard, part 2: classification of closed points]
Over an algebraically closed field of characteristic $p$, there is a unique formal group law up to isomorphism for each height.  Moreover, there is a representative $\Gamma_d$ of each isomorphism class with coefficients in $\F_p$ whose logarithm satisfies \[\log_{\Gamma_d}(x) \equiv x \pmod{x^{p^d}}.\]
\end{theorem}

\begin{theorem}[Landweber, part 2: classification of closed substacks]
Let $BP_*$ be the ring classifying formal group laws with $p$--typical logarithms.
\begin{enumerate}
\item It has the form $BP_* \cong \Z_{(p)}[v_1, v_2, \ldots, v_d, \ldots]$, where $v_d \equiv p \ell_d \pmod{\text{decomposables}}$.
\item The unique closed substack of $\moduli{fg} \times \Spec \Z_{(p)}$ of codimension $d$ is selected by $BP_* / (p, v_1, \ldots, v_{d-1})$, and its complementary open substack of dimension $d$ is selected by either of $v_d^{-1} BP_*$ or $v_d^{-1} \Z_{(p)}[v_1, \ldots, v_d]$.\footnote{It's worth pointing out how strange this is. In Euclidean geometry, open subspaces are always top-dimensional, and closed subspaces can drop dimension.}
\item A $BP_*$--module $M$ gives a flat sheaf on $\moduli{fg}$ exactly when $(p, v_1, v_2, \ldots, v_{d-1}, \ldots)$ is a regular sequence $M$ too.
\item In particular, $BP_*$ is itself such a module, and so gives rise to a homology theory $BP$ with $\M_{BP} \simeq \moduli{fg} \times \Spec \Z_{(p)}$.
\end{enumerate}
\end{theorem}

\begin{theorem}[Lubin--Tate: description of deformation spaces]
The deformation space of any height $d < \infty$ law $\Gamma$ over a perfect field $k$ of characteristic $p$ is smooth of geometric dimension $(d-1)$.  That is, it is noncanonically isomorphic to $\mathbb W(k)\ps{u_1, \ldots, u_{d-1}}$.  For $\Gamma = \Gamma_d$, the coordinates can be taken to be $v_{0 \le n < d}$.
\end{theorem}

% \begin{figure}
% \begin{tikzpicture}

% \end{tikzpicture}
% \caption{$\moduli{fg}$}
% \end{figure}

Having stood on the shoulders of all these arithmetic geometers, we can now put our program into practice.  We have a list of the closed points $\Gamma_d$ of $\moduli{fg} \times \Spec \Z_{(p)}$, and their deformation spaces lift to $\moduli{fgl}$ as smooth formal subschemes.  It follows from Landweber's theorem that we can construct homology theories $E_{\Gamma_d}$ for each of these formal groups.  Additionally, we can find regular sequences $(p, u_1, \ldots, u_{d-1}) \in (E_{\Gamma_d})_*$, and hence we can construct the regular quotient\todo{Really emphasize the role of the regular quotient.}\footnote{We think of $K(\Gamma_d)_* X$ as being a model for the stalk of $\M_{T(J)}(X)$ at $\Gamma_d$, though if $(E_{\Gamma_d})_* X$ has torsion this may not agree with $\Gamma_d^* \M_{T(J)}(X)$.} \[K(\Gamma_d) := E_{\Gamma_d} / (p, u_1, \ldots, u_{d-1}).\]  In the case that we pick the lift of $\Gamma_d$ with $p$--series $[p](x) = x^{p^d}$, these objects are typically written $E_d$ and $K(d)$, called Morava $E$--theory and Morava $K$--theory.




\subsection{$E$--theories and periodic self-maps}

Having constructed these ``stalk'' homology theories, I want to show that you can actually perform analyses of the kind I was describing at the beginning of today.  Our example case is a famous theorem: the solution of Ravenel's nilpotence conjectures by Devinatz, Hopkins, and Smith.  Their theorem concerns spectra which ``detect nilpotence'' in the following sense:

\begin{definition}
A ring spectrum $E$ \textit{detects nilpotence} if, for any ring spectrum $R$, the kernel of the Hurewicz homomorphism $E_*: \pi_* R \to E_* R$ consists of nilpotent elements.
\end{definition}

First, a word about why one would care about such a condition.  The following theorem is classical:
\begin{theorem}[Nishida]
Every homotopy class $\alpha \in \pi_{\ge 1} \S$ is nilpotent.
\end{theorem}

\noindent However, people studying $K$--theory in the '$70$s discovered the following phenomenon:

\begin{theorem}[Adams]
Let $M_{2n}(p)$ denote the mod--$p$ Moore spectrum with bottom cell in degree $2n$.  Then there is an index $n$ and a map $v: M_{2n}(p) \to M_0(p)$ such that $KU_* v$ acts by multiplication by the $n$\textsuperscript{th}\, power of the Bott class.\footnote{The minimal such $n$ is given by the formula $n = \begin{cases} p-1 & \text{when $p \ge 3$}, \\ 4 & \text{when $p = 2$}. \end{cases}$}
\end{theorem}

\noindent In particular, this means that $v$ cannot be nilpotent, since a null-homotopic map induces the zero map in any homology theory.  Just as we took the non-nilpotent endomorphism $p$ in $\pi_0 \End \S$ and coned it off, we can take the endomorphism $v$ in $\pi_{2p-2} \End M_0(p)$ and cone it off to form a new spectrum called $V(1)$.\footnote{$V(1)$ actually means a finite spectrum with $BP_* V(1) \cong BP_* / (p, v_1)$. At $p = 2$ this spectrum doesn't exist and this is a misnomer.}  Ravenel's burning question was whether the pattern continues: does $V(1)$ have a non-nilpotent self-map, and can we cone it off to form a new such spectrum with a new such map?  Can we then do that again, indefinitely?  In order to study this question, we are motivated to find spectra $E$ as above --- and in fact, we found one yesterday.

\begin{theorem}[Devinatz--Hopkins--Smith, hard]
The spectrum $T(J)$ detects nilpotence.
\end{theorem}

They also show that the $T(J)$ is the universal object which detects nilpotence, in the sense that any other ring spectrum can have this property checked stalkwise on $\M_{T(J)}$:

\begin{theorem}[Hopkins--Smith, easy]
A ring spectrum $E$ detects nilpotence if and only if $K(d)_* E \ne 0$ for all $0 \le d \le \infty$ and for all primes $p$.
\end{theorem}
\begin{proof}
If $K(d)_* E = 0$ for some $d$, then the non-nilpotent map $\S \to K(d)$ lies in the kernel of the Hurewicz homomorphism for $E$, so $E$ fails to detect nilpotence.

Hence, for any $d$ we must have $K(d)_* E \ne 0$.  Because $K(d)_*$ is a field, it follows by picking a basis of $K(d)_* E$ that $K(d) \sm E$ is a nonempty wedge of suspensions of $K(d)$.  So, for $\alpha \in \pi_* R$, if $E_* \alpha = 0$ then $(K(d) \sm E)_* \alpha = 0$ and hence $K(d)_* \alpha = 0$.  So, we need to show that if $K(d)_* \alpha = 0$ for all $n$ and all $p$ then $\alpha$ is nilpotent.  Taking Devinatz--Hopkins--Smith as given, it would suffice to show merely that $T(J)_* \alpha$ is nilpotent.  This is equivalent to showing that the ring spectrum $T(J) \sm R[\alpha^{-1}]$ is contractible or that the unit map is null: \[\S \to T(J) \sm R[\alpha^{-1}].\]

Pick a prime $p$ and recall the regular sequence of Landweber's theorem.  We define a spectrum $P(d+1)$ to be the regular quotient of $BP$ by $(p, v_1, \ldots, v_d)$.  A nontrivial result of Johnson and Wilson shows that if $T(J)_* X = 0$ for any $X$, then for any $d$ we have $K([0, d])_* X = 0$ and $P(d+1)_* X = 0$.\footnote{It is immediate that $T(J)_* X = 0$ forces $P(d+1)_* X = 0$ and $v_{d'}^{-1} P(d')_*(X) = 0$ for all $d' < d$.  What's nontrivial is showing that $v_{d'}^{-1} P(d')_*(X) = 0$ if and only if $K(d')_*(X) = 0$.}  Taking $X = R[\alpha^{-1}]$, have assumed all of these are zero except for $P(d+1)$.  But $\colim_d P(d+1) \simeq H\F_p \simeq K(\infty)$, and $\S \to K(\infty) \sm R[\alpha^{-1}]$ is assumed to be null as well.  By compactness of $\S$, that null-homotopy factors through some finite stage $P(d+1) \sm R[\alpha]$ with $d \gg 0$.
\end{proof}

As another example of the primacy of these methods, we can show the following interesting result.  Say that $R$ is a field spectrum when every $R$--module (in the homotopy category) splits as a wedge of suspensions of $R$.  It is easy to check (as mentioned in the proof above) that $K(d)$ is an example of such a spectrum.

\begin{theorem}
Every field spectrum $R$ splits as a wedge of Morava $K$--theories.
\end{theorem}
\begin{proof}
Set $E = \bigvee_{\text{primes $p$}} \bigvee_{d \in [0, \infty]} K(d)$, so that $E$ detects nilpotence.  The class $1$ in the field spectrum $R$ is non-nilpotent, so it survives when paired with some $K$--theory $K(d)$, and hence $R \sm K(d)$ is not contractible.  Because both $R$ and $K(d)$ are field spectra, the smash product of the two simultaneously decomposes into a wedge of $K(d)$s and a wedge of $R$s.  So, $R$ is a retract of a wedge of $K(d)$s, and picking a basis for its image on homotopy shows that it is a sub-wedge of $K(d)$s.
\end{proof}

\noindent This is interesting in its own right, because field spectra are exactly those spectra which have K\"unneth isomorphisms.  So, even if you weren't neck-deep in algebraic geometry, you might still have struck across these homology theories just if you like to compute things, since K\"unneth formulas make things computable.






\newpage
\section{Day 3: Chromatic localizations}

\begin{abstract}
We now try to superimpose some of the structure seen yesterday in $\moduli{fg}$ directly onto the category of finite spectra.  This summons certain Bousfield localizations, and we describe their primary application to the stable category.
\end{abstract}


\subsection{Classification of thick subcategories}

Our first goal for today is to apply these local methods once more to get a positive answer to Ravenel's question about finite spectra and periodic self-maps.  The solution to this problem passes through some now-standard machinery for triangulated $\otimes$--categories.

\begin{definition}
A subcategory of the category of a triangulated category (e.g., $p$--local finite spectra) is \textit{thick} if it is closed under weak equivalences, it is closed under retracts, and it has a $2$-out-of-$3$ property for cofiber sequences.
\end{definition}

\noindent Examples of thick subcategories include:
\begin{itemize}
\item The category $\CatOf{C}_d$ of $p$--local finite spectra which are $K(d-1)$--acyclic.  (For instance, if $d = 1$, the condition of $K(0)$--acyclicity is that the spectrum have purely torsion homotopy groups.)  These are called ``finite spectra of type at least $d$''.
\item The category $\CatOf{D}_d$ of $p$--local finite spectra $F$ which have a self-map $v: \Susp^N F \to F$, $N \gg 0$, inducing multiplication by a unit in $K(d)$--homology.  These are called ``$v_d$--self--maps''.
\end{itemize}
Hopkins and Smith show the following classification theorem:

\begin{theorem}[Hopkins--Smith, easy]
Any thick subcategory $\CatOf C$ of $p$--local finite spectra must be $\CatOf C_d$ for some $d$.
\end{theorem}
\begin{proof}
It is sufficient to show that any object $X \in \CatOf C$ with $X \in \CatOf C_d$ induces an inclusion $\CatOf C_d \subseteq \CatOf C$.  Let $Y \in \CatOf C_d$ be any other spectrum of type at least $d$.  Consider the endomorphism ring spectrum $R = F(X, X)$ and the fiber $f: F \to \S$ of its unit map.\todo{Make it clear what $f$ is. Draw the fiber sequence or something.}  The action of $f$ under $K(n)$--homology is an isomorphism exactly when $X$ is $K(n)$--acyclic, and because the $K(n)$--acyclicity of $X$ implies the $K(n)$--acyclicity of $Y$, it follows that $1 \sm f: Y \sm F \to Y \sm \S$ is always null on $K(n)$--homology for all $n$.  By a small variant of the local nilpotence detection theorem, it follows that \[Y \sm F^{\sm j} \xrightarrow{1 \sm f^{\sm j}} Y \sm \S^{\sm j}\] is null for $j \gg 0$, and hence that \[\operatorname{cofib}\left( Y \sm F^{\sm j} \xrightarrow{1 \sm f^{\sm j}} Y \sm \S^{\sm j} \right) \simeq Y \sm \operatorname{cofib} f^{\sm j} \simeq Y \vee (Y \sm \Susp F^{\sm j}),\] so that $Y$ is a retract.  However, using $\operatorname{cofib}(f) = X \sm DX \in \CatOf C$ and a smash version of the octahedral axiom
\begin{align*}
F \sm F^{\sm (j-1)} & \xrightarrow{f \sm 1} \S \sm F^{\sm (j-1)} \xrightarrow{1 \sm f^{\sm (j-1)}} \S \sm \S^{\sm (j-1)} & \Rightarrow & &  F \sm \operatorname{cofib} f^{\sm (j-1)} \to \operatorname{cofib} f^{\sm j} \to \operatorname{cofib} f \sm \S^{\sm (j-1)}
\end{align*}
one can inductively show that $\operatorname{cofib}(f^{\sm j})$, hence $Y \sm \operatorname{cofib}(f^{\sm j})$, and hence $Y$ all belong to $\CatOf C$ as well.
\end{proof}

They also show the \emph{considerably} harder theorem:

\begin{theorem}[Hopkins--Smith, hard]
A $p$--local finite spectrum is $K(d-1)$--acyclic exactly when it admits a $v_d$--self--map.
\end{theorem}
\begin{proof}[Executive summary of proof]
Given the classification of thick subcategories, if a property is closed under thickness then one need only exhibit a single spectrum with the property to know that all the spectra in the thick subcategory it generates also all have that property.  Inductively, they manually construct finite spectra $M_0(p^{i_0}, v_1^{i_1}, \ldots, v_{d-1}^{i_{d-1}})$ for sufficiently large\footnote{Compare this asymptotic condition with the assertion yesterday that there is no root of $v: M_8(2) \to M_0(2)$.} indices $i_*$ which admit a self-map $v$ governed by a commuting square
\begin{center}
\begin{tikzcd}
BP_* M_{|v_d| i_d}(p^{i_0}, v_1^{i_1}, \ldots, v_{d-1}^{i_{d-1}}) \arrow{r}{v} \arrow[-,double]{d} & BP_* M_0(p^{i_0}, v_1^{i_1}, \ldots, v_{d-1}^{i_{d-1}}) \arrow[-,double]{d} \\
\Susp^{|v_d| i_d} BP_* / (p^{i_0}, v_1^{i_1}, \ldots, v_{d-1}^{i_{d-1}}) \arrow{r}{- \cdot v_d^{i_d}} & BP_* / (p^{i_0}, v_1^{i_1}, \ldots, v_{d-1}^{i_{d-1}}).
\end{tikzcd}
\end{center}
These maps are guaranteed by very careful study of Adams spectral sequences.
\end{proof}


\subsection{Balmer spectra and chromatic localization}

As part of a broad attempt to analyze a geometric object through its modules, Paul Balmer has demonstrated the following theorem:

\begin{definition}
Given a triangulated $\otimes$--category $\CatOf C$, define a thick subcategory $\CatOf C' \subseteq \CatOf C$ to be a \textit{$\otimes$--ideal} when it has the additional property that $x \in \CatOf C'$ forces $x \otimes y \in \CatOf C'$ for any $y \in \CatOf C$.  Moreover, $\CatOf C'$ is said to be \textit{prime} when $x \otimes y \in \CatOf C'$ forces at least one of $x \in \CatOf C'$ or $y \in \CatOf C'$.  Define the \textit{spectrum} of $\CatOf C$ to be its collection of prime $\otimes$--ideals, topologized so that $U(x) = \{\CatOf C' \mid x \in \CatOf C'\}$ form a basis of opens.
\end{definition}

\begin{theorem}[Balmer]
The spectrum of $D^{\perf}(\CatOf{Mod}_R)$ is naturally homeomorphic to the Zariski spectrum of $R$.
\end{theorem}

Balmer's construction applies much more generally.  The category $\CatOf{Spectra}$ can be identified with $\CatOf{Modules}_{\S}$, and so one can attempt to compute the Balmer spectrum of $\CatOf{Modules}_{\S}^{\perf} = \CatOf{Spectra}^{\mathrm{fin}}$.  In fact, we just finished this.
\begin{theorem}
The Balmer spectrum of $\CatOf{Spectra}_{(p)}^{\mathrm{fin}}$ consists of the thick subcategories $\CatOf C_d$, and $\{\CatOf C_n\}_{n=0}^d$ are its open sets.
\end{theorem}
\begin{proof}
Using the characterization of $\CatOf C_d$ as the kernel of $K(d-1)_*$, we see that it is a prime $\otimes$--ideal: \[K(d-1)_*(X \sm Y) \cong K(d-1)_* X \otimes_{K(d-1)_*} K(d-1)_* Y\] is zero exactly when at least one of $X$ and $Y$ is $K(d-1)$--acyclic.
\end{proof}

In fact, our favorite functor\footnote{However, this functor is \emph{not} a map of triangulated categories, so this has to be interpreted lightly.} $T(J)_*: \CatOf{Spectra} \to \CatOf{QCoh}(\M_{T(J)})$ induces a homeomorphism of the Balmer spectrum of $\CatOf{Spectra}^{\mathrm{fin}}$ to that of $\moduli{fg}$.  However, Balmer's construction gives only a topological space, and not anything like a locally ringed space (or a space otherwise equipped locally with algebraic data).\footnote{We will address this in our situation, but in general this is an open question: given a ring spectrum $R$, how can one recognize these local categories of spectra in terms of $R$, without reference to auxiliary spectra like $T(J)$?  Or, just as importantly: what makes $T(J)$ a special $\S$--algebra?}  Recalling Landweber's theorem from yesterday, Bousfield's theory of homological localization allows us to extend it as follows:

\begin{theorem}[Bousfield]
Let $R_*$ denote the homology theory associated to a flat map $j: \Spec R \to \moduli{fg}$ by Landweber's theorem.  There is then a diagram\footnote{The meat of this theorem is in overcoming set-theoretic difficulties in the construction of $\CatOf{Spectra}_R$.  Bousfield accomplished this by describing a model structure on $\CatOf{Spectra}$ for which $R$--equivalences create the weak--equivalences.}
\begin{center}
\begin{tikzcd}[column sep=2.2cm,row sep=2cm]
\CatOf{Spectra}_R \arrow[red]{r}{R_* \quad \mathrm{conservative}} \arrow[leftarrow, shift left=0.20cm, red]{d}{L_R} & \CatOf{QCoh}(\Spec R) \arrow[shift left=0.20cm, red, leftarrow]{d}{j^*} \\
\CatOf{Spectra} \arrow[leftarrow,shift left=0.20cm, "\dashv"']{u}{i} \arrow[red]{ru}{R_*} \arrow[red]{r}{T(J)_*} & \CatOf{QCoh}(\M_{T(J)}), \arrow[leftarrow, shift left=0.20cm, "\dashv"']{u}{j_*}
\end{tikzcd}
\end{center}
such that $i$ is left-adjoint to $L_R$, $j^*$ is left-adjoint to $j_*$, $i$ and $j_*$ are inclusions of full subcategories, the red composites are all equal, and $R_*$ is conservative on $\CatOf{Spectra}_R$.
\end{theorem}

In the case when $R$ models the inclusion of the deformation space around the point $\Gamma_d$, we will denote the localizer by \[\CatOf{Spectra} \xrightarrow{\widehat L_d} \CatOf{Spectra}_{\Gamma_d}.\]  In the case when $R$ models the inclusion of the open complement of the unique closed substack of codimension $d$, we will denote the localizer by \[\CatOf{Spectra} \xrightarrow{L_d} \CatOf{Spectra}_d = \CatOf{Spectra}_{\moduli{fg}^{\le d}}.\]  We have set up our situation so that the following properties of these localizations either have easy proofs or are intuitive from the algebraic analogue of $j^* \vdash j_*$:
\begin{enumerate}
\item There is an equivalence \[L_d X \simeq (L_d \S) \sm X,\] analogous to $j^* M \simeq R \otimes M$ in the algebraic setting.  Because $L_{K(d)}$ is associated to the inclusion of a formal scheme (i.e., an ind-finite scheme), it has the formula \[\widehat L_d X \simeq \lim_I \left( M_0(v^I) \sm L_d X \right)\] analogous to $j^* M \simeq \lim_j (R/I^j \otimes M)$ in the complete algebraic setting.
\item Because the open substack of dimension $d$ properly contains both the open substack of dimension $(d-1)$ and the infinitesimal deformation neighborhood of the closed point of height $d$, there are natural factorizations
\begin{align*}
\operatorname{id} \to L_d \to L_{d-1}, & & \operatorname{id} \to L_d \to \widehat L_d.
\end{align*}
In particular, $L_d X = 0$ implies both $L_{d-1} X = 0$ and $\widehat L_d X = 0$.
\item The inclusion of the open substack of dimension $d-1$ into the one of dimension $d$ has relatively closed complement the point of height $d$.  Algebraically, this gives a gluing square (or Mayer-Vietoris square), and this is reflected in homotopy theory by a homotopy pullback square (the chromatic fracture square):
\begin{center}
\begin{tikzcd}
L_d \arrow{r} \arrow{d} \arrow[dr, phantom, "\lrcorner", very near start] & \widehat L_d \arrow{d} \\
L_{d-1} \arrow{r} & L_{d-1} \widehat L_d.
\end{tikzcd}
\end{center}
\end{enumerate}


\subsection{Chromatic dissembly}

There are also considerably more complicated facts known about these functors:
\begin{theorem}[Hopkins--Ravenel]
The homotopy limit of the tower \[\cdots \to L_d F \to L_{d-1} F \to \cdots \to L_1 F \to L_0 F\] recovers the $p$--local homotopy type of any finite spectrum $F$.\footnote{Spectra satisfying this limit property are said to be \textit{chromatically complete}, which is closely related to being \textit{harmonic}, i.e., being local with respect to $\bigvee_{d=0}^\infty K(d)$.  (I believe this a joke about ``music of the spheres''.)  It is known that nice Thom spectra (and in particular every suspension and finite spectrum) is harmonic, that every finite spectrum is chromatically complete, and that there exist some harmonic spectra which are not chromatically complete.}
\end{theorem}

\noindent This suggests a productive method for analyzing the homotopy groups of spheres: study the homotopy groups of each $L_d \S$ and perform the reassembly process encoded by this inverse limit.  Using the fracture square, one sees that it is also profitable to consider the homotopy groups of $\widehat L_d \S$.  In fact, the spectral version of $\M_E(F)$ considered on the first day furnishes us with a tool by which we can approach this:

\begin{theorem}[Bousfield, et al.]
The coskeletal filtration of $\mathcal D_E(F)$ gives a spectral sequence converging to the homotopy of its totalization, $F^\wedge_E$.\footnote{There is a subtlety here: the object $\mathcal D_E(F)$ must be able to be formed as a homotopy coherent diagram in order to produce the totalization. Essentially, this forces $E$ to be an $A_\infty$--ring spectrum. This holds for all the examples of ring spectra we have discussed.}  When $F$ is finite and $E$ models either of the cases above, this spectral sequence converges to $\pi_* L_E F$.  Furthermore, there is a line bundle $\omega$ on $\M_E$ such that\footnote{The identification of the $E_2$--page as computing stack cohomology is the first place where we really mean to employ the full technology of stacks in this talk.  Everywhere else, we have been essentially content to speak of simplicial presheaves.} \[E_2^{*, *} = H^*_{\mathrm{stack}}(\M_E; \M_E(F) \otimes \omega^{\otimes *}) \Rightarrow \pi_* L_E F.\]
\end{theorem}

The utility of this theorem is in the identification with stack cohomology.  In the case $E = E_{\Gamma_d}$, recall that $\M_{E_{\Gamma_d}}[0]$ is a smooth infinitesimal thickening of the spectrum of a field, so that \[\M_{E_{\Gamma_d}} = \left( \moduli{fg} \right)^\wedge_{\Gamma_d} \simeq \widehat{\mathbb A}^{d-1}_{\mathbb W(k)} \mmod \underline{\operatorname{Aut}}(\Gamma_d)\] as in the first example of $E = H\F_2$ on the first day.  But, in this specific case, there is an identification of stack cohomology with group cohomology: \[H^*_{\mathrm{stack}}(* \mmod \underline{G}; \mathcal M) = H^*_{\mathrm{group}}(G; M).\]  Another theorem from the arithmetic geometry literature gives \[\operatorname{Aut}(\Gamma_d) \cong \left( \mathbb{W}(k)\langle S \rangle \middle/ \left( \begin{array}{c} Sw = w^\phi S, \\ S^d = p \end{array} \right) \right)^\times,\] and so we have reduced the computation of all of the stable homotopy groups of spheres to a very difficult problem in profinite group cohomology --- but one which is arithmetically founded, so that arithmetic geometry might continue to lend a hand.

\begin{example}[Adams]
In the case $d = 1$, $\operatorname{Aut}(\Gamma_1) = \Z_p^\times$ and it acts on $\pi_* E_1 = \Z_p[u^\pm]$ by $\gamma \cdot u^n \mapsto \gamma^n u^n$.  At odd primes $p$ (so that $p$ is coprime to the torsion part of $\Z_p^\times$), one computes \[H^s(\operatorname{Aut}(\Gamma_1); \pi_* E_1) = \begin{cases}\Z_p & \text{when $s = 0$}, \\ \bigoplus_{j = 2(p-1)k} \Z_p\{u^j\} / (pk u^j) & \text{when $s = 1$}, \\ 0 & \text{otherwise}. \end{cases}\]  This, in turn, gives the calculation \[\pi_t \widehat L_1 \S^0 = \begin{cases} \Z_p & \text{when $t = 0$}, \\ \Z_p / (pk) & \text{when $t = t|v_1| - 1$}, \\ 0 & \text{otherwise}. \end{cases}\]  These groups are familiar to homotopy theorists: the $J$--homomorphism $J: BU \to BF$ described on the first day selects exactly these elements (for nonnegative $t$).
\end{example}








\section{-------------------------}




ideas for open questions which i ended up not using:


the Goerss--Hopkins--Miller theorem\footnote{Davis and Torii are responsible for the $(E_n \hat\wedge X)^{h\mathbb G_n} \simeq L_{K(n)} X$ equivalence.}

$K(n)$--local homotopy groups

the telescope conjecture

equivariant $E$--theory

connection to $TMF$ and higher orientations

character theories, transchromatic phenomena generally

Dieudonn\'e: Presentation of the endomorphism ring of the generic height $d$ law.


\newpage
\section{-----------------------------}

Nat's MPIM notes

(The first thing Nat says is that Vesna already introduced $tmf$ in the first two talks and that Tobi was going to introduce the chromatic program in the fifth talk.)

\section{Introduction: $K$--theory}
    \subsection{Quillen's theorem}
    \subsection{The (integral) Conner--Floyd isomorphism}
    \subsection{Total power operations in equivariant $K$--theory}
    \subsection{Hopf invariant one}

\section{Morava $E$--theory and the LEFT}
    \subsection{Examples of formal group laws}
    \subsection{Lubin--Tate deformation theory}
    \subsection{Definition of $E$--theory using LEFT}

\section{Morava $E$--theory as a rigid $E_\infty$--ring}
    \subsection{The Goerss--Hopkins--Miller theorem}
    \subsection{Devinatz--Hopkins on fixed point spectral sequences}
    \subsection{$TMF$ restricted to the supersingular locus}
    \subsection{The Goerss--Henn--Mahowald--Rezk resolution}

\section{Morava $E$--theory and algebraic geometry}
    \subsection{$\mathbb C \mathrm P^\infty_E$ as a formal scheme}
    \subsection{$BA^*_E$ and $BU(n)_E$ as formal schemes}
    \subsection{Strickland's theorem on finite symmetric groups}
    \subsection{Power operations and Ando's theorem}

\section{Morava $E$--theory and representation theory}
    \subsection{The case for equivariant and $p$--adic $K$--theory}
    \subsection{The classical Chern character}
    \subsection{The HKR character map}


\newpage

\section{Formal schemes for spaces}

The main goal of this talk is to communicate a way to organize computational results from algebraic topology in your head.  If you flip back through the literature in the 70s and 80s (and we will do some of that ourselves in a moment), you'll find yourself very envious of such a system.  People back then were writing these enormous papers with enormous manipulations of enormous formulas, and there was a real industry built around having sufficient facility with, say, the formula for the right--unit for Brown--Peterson cohomology, or with being adept with multi-index bookkeeping.  This was very hard work then, and it's fairly hard work now to go back and try to understand what these topologists were up to.  Attempting to untangle any of it will imbue you with an immediate appreciation for any kind of method that will allow you to compress one of these results into a small space.

I don't know who first considered the following method, but I do know that the majority of its appearance in the literature is connected, directly or indirectly, to Neil Strickland.  The essential idea is to directly apply algebraic geometry to the situation: rather than associating to a space $X$ the cohomology ring $E^* X$, we go one step further and associate the scheme $\Spec E^* X$ (over $\Spec E^*$).  There's a clear caveat here: algebraic geometry interprets commutative rings, so $E^*$ and $E^* X$ had better be commutative, and the easiest way to enforce this is by restricting attention to spaces with $E^* X$ \emph{even--concentrated}.  Secondly, it turns out to be useful to remember some of the topological structure associated to the original space: the sorts of $X$ we consider in homotopy theory are all ``CW'', and the ``C'' means that they're exhausted by their compact subspaces: $X = \colim_\alpha \{X_\alpha\}_\alpha$.  The cohomology $E^* X_\alpha$ of any one of these individual spaces is a finite-dimensional $E^*$--algebra,\footnote{Actually, some care is required here, since $X_\alpha$ need not all have even--concentrated cohomology even if $X$ does. In the examples of interest, this won't be an issue --- for instance, it suffices for $H_* X$ to be even and torsion--free. I'd advise you to ignore the wrinkle for now.} and so we form a \emph{formal scheme} from the system \[X_E := \Spf E^* X := \{\Spec E^* X_\alpha\}_{\alpha}.\]

The prototypical example of this construction is its value on $X = \CP^\infty$ for ordinary cohomology $E = H\F_p$.  As $X$ has a presentation as a cell complex, it's sufficient to take the subsystem of finite subcomplexes to define $X_E$.  In this case, the finite subcomplexes are $X_n = \CP^n$, with cohomology $H\F_p^* \CP^n = \F_p[x] / x^{n+1}$, and so altogether \[\CP^\infty_{H\F_p} = \A^1_{\F_p},\] where $\A^1_R = \Spf R\llbracket x \rrbracket$ is the ``formal affine line''.\footnote{For that matter, a prototypical formal scheme comes from taking the germ of a point in a Noetherian scheme.}  We can make two immediate further observations:

\begin{enumerate}
\item The condition that a cohomology theory $E$ admit an isomorphism $E^* \CP^\infty \cong E^*\llbracket x \rrbracket$ is called the \emph{complex--orientability} of $E$.  In our language, $E$ being complex orientable exactly means that $\CP^\infty_E$ is (non-canonically) isomorphic to a formal affine line.
\item The space $\CP^\infty = BU(1)$ has a map classifying the tensor of complex line bundles: \[\CP^\infty \times \CP^\infty \xrightarrow{\otimes} \CP^\infty.\]  Just by checking degrees, one can calculate that the induced map \[\CP^\infty_{H\F_p} \times \CP^\infty_{H\F_p} \to \CP^\infty_{H\F_p}\] acts on points by $(x, y) \mapsto x + y$, and so a yet better name for $\CP^\infty_{H\F_p}$ is $\G_a$.  In general, when $E$ is complex--orientable $\CP^\infty_E$ carries the structure of a commutative $1$--dimensional smooth formal group.\footnote{The reader is invited to check $\CP^\infty_{KU} \cong \G_m$.}
\end{enumerate}


\section{A second example: $BU(n)_E$}

To a certain crowd, illustrating features of this functor as applied to $\CP^\infty$ is old hat; anytime complex orientations are mentioned, formal group laws also arise, and we really weren't exploring anything beyond that.  The thesis I want to advance is that Neil's construction continues to be useful when applied to other spaces too, and the somewhat more serious examples we'll explore are the spaces $BU(n)$.

\subsection{$BU(n)_E \cong \A^n$}

The space $BU(n)$ classifies complex vector bundles of rank $n$; suppose that we have such a bundle $V$ over a space $X$.  Associated to $V$ we can form its fiberwise projectivization $\P(V)$, which is a $\CP^{n-1}$--bundle over $X$.  The space $\P(V)$ itself comes equipped with a canonical line bundle, and hence a map \[\P(V) \xrightarrow{p_V} X \times \CP^\infty.\]

\begin{theorem}
When $E$ is complex oriented and $E^* X$ is even, the induced map $E^* X$--module map takes the form
\begin{center}
\begin{tikzcd}
E^*(X \times BU(1)) \arrow{r} \arrow[double,-]{d} & E^* \P(V) \arrow[double,-]{d} \\
E^* X \otimes E^*\llbracket x \rrbracket \arrow{r} & E^* X \otimes E^*\llbracket x \rrbracket / \langle\text{$c_*(V)$, of degree $n$}\rangle.
\end{tikzcd}
\end{center}
\end{theorem}

\noindent Using this theorem, we define the \emph{Chern classes of $V$} by \[0 = x^n - c_1(V) x^{n-1} + c_2(V) x^{n-2} + \cdots + (-1)^n c_n(V).\]  This polynomial is called $c_*(V)$, the \emph{total Chern class of $V$}; it is a monic polynomial generating the ideal corresponding to the quotient ring $E^* \P(V)$.  A second basic theorem declares that these classes $c_j$ account for all of $E^* BU(n)$:

\begin{theorem}
A complex orientation of $E$ begets an isomorphism $E^* BU(n) \cong E^*\llbracket c_1, \ldots, c_n \rrbracket$.
\end{theorem}

\noindent In our language, this allows us to identify the formal scheme $BU(n)_E$ as the smooth formal scheme $BU(n)_E \cong \A^n$.


\subsection{$BU(n)_E \cong \Div_n^+ \CP^\infty_E$}

We can do better than this.  Applying our formal scheme functor to $p_V$, the same theorem asserts that $\P(V)_E \to X_E \times \CP^\infty_E$ is a closed inclusion, i.e., an effective divisor of degree $n$ on $\CP^\infty_E$, or a $E^* X$--point of $\Div_n^+ \CP^\infty_E$.

\begin{theorem}
A complex orientation of $E$ begets an isomorphism $BU(n)_E \cong \Div_n^+ \CP^\infty_E$.
\end{theorem}

\noindent Let's take the time to show that this is a serious description, carrying much more information that you might think. To begin, recall that iterated projectivization can be used to prove the following essential theorem:

\begin{theorem}[Splitting principle]
Suppose $V \downarrow X$ is a rank $n$ complex vector bundle on $X$.  There exists a natural space $f: Y \to X$ over $X$ for which \ldots
\begin{enumerate}
\item \ldots the induced map $f^*: E^* X \to E^* Y$ is an injective map of rings.
\item \ldots the pullback bundle $f^* V$ has a canonical splitting into complex lines: \[f^* V \cong \bigoplus_{j=1}^n \L_j.\]  That is, the classifying map $X \to BU(n)$ lifts across the direct sum map \[\overset{\text{$n$ times}}{\overbrace{BU(1) \times \cdots \times BU(1)}} \xrightarrow{\oplus} BU(n).\]
\end{enumerate}
\end{theorem}

Applying the splitting principle to $V$ and using properties of the total Chern class $c_*$, we then have
\[
c_*(f^* V) = x^n - f^* c_1(V) x^{n-1} + \cdots + (-1)^n f^* c_n(V) = c_*\left( \bigoplus_{j=1}^n \L_j \right) = \prod_{j=1}^n c_*(\L_j) = \prod_{j=1}^n (x - c_1(\L_j)).
\]
These are called the ``Chern roots'' of $c(f^* V)$, and it's now plain that the splitting principle is a topological lift of the factorization of the Chern polynomial.  The space $Y$ enlarges the cohomology ring to be sufficiently solveable so that roots exist, and then additionally the roots are realized by complex lines.  This digression is meant to provide some intuition about how the isomorphism $BU(n)_E \cong \Div_n^+ \CP^\infty_E$ behaves: the point corresponding to a vector bundle $V$ is mapped to the divisor which, after sufficient base extension, is given by the formal sum of its Chern roots.

Additionally, the spaces $BU(n)$ come with formal sum and tensor product operations:
\begin{align*}
BU(n) \times BU(m) & \xrightarrow{\oplus} BU(n+m), & BU(n) \times BU(m) & \xrightarrow{\otimes} BU(n \cdot m).
\end{align*}
The first of these is easy to account for: the total Chern class has $c_*(V \oplus W) = c_*(V) \cdot c_*(W)$, so the induced map
\begin{center}
\begin{tikzcd}
BU(n) \times BU(m) \arrow{r}{\oplus} \arrow[-,double]{d} & BU(n+m) \arrow[-,double]{d} \\
\Div_n^+ \CP^\infty_E \times \Div_m^+ \CP^\infty_E \arrow[dashed]{r}{+} & \Div_{n+m}^+ \CP^\infty
\end{tikzcd}
\end{center}
sends a pair of divisors to their formal sum.  The tensor product is easiest to describe through the splitting principle:
\[
c(V \otimes W) = c\left( \left( \bigoplus_{j=1}^n \L_j\right) \otimes \left( \bigoplus_{k=1}^m \H_k \right)\right) = \prod_{j, k} c(\L_j \otimes \H_k).
\]
From the example at the top of the hour, we know what the Chern polynomial of a tensor product of lines corresponds to: we're using the group structure of $\CP^\infty_E$ to build the formal sum \[\left( \sum_{j=1}^n [a_j] \right) \cdot \left( \sum_{k=1}^m [b_k]\right) = \sum_{j,k} [a_j + b_k].\]  Collectively, these isomorphisms efficiently describe a ring scheme structure on $\coprod_n \Div_n^+ \CP^\infty_E$ reflecting all of the structure on the cohomology rings $E^* BU(n)$.



\section{$\OS{kU}{2k}$}

Given these descriptions, it's easy to take the colimit in $n$ to get a description of $BU_E$: just as $BU$ classifies stable vector bundles of virtual rank zero, $BU_E \cong \Div_0 \CP^\infty_E$ classifies stable divisors of virtual weight zero.  Eliminating this weight condition, we also have $(BU \times \Z)_E \cong \Div \CP^\infty_E$.  These two spaces suggest a new avenue of generalization, as they are both spaces in the connective complex $K$--theory spectrum:
\begin{align*}
BU \times \Z & \simeq \OS{kU}{0}, & BU & \simeq \OS{kU}{2}.
\end{align*}
The next space in this sequence is also very accessible.  It lies in a fiber sequence\todo{Does is map $\OS{kU}{4} \to \OS{kU}{2}$ a map of of infinite loopspaces?}
\begin{center}
\begin{tikzcd}
BSU \arrow{r} \arrow[-,double]{d} & BU \arrow[-,double]{d} \arrow{r}{\det} & BU(1) \arrow[-,double]{d} \\
\OS{kU}{4} \arrow{r} & \OS{kU}{2} \arrow{r} & \CP^\infty.
\end{tikzcd}
\end{center}
For complex--orientable $E$, the associated Serre spectral sequence is collapsing and we have an induced short exact sequence of group schemes
\begin{center}
\begin{tikzcd}
BSU_E \arrow{r} \arrow[-,double]{d} & BU_E \arrow{r} \arrow[-,double]{d} & BU(1)_E \arrow[-,double]{d} \\
\SDiv_0 \CP^\infty_E \arrow{r} & \Div_0 \CP^\infty_E \arrow{r}{\sigma} & \CP^\infty_E,
\end{tikzcd}
\end{center}
where $\sigma$ is the summation map and ``$\SDiv$'' denotes ``special divisors'', i.e., those which sum to zero.

After this space, things get complicated quickly.  The fiber sequence \[K(\Z, 3) \to BU[6, \infty) \to BSU\] has a somewhat accessible Serre spectral sequence, but the higher analogues do not.  In his PhD thesis, Bill Singer completed this calculation for mod--$p$ cohomology using carefully iterated Eilenberg--Moore spectral sequences:

\begin{theorem}[Bill Singer; Bob Stong]
Take $E = H\F_2$.  There is an isomorphism \[H\F_2^*(BU[2k,\infty)) = \frac{H\F_2^*(BU)}{\F_2[\theta_{2i} \mid \sigma_2(i - 1) < k - 1]} \otimes \operatorname{Op}[\Sq^3 \iota_{2k-3}],\] where ``$\operatorname{Op}[\Sq^3 \iota_{2k-3}]$'' denotes the smallest sub-Steenrod-Hopf-algebra of $H\F_2^*(K(\Z, 2k-3))$ containing $\Sq^3 \iota_{2k-3}$ and $\theta_{2i} \equiv c_i$ modulo decomposables.
\end{theorem}

This presentation does not suggest any geometric description.  Instead, using as motivation the fact that ``$\Div$'' constructs a sort of free group scheme, Ando, Hopkins, and Strickland went looking for interesting free constructions laying around.  Taking powers of the natural map $(\L - 1): BU(1) \to BU \simeq \OS{kU}{2}$ gives an interesting map \[BU(1)^{\times k} \xrightarrow{f_k} \OS{kU}{2k} \simeq BU[2k, \infty).\]  Some properties of this map are evident: it is symmetric under permuting the domain, and restricting it to the basepoint of any of the factors collapses the map.  There is an interesting third property, most easily visible by postcomposing to $BU$.  There, the associated divisor (i.e., point in $BU_E$) takes the form $\<a_1, \ldots, a_n\> := \prod_i ([a_i] - [0])$.  We then compute:
\begin{align*}
\<a_1, \ldots, a_{n+1}\> & = ([0] - [a_1])([0] - [a_2])([0] - [a_3])\<a_4, \ldots, a_{n+1}\> \\
& = ([0] - [a_1])[a_2]([0] - [a_3])\<a_4, \ldots, a_{n+1}\> + \<a_1, a_3, \ldots, a_{n+1}\> \\
& = ([0] - [a_1])([a_2] - [a_2 + a_3])\<a_4, \ldots, a_{n+1}\> + \<a_1, a_3, \ldots, a_{n+1}\> \\
& = \<a_1, a_2, a_4, \ldots, a_{n+1}\> - \<a_1, a_2 + a_3, a_4, \ldots, a_{n+1}\> + \<a_1, a_3, a_4, \ldots, a_{n+1}\> \\
\Rightarrow \<a_2, \ldots, a_{n+1}\> - \<a_1 + a_2, a_3, \ldots, a_{n+1}\> & = \<a_1, a_2, a_4, \ldots, a_{n+1}\> - \<a_1, a_2 + a_3, a_4, \ldots, a_{n+1}\>,
\end{align*}
a kind of cocycle condition.  The most important step of this computation is the transition from the second to the third line: we used the fact that $\Div_0 \CP^\infty_E$ is an ideal for $\Div \CP^\infty_E$.  This informs the following lucky guess:

\begin{theorem}[Ando, Hopkins, Strickland]
For even--periodic cohomology theories $E$ and $k \le 3$,\footnote{At $k = 2$, this scheme is not obviously equivalent to the $\SDiv_0$ description above. To explain: the map $\delta$ factors through $\ker \sigma = \SDiv_0 \G$; we will define an inverse $\phi$. Set $\phi_n(\underline a)$ for a tuple $\underline a \in \G^n$ to be $\phi_n(\underline a) = \sum_{j=1}^n [\sigma(\underline a_{< j}), a_j].$ This turns out to be $\Sigma_n$--invariant, so one can write $\phi_\infty$. This map has $\phi_\infty(\underline a + \underline b) = \phi_\infty(\underline a) + \phi_\infty(\underline b) + [\sigma(\underline a), \sigma(\underline b)]$, so for $\underline a$ and $\underline b$ in $\ker(\sigma)$ it is a homomorphism. This is the desired inverse.} there is a diagram
\begin{center}
\begin{tikzcd}
BU(1)^{\times k}_E \arrow{rr} \arrow[dashed]{rd} & & BU[2k, \infty)_E \\
& C_k := \Sym_{\Div \CP^\infty_E}^k (\Div_0 \CP^\infty_E) \arrow{ru}{\simeq}.
\end{tikzcd}
\end{center}
\end{theorem}

This is a hard theorem: not only does that map have to be checked to be an isomorphism, but the mere existence of the symmetric power scheme needs to be checked.  It's also an incredible theorem: suppose that $E$ is an elliptic cohomology theory, so that $\CP^\infty_E$ comes with a chosen isomorphism to the formal group $\widehat{C}$ of some elliptic curve $C$.  \todo{Expand this?}  The ``theorem of the cube'' in algebraic geometry applied to $C$ furnishes us with a canonical point in $MU[6, \infty)_E$, i.e., a canonical multiplicative map $MU[6, \infty) \to E$.  Morally, as $\TMF$ is the ``universal elliptic cohomology theory'', one can take a homotopy inverse limit over the various choices of $C$ to get a map \[MU[6, \infty) \xrightarrow{\sigma} \TMF.\]  This map indeed exists and is the complex--geometric version of ``the $\sigma$--orientation'' or ``Witten's string genus''.  The construction of this canonical point in $MU[6, \infty)_E$ uses in an essential way the schematic description, and it's difficult to conceive of finding the homotopy theoretic instantiation of this map without employing this language.

You'll also notice that we didn't gain many new cases with this theorem: we already understood $\OS{kU}{2k}$ for $k \le 2$, and the Ando--Hopkins--Strickland theorem applies to $k \le 3$.  At $k = 4$, we can already see what's getting in the way: the odd--degree class $\Sq^7 \Sq^3 \iota_{2k-3}$ becomes nonzero for the first time when $k = 4$, and the connection to formal geometry collapses in the presence of odd--degree information.  Nonetheless, the schemes $C_k$ continue to exist, and one can investigate them in their own right.

\begin{theorem}[Hughes, Lau, P.]
For $E = H\F_2$, the Cartier--dual scheme $C^k = \mathbb{D}(C_k)$ has an explicit and efficient presentation which can be computed as far as out as one cares.  (It isn't very pretty, though.)
\end{theorem}

Formal geometry or not, the class $f_k$ still exists, and it induces a map \[\mathcal{O} C^k \xrightarrow{f_k'} (H\F_2)_* BU[2k, \infty).\]  Given our explicit presentation, we can attempt to analyze this map.  Since the source is an even--concentrated Hopf algebra, its image in the target will also consist of even classes.  However, Singer's calculation indicates that restricting to the subalgebra of even classes in the target is not sufficient to make $f_k'$ an isomorphism.  Instead, there appears to be one other item to take into account: the Steenrod algebra $\mathcal{A}_* = \mathcal{O} \underline{\operatorname{Aut}}(\G_a)$ naturally coacts on both sides.

\begin{conjecture}[Hughes, Lau, P.]
The map $f'_k$ is $\underline{\Aut}(\G_a)$--equivariant.  Restricting the target to the Steenrod--Hopf--subalgebra of even classes \emph{which have even diagonals}, this map becomes an isomorphism.
\end{conjecture}

\noindent We've verified this computationally in thousands of bidegrees.  I can't imagine it isn't true, but I don't have a proof.  This modest conjecture naturally leads to a more seriously speculative question: is there an infinite loopspace $X_{2k}$ over $\OS{kU}{2k}$ realizing this factorization?  I have no real feelings about this either way, but I do have a philosophical soapbox to stand on.  The platform of this talk is basically that algebraic geometry can be used to capture a lot of what we do---and can even lead us to proofs of important ideas in homotopy theory, as with the $\sigma$--orientation.  Faced with the fact that these two computations don't line up, we're forced to admit one of two things: either formal geometry isn't quite capturing the natural object of complex $K$--theory and the formal geometry needs to be augmented, or complex $K$--theory isn't quite capturing the natural algebraic geometry and the spectrum needs to be augmented.

I'm tempted to give the latter viewpoint a fair shake.  Geometers seem a little confused about what, morally, comes after $BU[6, \infty)$ and $B\mathrm{String} = BO[8, \infty)$.  The Thom spectra for the spaces that come after also don't really seem to fit as nicely into homotopy theory; it's known, for instance, that $MO[9, \infty)$ can't participate in a (suitably structured) orientation for the height $3$ Morava $E$--theory.  It sure would be interesting if there were some other candidate spaces $X_{2k}$ with a tighter bond to algebraic geometry and so a better shot at achieving these goals.

Here are three immediate stray thoughts about these proposed spaces:
\begin{enumerate}
\item The spaces $X_{2k}$ cannot themselves assemble into a single infinite loopspace. A result from the 1970s of Adams and Priddy shows that any spectrum with $BU[2k, \infty)$ as its zeroth space must be a shift of $kU$.  This is a neat paper; it works by ``running the Adams spectral sequence backwards''.  Borrowing cues from it could turn up interesting results about, say, what the homotopy of $X_{2k}$ must look like.
\item Old work of Steve Wilson gives a description of all sufficiently nice $H$--spaces local to a prime: they are produces of spaces appearing as $\OS{BP\<m\>}{k}$ in the $\Omega$--spectrum for truncated Brown--Peterson theory.  It would probably be instructive to understand the cohomologies of these spaces (a calculation due to Kathleen Sinkinson) and then to compare them with the ring of functions on $C^k$.
\item Incredibly, there are tools around (due to Alexander Zabrodsky) to delete odd classes from $H$--space \emph{while preserving their $H$--spaceiness}.  These kinds of techniques could be useful here, but I suspect they'll be too crude to yield the kind of interesting result we're looking for.
\end{enumerate}


\end{document}
