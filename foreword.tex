% -*- root: main.tex -*-

\subsection*{Foreword (Matthew Ando)}

This book does a remarkable job of introducing some of the 
interaction between algebraic topology and algebraic geometry which
these days, thanks to Doug Ravenel, goes by the name ``chromatic
stable homotopy theory.''

Chromatic homotopy theory had its origins in the work of Novikov and
Quillen, who first investigated the relationship between complex
cobordism and formal groups and perceived its potential for
investigating the stable homotopy groups of spheres. That this works
as well as it does still boggles my mind, and there is still research
to be done to understand why this is so (for example the recent work
of Beardsley, mentioned in the appendix).  

We are fortunate that Jack Morava perceived that the work of Novikov
and Quillen hinted at a deep relationship between the structure of
the stable category and the structure of the stack of formal groups,
and that he was persuasive enough to get others including Landweber,
Miller, Ravenel, and Wilson excited about this approach. The
remarkable activity that followed culminated in  Ravenel's periodicity
conjectures and their resolution by Devinatz, Hopkins, and Smith. 

The chromatic homotopy theorists of the 1970s took Morava's
vision as inspiration and proved amazing results, but in their
published work they usually did not make use of modern
algebro-geometric methods, such as the theory of stacks, which was
more or less simultaneously under development (Though in
``Forms of $K$-theory'', which dates as far back as 1973, Morava sketches
a stacky proof of Landweber's exact functor theorem).

Around 1990, the study of chromatic stable homotopy underwent a
qualitative change. Mike Hopkins took the lead in
showing that algebraic geometry and the theory of stacks
provide powerful tools for proving theorems in chromatic homotopy
theory; at the same time, the conceptual picture of the subject became
much simpler.  The simplifications that resulted made it possible for
many more people including me to enter the subject. 

I was fortunate to have Haynes Miller and Mike Hopkins as
teachers.  I was also fortunate to have Adams's ``blue book'' and
Ravenel's ``green book.'' Until recently, students entering the
subject since the 1990's have not had access to comparable sources
which introduce them to the mix of algebraic topology and algebraic
geometry which form the context for  
modern chromatic stable homotopy theory (Strickland's 
lovely 
``Formal Schemes and Formal Groups'' is a notable exception).  This has
begun to change, and there are several expositions of aspects of the
subject: the list in the acknowledgements of this volume are a good
starting point.  

Which brings me to this book.  I had the good fortune to meet
Eric as an undergraduate and convince him to work on some problems I
was interested in. The things that make Eric fun to work with are
well reflected in this book.  It has a down-to-earth and inviting
style (no small achievement in a book about 
functorial algebraic geometry).  It is elegant,
precise, and incisive, and it is strong on both theory and
calculation.  An important feature is of the book is that it takes the
time to give elegant proofs of some ``theory-external'' results:
theorems you might care about even if chromatic stable homotopy theory isn't
your subject.  

There is a huge amount yet to be discovered: the appendix indicates
some possible directions for future research.  It is great to see this
material assembled here to help the next generation of researchers get
started on an exciting subject.

\vspace{2\baselineskip}
\hspace{3em} ------------Matthew Ando

\hspace{7em} October 2\textsuperscript{nd}, 2017
