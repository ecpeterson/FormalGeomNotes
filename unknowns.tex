% -*- root: main.tex -*-

\chapter{Loose Ends}




Thus ends our technical discussion of the interaction of algebraic geometry and algebraic topology.  Before closing the volume entirely, we take the time to put some of these theorems and ideas into context, in the senses of adjacent mathematics, mathematical history, and future directions as yet unexplored.



\section{The road thus far}

\begin{center}
\Huge Mike is writing this.
\end{center}






\section{The road ahead}\label{OpenQuestionsSection}

\todo[inline]{Make sure you put references in for all these claims.}



In this section, we discuss various topics (sometimes very sketchily, sometimes in more detail) which are currently poorly understood but which also are now coming over the horizon.  With any luck, a number of these will be resolved by the time a second printing of this book becomes a possibility, though many others will probably take the better part of a century to fully resolve.  Accordingly, this is obviously a rather eclectic collection of loose ends; we have made only the barest effort to be comprehensive, and we have certainly missed huge swaths of relevant active research subjects out of preference for the author's own interests.  While the inclusion of this section may ``date'' the book as the problems pass from unresolved to resolved, or as conjectures are confronted by counterevidence, for pedagogical reasons alone it seems important to name the kinds of questions that one might pursue using the tools we have built up.

These are sorted very roughly by their order of appearance in the main thread of the book.








\subsection*{Splitting bordism spectra}

Our first major project in this text, whose culmination we recorded in \Cref{MOSplitsIntoHF2s}, was to split $MO$ into a wedge of Eilenberg--Mac Lane spectra.  This is actually the first of several similar splittings of bordism spectra, all proven by similar\footnote{Similar, at least, at some macroscopic scale.} methods:
\begin{enumerate}
    \item First, one calculates the homology $\HFtwo_* MX$ of the bordism spectrum as a comodule over the dual Steenrod algebra.
    \item Then, one shows that the associated sheaf $(\HFtwo_* MX)\widetilde{}$ is the pushforward of a simpler sheaf for some smaller structure group $G \le \InternalAut_1(\G_a)$.  (This amounts to understanding the Steenrod subcomodule generated by the unit class.)
    \item Finally, one demonstrates some algebraic structure theorem for sheaves arising through such pushforwards.  These are referred to generally as ``Milnor--Moore-type theorems''.
    \item These results then similarly assemble to yield a splitting at the level of $2$--complete spectra through Adams spectral sequence methods.\footnote{Replacing $2$ by $p$ everywhere, one can also produce odd-primary results.}
\end{enumerate}

For instance, by studying the homology of $M\SO$, one finds that the unit class in $\HFtwo_* M\SO$ generates the submodule~\cite[Lemma 20.38]{Switzer} \[1 \cdot \mathcal A_* \mmod (\Sq^1) \subseteq \HFtwo_* M\SO,\] and an appropriate generalization of Milnor--Moore then gives a splitting \[M\SO \simeq \left( \bigvee_j \Susp^{n_j} \HFtwo \right) \vee \left( \bigvee_k \Susp^{m_k} H\Z \right).\]  Similarly, the homology of $M\Spin$ reveals an inclusion~\cite{ABS,ABP,GiambalvoPengelley} \[1 \cdot \mathcal A_* \mmod (\Sq^1, \Sq^2) \subseteq \HFtwo_* M\Spin,\] and a further generalization of Milnor--Moore ultimately produces a splitting\footnote{However, this time the splitting is not one of ring spectra.} \[M\Spin \simeq \left( \bigvee_j kO[n_j, \infty)\right) \vee \left( \bigvee_k \Susp^{m_k} H\F_2 \right).\]  When studying $M\String$, one finds that there is an inclusion \[1 \cdot \mathcal A_* \mmod (\Sq^1, \Sq^2, \Sq^4) \subseteq \HFtwo_* M\String,\] but at this point the Adams spectral sequence becomes too intricate to analyze effectively and no analogous splitting of $M\String$ has yet been produced.\footnote{See, however, recent work of Laures--Schuster~\cite{LauresSchuster}.}\footnote{There are also complex analogues of these results:
\[
MU \simeq \bigvee_j \Susp^{n_j} BP, \quad
M\SU \simeq \left(\bigvee_j \Susp^{n_j} BP\right) \vee \left(\bigvee_k \Susp^{m_k} BoP \right),
\]
where $BoP$ is a certain amalgam of $BP$ and $kO$ described in work of Pengelley~\cite{Pengelley}.}

In addition to these splittings, there is also a collection of Landweber-type results that fall along similar lines:
\begin{itemize}
    \item Conner and Floyd~\cite{ConnerFloyd} demonstrate the following:
    \begin{align*}
    MU_*(X) \otimes_{MU_*} KU_* & \xrightarrow\cong KU_*(X), \\
    M\mathit{Sp}_*(X) \otimes_{M\mathit{Sp}_*} KO_* & \xrightarrow\cong KO_*(X).
    \end{align*}
    \item Hopkins and Hovey~\cite[Theorem 1]{HopkinsHovey} demonstrate the following:
    \begin{align*}
    M\Spin_*(X) \otimes_{M\Spin_*} KO_* & \xrightarrow\cong KO_*(X), \\
    M\Spin^c_*(X) \otimes_{M\Spin^c_*} KU_* & \xrightarrow\cong KU_*(X),
    \end{align*}
    where $\Spin^c := \Spin \times_{O(1)} U(1)$ is a certain complex analogue of the $\Spin$ group.
    \item Ochanine~\cite{OchanineSUModules} demonstrates the following non-result: \[M\SU_*(X) \otimes_{M\SU_*} KO_* \xrightarrow{\not\cong} KO_*(X).\]  In particular, this highlights the importance of the Hopkins--Hovey result above: although $KO$ also receives an $MSU$--orientation, this information is not enough to recover $KO$ by Landweber-type techniques.
    \item Landweber, Ravenel, and Stong~\cite{LandweberEll,LRS} originally constructed a variant of elliptic cohomology according to the following formula:  \[M\SO_*(X) \otimes_{M\SO_*} \left.\Z\left[\frac{1}{6}, \delta, \eps, \Delta, \Delta^{-1}\right] \middle/ \left(2^6 \eps (\delta^2 - \eps)^2 - \Delta\right)\right. =: \mathit{Ell}^*(X).\]
\end{itemize}

These spilttings and these flatness isomorphisms---and the absence of a known splitting in the $\String$ case!---are of great interest to a homotopy theorist.  In the more obvious direction, they reveal a great deal about bordism spectra in terms of well-understood spectra, but less obviously they also promise to let information flow in the opposite direction.  As a thought experiment, imagine an alternative history where we had never encountered real $K$--theory, but we had nonetheless embarked on a study of bordism theories.  An intensive study of $\Spin$--bordism would have led us inexorably toward $kO$, and once we had isolated $kO$ from the rest of $M\Spin$ we could then attempt to rig a geometric model for this dramatically simpler spectrum, ultimately leading us to the highly interesting and rewarding theory of vector bundles~\cite[pg.\ 338]{HoveyVnEltsOfRings}.  Similarly, one might hope that a splitting of $M\String$ would furnish us with spectra that themselves stand a high chance of admitting interesting geometric models, perhaps via as-yet undiscovered geometric constructions.  Many of the original spectra under $M\String$ that were considered, like $KO^{\Tate}$ and $EO_2$, have both nonconnected and completed homotopy, and so are unlikely to have direct geometric interpretation---but these complaints do not apply to $\tmf$.  Indeed, to a large extent, the start of this program (though not the promised geometric model) has been realized by the existence of $\tmf$ and of the $\sigma$--orientation: it is known that the $\sigma$--orientation produces a split inclusion \[\mathcal A^* \mmod \mathcal A^*(2) \cong \HFtwo^* \tmf \to \HFtwo^* M\String\] which is supported on the unit class.  Since $\mathcal A^* \mmod \mathcal A^*(2)$ is an indecomposable Steenrod module, any splitting of $\HFtwo_* M\String$ into indecomposable components will attach the unit class to a version of $\tmf$.\footnote{However, as in the case of the Atiyah--Bott--Shapiro orientation $M\Spin \to kO$, even this piece cannot participate in a splitting of ring spectra, as shown by McTague~\cite{McTague}.}  The idea, then, is to take this as evidence that $\tmf$ wants to admit a geometric model, and we need only uncover what geometry does the job.  In the more restrictive case of $\mathit{Ell}$, there actually \emph{is} a geometric model, as uncovered by Kreck and Stolz~\cite{KreckStolz}.  They produced a geometrically defined integral cohomology theory using symplectic vector $2$--bundles that agrees with the Landweber--Ravenel--Stong functor after inverting $6$.  Even ignoring the specific case of $\tmf$, these generic methods may continue to point the way for geometry associated to the bordism spectra $MO[k, \infty)$ for still larger values of $k$.  There is a great deal of ongoing work surrounding this problem, especially in the research programs of Hovey and Laures.

Even if $KO$ and $\widehat L_2 \TMF$ appear un-geometric, they appear to belong to the beginning of a recognizable pattern: $KO^\wedge_p = E_1^{hC_2}$ and $\widehat L_2 \tmf = E_2^{hM}$ both arise as fixed point spectra for maximal finite subgroups of their respective stabilizer groups.  One might therefore analogously define \textit{higher real $K$--theories} by the formula $EO_d = E_d^{hM}$ for a maximal finite subgroup $M \le \S_d$.  These spectra are interesting in their own right, but in light of the discussion above one might furthermore search for a connection between them and higher bordism spectra.  An extremely interesting result of Hovey~\cite[Proposition 2.3.4]{HoveyVnEltsOfRings} says that this likely cannot be made to work out: for $p > 3$ and $k$ arbitrary, there is no map of ring spectra \[MO[k, \infty) \to EO_{p-1}.\]  The meat of this theorem comes from two competing forces:
\begin{itemize}
    \item The homotopy of $EO_{p-1}$ is known by computations of Hopkins and Miller to contain certain $\alpha$--family elements.  Since the homotopy of $MO[k, \infty)$ agrees with that of the sphere below $k$ and destroys the $\alpha$ elements above $k$, this forces the value of $k$ to be large enough to accommodate the elements present in $\pi_* EO_{p-1}$.
    \item Because $EO_{p-1}$ is local, any orientation of it factors through $\widehat L_{p-1} MO[k, \infty)$.  The Ravenel--Wilson acyclicity results for Eilenberg--Mac Lane spaces show that the natural map $\widehat L_{p-1} MO[k, \infty) \to \widehat L_{p-1} MO[p, \infty)$ is an equivalence for any $k \ge p$.  It follows that such an $MO[k, \infty)$--orientation of $EO_{p-1}$ is equivalent data to an $MO[p, \infty)$--orientation.
\end{itemize}
For $p > 3$, these bounds pass through each other and there is no satisfying value of $k$.  At the same time, because of the tight connection between stable homotopy theory and the Morava $E$--theories, this might be read as a failure of the connective bordism spectra $MO[k, \infty)$ more than a failure of higher real $K$--theories.  It is interesting to try to imagine a continuation of the sequence $MO$, $MSO$, $M\Spin$, $M\String$ which does not suffer from Hovey's negative result.\footnote{Some footholds for the \emph{algebraic} geometry of the large $k$ case were worked out by me and my coauthors~\cite{HLP}.}\footnote{One suggested continuation of the analogous complex sequence comes from the $2$--primary identifications $BU \simeq Y_1$, $BSU \simeq Y_2$, and $BU[6, \infty) \simeq Y_3$, using the notation of \Cref{WilsonSpaces}.}








\subsection*{Blueshift and redshift}

This procedure of passing to (higher-order) vector bundles, as in the Kreck--Stolz construction, is embodied by passing from a ring spectrum to its algebraic $K$--theory spectrum.  There are on-going research programs with the goal of demonstrating that the algebraic $K$--theory of a ring spectrum of ``chromatic complexity $d$'' is itself of chromatic complexity one larger, as in the transition from $kO$ to $\tmf$.  This is one instance of a family of operations that (conjecturally) modulate chromatic height; such operations that raise height are called \textit{redshifting}, and operations that lower height are called \textit{blueshifting}, in reference to the periodicity perspective of \Cref{CdEqualsDd}.

As an example of a blueshifting construction, an important observation of Ando, Morava, and Sadofsky~\cite{AMS} is that the $C_p$--Tate construction on the ring spectrum $E(d)$ is a variant of $E(d-1)$.\footnote{Somewhat more specifically, it gives $E(d-1)$ with homotopy tensored up by a quite \emph{large} ring.}  In fact, Hovey, Sadofsky, and Kuhn show that something similar happens quite generally: if $R$ is an $E(d)$--local ring spectrum and $G$ is a finite $p$--group, then the Tate object $R^{tG}$ is an $E(d-1)$--local ring spectrum.  For a striking example of this phenomenon, we consider the case of the $2$--adic sphere spectrum, which is $K(\infty)$--local.  We specialize to the case $G = C_2$ in order to perform an explicit analysis of $\S^{tC_2}$, which belongs to a stable fiber sequence \[\S^{hC_2} \to \S^{tC_2} \to \Susp \S_{hC_2}.\]  The orbit spectrum can be identified as \[\S_{hC_2} \simeq \Susp^\infty_+ EC_2 / C_2 = \Susp^\infty_+ BC_2 \simeq \Susp^\infty_+ \RP^\infty,\] and similarly we can identify the fixed point spectrum as \[\S^{hC_2} \simeq F_{C_2}(\Susp^\infty_+ EC_2, \S) \simeq D\Susp\infty_+ \RP^\infty.\]  A remarkable theorem of Atiyah~\cite{Atiyah} presents the Spanier--Whitehead dual of projective space.  First, the calculation $T(\mathcal L \downarrow \RP^n) = \RP^{n+1}$ of \Cref{RPnThomExample} extends by the formula $T(m \mathcal L \downarrow \RP^n) = \Susp^\infty_+ \RP^{n+m} / \RP^{m-1}$ to positive values of $m$, and the same formula can be used to \emph{define} complexes $\RP^n_m$ for negative values of $m$.  Then, Atiyah demonstrates the formula \[D \RP^{n-1}_m \simeq \Susp \RP^{m-1}_n.\]  In our case, we thus calculate
\begin{center}
\begin{tikzcd}
\S^{hC_2} \arrow{r} \arrow[equal]{d} & \S^{tC_2} \arrow{r} \arrow[equal]{d} & \Susp \S_{hC_2} \arrow[equal]{d} \\
\Susp \RP^{-1}_{-\infty} \arrow{r} & \Susp \RP^\infty_{-\infty} \arrow{r} & \Susp \RP^\infty.
\end{tikzcd}
\end{center}
By shifting this presentation slightly, we highlight an interesting comparison map.  Note that the bottom cell of $\RP^n_0$ is unattached, and hence there is a natural wrong-way map \[\RP^n_0 \to \S^0.\]  Taking Spanier--Whitehead duals and applying Atiyah's formula, we produce a map \[\S^0 \to D \RP^{(n+1)-1}_{-(0)} = \Susp \RP^{(0)-1}_{-(n+1)} \to \Susp \RP^\infty_{-(n+1)}.\]  Taking the inverse limit gives a map \[\S^0 \to \Susp \RP^\infty_{-\infty},\] and Lin showed this map to be a $2$--adic equivalence~\cite{Lin}.\footnote{More generally, Carlsson proved a generalization of this to all other groups, known as the Segal conjecture.}  From the perspective of blueshift, the qualitative form of this is not entirely unexpected: the sphere has infinite chromatic complexity, and so blueshifting it by one stage yields another spectrum of infinite chromatic complexity.  The precise form, however, is quite surprising: we produced exactly the sphere spectrum again!

This alternative presentation of the sphere spectrum as a Tate spectrum yields a filtration spectral sequence \[\pi_* \S^* \Rightarrow \pi_* (S^{-1})^\wedge_2.\]  In terms of this spectral sequence, any stable stem is represented by a (coset of) element(s) on the $E_1$--page, and this reverse reading of the spectral sequence is known as the \textit{Mahowald root invariant}~\cite{MahowaldShick}.  The root invariant appears to possess deeply interesting redshifting properties---not wholly surprising, since we are reading a blueshifting construction in reverse.  For instance, the root of $p$ lies in the $\alpha$--family, and the root of an $\alpha$--family element lies in the $\beta$--family~\cite{BehrensRootInv}.  The root invariant has many striking connections to other areas of topology~\cite{MahowaldRavenel}.

In addition to the chromatic properties waiting to be explored, the Tate construction itself has some truly puzzling features.  Mahowald's original perspective on the Tate construction was through the definition \[R^{tC_2} := \lim_{n \to \infty} \Susp R \sm P^\infty_{-n},\] whereas the general construction of Greenlees--May~\cite{GreenleesMay}\footnote{Stroilova's PhD thesis~\cite{Stroilova} gives a variation of this construction that converts an $E(n)$--local ring spectrum to an $E(n-k)$--local ring spectrum.} uses the formula \[R^{tG} = (F(EG_+, R) \sm \widetilde{EG})^G,\] whose pieces in the case at hand consist of $EG_+ = S(\infty \L)$ and $\widetilde{EG} = S^{\infty \mathcal L}$, which rearranges to give \[R^{tC_2} := \colim_{m \to \infty} F(P^\infty_{-m}, R).\]  Something quite spooky has happened: a colimit and a limit were inexplicably interchanged in the formulas \[\RP^\infty_{-\infty} = \colim_n \lim_m \RP^n_{-m} = \lim_m \colim_n \RP^n_{-m}.\]  Understanding why these extremely different constructions give the same answer will likely yield some very important background theory.\todo{Footnote todo: \textbf{Find some proofs and disproofs.}}\footnote{There is a superficially related famous problem known as the Telescope Conjecture.  Namely, there are various ``finitary'' flavors of chromatic localization, which are typically less categorically robust but more computable.  They assemble into a diagram:
\[\begin{array}{ccccc}
E & \to & L_d^{\fin} E & \to & L_d E \\
\downarrow & & \downarrow & & \downarrow \\
L_{X(d)} E & \to & \widehat L_d^{\fin} E & \to & \widehat L_d E,
\end{array}\]
where $X(d)$ is a finite complex of type exactly $d$, $v$ is a $v_d$--self-map of $X(d)$, $T(d) = X(d)[v^{-1}]$ is the localizing telescope, $\widehat L_d^{\fin}$ is Bousfield localization with respect to $T(d)$ (which can be shown to be independent of choice of $X(d)$ and of $v$), and $L_d^{\fin}$ denotes localization with respect to the class of \emph{finite} $E(d)$--acyclics.  Much is known about these functors: for instance, $L_{X(d)} L_d = \widehat L_d$, there is a chromatic fracture square relating $L_d^{\fin}$ to $\widehat L_{\le d}^{\fin}$, and $L_d^{\fin} E \simeq L_d E$ if and only if $\widehat L_{\le d}^{\fin} E \simeq \widehat L_{\le d} E$.  One major question about these functors remains open, corresponding the last unsettled nilpotence and periodicity conjecture of Ravenel~\cite[Conjecture 10.5]{RavenelLocalizationWRTPeriodic}: is the map $\widehat L_d^{\fin} E \to \widehat L_d E$ an equivalence?  Multiple proofs and disproofs have been offered, but the literature remains unsettled.  Our interest in this problem here is in the formula $\widehat L_d^{\fin} E = \colim\left(X(d) \sm E \xrightarrow{v} X(d) \sm E\right)$: this formulas has a colimit, whereas the formula for the right-hand side given in \Cref{FormulaForKnLocalization} has a limit.}

The blueshifting behavior of the Tate construction begs for insertion into the framework for understanding chromatic homotopy theory described in the main thread of this text.  For instance, the formal-geometric perspective on the isomorphism in Lin's theorem is that the residue map $\F_2(\!(x)\!) \to \F_2\{x^{-1}\}$ is a quasi-isomorphism of continuous Steenrod comodules~\cite[Remark 8.34]{StricklandFSFG}.\footnote{Reader beware: it is, of course, actually illegal to commute ordinary homology past the inverse limit.}  The framework of $p$--divisible groups appearing around the edges of this book seems to show a lot of promise for serving as a general organizing principle for these ``transchromatic'' results, but this largely has not been worked out.\footnote{For instance, contact between $p$--divisible groups and the Tate construction is visible in work of Greenlees and Strickland~\cite{GreenleesStrickland}, \cite[pg.\ 10]{StricklandFPFP}, and $p$--divisible groups have played a central role in the height-modulating phenomena of Hopkins--Kuhn--Ravenel character theory (see \Cref{CharacterTheorySection}, \cite{HKR}) and in its extensions by Stapleton (see, for example, \cite{Stapleton}).}






\subsection*{Why formal groups?}

A question we have cheerfully left unresolved is: what is so special about $MU$ that makes it such an effecive tool in studying stable homotopy theory?  There are many approaches to this question that appear to lead in many different directions; we address several of them in turn.

First, one can ask this question just considering $MU$ as a ring spectrum.  From this perspective, we summarized the most important resuls about $MU$ in \Cref{MUInducesSpectrumHomeo}: the context functor perfectly detects the Balmer spectrum of the global stable category.  Given a general ring spectrum $R$, we can ask two questions analogous to this result:
\begin{enumerate}
\item Is it possible to find an $R$--algebra $S$ whose context functor induces a homeomorphism of Balmer spectra $\Spec(\CatOf{Modules}_R^{\perf}) \to \Spec(\CatOf{QCoh}(\context{S/R}))$?\footnote{Note that the right-hand side is a completely algebraic construction: these are simplicial sheaves of modules.}
\item Given a thick $\otimes$--ideal $\alpha$, is there a complementary localizer $L_\alpha\co \CatOf{Modules}_R \to \CatOf{Modules}_{R,(\alpha)}$?  Can these localizers be presented via Bousfield's framework as homological localizations for auxiliary $S$--algebra spectra $S_\alpha$ (cf.\ \Cref{BousfieldLocalizationThm})?  Do the contexts $\context{S_\alpha}$ admit compatible localizers with $\context{S}$?
\end{enumerate}
For $R = \S$, this is precisely the role that the $R$--algebra $S = MU$ and the $S$--algebras $S_d = E(d)$ play.\footnote{A potentially useful observation is that this does not appear to be a question in the domain of highly structured ring spectra.  After all, the ring spectra $E(d)$ are not known to be $E_\infty$.}

There are some obvious restrictions on the $R$--algebra $S$: for instance, because we intend to use $S$ to form a context, we require that $\pi_* S$ (and $\pi_* S^{\sm_R (j)}$ generally) be even.  There are, in fact, two results in the literature that shed light on exactly this requirement.  First, Priddy showed that by iteratively attaching cells to the $p$--local sphere in order to ensure that it has only even-dimensional homotopy, one arrives at $BP$~\cite{Priddy}.  More recently, Beardsley showed that $MU$ arises similarly by attaching certain $A_\infty$--algebra cells to kill the odd homotopy on the integral sphere~\cite{Beardsley}.\footnote{He also identifies certain intermediate spectra in this process as the spectra $X(n)$ that arise for Devinatz, Hopkins, and Smith in their proof of the nilpotence conjectures~\cite{DHS}.}  It would be satisfying to understand whether some construction like this holds in any sort of generality.

For a second perspective, using the adjunction \[\CatOf{RingSpectra}(MU, E) \cong \CatOf{Spectra}_{\S/}(\Susp^{-2} \Susp^\infty \CP^\infty, E),\] we might ask: what's so special about the space $\CP^\infty$?  Much of the contents of this book supports the following perspective: given a space $X$, we consider the $E_\infty$--ring pro-spectrum $DX_+ = \{F(X_\alpha, \S)\}_\alpha$.  Because each $X_\alpha$ is a compact object, base-change along the unit map $\eta\co \S \to E$ is computed by the following formula:
\begin{align*}
\eta^* DX_+ = E \sm \{F(X_\alpha, \S)\}_\alpha = \{E \sm F(X_\alpha, \S)\}_\alpha = \{F(X_\alpha, E)\}_\alpha.
\end{align*}
Applying the functor $\Spf \circ \pi_0$ to this pro-system yields the formal scheme $X_E$ considered in \Cref{FullDefnOfXE}.  One of the overarching themes of this book has been to think of the objects $X$ and $DX_+$ as spectral incarnations of some algebro-geometric recipe, which base-change along $\eta$ to form classical algebro-geometric constructions which have been ``bound'' to $E$.  The situation, as pointed out around \Cref{MUstarVsMUAsModuli}, is somewhat analogous to that of Lubin and Tate's explicit local class field theory, where a certain recipe associates to a local number field a governing formal group in terms of which much of the structure of the number field can be cast.\footnote{Jack Morava is very insistent on parameterizing his $K$--theories not by a formal group but by a local number field $L$ and its Lubin--Tate $p$--divisible group, a trend not picked up on by most other algebraic topologists.  In connection with the ``topological Langlands program'' hinted at below, he is also highly interested in the Weil--Shafarevich theorem\cite[Appendix III]{Weil}, which shows that all such number fields arise as maximal tori in certain division algebras (dependent on the degree of the extension), and their Galois groups can be understood through this embedding.}  In the setting of algebraic topology, such a ``recipe'' is embodied by $\CP^\infty$ itself.  Understanding in what sense $\CP^\infty$ is encoding anything (or any of the other spaces discussed in this text) is an important challenge for homotopy theorists in years to come.\footnote{Speaking completely from a place of personal opinion: I have a hard time believing that ``spectral formal groups'', in the sense of derived algebraic geometry, exist in any real generality.  Surely it is not an accident that all the known formal groups arising from homotopy theory tightly derive from the one associated to $\CP^\infty$.  Thinking of $MU$ as some manner of topological enrichment of $\moduli{fg}$, that $MU$--orientations are entirely controlled by a cohomology theory's behavior on $\CP^\infty$ seems to prohibit the existence of any other formal groups associated to a given ring spectrum.}









\subsection*{Adams filtration asymptotics}

Various large-scale behaviors of the $MU$--Adams spectral sequence are poorly understood and would yield interesting information about stable homotopy theory as a whole.  For instance, the following question is pulled from Mike Hopkins~\cite[Section 10]{HopkinsOnRavenel}: let $g(n)$ denote the largest $MU$--Adams filtration degree of an element of $\pi_n \S$ (i.e., $g$ traces the vanishing curve on the $E_\infty$ page of the $MU$--Adams spectral sequence).  The nilpotence conjectures are all equivalent to the statement \[\lim_{n \to \infty} \frac{g(n)}{n} = 0,\] but little is known about the asymptotics of $g$ beyond this statement.  For instance, one might ask: for what $\eps > 0$ does the asymptotic formula $g(n) = O(n^{\eps})$ hold, and what is the infimum $\eps_{\inf}$ over such values?  Various values of the infimum have various consequences: sufficiently small values entail the Telescope Conjecture (see the footnotes above), and sufficiently large values entail its failure.  Hopkins and Smith claim a plausibility argument that $\eps_{\inf} = 1/2$, which is a Goldilocks value~\cite{Dicke} where no consequence for the Telescope Conjecture can be deduced.\footnote{Variations on this question with other homology theories are also interesting.  For instance, a consequence of the Hopkins--Smith periodicity theorems is that a finite complex is type $d$ if and only if $\lim_{n \to \infty} g(n) / n = (2(p^d-1))^{-1}$ holds for $g$ formed from the $H\Z/p$--based Adams spectral seqeunce~\cite[Section 3.5]{HopkinsICMZurich}.}

Another important observation is that $\eta$ is not nilpotent in the $E_2$--term of the $MU$--Adams spectral sequence, and so generally the nilpotence theorems---the most impactful theorems known about the global stable category---do not hold in the algebraic model $\CatOf{QCoh}(\moduli{fg})$.  Rather, the nilpotency of $\eta$ is enforced by a differential further in the spectral sequence.  This differential is actually also algebraic, but of a different nature: it can be deduced from the $C_2$--fixed point spectral sequence for $KU^{hC_2} \simeq KO$.  Understanding what bouquet of extra algebraic techniques account for the general nilpotency of stable homotopy elements would be very interesting.

Yet another interesting observation about this same homotopy fixed point spectral sequence is that it has a horizontal vanishing line on the $E_4$ page.  This phenomenon is quite generic~\cite{MathewMeier}, and in particular it is also true of the descent spectral sequence for $\TMF$.  This is quite intriguing: the moduli stack of elliptic curves is Artinian (as are most ``well-behaved'' moduli stacks considered in arithmetic geometry), meaning that it has finite stabilizer groups.  Such stacks are especially amenable to geometric study.  However, this same finiteness is the source of the infinite cohomological dimension of the moduli of elliptic curves: after all, every nontrivial finite group has infinite group cohomology with coefficients in the trivial integral representation.  In this sense, the \emph{derived} moduli of elliptic curves enjoys both of these benefits: it has finite stabilizer groups, and it simultaneously is, in a certain sense, of \emph{finite} cohomological dimension.  Surely this is useful for something: there must be some facts that arithmetic geometers wish were true, but which are stymied by the infinite cohomological dimension of $\moduli{ell}$.  Finding a use for this may well allow information to flow out of homotopy theory and into arithmetic geometry, itself an exciting prospect.

Finally, a precise understanding of vanishing curves in other Adams spectral sequences---for instance Gonz\'{a}lez's results for the $BP\<1\>$--Adams spectral sequence~\cite{Gonzalez}---also give rise to sparsity results in the $MU$--Adams spectral sequence, and hence control of the overall behavior.







\subsection*{Local analysis on $\moduli{fg}$}

The stabilizer action $\S_d \actson E_d^*$ is enormously, inhumanly complicated, as even a passing look at the computations originally pursued by Miller, Ravenel, and Wilson~\cite{MRW} will make clear, nevermind their many extensions by other authors over the decades since.  However quantitatively inaccessible, this action may yet be amenable to qualitative analysis, and there are a great many outstanding conjectures about its behavior in this sense.  The largest one is the chromatic splitting conjecture~\cite[Conjecture 4.2]{HoveyCSC}, which asserts the following claims, listed in ascending order of severity:
\begin{enumerate}
    \item $H^*(\S_d; \W(\F_{p^d})[u^\pm])$ contains an exterior algebra $\Lambda[x_1, \ldots, x_d]$, where $x_j$ has cohomological degree $1$ and transforms in the $(1 - 2j)${\th} character of $\Gm$.\footnote{One of these elements has a concrete description: $x_1$ is described by the determinant homomorphism, which sends a stabilizer element in $\S_d$ to the determinant of its matrix representation as described in \Cref{FormOfStabilizerGroupEarly}.}
    \item Each nonzero class $x_{i_1} \cdots x_{i_j}$ in the exterior algebra pushes forward to a nonzero class in $H^*(\S_d; LT_d[u^\pm])$, and it survives the $E_d$--Adams spectral sequence to give a nonzero homotopy class $x_{i_1} \cdots x_{i_j}\co \S_p^{j-2i_+} \to \widehat L_d \S$.
    \item The composite \[\S_p^{j-2i_+} \xrightarrow{x_{i_1} \ldots x_{i_j}} \widehat L_d \S^0 \to \Susp F(L_{d-1} \S^0, L_d \S_p^0)\] factors through \[L_{d - \max i_k} \S_p^{j-2i_+}.\]
    \item The maps above split $F(L_{d-1} \S^0, L_d \S^0_p)$ into $2^d-1$ summands.
    \item The cofiber sequence \[F(L_{d-1} \S^0, L_d \S^0_p) \to L_{d-1} \S^0_p \to L_{d-1} \widehat L_d \S^0\] splits, so that \[L_{d-1} \widehat L_d \S^0 \simeq L_{d-1} \S^0_p \vee \Susp F(L_{d-1} \S^0, L_d \S^0_p).\]
\end{enumerate}
By exhaustive computation, this conjecture has been verified in the case $d = 1$ and in the case $d = 2$, $p \ge 5$.  Very recently, Beaudry has shown that the final claim of the conjecture is \emph{false} in the case $d = 2$ and $p = 2$~\cite{Beaudry}, but leaves open the other statements in the full conjecture.  Essentially everything else is unknown.

A subtle point in the above statement of the splitting conjecture is that the classes $x_j$ lie in the cohomology of $\W(\F_{p^d})$, which is \emph{not} the Lubin--Tate ring.  Remarkably, the natural map \[H^*(\S_d; \W(\F_{p^d})) \to H^*(\S_d; LT_d)\] has turned out to be an isomorphism in the cases where the conjecture has been verified, and this has turned out to be a linchpin in the rest of the computation---and we do not have a conceptual reason for either of these facts.  Specializing to the case $d = 2$, Goerss has observed that this statement is equivalent to several others, any of which could be a hint toward a conceptual explanation:
\begin{itemize}
    \item The natural map \[H^*(\S_2; \F_{p^2}) \to H^*(\S_2; \F_{p^2}\ps{u_1})\] is an isomorphism.
    \item The Frobenius map \[\Frob\co H^*(\S_2; \F_{p^2}\ps{u_1}) \to H^*(\S_2; \F_{p^2}\ps{u_1})\] is an isomorphism.
    \item The multiplication--by--$v_1^k$ map \[v_1^k\co H^*(\S_2; \F_{p^2}\{v_1^k\}) \to H^*(\S_2; \F_{p^2})\] is zero.
\end{itemize}
In a different direction, one could hope to check that the natural map is an isomorphism just on the relevant torsion-free parts, by finding methods by which to study the maps
\begin{align*}
H^*(\S_d; \W(\F_{p^d})) \otimes \Q \to H^*(\S_d; LT_d) \otimes \Q \to H^*(\S_d; LT_d \otimes \Q).
\end{align*}
As announced by Morava~\cite[Remark 2.2.5]{MoravaCobordismComodules}, methods from the theory of $p$--adic analytic Lie groups show that the source of these maps has exactly the desired form, but comparing the rationalized cohomology with the cohomology of the rational representation is a delicate affair, since the group cohomology of $\S_d$ is being taken in a ``profinite'' sense.

In any event, this only addresses the ``easiest'' part of the conjecture, and the other parts take considerable effort even to parse properly.  The last part is the most interesting: it is a statement about $L_{d-1} \widehat L_d \S$, and hence about the interplay between two different chromatic heights.  Torii has spent much of his career analyzing algebraic models of this phenomenon---in particular, the action of $\S_{d-1}$ on the ``punctured Lubin--Tate ring'' $\F_{p^d}(\!(u_d)\!)[u^\pm]$~\cite{Torii1,Torii2,Torii4,Torii3,Torii5}.  Recent activity on this front indicates that there is much to mine from this vein, and that Torii's program deserves more attention than it has so far garnered.

A related point of interest is the action of $\S_d$ on $\F_{p^d}\ps{u_d}[u^\pm]$.  The action of $\S_d$ on the special fiber of the Lubin--Tate ring is somewhat well-understood~\cite{RavenelCohomologyStabAlgs}; for instance, in the case $d = 2$ we have a calculation \[H^*(\S_2; \F_{p^2}) \cong \left.(\F_{p^2}[\zeta][u^{\pm(p^2-1)}])\{1, h_0, h_1, g_0, g_1, t\}\middle/\left(\begin{array}{c} h_0 g_1 = t, \\ h_1 g_0 = t \end{array}\right)\right. ,\] with all unlabeled products of the Roman elements equal to zero.  This forms the input to a Bockstein spectral sequence \[H^*(\S_2; \m^j / \m^{j+1}) \Rightarrow H^*(\S_2; \F_{p^2}\ps{u_1}[u^\pm]),\] where $\m = (u_1)$ is the maximal ideal in this local graded ring, and hence the representation $\m^j / \m^{j+1}$ is (a twist of) the representation $\F_{p^2}$.  This Bockstein spectral sequence displays a number of intriguing phenomena\footnote{For $d > 2$ there are a succession of similar Bockstein spectral sequences, where the Lubin--Tate generators are reintroduced to the Lubin--Tate ring one at a time.  The first of these spectral sequences always displays these same intriguing phenomena, but the later ones are much more poorly understood.}---for instance, it has ``periodic'' differentials~\cite{Sadofsky}.  One such periodic family is specified by
\begin{align*}
d_1 u^{1-p^2} & = (u_1 u^{1-p}) \cdot h_1, \\
d_p u^{p(1-p^2)} & = (u_1 u^{(1-p)})^p \cdot u^{(p-1)(1-p^2)} \cdot h_0, \\
d_{p^n+p^{n-1}-1} u^{p^n(1-p^2)} & = 2 (u_1 u^{(1-p)})^{p^n + p^{n-1} - 1} \cdot u^{(p^n - p^{n-1})(1 - p^2)} \cdot h_0.
\end{align*}
Even more interestingly, there are explicit lifts of these cohomology classes to cochains in the cobar complex that witness these differentials, given as follows: \[v_2^{(n)} = (u^{1-p^2})^{p^n} \prod_{j=0}^{n-2} \left(1 - u_1^{(p+1)(p^{n-1} - p^j)} \right) \pmod{u_1^{p^n + p^{n-1}}}.\]  As $n$ grows large, this formula looks curiously like an analytic Weierstrass product.  It would be interesting to know what function it names, which could perhaps help illuminate the behavior of the spectral sequence itself.





\subsection*{$p$--adic Interpolation}

In projective geometry, one studies a projective variety by calculating its global sections against different line bundles: \[X \mapsto [(\L \in \Pic(X)) \mapsto H^0(X; \L)].\]  Done correctly, this can be used to recover a graded ring $R$ with a natural map $\operatorname{Proj}(R) \to X$ which is definitionally an isomorphism in the case that $X$ is affine---that is, this construction captures the isomorphism type of $X$.

This is somewhat analogous to Whitehead's theorem in homotopy theory: the Picard group of the stable category (i.e., the group of isomorphism classes of spectra $Y$ such that there exists a $Y^{-1}$ with $Y \sm Y^{-1} \simeq \S^0$) consists of precisely the stable spheres.  However, in the $K(1)$--local category there are many more invertible objects---in fact, the Picard group of this category has the form \[\operatorname{Pic}(\CatOf{QCoh}(\CatOf{Spectra}_{\G_m})) \cong \Z_p \times \Z / (2p-2),\] where the right-hand factor exactly tracks the degree of a generator of the $K(1)$--homology of an invertible spectrum.  Although Whitehead's theorem only requires testing against the ``standard spheres'', now a mere subgroup of the \emph{much} larger Picard group, the homotopy groups of $K(1)$--local spectra as graded over the larger group display considerable extra structure.  For instance, the homotopy groups of the $K(1)$--local sphere graded over this larger group are described by the same formula given in \Cref{piLK1SExample}: \[\pi_t \widehat L_1 \S^0 = \begin{cases} \Z_p & \text{when $t = 0$}, \\ \Z_p / (pk) & \text{when $t = k|v_1| - 1$}, \\ 0 & \text{otherwise}, \end{cases}\] where the symbol $k$ can now be taken to be any $p$--adic integer.  This exactly enforces a kind of $p$--adic continuity of this family of groups.

The idea is that these kinds of observations can be used to bundle the behavior of the homotopy of the $\Gamma_d$--local sphere into a digestable format.  The computation of the homotopy of the $\Gamma_2$--local sphere has been fully executed by Shimomura and collaborators~\cite{Shimomura,ShimomuraYabeM20,ShimomuraYabeL2S,BehrensRevisited}, but it is \emph{exceedingly} complicated.  The role of continuity properties is to reduce the seemingly erratic behavior of arbitrary functions down to a specification on a dense set.  Further properties---analyticity, say, or more seriously a direct relationship to $p$--adic analytic number theory---could reduce the statement to something genuinely tractible.  This idea is meant to be analogous to the relative ease of studying number theoretic $L$--functions over trying to understand them through painstaking computation of their special values.  Such a program was initiated by Hopkins~\cite{StricklandpAdicInterpolation}, from which a partial collection of results have emerged.\footnote{Behrens has also pursued a program encoding this problem in terms of modular forms~\cite{BehrensCongruences,BehrensModularDescription,BehrensBuildings}.}  Most strikingly, Mitchell \cite{MitchellIwasawa,HahnMitchell} has fully elucidated this program in the $\G_m$--local category, Hovey and Strickland have shown a restricted continuity result in the general setting~\cite[Section 14]{HoveyStrickland}\footnote{Also interesting are some negative results, such as: the number of Picard-places where the homotopy of $\widehat L_2 M(p)$ has infinite order has positive Haar measure~\cite[Section 15.2]{HoveyStrickland}.}, and the original work of Hopkins, Mahowald, and Sadofsky~\cite{HMS} shows that the following pair of squares are both pullbacks:
\begin{center}
\begin{tikzcd}
\CatOf{Spectra}_{\Gamma_d} \arrow{r} \arrow[bend left=12]{rr} & \CatOf{QCoh}((\moduli{fg})^\wedge_{\Gamma_d}) \arrow[densely dotted]{r} & \CatOf{QCoh}(k) \\
\operatorname{Pic}(\CatOf{Spectra}_{\Gamma_d}) \arrow{r} \arrow{u} & \operatorname{Pic}(\CatOf{QCoh}((\moduli{fg})^\wedge_{\Gamma_d})) \arrow{r} \arrow{u} & \operatorname{Pic}(\CatOf{QCoh}(k)) \arrow{u},
\end{tikzcd}
\end{center}
allowing for the easy detection of invertible $\Gamma_d$--local spectra.\footnote{There are interesting basic open questions about the behavior of the horizontal map $\operatorname{Pic}(\CatOf{Spectra}_{\Gamma_d}) \to \operatorname{Pic}(\CatOf{QCoh}((\moduli{fg})^\wedge_{\Gamma_d}))$.  Is it injective?  Is it surjective?  It can be shown to be injective in the case of $p \gg \operatorname{ht} \Gamma$, and it is known to fail to be injective when $p$ is small---but then the precise degree to which it fails to be injective becomes of interest.  Picard elements in the kernel of this map are called \textit{exotic}, and they are responsible for rotating differentials in certain Adams spectral sequences with cyclic symmetries.}

There is further hope that the analogy to special values of $L$--functions can be strengthened to a precise connection: the orders of the $\G_m$--local homotopy groups of the sphere (or, equivalently, the orders of the stable image of $j$ elements) are controlled by Bernoulli denominators, which also appear as the negative special values of the Riemann $\zeta$--function.\footnote{It would be truly amazing to have an analogue of the Beilinson conjectures in this topological Langlands program.}  This portends to be more than coincidence: the conjectured $p$--adic local Langlands correspondence promises a comparison between certain ``nice'' representations of the groups $\mathit{GL}_d(\Q_p)$, $\operatorname{Gal}(\overline{\Q_p} / \Q_p)$, and $\S_d$ in such a way that the $L$--functions naturally associated to each representation are equal across the correspondence.  At $d = 1$, the relevant spaces of representations are identically equal (and, in particular, the groups $\S_1$, $\mathrm{GL}_1(\Q_p)$, and $\operatorname{Gal}(\overline{\Q_p} / \Q_p)^{\mathrm{ab}}$ are themselves all equal), and the $L$--function associated to the $\S_1$--representation $E_{\G_m}(\S^0)$ is, indeed, the Riemann $\zeta$--function.  It is rather a lot to hope that this correspondence continues at higher heights, but with the current rate of progress on the $p$--adic Langlands correspondence (see, e.g., \cite{KnightThesis}) this promises to soon be testable, if not provable.\footnote{An interesting feature of the local Langlands correspondence is that it has \emph{two} geometric instantiations, stemming from the Lubin--Tate moduli of infinite level and from the \emph{Drinfel'd moduli} of infinite height.  Drinfel'd modules, the constituent poits of the Drinfel'd moduli, have the many desirable properties, but among their stranger properties is that they are naturally positive objects---a situation that a modern homotopy theorist essentially never enters.  Any kind of introduction of equicharacteristic algebraic geometry into homotopy theory would be a welcome invitation and possible point of homotopical contact between these two kinds of halves of the local Langlands program.  (One such place this is coming into view is in the ``ultrachromatic program'' of Barthel, Behrens, Schlank, and Stapleton.)}
\todo{Include a footnote summarizing: Comparison of comodules $M$ for the isogenies pile with the action of $M_n(\Z_p)$ on $M \otimes_{E_n^*} D_\infty$ (this is a modern result due to Tomer, Tobi, Lukas, and Nat).  This is basically Nat's rational claim: start with a sheaf on the isogenies pile.  Tensor everything with $\Q$.  That turns this thing into a rational algebra under the Drinfel'd ring together with an equivariant action of $\GL_n \Q_p$.}







\todo{Summarize this more briefly.}
The program above is not the only way that homotopy theory appears to admit $p$--adic interpolation, as the following striking result of Yanovski demonstrates.  A \textit{generalized homotopy cardinality function} is a function $\chi$, defined only on spaces generated under finite colimits by $\pi$--finite spaces and valued in rational numbers, which satisfies
\begin{enumerate}
    \item Homotopy invariance: if $A \simeq B$, then $\chi(A) = \chi(B)$.
    \item Normalization: $\chi(*) = 1$.
    \item Additivity: $\chi(A \cup_C B) = \chi(A) + \chi(B) - \chi(C)$.
    \item Multiplicativity: $\chi(A \times B) = \chi(A) \cdot \chi(B)$.
\end{enumerate}
There is a natural family of examples of such functions: at any prime $p$, set \[\chi_{d,p}(X) = \dim K(d)^0(X) - \dim K(d)^1(X).\]  The special cases $\chi_{0, p}$ and $\chi_{\infty, p}$ recover the Euler characteristic in the cases where they converge, but this quickly gets fussy.  For instance, $B\Z/p$ is a $\pi$--finite space but is cohomologically infinite dimensional---but its Euler characteristic is \textit{regularized} to give closed sums like \[\chi_{\infty, p}(B\Z/p) = 1 - (p-1) + (p-1)^2 - \cdots = \frac{1}{p}.\]  The finite-height versions are much better behaved: using the results of \Cref{CoopnsForMoravaKandHA}, we can quickly compute
\begin{align*}
\chi_{d,p}(B\Z/p) & = p^d, &
\chi_{d,p}(\OS{H\Z/p}{m}) & = p^{\binom{d}{m}},
\end{align*}
without any handwaving about summation.

In another direction, the function \[|X| = \sum_{x_0 \in \pi_0 X} \prod_{n=1}^\infty |\pi_n (X, x_0)|^{(-1)^n} \in \Q_{\ge 0}\] was shown to be a homotopy cardinality function (on $p$--local $\pi$--finite spaces) by Baez and Dolan~\cite{BaezDolan}.  In the same toy setting as above, we compute \[|B\Z/p| = \frac{1}{p}.\]  This is quite a curiosity, to get the same value as with the regularized sum, and Baez and Dolan conjectured that whenever these formulas could be simultaneously made sense of, they would agree.  Yanovsky demonstrated this conjecture to be true: taking $p \ge 2$, he showed that the function $n \mapsto \chi_{n, p}(X)$ extends uniquely to a dyadic analytic function $\widehat L_{X,p}\co \Z^\wedge_2 \to \Z^\wedge_2$ and that the Baez--Dolan function appears as $\widehat L_{X,p}(-1)$.  This justifies $p$--adic analytic continuation formulas relating the two invariants, such as \[\widehat L_{\OS{H\Z/p}{m}, p}(-1) = p^{\binom{-1}{m}} = p^{(-1)^m} = |\OS{H\Z/p}{m}|.\]







\subsection*{Interaction with Dieudonn\'e theory}

In \Cref{SectionDieudonneModules}, we gave a pair of definitions of a Dieudonn\'e module, one contravariant and one covariant, and in \Cref{DieudonneDualityThm} we announced that the resulting modules were linearly dual to one another.  This is a remarkably difficult theorem to prove, it is hard to find an accessible proof in the literature,\footnote{In fact, Grothendieck introduced the de Rham--Dieudonn\'e functor and crystalline Dieudonn\'e theory in the same landmark paper, but elected not to provide a proof of an equivalence of his methods with past ones.} and a \emph{geometric} proof appears not to exist.  This doesn't have the same grandeur as the preceding open problems, but rabbit hole still seems to be quite a lot deeper than one might first expect.

To set the stage, we summarize the proof of Mazur and Messing~\cite[Section II.15]{MazurMessing}.  First, the functor of curves is both representable and corepresentable in the category of formal group schemes by the \textit{formal (co)Witt scheme}~\cite[Chapter 3]{ZinkCartierTheory}, \cite[Section III.4]{LazardCFGs}:
\begin{align*}
\CatOf{FormalSchemes}(\A^1, \G)^{\ptyp} & \cong \CatOf{FormalGroups}(\widehat{\W}_p, \G) \\
& \cong \CatOf{FormalGroups}(\G^*, \widehat{C\mathbb W}_p).
\end{align*}
There is a canonical short exact sequence \[0 \to \G_a \to \widehat{C\W}_p \xrightarrow{V} \widehat{C\W}_p \to 0,\] and a co-curve $\gamma^*\co \G \to \widehat{C\W}_p$ gives a pullback sequence
\begin{center}
\begin{tikzcd}
0 \arrow{r} & \G_a \arrow{r} & \widehat{\W}_p \arrow["V"]{r} & \widehat{\W}_p \arrow{r} & 0 \\
0 \arrow{r} & \G_a \arrow{r} \arrow[equal]{u} & E \arrow{r} \arrow{u} & \G \arrow{r} \arrow["\gamma^*"]{u} & 0.
\end{tikzcd}
\end{center}
This latter sequence is a \textit{rigidified extension} of $\G$ by $\G_a$, as in \Cref{ExtensionsPresentationOfDieudonne}.  The conclusion of Mazur and Messing is that this assignment is an isomorphism.

We suspect that the two Dieudonn\'e functors can be connected via a \textit{residue pairing} between forms on $\G$ over $\W(k)$ and curves on $\G_0$ over $k$.  For inspiration, the residue pairing for an equicharacteristic local field \[\<-,-\>\co k(\!(z)\!) \times k(\!(z)\!)^* \to k\] is given by the formula \[\<g, f\> = \operatorname{Res}_{z=0}\left(g \cdot \frac{\mathrm df}{f}\right) = \operatorname{Res}_{z=0}(g \cdot \mathrm d\log f).\]  There are several ingredients in its construction for which we must find formal group analogues.  First, we must deal with the different ground objects $\W_p(k)$ and $k$ for $\G$ and $\G_0$:
\begin{lemma}[{\cite[Lemma VII.7.5]{LazardCFGs}}]
There is a $(\W, F)$--linear section $\sigma$ of the base-change map $D_* \G \to D_* \G_0$.
\end{lemma}
\begin{proof}[Construction]
Cartier constructs an auxiliary Dieudonn\'e module $M$ by picking a presentation\footnote{The formal group so constructed does not actually depend upon the presentation.} \[D_* \G_0 = \left.\W_p(k)\{\gamma, V\gamma, \ldots, V^{d-1} \gamma\} \middle/ \left(F V^i \gamma = \sum_{j=0}^{d-1} c_{i,j} V^j \gamma\right) \right.\] and forming the module \[M := \left.\W_p(\W_p(k))\<\!\<V\>\!\>\{\widetilde\gamma_0, \ldots, \widetilde\gamma_{d-1}\} \middle/ \left(F \widetilde \gamma_i = \sum_{j=0}^{d-1} \Delta(c_{i,j}) \widetilde\gamma_j\right) \right. .\]  There is a $(\W, F)$--linear map $\tau\co D_* \G_0 \to M$ given by base-change along $\Delta$, and Cartier shows that this map has a universal property~\cite[VII.2.9]{LazardCFGs}: $(\W, F)$--linear objects under $D_* \G_0$ agree with Dieudonn\'e modules under $M$ by restriction along $\tau$.  Reapplying the change-of-rings functor $\pi_* M$ gives the Cartier--Dieudonn\'e module for the universal additive extension of $\G_0$~\cite[V.6.22 and VII.2.28]{LazardCFGs}, and hence $D_* \G$ is a Dieudonn\'e module under $\pi_* M$, and hence under $M$.  This gives the desired map \[\sigma\co D_* \G_0 \to D_* \G. \qedhere\]
\end{proof}

Second, we must produce an analogue of logarithmic differentiation, which is responsible for the bilinearity on the side of the differential form.  This arises naturally when trying to find analogues of differentiation internal to formal groups:
\begin{enumerate}
    \item The fiber sequence \[T_0 V \to T_* V \to V\] described around \Cref{ConstructionTangentAffineScheme} does not naturally split: an arbitrary formal variety has no intrinsic notion of ``constant vector field''~\cite[V.11.12]{LazardCFGs}.  However, in the presence of a formal group structure $\widehat{\mathbb H}$ on $V$, the tangent space acquires a natural splitting analogous to the splitting in classical Lie theory:\footnote{This topic also arises when trying to understand what is special about curves in the image of Cartier's section $\sigma$: they are \emph{horizontal} in a related sense.}
    \begin{align*}
    \mathbb H \times (\G_a \otimes \Lie \widehat{\mathbb H}) & \xrightarrow{\cong} T_* \widehat{\mathbb H}, \\
    (x, \xi) & \mapsto x +_{\widehat{\mathbb H}} \eps \cdot \xi.
    \end{align*}
    This gives rise to an invariant notion of differentiation: given two formal groups $\G$, $\widehat{\mathbb H}$ as well as any pointed map $f\co \widehat{\mathbb H} \to \G$ of formal varieties, there is a function \[D_{\widehat{\mathbb H}}^{\G} f\co \widehat{\mathbb H} \to \G_a \otimes (\Lie \widehat{\mathbb H})^* \otimes \Lie \G\] characterized by \[T_* f(x +_{\widehat{\mathbb H}} \eps \cdot \xi) = f(x) +_{\widehat{\mathbb H}} \eps \cdot (D_{\G}^{\widehat{\mathbb H}} f)(x) \xi.\]
    \item By taking $\G_a$ as a model for $\A^1$, any curve $\gamma\co \A^1 \to \G$ can be interpreted as a map of formal varieties $\gamma\co \G_a \to \G$.  Applying the recipe above gives rise to a function $\A^1 \to \G_a \otimes \Lie \G$, i.e., a series with coefficients in $\Lie \G$.  This series can actually be given explicitly~\cite[V.7.3]{LazardCFGs}: \[\gamma \mapsto \sum_{n=1}^\infty t^{n-1} \Lie(F_n \gamma).\]  In particular, for $\G = \G_a$, this computes the classical derivative of $\gamma$ expanded in the canonical coordinate~\cite[V.7.13]{LazardCFGs}.
    \item This assignment is compatible with the Dieudonn\'e module structures on curves on $\G$ and curves on $\G_a \otimes \Lie \G$.  The inverse Dieudonn\'e functor then gives rise to a map \[\breve D_{\G}\co \G \to \G_a \otimes \Lie \G,\] called the \textit{reduced derivative}.  This construction is natural in $\G$, so if $\G$ had a logarithm, there would be a commuting square
    \begin{center}
    \begin{tikzcd}[column sep=4em]
    \G \arrow["\breve D_{\G}"]{d} \arrow["\log_{\G}"]{r} & \G_a \otimes \Lie G \arrow["\breve D_{\G_a \otimes \Lie G}"]{d} \\
    \G_a \otimes \Lie \G \arrow["\log_{\G_a \otimes \Lie \G}", equal]{r} & \G_a \otimes \Lie G.
    \end{tikzcd}
    \end{center}
    The bottom logarithm is the identity, since the source is already additive.  If we evaluate these maps on a curve $\gamma \in C_* \G$, we arrive at the equation~\cite[V.8.5]{LazardCFGs} \[\breve D_{\G} \circ \gamma = \breve D_{\G_a \otimes \Lie G} \circ \log_{\G} \circ \gamma = \frac{\mathrm d}{\mathrm dt} \left( (\log_\G \circ \gamma)(t) \right).\]  Hence, $\breve D_{\G}$ plays the role of a logarithmic derivative.
\end{enumerate}

The conjecture, then, is that these pieces go much of the way towards describing a local residue pairing internal to the theory of formal groups, but there are still several details left to the enterprising reader, such as: should the de Rham complex be replaced?\footnote{Relatedly, can $\breve D_{\G}$ or $D_{\widehat{\mathbb H}}^{\widehat{\mathbb G}}$ be used to give a coordinate-free description of the deformation complex of \Cref{DeformationComplex}?}  Where do \emph{primitives} in de Rham cohomology enter play?  How do the formulas for $\sigma$ in terms of $\breve D_{\G}$~\cite[VII.6.14]{LazardCFGs} enter this story---presumably in demonstrating perfection?

In a more topological direction, one wonders to what extent it is critical to use $\HFp$ in the machinery built up in \Cref{LEFTCooperations}.  In particular, could $\HFp$ be replaced by another field spectrum to build Morava $K$--theoretic analogues of Brown--Gitler spectra?  There is some existing work on this in the height $1$ case, due to Bousfield~\cite{BousfieldLambdaRings}, but the matter remains ultimately unsettled.





\subsection*{Structured ring spectra}

As indicated in the introduction to \Cref{PowerOpnsChapter}, the algebraic geometry of $E_\infty$--ring spectra is still very much in flux, and accordingly there are a lot of interesting open questions about their interaction with the story presented here.

One of the most obvious ones, almost directly cribbed out of \Cref{ConstructionOfTMFSection}, is whether $\moduli{fg}$ itself admits a topological enrichment~\cite{GoerssRealizingFamilies}.  This question is quite flexible: varying the Grothendieck topology chosen on $\moduli{fg}$ will almost certainly affect the positivity of the answer, and there are versions of the question that apply to $E_\infty$ rings, $A_\infty$ rings, or just spectra.  Still, even admitting the existence of a huge family of implicit questions, very little is known.  On the looser end, we do not have an example of a formal group that cannot arise as the formal group associated to a complex-orientable cohomology theory.  If we had such an example of a prohibited formal group, then if the selecting map $\Spec R_0 \to \moduli{fg}$ were furthermore---for example---flat, then we could conclude that there cannot exist a topological enrichment of the flat site.  We do have one rather extreme example: one cannot adjoin $p${\th} roots to $E_\infty$--rings~\cite{SchwaenzlRolandVogt,Devalapurkar}, from which it follows that certain forms of $K$--theory do not have $E_\infty$ structures, and hence the fpqc site of $\moduli{fg}$ does not admit a topological enrichment with a sheaf of $E_\infty$--rings.  A recent result of Lawson also shows that $BP$ does \emph{not} admit an $E_\infty$ structure---at least at $p = 2$, but his techniques are expected to yield results at odd primes as well~\cite{LawsonSecondaryPowerOps}.  This doesn't yield any information over what the above example of $KU[\zeta_p]$ shows, but it is worth pointing out that May asked the question of whether $BP$ admits an $E_\infty$ structure over three decades prior~\cite{MayProblemsInLoopspaceTheory} (and it has had a long and storied intervening history), and the delay in its resolution is some indication of the difficulty of this problem.  Some other richer stacks than $\moduli{fg}$ are known not to admit enrichment: for instance, for this same reason it follows that there cannot exist a topological enrichment of the moduli of (almost any piece of) formal groups equipped with level--$p$ structures,\footnote{In particular, this appears to inhibit us from producing a spectrum embodying the Lubin--Tate tower at infinite level, which is a bummer from the perspective of the local Langlands program discussed above.} and Lawson has shown that there cannot exist spectra associated to certain stacks of formal $A$--modules~\cite{LawsonRealizability}.  Meanwhile, other nearby stacks do admit enrichemnts: there are spectra $\TAF$, an abbreviation for ``topological automorphic forms'', which generalize $\TMF$~\cite{BehrensLawson}.

Meanwhile, even though we know that the spectrum $E_\Gamma$ admits an $E_\infty$--ring structure, and despite our analysis of \Cref{PowerOpnsSection}, it is not known whether there is an $E_\infty$--orientation.  This question has received plenty of attention: most recently, Hopkins and Lawson have produced a spectral sequence computing the space of such maps, the first differential of which encodes the norm coherence condition of \Cref{PowerOpnsSection}~\cite{HopkinsLawson}.  Their work organizes a process that is manually understandable at low heights: the natural map $\mathbb P(\Susp^{-2} \CP^\infty) \to MU$ is a rational isomorphism of $E_\infty$--rings, but at height $1$ it is not, essentially because the collection of power operations for $\G_m$--local $E_\infty$--rings acts freely on the source but not on the target.  This can be corrected for with a single rationally-acyclic term, altogether begetting a two-term free resolution of $MU$ as an $L_1$--local $E_\infty$--ring spectrum.  Again, this resolution is insufficient when localized at $\Gamma_2$, and the process continues ad infinitum as in the Hopkins--Lawson paper.  Understanding this spectral sequence even in the height $2$ case, as specialized to $E_{\Gamma_2}$, would be an extremely interesting exercise, as it would illuminate the single extra condition at height $2$ needed to enrich a norm coherent orientation to an $E_\infty$--orientation.







Let us turn our attention away from the (rather burning) question of realizability.  As we observed in the main proof of \Cref{JuvitopTalkSection}, the $E_\infty$--ring structure on a spectrum can be mined for interesting arithmetic information in rather unexpected ways: computation of the Miller invariant for $KO$ yields the Bernoulli numbers, and the same for $\TMF$ yields the normalized Eisenstein series.  The original computation of the Miller invariant for $MU$ was performed by studying the $S^1$--transfer map, and it was concluded that the results are essentially the same for any complex-orientable cohomology theory: there is a set of ``universal Bernoulli numbers'' lying over $\moduli{fgl}$~\cite{MillerBernoulliNos}, and the Miller invariant for other complex-orientable theories is always computed by the pushforward of these universal values along the classifying map.  Baker, Carlisle, Gray, Hilditch, Richter, and Wood~\cite{BCGHRW} have shown that one can produce other interesting number theoretic phenomena by instead using the iterated $S^1$--transfer (or, equivalently, the transfer for a torus).  Since Morava's theories $E_\Gamma$ are known to be $E_\infty$--rings, one wonders what functions on Lubin--Tate space these iterated transfers select.

Hopkins has also highlighted the potential connections between $\gl_1 \tmf$, number theory, and manifold geometry~\cite{HopkinsTheStringOrientation}, in analogy to the connections between $\gl_1 kO$ (through the image of $j$ spectrum), number theory (through Bernoulli numbers), and manifold geometry (through the Hopf invariant one problem).  One place the analysis of $\gl_1 KO$ could be strengthened (and which would, presumably, also strengthen the analysis of $\gl_1 \tmf$) would be to \emph{uniquely} characterize the Bernoulli sequences appearing in \Cref{JuvitopTalkSection}.  More than one sequence satisfies those simultaneous congruences~\cite{SprangNaumann}, but the natural one appearing in the Bernoulli numbers is in some sense the ``smallest'' one.  It would be interesting to have a non-tautological encoding of this statement, in such a way that certain orientations in other contexts might also be singled out as preferable.  It would also be interesting if this could be encoded as a ``real place'' condition, which had a topological incarnation in terms of a smooth cohomology theory like differential real $K$--theory.

An unpublished theorem of Hopkins and Lurie states that the \textit{discrepancy spectrum}, defined by the fiber sequence \[F_\Gamma \to \gl_1 E_\Gamma \to L_d \gl_1 E_\Gamma,\] satisfies $F_\Gamma \simeq \Susp^d I_{\Z}$, where $I_{\Z}$ is the Anderson dualizing spectrum.\footnote{This follows from the positive resolution of \cite[Conjecture 5.4.14]{HopkinsLurie}, which states $\CatOf{Spectra}(H\Z/p, \gl_1 E_\Gamma) = \Susp^d H\Z/p$, together with the characterization that a coconnective spectrum $R$ (such as $F_\Gamma$) satisfies $\CatOf{Spectra}(H\Z, R) \cong H\Z$ if and only if $R$ is the Anderson dualizing spectrum.}  The Anderson dualizing spectrum fits into the fiber sequence \[I_{\Z} \to I_{\Q} \to I_{\Q/\Z},\] where the rational dualizing spectrum $I_{\Q} = H\Q$ and Brown--Comenetz dualizing spectrum $I_{\Q/\Z}$ are defined so as to satisfy the relations
\begin{align*}
\pi_0 F(E, I_{\Q}) & = \CatOf{AbelianGroups}(\pi_0 E, \Q), \\
\pi_0 F(E, I_{\Q/\Z}) & = \CatOf{AbelianGroups}(\pi_0 E, \Q/\Z).
\end{align*}
By setting $E = \S$, it follows that the homotopy groups of $I_{\Z}$ mostly match those of $\S$ itself (or, rather, their Pontryagin duals), so computing them is equivalent to computing the homotopy groups of spheres.  The Hopkins--Lurie theorem gives an interesting approach to performing this computation: the chromatic fracture square for $L_d \gl_1 E_\Gamma$ couples to the logarithm to rewrite this square in terms of $\widehat L_j$--localizations of $E_\Gamma$, intertwined by very complicated power operation formulas.  It would be an extremely interesting exercise to see this play out even at low heights.

Finally there is a substantially different approach to associating algebro-geometric objects to the elliptic cohomology of spaces, as pioneered in work of Grojnowski~\cite{Grojnowski} and of Ginzburg--Kapranov--Vasserot~\cite{GKV}, which has been used to enormous effect by Lurie in his wide-ranging program to realize various pieces of elliptic cohomology completely internally to the algebric geometry of $E_\infty$--ring spectra~\cite{LurieSurveyOfEll}.\footnote{The $\sigma$--orientation arises essentially from considering the objects associated to $BU(n)$ and $BO(n)$ using these machines (see \cite[Section 5.1]{LurieSurveyOfEll} especially) and identifying certain naturally occuring local systems (of spectra).}  Any serious student of elliptic cohomology should also become familiar with this framework.  It has one feature that is especially curious and worth emphasizing here: the study of power operations is incidental to the Lurie-style approach of working with $E_\infty$ ring spectra, but his framework uses in a critical way the geometry of $p$--divisible groups.  It must therefore be the case that the structure of a $p$--divisible group---and, more presumptuously and more specifically, its subgroup lattice as in \Cref{PowerOpnsSection}---must entirely encode its theory of power operations.  It is very much worthwhile to understand this connection in more generality than just in the setting of Morava $E$--theory.







\subsection*{Equivariance}

Since the resolution of the Kervaire invariant one question by Hill, Hopkins, and Ravenel~\cite{HHR}, the field of equivariant homotopy theory has enjoyed a serious revitalization after a rather quiet preceding period.  There is a lot to study here, and most of it is better elucidated elsewhere, but one of the central points of the Hill--Hopkins--Ravenel program is exactly the employ of formal geometry.  Namely, one of their main tools is the moduli of formal $\Z[\!\!\sqrt{i}]$--modules, which are formal groups equipped with a certain $C_8$--action affecting multiplication by the associated powers of $\!\!\sqrt{i}$ on the tangent space.  They take as a model for this a certain bouquet of copies of $MU$, which selects four coordinates on a given formal group, intertwined by this action of $\!\!\sqrt{i}$.  This approach to this problem has historical precedent: the Kervaire invariant one problem is the $p = 2$ case of a family of prime-indexed problems, and the versions satisfying $p \ge 5$ were simultaneously resolved by Ravenel by similar methods with formal $A$--modules~\cite{RavenelNonexistenceArfInvariantElts}.  This leaves one case open: the Kervaire problem at $p = 3$ remains unresolved.  From the perspective of formal $A$--modules, this is somewhat predictable: at least primes, a certain map is $p$--adically continuous but fails to be continuous at small primes.\footnote{Behrens gave a talk at the MSRI Hot Topics session on the Kervaire problem in 2010.  He addresses this point at the timecode 1h04m20s in the recorded video.}

In a different direction, the moduli of formal $A$--modules has been used to great effect by Salch to make computations deep into the chromatic layers~\cite{Salch}.  In essence, the theory of formal $A$--modules grant access to multiplicative ``height amplification'' theorems: deep knowledge of the homotopy groups of spheres at height $2$ can yield extensive information at height $2 \cdot 2$, for instance.  Salch's methods also promise to yield connections to the $L$--functions perspective described above, fitting into a larger program that describes the homotopy groups of higher-height localizations of spheres in much the way that polyzeta functions are associated to arithmetic schemes of cohomological dimension larger than $1$.  All this is quite frustrating, then, in the face of Lawson's result that the moduli of formal $A$--modules does not admit a topological enrichment~\cite{LawsonRealizability}.

In a yet different direction, Lurie's approach to the reinterpretation of the $\sigma$--orientation through spectral algebraic geometry relies on a concept of ``$2$--equivariance''---spectra equipped with a notion of equivariance not just against a pointed homotopy $1$--type (i.e., a $BG$ for some $G$) but against general homotopy $2$--types.  Relatively little about this has been written down, but one can find an overview in his survey~\cite[Section 5.3]{LurieSurveyOfEll}.



\subsection*{Index theorems}

In closing the book, we finally address the lurking physical inspiration for the $\sigma$--orientation.  One of the most remarkable developments of modern physics, from the perspective of a mathematician, is their ability to ``guess'' the Atiyah--Singer index theorem~\cite[Section 8.6]{Takhtajan}: there is a path integral formulation for the supertrace of the Dirac operator on a $\Spin$ manifold $M$ which expresses it as the time evolution of a massive supersymmetric particle evolving through $M$, and standard expansions for this physical system lead directly to the discovery of the $\widehat A$--genus and, ultimately, the index formula \[\operatorname{ind}\, /\!\!\!\partial_+ = \operatorname{Tr}_s \mathrm e^{-H} = (2 \pi \mathrm i)^{-\frac{n}{2}} \int_M \widehat A(M).\]  The $\sigma$--orientation was uncovered by Witten when studying the free loopspace $\L M$ of the manifold $M$, which is indeed a natural space to consider when studying string theory on $M$.  The condition that $M$ be $\String$ is exactly the precondition that $\L M$ be $\Spin$, so that an analogous Dirac operator exists.  The $\sigma$--orientation then arises when trying to express a formula for the index of the Dirac operator on the free loopspace in terms of invariants of $M$ and using analogous path integral methods~\cite{SegalEll}.

Because of the predictive power of the physical appartus, one might hope to uncover other types of analytic data from it---for instance, a geometric theory of elliptic cohomology.  In addition to the variant of bordism theory described far above, the most long-standing program is due to Stolz and Teichner~\cite{StolzTeichnerWhatIs,StolzTeichnerSusy}, though it is not alone even among physically-inspired theories (for instance, see \cite{DouglasHenriques} as well as many other papers by the same authors).

One of the remarkable features of loop groups is that this process does not appear to be iterable: the theory of compact Lie groups is exceedingly nice (for instance: they have surjective exponential maps), the theory of loopgroups formed from compact Lie groups is only slightly less nice (for instance: they have dense exponential maps), and any extension of this seems to immediately go to pot.\footnote{For instance, little positive is known about the group of free ``tori'', i.e., the loop group of a loop group.}  Finding the ``next step'' after loop groups is an extremely difficult problem that requires such inventive thinking that it could only open enormous avenues for generalization.








