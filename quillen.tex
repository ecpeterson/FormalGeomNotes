% -*- root: main.tex -*-

\chapter{Complex Bordism}\label{ComplexBordismChapter}


Having totally dissected unoriented bordism, we can now turn our attention to other sorts of bordism theories, and there are many available: oriented, \(\Spin\), \(\String\), complex, \ldots---the list continues.  We would like to replicate the results from \Cref{UnorientedBordismChapter} for these other cases, but upon even a brief inspection we quickly see that only one of the bordism theories mentioned supports this program.  Specifically, the space \(\RP^\infty = BO(1)\) was a key player in the unoriented bordism story, and the only other similar ground object is \(\CP^\infty = BU(1)\) in complex bordism.  This informs our choice to spend this Case Study focused on it.  To begin, the contents of \Cref{LectureThomSpectra} can be replicated essentially \textit{mutatis mutandis}, resulting in the following theorems:

\begin{theorem}[{cf.\ \Cref{JIsMonoidal} and surrounding discussion}]\label{ComplexJHomomorphism}
There is a map of infinite--loopspaces \[J_{\C}\co BU \to B \GL_1 \S\] called the \index{J homomorphism@\(J\)--homomorphism}\textit{complex \(J\)--homomorphism}. \qed
\end{theorem}

\begin{definition}[{cf.\ \Cref{DefnOfMO}}]\label{DefnComplexOrientation}
The associated Thom spectrum is written ``\(MU\)'' and called \index{bordism!complex, \(MU\)}\textit{complex bordism}.  A map \(MU \to E\) of ring spectra is said to be a \index{orientation!complex}\textit{complex orientation of \(E\)}.
\end{definition}

\begin{theorem}[{cf.\ \Cref{GeneralThomIsom}}]\label{ThomIsomOverC}\index{Thom isomorphism}
For a complex vector bundle \(\xi\) on a space \(X\) and a complex-oriented ring spectrum \(E\), there is a natural equivalence \[\pushQED{\qed}
E \sm T(\xi) \simeq E \sm \Susp^\infty_+ X. \qedhere
\popQED\]
\end{theorem}

\begin{corollary}[{cf.\ \Cref{HF2RPinftyExample}}]\label{CPinftyNiceCalculation}
In particular, for a complex-oriented ring spectrum \(E\) it follows that \(E^* \CP^\infty\) is isomorphic to a one--dimensional power series ring. \qed
\end{corollary}

We would like to then review the results of \Cref{TheSteenrodAlgebraSection} and conclude (by reinterpreting \Cref{CPinftyNiceCalculation}) that \(\CP^\infty_E\) gives a \(1\)--dimensional formal group over \(\Spec E_*\).  In order to make this statement honestly, however, we are first required to describe more responsibly the algebraic geometry we outlined in \Cref{SectionSchemesOverF2}.  Specifically, the characteristic \(2\) nature of the unoriented bordism ring was a major simplifying feature which made it wholly amenable to study by \(\HFtwo\).  In turn, \(\HFtwo\) has many nice properties---for example, it has a duality between homology and cohomology, and it supports a K\"unneth isomorphism---and these are reflected in the extremely simple algebraic geometry of \(\Spec \F_2\).  By contrast, the complex bordism ring is considerably more complicated, not least because it is a characteristic \(0\) ring, and more generally we have essentially no control over the behavior of the coefficient ring \(E_*\) of some other complex-oriented theory.  Nonetheless, once the background theory and construction of ``\(X_E\)'' are taken care of in \Cref{FormalVarietiesLecture}, we indeed find that \(\CP^\infty_E\) is a \(1\)--dimensional formal group over \(\Spec E_*\).

However, where we could explicitly calculate \(\RP^\infty_{\HFtwo}\) to be \(\G_a\), we again have little control over what formal group \(\CP^\infty_E\) could possibly be.  In the universal case, \(\CP^\infty_{MU}\) comes equipped with a natural coordinate, and this induces a map \[\Spec MU_* \to \moduli{fgl}\] from the spectrum associated to the coefficient ring of complex bordism to the moduli of formal group laws.  The conclusion of this Case Study in \Cref{QuillensTheorem} (modulo an algebraic result, shown in the next Case Study as \Cref{LazardsTheorem}) states that this map is an isomorphism, so that \(\CP^\infty_{MU}\) carries the universal---i.e., maximally complicated---formal group law.  Our route for proving this passes through the foothills of the theory of ``\(p\){\th} power operations'', which simultaneously encode many possible natural transformations from \(MU\)--cohomology to itself glommed together in a large sum, one term of which is the literal \(p\){\th} power.  Remarkably, the identity operation also appears in this family of operations, and the rest of the operations are in some sense controlled by this naturally occuring formal group law.  A careful analysis of this sum begets the inductive proof in \Cref{QuillenSurjective} that \(\sheaf O_{\moduli{fgl}} \to MU_*\) is surjective.

The execution of this proof requires some understanding of cohomology operations for complex-oriented cohomology theories generally.  Stable such operations correspond to homotopy classes \(MU \to E\), i.e., elements of \(E^0 MU\), which correspond via the Thom isomorphism to elements of \(E^0 BU\).  This object is the repository of \(E\)--characteristic classes for complex vector bundles, which we describe in terms of divisors on formal curves.  This amounts to a description of the formal schemes \(BU(n)_E\), which underpins our understanding of the whole story and which significantly informs our study of connective orientations in \Cref{ChapterSigmaOrientation}.









\section{Calculus on formal varieties}\label{FormalVarietiesLecture}

In light of the introduction, we see that it would be prudent to develop some of the theory of formal schemes and formal varieties outside of the context of \(\F_2\)--algebras.  However, writing down a list of definitions and checking that they have good enough properties is not especially enlightening or fun.  Instead, it will be informative to understand where these objects come from in algebraic geometry, so that we can carry the accompanying geometric intuition along with us as we maneuver our way back toward homotopy theory and bordism.  Our overarching goal in this Lecture is to develop a notion of calculus (and analytic expansions in particular) in the context of affine schemes.  The place to begin is with definitions of cotangent and tangent spaces, as well as some supporting vocabulary.
\begin{definition}[{cf.\ \Cref{DefnAffineF2Scheme}}]
Fix a ring \(R\).  For an \(R\)--algebra \(A\), the functor
\begin{align*}
\Spec A \co \CatOf{Algebras}_R & \to \CatOf{Sets}, \\
T & \mapsto \CatOf{Algebras}_R(A, T)
\end{align*}
is called the \index{scheme!affine}\textit{spectrum of \(A\)}.  A functor \(X\) which is naturally isomorphic to some \(\Spec A\) is called an \textit{affine (\(R\)--)scheme}, and \(A = \sheaf O_{\Spec A}\) is called its \index{ring of functions}\textit{ring of functions}.  A subfunctor \(Y \subseteq X\) is said to be a \index{scheme!closed}\textit{closed\footnote{The word ``closed'' is meant to suggest properties of these inclusions: in suitable senses, they are closed under finite unions and arbitrary intersections.  The complementary concept of ``open'' is harder to describe: open subschemes of affine schemes are merely ``covered'' by finitely many affines, which requires a discussion of coverings, which we remit to the actual algebraic geometers~\cite[Definition 8.1]{StricklandFPFP}.} subscheme} when an identification\footnote{This property is independent of choice of chart.} \(X \cong \Spec A\) induces a further identification
\begin{center}
\begin{tikzcd}
Y \arrow{r} \arrow[leftarrow, "\simeq"]{d} & X \arrow[leftarrow, "\simeq"]{d} \\
\Spec (A/I) \arrow{r} & \Spec A.
\end{tikzcd}
\end{center}
\end{definition}

\begin{definition}\label{DefnOfCoTangentSpaces}
Take \(S = \Spec R\) to be our base scheme, let \(X = \Spec A\) be an affine scheme over \(S\), and consider an \(S\)--point \(s\co S \to X\) of \(X\).  This is automatically a closed subscheme, so that \(s\) is presented as \(\Spec A/I \to \Spec A\) for some ideal \(I\).  The \index{cotangent space}\textit{cotangent space} \(T^*_s X\) is defined by the quotient \(R\)--module \[T^*_s X := I / I^2,\] consisting of functions vanishing at \(s\) as considered up to first order.  Examples of these include the linear parts of curves passing through \(s\), so we additionally define the \index{tangent space}\textit{tangent space} \(T_s X\) by \[T_s X = \CatOf{Schemes}_{\Spec R/}(\Spec R[\eps] / \eps^2, X),\] i.e., maps \(\Spec R[\eps] / \eps^2 \to X\) which restrict to \(s\co S \to X\) upon setting \(\eps = 0\).
\end{definition}

\begin{remark}
In the situation above, there is a naturally occurring map \[T_s X \to \CatOf{Modules}_R(T^*_s X, R).\]  Namely, a map \(\sheaf O_X \to R[\eps] / \eps^2\) induces a map \(I \to (\eps)\), and hence induces a further map \[I / I^2 \to (\eps) / (\eps^2) \cong R,\] which can be interpreted as a point in \((T^*_s X)^*\).
\end{remark}

Harkening back to \Cref{FirstAppearanceOfInternalAut}, the definition of the \(R\)--module tangent space begs promotion to an \(S\)--scheme.
\begin{lemma}\label{ConstructionTangentAffineScheme}
There is an affine scheme \(T_s X\) defined by \[(T_s X)(u\co \Spec T \to S) := \left\{ f \mid f \in T_{u^* x} u^* X \right\}.\]
\end{lemma}
\begin{proof}[Proof sketch]
We specialize an argument of Strickland~\cite[Proposition 2.94]{StricklandFSFG} to the case at hand.\footnote{Strickland also shows that mapping schemes between formal schemes exist considerably more generally~\cite[Theorem 4.69]{StricklandFSFG}.  The source either has to be ``finite'' in some sense, in which case the proof proceeds along the lines presented here, or it has to be \index{scheme!coalgebraic}\textit{coalgebraic}, which is an important technical tool that we discuss much later in \Cref{DefnCoalgebraicFormalScheme}.}  We start by seeking an \(R\)--algebra \(B\) that corepresents the entire \textit{tangent bundle}, \(T_* X\): we ask that \(R\)--algebra maps \(B \to T\) biject with pairs of a map \(u\co R \to T\) and a \(T\)--algebra map
\begin{align*}
f\co A \otimes_R T & \to R[\eps] / \eps^2 \otimes_R T.
\intertext{Such a map \(f\) is equivalent data to an \(R\)--algebra map}
A & \to R[\eps] / \eps^2 \otimes_R T.
\end{align*}
Using the sequence of inclusions
\begin{align*}
\CatOf{Algebras}_{R/}(A, R[\eps] / \eps^2 \otimes_R T) & \subseteq \CatOf{Modules}_R(A, R[\eps] / \eps^2 \otimes_R T) \\
& \cong \CatOf{Modules}_R(A \otimes_R (R[\eps] / \eps^2)^*, T) \\
& \cong \CatOf{Algebras}_{R/}(\Sym_R(A \otimes_R (R[\eps] / \eps^2)^*), T),
\end{align*}
we see that we can pick out the original mapping set by passing to a quotient of the domain.  After some thought, we arrive at the equation \[T_* X = \InternalHom{Schemes}_{/S}(\Spec R[\eps] / \eps^2, X) = \Spec \frac{A\{1, \operatorname{d\mathit{a}} \mid a \in A\}}{\left( \begin{array}{c} \text{\(\operatorname{d\mathit{r}} = 0\) for \(r \in R\)}, \\ \operatorname{d}(a_1 a_2) = \operatorname{d\mathit{a}_1} \cdot a_2 + a_1 \cdot \operatorname{d\mathit{a}_2} \end{array} \right)} .\]  To extract the scheme \(T_s X\) from this, we construct the pullback \[T_s X := \InternalHom{Schemes}_S(\Spec R[\eps] / \eps^2, X) \times_X S,\] where the structure maps are given on the left by setting \(\eps = 0\) and on the right using the point \(s\).  Expanding the formulas again shows that the coordinate ring of this affine scheme is given by \[\sheaf O_{T_s X} = A / I^2 \cong R \oplus T^*_s X. \qedhere\]
\end{proof}

\begin{definition}
The ring of functions appearing in the proof above fits into an exact sequence
\begin{align*}
0 & \to \Omega_{A/R} \\
& \to \left. A\{1, \operatorname{d\mathit{a}} \mid a \in A\} \middle/ \left( \begin{array}{c} \text{\(\operatorname{d\mathit{r}} = 0\) for \(r \in R\)}, \\ \operatorname{d}(a_1 a_2) = \operatorname{d\mathit{a}_1} \cdot a_2 + a_1 \cdot \operatorname{d\mathit{a}_2} \end{array} \right) \right. \\
& \to A\{1\} \to 0.
\end{align*}
The kernel \(\Omega_{A/R}\) is called the module of \index{Kahler differentials@K\"ahler differentials}\textit{K\"ahler differentials} (of \(A\), relative to \(R\)).  The map \(\operatorname d\co A\to \Omega^1_{A/R}\) is the universal \(R\)--linear derivation into an \(A\)--module, i.e., \[\CatOf{Derivations}_R(A, M) = \CatOf{Modules}_A(\Omega^1_{A/R}, M).\]
\end{definition}

The upshot of this calculation is that \(\Spec A/I^2\) is a natural place to study the linear behavior of functions on \(X\) near \(s\).  We have also set the definitions up so that we can easily generalize to higher-order approximations:
\begin{definition}\label{JetSpacesDefn}
More generally, the \textit{\(n\){\th} jet space} of \(X\) at \(s\), or the \textit{\(n\){\th} order neighborhood} of \(s\) in \(X\), is defined by \[\InternalHom{Schemes}_S(\Spec R[\eps] / \eps^{n+1}, X) \times_X S \cong \Spec A / I^{n+1}.\]  Each jet space has an inclusion from the one before, modeled by the closed subscheme \(\Spec A/I^n \to \Spec A / I^{n+1}\).
\end{definition}

In order to study analytic expansions of functions, we bundle these jet spaces together into a single object embodying formal expansions in \(X\) at \(s\):
\begin{definition}\label{DefnCompletion}
Fix a scheme \(S\).  A \index{formal scheme}\textit{formal \(S\)--scheme} \(X = \{X_\alpha\}_\alpha\) is an ind-system of \(S\)--schemes \(X_\alpha\).\footnote{This definition, owing to Strickland~\cite[Definition 4.1]{StricklandFSFG}, is somewhat idiosyncratic.  Its generality gives it good categorical properties, but it is somewhat disconnected from the formal schemes familiar to algebraic geometers, which primarily arise through linearly topologized rings~\cite[pg.\ 194]{Hartshorne}.  For functor-of-points definitions that hang more tightly with the classical definition, the reader is directed toward Strickland's solid formal schemes~\cite[Definition 4.16]{StricklandFSFG} or to Beilinson and Drinfel'd~\cite[Section 7.11.1]{BeilinsonDrinfeld}.}  Given a closed subscheme \(Y\) of an affine \(S\)--scheme \(X\), we define the \textit{\(n\){\th} order neighborhood of \(Y\) in \(X\)} to be the scheme \(\Spec A/I^{n+1}\).  The \index{formal scheme!formal neighborhood}\textit{formal neighborhood of \(Y\) in \(X\)} is then defined to be the formal scheme \[X^\wedge_Y := \Spf A^\wedge_I := \left\{ \Spec A/I \to \Spec A/I^2 \to \Spec A/I^3 \to \cdots \right\}.\]  In the case that \(Y = S\), this specializes to the system of jet spaces as in \Cref{JetSpacesDefn}.
\end{definition}

Although we will make use of these definitions generally, the following ur-example captures the most geometrically-intuitive situation.

\begin{example}\label{MapsOfFVarsArePowerSeries}
Picking the affine scheme \(X = \Spec R[x_1, \ldots, x_n] = \mathbb A^n\) and the point \(s = (x_1 = 0, \ldots, x_n = 0)\) gives a formal scheme known as \index{formal scheme!affine n space@affine \(n\)--space}\textit{formal affine \(n\)--space}, given explicitly by \[\A^n = \Spf R\llbracket x_1, \ldots, x_n\rrbracket.\]  Evaluated on a test algebra \(T\), \(\A^1(T)\) yields the ideal of nilpotent elements in \(T\) and \(\A^n(T)\) its \(n\)--fold Cartesian power.  Pointed maps \(\A^n \to \A^m\) naturally biject with \(m\)--tuples of \(n\)--variate power series with no constant term.\footnote{In some sense, this example is a full explanation for why anyone would even think to involve formal geometry in algebraic topology (nevermind how useful the program has been in the long run).  Calculations in algebraic topology have long been expressed in terms of power series rings, and with this example we are provided geometric interpretations for such statements.}
\end{example}

Part of the point of the geometric language is to divorce abstract rings (e.g., \(E^0 \CP^\infty\)) from concrete presentations (e.g., \(E^0\ps{x}\)), so we additionally reserve some vocabulary for the property of being isomorphic to \(\A^n\):
\begin{definition}\label{DefnFormalVariety}
A \index{formal scheme!formal variety}\textit{formal affine variety} (of dimension \(n\), over a base \(R\)) is a formal scheme \(V\) which is (noncanonically) isomorphic to \(\A^n\)---or, equivalently, \(\sheaf O_V\) is (noncanonically) isomorphic to a power series ring.  The two maps in an isomorphism pair \[V \xrightarrow{\simeq} \A^n, \quad V \xleftarrow{\simeq} \A^n\] are called a \index{coordinate!system}\textit{coordinate (system)} and a \index{parameter!system}\textit{parameter (system)} respectively.  Finally, an \(S\)--point \(s\co S \to X\) is called \index{formal scheme!smooth}\textit{formally smooth} when \(X^\wedge_s\) gives a formal variety.
\end{definition}

This definition allows local theorems from analytic differential geometry to be imported in coordinate-free language.  For instance, there is the following version of the \index{inverse function theorem}inverse function theorem:
\begin{theorem}[Inverse function theorem, {\cite[Theorem I.8.1]{LazardCFGs}}]\label{InverseFunctionTheoremForFVars}
A pointed map \(f\co V \to W\) of finite--dimensional formal varieties is an isomorphism if and only if the induced map \(T_0 f\co T_0 V \to T_0 W\) is an isomorphism of \(R\)--modules. \qed
\end{theorem}

Coordinate-free theorems are only really useful if we can verify their hypotheses by coordinate-free methods as well.  The following two results are indispensible in this regard:
\begin{theorem}\label{DetectingFormalVarieties}
% \citeme{This is 9.6.4 in the Crystals notes}
Let \(R\) be a Noetherian ring and \(F\co \CatOf{Algebras}_{R/} \to \CatOf{Sets}_{*/}\) be a functor such that \(F(R) = *\), \(F\) takes surjective maps to surjective maps, and there is a fixed finite free \(R\)--module \(M\) such that \(F\) carries square-zero extensions of Noetherian \(R\)--algebras \(I \to B \to B'\) to product sequences \[* \to I \otimes_R M \to F(B) \to F(B') \to *.\]  Then, a basis \(M \cong R^n\) determines an isomorphism \(F \cong \A^n\).
\end{theorem}
\begin{proof}[Proof sketch]
In the motivating case where \(F \cong \A^n\) is given, we can define \(M\) to be \[M := F(R[\eps] / \eps^2) = (\eps)^{\times n} = R\{\eps_1, \ldots, \eps_n\}.\]  In fact, this is always the case: the square-zero extension \[(\eps) \to R[\eps] / \eps^2 \to R\] induces a product sequence and hence an isomorphism \[* \to (\eps) \otimes_R M \xrightarrow{\cong} F(R[\eps] / \eps^2) \to * \to *.\]  A choice of basis \(M \cong R^{\times n}\) thus induces an isomorphism \[F(R[\eps] / \eps^2) = (\eps) \otimes_R M = M \cong R^{\times n} = \A^n(R[\eps] / \eps^2).\]  Lastly, induction shows that if the maps between the outer terms of the set-theoretic product sequence exist, then so must the middle:
\begin{center}
\begin{tikzcd}
* \arrow{r} & I \otimes_R M \arrow{r} & F(B) \arrow{r} & F(B') \arrow{r} & * \\
* \arrow{r} & I \otimes_R M \arrow[equal]{u} \arrow{r} & \A^n(B) \arrow{r} \arrow[densely dotted, "\simeq"]{u} & \A^n(B') \arrow{r} \arrow["\simeq"]{u} & *.
\end{tikzcd}
\end{center}
\vspace{-\baselineskip}
\end{proof}

\begin{corollary}[{\cite[Th\'eor\`eme III.2.1]{GrothendieckSGAI}}]
We define a \textit{nilpotent thickening}\textit{nilpotent thickening}\index{scheme!nilpotent thickening} to be a closed inclusion of schemes whose associated ideal sheaf is nilpotent.  An \(S\)--point \(s\co S \to X\) of a Noetherian scheme is formally smooth exactly when \(T_s X\) is a projective \(R\)--module and for any nilpotent thickenings \(S \to \Spec B \to \Spec B'\) and any solid diagram
\begin{center}
\begin{tikzcd}[column sep=3em]
S \arrow{r} \arrow["s"]{rd} & \Spec B \arrow{r} \arrow{d} & \Spec B' \arrow[densely dotted]{ld} \\
& X
\end{tikzcd}
\end{center}
there exists a dotted arrow extending the diagram. \qed
\end{corollary}

With all this algebraic geometry in hand, we now return to our original motivation: extracting formal schemes from the rings appearing in algebraic topology.

\begin{definition}[{cf.\ \Cref{FullDefnOfXHF2}}]\label{FullDefnOfXE}
\index{formal scheme!from a space}Let \(E\) be an even-periodic ring spectrum, and let \(X\) be a CW--space.  Because \(X\) is compactly generated, it can be written as the colimit of its compact subspaces \(X^{(\alpha)}\), and we set\footnote{The properties satisfied by this construction suffer greatly if it is not the case that the \(E\)--cohomology of \(X\) is even-concentrated, i.e., if \(E^* X \cong E^* \otimes_{E^0} E^0 X\) is violated.  In fact, even the intermediate failure of \(E^* X_\alpha \cong E^* \otimes_{E^0} E^0 X_\alpha\) can cause issues, but passing to a clever cofinal subsystem often alleviates them.  For instance, such a subsystem exists if \(H\Z_* X\) is free and even~\cite[Definition 8.15, Proposition 8.17]{StricklandFSFG}.}\footnote{In cases of ``large'' spaces and cohomology theories, the technical points underlying this definition are necessary: \(BU_{KU}\) is an instructive example, as it is \emph{not} the formal scheme associated to \(KU^0(BU)\) by any adic topology.} \[X_E := \Spf E^0 X := \{\Spec E^0 X^{(\alpha)}\}_\alpha.\]
\end{definition}

Consider the example of \(\CP^\infty_E\) for \(E\) a complex-oriented cohomology theory.  We saw in \Cref{CPinftyNiceCalculation} that the complex-orientation determines an isomorphism \(\CP^\infty_E \cong \A^1\) (i.e., an isomorphism \(E^0 \CP^\infty \cong E^0\ps{x}\)).  However, the object ``\(E^0 \CP^\infty\)'' is something that exists independent of the orientation map \(MU \to E\), and the language of \Cref{DefnFormalVariety} allows us to make the distinction between the property and the data:
\begin{lemma}
A cohomology theory \(E\) is \index{orientation}\textit{complex orientable} (i.e., it is able to receive a ring map from \(MU\)) precisely when \(\CP^\infty_E\) is a formal curve (i.e., it is a formal variety of dimension \(1\)).  A choice of \index{orientation}orientation \(MU \to E\) determines a coordinate \(\CP^\infty_E \cong \A^1\) via the first \index{Chern class}Chern class associated to the orientation. \qed
\end{lemma}

\begin{remark}\label{EvenPeriodicImplicesCplxO}
An even-periodic ring spectrum is automatically complex-orientable: the Atiyah--Hirzebruch spectral sequence for the cohomology of \(\CP^\infty\) collapses.
\end{remark}

As in \Cref{RPinftyExampleForReal}, the formal scheme \(\CP^\infty_E\) has additional structure: it is a group.  We close this Lecture with some remarks about such objects.

\begin{definition}\label{DefnFormalGps}
A \index{formal group}\textit{formal group} is a formal variety endowed with an abelian group structure.\footnote{Formal groups in dimension \(1\) are automatically commutative if and only if the ground ring has no elements which are simultaneously nilpotent and torsion~\cite[Theorem I.6.1]{Hazewinkel}.}  If \(E\) is a complex-orientable cohomology theory, then \(\CP^\infty_E\) naturally forms a (\(1\)--dimensional) formal group using the map classifying the tensor product of line bundles.
\end{definition}

\begin{remark}
As with formal schemes, a formal group can arise as the formal completion of an algebraic group at its identity point.  It turns out that there are many more formal groups than come from this procedure, a phenomenon that is of keen interest to stable homotopy theorists---see \Cref{OpenQuestionsSection}.
\end{remark}

We give the following Corollary as an example of how nice the structure theory of formal varieties is---in particular, formal groups often behave like physical groups.

\begin{corollary}
The formal group addition map on \(\G\) determines the inverse law.
\end{corollary}
\begin{proof}
Consider the shearing map
\begin{align*}
\G \times \G & \xrightarrow{\sigma} \G \times \G, \\
(x, y) & \mapsto (x, x + y).
\end{align*}
The induced map \(T_0 \sigma\) on tangent spaces is evidently invertible, so by \Cref{InverseFunctionTheoremForFVars} there is an inverse map \((x, y) \mapsto (x, y - x)\).  Setting \(y = 0\) and projecting to the second factor gives the inversion map.
\end{proof}



\begin{definition}\label{FGLDefinition}
Let \(\G\) be a formal group.  In the presence of a coordinate \(\phi \co \G \cong \A^n\), the addition law on \(\G\) begets a map
\begin{center}
\begin{tikzcd}
\G \times \G \arrow{r} \arrow["\simeq"']{d} & \G \arrow["\simeq"]{d} \\
\A^n \times \A^n \arrow{r} & \A^n,
\end{tikzcd}
\end{center}
and hence a \(n\)--tuple of \((2n)\)--variate power series ``\(+_\phi\)'', satisfying
\begin{align*}
\underline{\smash x} +_\phi \underline{\smash y} & = \underline{\smash y} +_\phi \underline{\smash x}, \tag{commutativity} \\
\underline{\smash x} +_\phi \underline{\smash 0} & = \underline{\smash x}, \tag{unitality} \\
\underline{\smash x} +_\phi (\underline{\smash y} +_\phi \underline{\smash z}) & = (\underline{\smash x} +_\phi \underline{\smash y}) +_\phi \underline{\smash z}. \tag{associativity}
\end{align*}
Such a (tuple of) series \(+_\phi\) is called a \index{formal group!law}\textit{formal group law}, and it is the concrete data associated to a formal group.
\end{definition}

Let us now consider two examples of complex-orientable ring spectra \(E\) and describe these invariants for them.

\begin{example}\label{HZGivesGa}
There is an isomorphism \(\CP^\infty_{H\Z P} \cong \G_a\).  This follows from reasoning identical to that given in \Cref{RPinftyExampleForReal}.
\end{example}

\begin{example}\label{CPinftyKUExample}
There is also an isomorphism \(\CP^\infty_{KU} \cong \G_m\).  The standard choice of first Chern class is given by the topological map \[c_1\co \Susp^{-2} \Susp^\infty \CP^\infty \xrightarrow{1 - \beta \L} KU,\] and a formula for the first Chern class of the tensor product is thus
\begin{align*}
c_1(\L_1 \otimes \L_2) & = 1 - \beta(\L_1 \otimes \L_2) \\
& = -\beta^{-1} \left( (1 - \beta \L_1) \cdot (1 - \beta \L_2) \right) + (1 - \beta \L_1) + (1 - \beta \L_2) \\
& = c_1(\L_1) + c_1(\L_2) - \beta^{-1} c_1(\L_1) c_1(\L_2).
\end{align*}
In this coordinate on \(\CP^\infty_{KU}\), the group law is then \(x_1 +_{\CP^\infty_{KU}} x_2 = x_1 + x_2 - \beta^{-1} x_1 x_2\).  Using the coordinate function \(1 - t\), this is also the coordinate that arises on the formal completion of \(\Gm\) at \(t = 1\):
\begin{align*}
x_1(t_1) +_{\Gm} x_2(t_2) & = 1 - (1 - t_1)(1 - t_2) \\
& = t_1 + t_2 - t_1 t_2.
\end{align*}
\end{example}

As an application of all these tools, we will show that the \emph{rational} theory of formal groups is highly degenerate: every rational formal group admits a \index{logarithm}\textit{logarithm}, i.e., an isomorphism to \(\G_a\).  Suppose now that \(R\) is a \(\Q\)--algebra and that \(A = R\llbracket x \rrbracket\) is the coordinatized ring of functions on a formal line over \(R\).  The special feature of the rational curve case is that differentiation gives an isomorphism between the \index{Kahler differentials@K\"ahler differentials}K\"ahler differentials \(\Omega^1_{A/R}\) and the ideal \((x)\) of functions vanishing at the origin (i.e., the ideal sheaf selecting the closed subscheme \(0\co \Spec R \to \Spf A\)).  Its inverse is formal integration: \[\int \co \left(\sum_{j=0}^\infty c_j x^j \right) \dx \mapsto \sum_{j=0}^\infty \frac{c_j}{j+1} x^{j+1}.\]  In pursuit of the construction of a logarithm for a formal group \(\G\) over \(R\), we now take a cue from classical Lie theory:
\begin{definition}
A \(1\)--form \(\omega \in \Omega^1_{A/R}\) is said to be \index{Kahler differentials@K\"ahler differentials!invariant}\textit{invariant (under a group law \(+_\phi\))} when \(\omega = \tau_y^* \omega\) for all translations \(\tau_y(x) = x +_\phi y\).  We write \(\omega_{\G} \subseteq \Omega^1_{A/R}\) for the subsheaf of such invariant \(1\)--forms.
\end{definition}

\begin{lemma}\label{InvDifflsGiveLogsRationally}
For \(R\) a \(\Q\)--algebra, integration gives an isomorphism \[\int\co \omega_{\G} \to \CatOf{FormalGroups}(\G, \G_a).\]
\end{lemma}
\begin{proof}
Suppose that \(f \in \omega_{\G}\) is an invariant differential, which in terms of a coordinate \(x\) indicates the equation
\begin{align*}
f(x) \dx = \omega & = \tau_y^* \omega = f(x +_\phi y) \operatorname{d}(x +_\phi y). \\
\intertext{Writing \(F(x) = \int f(x) \dx\) for its formal integral, we use the multivariate chain rule to extend this equation to}
\frac{\partial F(x)}{\partial x} \dx = f(x) \dx = \omega & = \tau_y^* \omega = f(x +_\phi y) \operatorname{d}(x +_\phi y) = \frac{\partial F(x +_\phi y)}{\partial x} \dx.
\end{align*}
It follows that \(F(x +_\phi y)\) and \(F(x)\) differ by a constant.  Checking at \(x = 0\) shows that the constant is \(F(y)\), hence \[F(x +_\phi y) = F(x) + F(y).\]  This chain of steps can be read in reverse: starting with an \(F(x)\) satisfying this last equation, differentiating against \(x\) yields the long equation above, and hence an invariant differential \(\omega = \frac{\partial F(x)}{\partial x} \dx\).
\end{proof}

\begin{lemma}\label{InvDifflsAreDeterminedByConstantTerm}
For \(R\) any ring, restriction to the identity point yields an isomorphism \(\omega_{\G} \cong T_0^* \G\).
\end{lemma}
\begin{proof}
We set out to analyze the space of invariant differentials in terms of a coordinate \(x\).  As above, a differential \(\omega\) admits expression as \(\omega = f(x) \dx\), and the invariance condition above becomes \[f(x) \dx = f(y +_\phi x) \operatorname{d}(y +_\phi x) = f(y +_\phi x) \frac{\partial(y +_\phi x)}{\partial x} \dx.\]  Restricting this equation to the origin by setting \(x = 0\), we produce the condition \[f(0) = f(y) \cdot \left. \frac{\partial(y +_\phi x)}{\partial x} \right|_{x=0}.\]  The partial differential is a multiplicatively invertible power series, and hence we may rewrite \(f(y)\) as \[f(y) = f(0) \cdot \left(\left. \frac{\partial(y +_\phi x)}{\partial x} \right|_{x=0}\right)^{-1}.\]  This shows that the assignment \(f(y) \mapsto f(0)\) is bijective, establishing the desired isomorphism.
\end{proof}

\begin{theorem}\label{RationalFGLsHaveLogarithms}
If \(R\) is a \(\Q\)--algebra, there is a natural logarithm \[\log_{\G}\co \G \to \G_a \otimes T_0 \G.\]
\end{theorem}
\begin{proof}
First, \Cref{InvDifflsAreDeterminedByConstantTerm} shows that a choice of section of the cotangent space at the identity of \(\G\) uniquely specifies an invariant differential.  Then, \Cref{InvDifflsGiveLogsRationally} shows that this uniquely specifies a logarithm function.  This describes an isomorphism \[T_0^* \G \to \CatOf{FormalGroups}(\G, \G_a),\] which we transpose to give the desired isomorphism \[\log_{\G}\co \G \to \G_a \otimes T_0 \G. \qedhere\]
\end{proof}

\begin{example}\label{GmAndItsLogExample}
Consider the formal group law \(x_1(t_1) +_{\G_m} x_2(t_2) = t_1 + t_2 - t_1 t_2\) studied in \Cref{CPinftyKUExample}.  Its associated rational logarithm is computed as \[\log_{\G_m}(t_2) = f(0) \cdot \int \frac{1}{1 - t_2} \operatorname{d\mathit{t}_2} = -f(0) \log(1 - t_2) = -f(0) \log(x_2),\] where ``\(\log(x_2)\)'' refers to Napier's classical natural logarithm of \(x_2\).
\end{example}









\section{Divisors on formal curves}\label{CurveDivisorsSection}

We continue to develop vocabulary and accompanying machinery used to give algebro-geometric reinterpretations of the results in the introduction to this Case Study.  In the previous section we deployed the language of formal schemes to recast \Cref{CPinftyNiceCalculation} in geometric terms, and we now turn towards reencoding \Cref{ThomIsomOverC}.  In \Cref{DefnQCohSheaves} and \Cref{CorrespondenceQCohAndModules} we discussed a general correspondence between \(R\)--modules and quasicoherent sheaves over \(\Spec R\), and the isomorphism of \(1\)--dimensional \(E^* X\)--modules appearing in \Cref{ThomIsomOverC} moves us to study sheaves over \(X_E\) which are \(1\)--dimensional---i.e., \index{line bundle}line bundles.  In fact, for the purposes of \Cref{ThomIsomOverC}, we will find that it suffices to understand the basics of the geometric theory of line bundles \emph{just over formal curves}.  This is our goal in this Lecture, and we leave the applications to algebraic topology aside for later.  For the rest of this section we fix the following three pieces of data: a base formal scheme \(S\), a formal curve \(C\) over \(S\), and a distinguished point \(\zeta\co S \to C\) on \(C\).

To begin, we will be interested in a very particular sort of line bundle over \(C\): for any function \(f\) on \(C\) which is not a zero-divisor, the \index{sheaf!ideal}(ideal) subsheaf \(\sheaf I_f = f \cdot \sheaf O_C\) of functions on \(C\) which are divisible by \(f\) forms a \(1\)--dimensional \(\sheaf O_C\)--submodule of the ring of functions \(\sheaf O_C\) itself---i.e., a line bundle on \(C\).  By interpreting \(\sheaf I_f\) as an ideal sheaf, this gives rise to a second interpretation of this data in terms of a closed subscheme \[\Spec \sheaf O_C / f \subseteq C,\] which we will refer to as the \index{divisor}\textit{divisor} associated to \(\sheaf I_f\).  In general these can be somewhat pathological, so we specialize further to an extremely nice situation:

\begin{definition}[{\cite[Section 5.1]{StricklandFSFG}}]
A formal subscheme \(D\) of a formal scheme \(X = \colim_\alpha X_\alpha\) is said to be \textit{closed}\index{formal scheme!closed subscheme} when it pulls back along any of the defining affine schemes \(X_\alpha \to X\) to give a closed subscheme of \(X_\alpha\).  An \textit{effective Weil divisor} \(D\) on a formal curve \(C\) is a closed subscheme of \(C\) whose structure map \(D \to S\) presents \(D\) as finite and free.  We say that the \textit{rank} of \(D\) is \(n\) when its ring of functions \(\sheaf O_D\) is free of rank \(n\) over \(\sheaf O_S\).
\end{definition}

\begin{lemma}[{\cite[Proposition 5.2]{StricklandFSFG}, see also \cite[Example 2.10]{StricklandFSFG}}]
There is a scheme \(\Div_n^+ C\) of effective Weil divisors of rank \(n\).  It is a formal variety of dimension \(n\).  In fact, a coordinate \(x\) on \(C\) determines an isomorphism \(\Div_n^+ C \cong \A^n\) where a divisor \(D\) is associated to a monic polynomial \(f_D(x)\) with nilpotent lower-order coefficients.
\end{lemma}
\begin{proof}[Proof sketch]
To pin down the functor we wish to analyze, we make the definition \[\Div_n^+(C)(u\co \Spec T \to S) = \left\{\text{\(D\) an effective divisor on \(u^* C\) of rank \(n\)}\right\}.\]  To show that this is a formal variety, we pursue the final claim and select a coordinate \(x\) on \(C\), as well as a point \(D \in \Div_n^+(C)(T)\).  The coordinate presents \(u^* C\) as \[u^* C = \Spf T\llbracket x \rrbracket,\] and the characteristic polynomial \(f_D(x)\) of \(x\) in \(\sheaf O_D\) presents \(D\) as the closed subscheme \[D = \Spf T\llbracket x \rrbracket / (f_D(x))\] for \(f_D(x) = x^n + a_{n-1} x^{n-1} + \cdots + a_0\) monic.  Additionally, for any prime ideal \(\p \subset T\) we can form the field \(T_{\p} / \p\), over which the module \(\sheaf O_D \otimes_T T_{\p} / \p\) must still be of rank \(n\).  It follows that \[f_D(x) \otimes_T T_{\p} / \p \equiv x^n,\] hence that each \(a_j\) lies in the intersection of all prime ideals of \(T\), hence that each \(a_j\) is nilpotent.

In turn, this means that the polynomial \(f_D\) is selected by a map \(\Spec T \to \A^n\).  Conversely, given such a map, we can form the polynomial \(f_D(x)\) and the divisor \(D\).
\end{proof}

\begin{remark}\label{DescriptionOfSqCupMapOnPolynomials}
This Lemma effectively connects several simple dots: especially nice polynomials \(f_D(x) \in \sheaf O_C\), their vanishing loci \(D \subseteq C\), and the ideal sheaves \(\sheaf I_D\) of functions divisible by \(f\)---i.e., functions with a partially prescribed vanishing set.  Basic operations on polynomials affect their vanishing loci in predictable ways, and these operations are also reflected on the level of divisor schemes.  For instance, there is a unioning map
\begin{align*}
\Div_n^+ C \times \Div_m^+ C & \to \Div_{n+m}^+ C, \\
(D_1, D_2) & \mapsto D_1 \sqcup D_2.
\end{align*}
At the level of ideal sheaves, we use their \(1\)--dimensionality to produce the formula \[\sheaf I_{D_1 \sqcup D_2} = \sheaf I_{D_1} \otimes_{\sheaf O_C} \sheaf I_{D_2}.\]  Under a choice of coordinate \(x\), the map at the level of polynomials is given by \[(f_{D_1}, f_{D_2}) \mapsto f_{D_1} \cdot f_{D_2}.\]
\end{remark}

Next, note that there is a canonical isomorphism \(C \xrightarrow{\cong} \Div_1^+ C\).  Iterating the above addition map gives a map \[C^{\times n} \xrightarrow{\sqcup} \Div_n^+ C.\]
\begin{lemma}\label{SymmetricAffinesExist}
The object \(C^{\times n}_{\Sigma_n}\) exists as a formal variety, and it participates in the following triangle:
\begin{center}
\begin{tikzcd}
& C^{\times n} \arrow{ld} \arrow["\sqcup"]{d} \\
C^{\times n}_{\Sigma_n} \arrow[densely dotted, "\cong"]{r} & \Div_n^+ C.
\end{tikzcd}
\end{center}
\end{lemma}
\begin{proof}
The first assertion is a consequence of Newton's theorem on \index{symmetric function!Newton's theorem}symmetric polynomials: the subring of symmetric polynomials in \(R[x_1, \ldots, x_n]\) is itself polynomial on generators \[\sigma_j(x_1, \ldots, x_n) = \sum_{\substack{S \subseteq \{1, \ldots, n\} \\ |S| = j}} x_{S_1} \cdots x_{S_j},\] and hence \[R[\sigma_1, \ldots, \sigma_n] \subseteq R[x_1, \ldots, x_n]\] gives an affine model of the quotient map \(\A^n \to (\A^n)_{\Sigma_n}\).  Picking a coordinate on \(C\) allows us to import this fact into formal geometry to deduce the existence of \(C^{\times n}_{\Sigma_n}\).  The factorization then follows by noting that the iterated \(\sqcup\) map is symmetric.  Finally, \Cref{DescriptionOfSqCupMapOnPolynomials} shows that the horizontal map pulls the coordinate \(a_j\) back to \(\sigma_j\), so the third assertion follows.
\end{proof}

\begin{remark}
The map \(C^{\times n} \to C^{\times n}_{\Sigma_n}\) is an example of a map of schemes which is \index{sheaf!surjection}surjective \emph{as a map of sheaves}.  This is somewhat subtle: for any given test ring \(T\), it is not necessarily the case that \(C^{\times n}(T) \to C^{\times n}_{\Sigma_n}(T)\) is surjective on \(T\)--points.  However, for a fixed point \(f \in C^{\times n}_{\Sigma_n}(T)\), we are guaranteed a flat covering \(T \to \prod_j T_j\) such that there are individual lifts \(\widetilde f_j\) of \(f\) over each \(T_j\).\footnote{This amounts to the claim that not every polynomial can be written as a product of linear factors.  For instance, the divisor on \(C = \Spf \R\ps{x}\) defined by the equation \(x^2 + 1\) splits as \((x + i)(x - i)\) after base-change along the flat cover \(\Spec \C \to \Spec \R\).}
\end{remark}

Now we use the pointing \(\zeta\co S \to C\) to interrelate divisor schemes of varying ranks.  Together with the \(\sqcup\) operation, \(\zeta\) gives a composite
\begin{center}
\begin{tikzcd}
\Div_n^+ C \arrow{r} & C \times \Div_n^+ C \arrow{r} & \Div_1^+ C \times \Div_n^+ C \arrow{r} & \Div_{n+1}^+ C, \\
D \arrow[|->,r] & (\zeta, D) \arrow[|->,r] & (\<\zeta\>, D) \arrow[|->,r] & {\<\zeta\> \sqcup D}.
\end{tikzcd}
\end{center}

\begin{definition}\label{StableDivisorSchemeDefn}
We define the following variants of ``stable divisor schemes'':
\begin{align*}
\Div^+ C & = \coprod_{n \ge 0} \Div_n^+ C, \\
\Div_n C & = \colim \left( \Div_n^+ C \xrightarrow{\<\zeta\> + -} \Div_{n+1}^+ C \xrightarrow{\<\zeta\> + -} \cdots \right), \\
\Div C & = \colim \left( \Div^+ C \xrightarrow{\<\zeta\> + -} \Div^+ C \xrightarrow{\<\zeta\> + -} \cdots \right) \\
& \cong \coprod_{n \in \Z} \Div_n C.
\end{align*}
\end{definition}

\noindent The first of these constructions is very suggestive: it looks like the free commutative monoid formed on a set, and we might hope that the construction in formal schemes enjoys a similar universal property.  In fact, all three constructions have universal properties:

\begin{theorem}[{cf.\ \Cref{FreeFormalGroupOnACurve}}]\label{DivConstructionsAreFree}
The scheme \(\Div^+ C\) models the free formal commutative monoid on the unpointed formal curve \(C\).  The scheme \(\Div C\) models the free formal group on the unpointed formal curve \(C\).  The scheme \(\Div_0 C\) \emph{simultaneously} models the free formal commutative monoid and the free formal group on the \emph{pointed} formal curve \(C\). \qed
\end{theorem}
\noindent We will postpone the proof of this Theorem until later, once we've developed a theory of coalgebraic formal schemes.

\begin{remark}\label{DivHasPushforwards}
Given \(q\co C \to C'\) a map of formal curves over \(S\) and \(D \subseteq C\) a divisor on \(C\), the construction of \(\Div_n^+ C\) as a symmetric space as in \Cref{SymmetricAffinesExist} shows that there is a corresponding \index{divisor!pushforward}\textit{pushforward} divisor \(q_* D\) on \(C'\).  This can be thought of in some other ways---for instance, this map at the level of sheaves~\cite[Ch.\ IV, Exercise 2.6]{Hartshorne} is given by \[\det q_* \sheaf I_D \cong (\det q_* \sheaf O_C) \otimes \sheaf I_{q_* D}.\]  We can also use \Cref{DivConstructionsAreFree} in the stable case: the composite map \[C \xrightarrow{q} C' \cong \Div_1^+ C' \to \Div C'\] targets a formal group scheme, and hence universality induces a map \[q_*\co \Div C \to \Div C'.\]  On the other hand, for a general \(q\) the pullback \(D \times_{C'} C\) of a divisor \(D \subseteq C'\) will not be a divisor on \(C\).  It is possible to impose conditions on \(q\) so that this is so, and in this case \(q\) is called an \index{isogeny}\textit{isogeny}.  We will return to this in \Cref{IsogeniesSection}.
\end{remark}

Our final goal for the section is to broaden this discussion to line bundles on formal curves generally, using this nice case as a model.  The main classical theorem is that line bundles, sometimes referred to as \index{divisor!Cartier}\textit{Cartier divisors}, arise as the group-completion and sheafification of zero-loci of polynomials, referred to (as above) as \textit{Weil divisors}.  In the case of a \emph{formal} curve, sheafification has little effect, and so we seek to exactly connect line bundles on a formal curve with formal differences of Weil divisors.  To begin, we need some vocabulary that connects the general case to the one studied above.

\begin{definition}[{cf.\ \cite[Section 14.2]{Vakil}}]\label{DivisorialDefn}
Suppose that \(\L\) is a line bundle on \(C\) and select a section \(u\) of \(\L\).  There is a largest closed subscheme \(D \subseteq C\) where the condition \(u|_D = 0\) is satisfied.  If \(D\) is a divisor, \(u\) is said to be \index{divisor!divisorial section}\textit{divisorial} and we write \(\div u := D\).
\end{definition}

\begin{lemma}[{cf.\ \cite[Exercise 14.2.E]{Vakil} and \cite[Proposition II.6.13]{Hartshorne}}]
A divisorial section \(u\) of a line bundle \(\L\) induces an isomorphism \(\L \cong \sheaf I_D\). \qed
\end{lemma}

Line bundles which admit divisorial sections are therefore those that arise through our construction above, i.e., those which are controlled by the zero locus of a polynomial.  In keeping with the classical inspiration, we expect generic line bundles to be controlled by the zeroes \emph{and poles} of a rational function, and so we introduce the following class of functions:

\begin{definition}[{\cite[Definition 5.20 and Proposition 5.26]{StricklandFSFG}}]
The ring of meromorphic functions on \(C\), \(\sheaf{M}_C\), is obtained by inverting all coordinates in \(\sheaf{O}_C\).\footnote{In fact, it suffices to invert any single one~\cite[Lemma 5.21]{StricklandFSFG}.}  Additionally, this can be augmented to a scheme \(\operatorname{Mer}(C, \mathbb G_m)\) of meromorphic functions on \(C\) by \[\operatorname{Mer}(C, \mathbb G_m)(u\co \Spec T \to S) := (\sheaf{M}_{u^* C})^\times.\]
\end{definition}

Thinking of a meromorphic function as the formal expansion of a rational function, we are moved to study the monoidality of \Cref{DivisorialDefn}.

\begin{lemma}
If \(u_1\) and \(u_2\) are divisorial sections of \(\sheaf L_1\) and \(\sheaf L_2\) respectively, then \(u_1 \otimes u_2\) is a divisorial section of \(\sheaf L_1 \otimes \sheaf L_2\) and \(\div(u_1 \otimes u_2) = \div u_1 + \div u_2\). \qed
\end{lemma}

\begin{definition}
A \textit{meromorphic divisorial section} of a line bundle \(\sheaf L\) is a decompositon \(\sheaf L \cong \sheaf L_1 \otimes \sheaf L_2^{-1}\) together with an expression of the form \(u_+ / u_-\), where \(u_+\) and \(u_-\) are divisorial sections of \(\sheaf L_1\) and \(\sheaf L_2\) respectively.  We set \(\div(u_+ / u_-) = \div u_+ - \div u_-\).
\end{definition}

In the case of a formal curve, the fundamental theorem is that meromorphic functions, line bundles, and stable Weil divisors all essentially agree.  A particular meromorphic function spans a \(1\)--dimensional \(\sheaf O_C\)--submodule sheaf of \(\sheaf M_C\), and hence it determines a line bundle.  Conversely, a line bundle is determined by local gluing data, which is exactly the data of a meromorphic function.  However, it is clear that there is some overdeterminacy in this presentation: scaling a meromorphic function by a nowhere vanishing entire function will not modify the submodule sheaf.  We now make the observation that the function \(\div\) gives an assignment from meromorphic functions to stable Weil divisors which is \emph{also} insensitive to rescaling by a nowhere vanishing function.  These inputs are arranged in the following theorem:

\begin{theorem}[{\cite[Proposition 5.26]{StricklandFSFG}, \cite[Proposition 33.4]{StricklandFPFP}, cf.\ also \cite[Proposition II.6.11]{Hartshorne}}]
In the case of a formal curve \(C\), there is a short exact sequence of formal groups
\[\pushQED{\qed}
0 \to \InternalHom{FormalSchemes}(C, \mathbb G_m) \to \operatorname{Mer}(C, \mathbb G_m) \to \Div(C) \to 0. \qedhere
\popQED\]
\end{theorem}











\section{Line bundles associated to Thom spectra}\label{ProjectivizationLecture}

In this Lecture, we will exploit all of the algebraic geometry previous set up to deduce a load of topological results.

\begin{definition}\label{DefnThomSheaf}
Let \(E\) be a complex-orientable theory and let \(V \to X\) be a complex vector bundle over a space \(X\).  According to \Cref{ThomIsomOverC}, the cohomology of the Thom spectrum \(E^* T(V)\) form a \(1\)--dimensional \(E^* X\)--module.  Using \Cref{CorrespondenceQCohAndModules}, we construct a line bundle over \(X_E\) \[\ThomSheaf{V} := \widetilde{E^* T(V)},\] called the \index{Thom sheaf}\textit{Thom sheaf} of \(V\).
\end{definition}

\begin{remark}
One of the main utilities of this definition is that it only uses the \emph{property} that \(E\) is complex-orientable, and it begets only the \emph{property} that \(\ThomSheaf{V}\) is a line bundle.
\end{remark}

This construction enjoys many properties already established.
\begin{corollary}\label{PropertiesOfThomSheaves}
A vector bundle \(V\) over \(Y\) and a map \(f \co X \to Y\) induce an isomorphism \[\ThomSheaf{f^* V} \cong (f_E)^* \ThomSheaf{V}.\]  There is also is a canonical isomorphism \[\ThomSheaf{V \oplus W} = \ThomSheaf{V} \otimes \ThomSheaf{W}.\]  Finally, this property can then be used to extend the definition of \(\ThomSheaf{V}\) to virtual bundles: \[\ThomSheaf{V - W} = \ThomSheaf{V} \otimes \ThomSheaf{W}^{-1}.\]
\end{corollary}
\begin{proof}
The first claim is justified by the naturality of \Cref{GeneralThomIsom}, the second is justified by \Cref{ThomSpacesAreMonoidal}, and the last is a direct consequence of the first two.
\end{proof}

We use these properties to work the following Example, which connects Thom sheaves with the major players from \Cref{FormalVarietiesLecture}.

\begin{example}[{\cite[Section 8]{AHSHinfty}}]\label{Pi2AndInvariantDiffls}
Take \(\L\) to be the \index{canonical bundle}canonical line bundle over \(\CP^\infty\).  Using the same mode of argument as in \Cref{RPnThomExample}, the zero-section \[\Susp^\infty \CP^\infty \xrightarrow{\simeq} T_2(\L)\] gives an identification \[E^0 \CP^\infty \supseteq \widetilde E^0 \CP^\infty \xleftarrow{\simeq} \widetilde E^0 T_2(\L)\] of \(\widetilde E^0 T_2(\L)\) with the augmentation ideal in \(E^0 \CP^\infty\).  At the level of Thom sheaves, this gives an isomorphism \[\sheaf I(0) \xleftarrow{\simeq} \ThomSheaf{\L}\] of \(\ThomSheaf{\L}\) with the sheaf of functions vanishing at the origin of \(\CP^\infty_E\).  Pulling \(\L\) back along \[0\co * \to \CP^\infty\] gives a line bundle over the one-point space, which on Thom spectra gives the inclusion \[\Susp^\infty \CP^1 \to \Susp^\infty \CP^\infty.\]  Stringing many results together, we can now calculate:
\begin{align*}
\widetilde{\pi_2 E} & \cong \widetilde{E^0 \CP^1} \tag{\(S^2 \simeq \CP^1\)} \\
& \cong \ThomSheaf{0^* \L} \tag{\Cref{DefnThomSheaf}} \\
& \cong 0^* \ThomSheaf{\L} \tag{\Cref{PropertiesOfThomSheaves}} \\
& \cong 0^* \sheaf I(0) \tag{preceding calculation} \\
& \cong \sheaf I(0) / (\sheaf I(0) \cdot \sheaf I(0)) \tag{definition of \(0^*\) from \Cref{PushAndPullForQCohOnAffines}} \\
& \cong T^*_0 \CP^\infty_E \tag{\Cref{DefnOfCoTangentSpaces}} \\
& \cong \omega_{\CP^\infty_E}, \tag{proof of \Cref{InvDifflsAreDeterminedByConstantTerm}}
\end{align*}
where \(\omega_{\CP^\infty_E}\) denotes the sheaf of \index{Kahler differentials@K\"ahler differentials!invariant}invariant differentials on \(\CP^\infty_E\).  Consequently, if \(k \cdot \eps\) is the trivial bundle of dimension \(k\) over a point, then \[\widetilde{\pi_{2k} E} \cong \ThomSheaf{k \cdot \eps} \cong \ThomSheaf{k \cdot 0^* \L} \cong \ThomSheaf{0^* \L}^{\otimes k} \cong \omega_{\CP^\infty_E}^{\otimes k}.\]  Finally, given an \(E\)--algebra \(f \co E \to F\) (e.g., \(F = E^{X_+}\)), then we have \[\widetilde{\pi_{2k} F} \cong f_E^* \omega_{\CP^\infty_E}^{\otimes k}.\]
\end{example}

Outside of this Example, it is difficult to find line bundles \(\ThomSheaf{V}\) which we can analyze so directly.  In order to get a handle on on \(\ThomSheaf{V}\) in general, we now seek to strengthen this bond between line bundles and vector bundles by finding inside of algebraic topology the alternative presentations of line bundles given in \Cref{CurveDivisorsSection}.  In particular, we would like a topological construction on vector bundles which produces divisors---i.e., finite schemes over \(X_E\).  This has the scent of a certain familiar topological construction called projectivization, and we now work to justify the relationship.

\begin{definition}
Let \(V\) be a complex vector bundle of rank \(n\) over a base \(X\).  Define \(\P(V)\), the \index{projectivization}\textit{projectivization of \(V\)}, to be the \(\CP^{n-1}\)--bundle over \(X\) whose fiber over \(x \in X\) is the space of complex lines in the original fiber \(V|_x\).
\end{definition}

\begin{theorem}\label{CohomologyOfProjectivization}
Take \(E\) to be \emph{complex-oriented}.  The \(E\)--cohomology of \(\P(V)\) is given by the formula \[E^* \P(V) \cong \left. E^*(X) \llbracket t \rrbracket \middle/ c(V) \right.\] for a certain monic polynomial \[c(V) = t^n - c_1(V) t^{n-1} + c_2(V) t^{n-2} - \cdots + (-1)^n c_n(V).\]
\end{theorem}
\begin{proof}
We fit all of the fibrations we have into a single diagram:
\begin{center}
\begin{tikzcd}
& \C^\times \arrow[equal]{dd} \arrow{rd} \\
\C^n \arrow{dd} & & \C^n \setminus \{0\} \arrow[crossing over]{ll} \arrow{r} \arrow{dd} & \CP^{n-1} \arrow{r} \arrow{dd} & \CP^\infty \arrow[equal]{dd} \\
& \C^\times \arrow{rd} \\
V \arrow{d} & & V \setminus \zeta \arrow{ll} \arrow{r} \arrow{d} & \P(V) \arrow{r} \arrow{d}{\pi} & \CP^\infty \arrow{d} \\
X \arrow[equal]{rr} \arrow[bend left, densely dotted]{u}{\zeta} & & X \arrow[equal]{r} & X \arrow{r} & *.
\end{tikzcd}
\end{center}
We read this diagram as follows: on the far left, there's the vector bundle we began with, as well as its zero-section \(\zeta\).  Deleting the zero-section gives the second bundle, a \(\C^n \setminus \{0\}\)--bundle over \(X\).  Its quotient by the scaling \(\C^\times\)--action gives the third bundle, a \(\CP^{n-1}\)--bundle over \(X\).  Additionally, the quotient map \(\C^n \setminus \{0\} \to \CP^{n-1}\) is itself a \(\C^\times\)--bundle, and this induces the structure of a \(\C^\times\)--bundle on the quotient map \(V \setminus \zeta \to \P(V)\).  Thinking of these as complex line bundles, they are classified by a map to \(\CP^\infty\), which can itself be thought of as the last vertical fibration, fibering over a point.

Note that the map on \(E\)--cohomology between these two last fibers is surjective.  It follows that the Serre spectral sequence for the third vertical fibration is degenerate, since all the classes in the fiber must survive.\footnote{This is called the Leray--Hirsch theorem.}  We thus conclude that \(E^* \P(V)\) is a free \(E^*(X)\)--module on the classes \(\{1, t, t^2, \ldots, t^{n-1}\}\) spanning \(E^* \CP^{n-1}\), where \(t\) encodes the chosen complex-orientation of \(E\).  To understand the ring structure, we need only compute \(t^{n-1} \cdot t\), which must be able to be written in terms of the classes which are lower in \(t\)--degree: \[t^n = c_1(V) t^{n-1} - c_2(V) t^{n-2} + \cdots + (-1)^{n-1} c_n(V)\] for some classes \(c_j(V) \in E^* X\).  The main claim follows.
\end{proof}

In coordinate-free language, we have the following Corollary:
\begin{corollary}[{\Cref{CohomologyOfProjectivization} redux}]
Take \(E\) to be \emph{complex-orientable}.  The map \[\P(V)_E \to X_E \times \CP^\infty_E\] is a closed inclusion of \(X_E\)--schemes, and the structure map \(\P(V)_E \to X_E\) is free and finite of rank \(n\).\footnote{That this map is relative to $X_E$ is why we were free to consider only curves in the previous Lecture.}  It follows that \(\P(V)_E\) is a \index{divisor}divisor on \(\CP^\infty_E\) considered over \(X_E\), i.e.,
\[\pushQED{\qed}
\P(V)_E \in \left(\Div_n^+(\CP^\infty_E)\right)(X_E). \qedhere
\popQED\]
\end{corollary}

\begin{definition}
The classes \(c_j(V)\) of \Cref{CohomologyOfProjectivization} are called the \index{Chern class}\textit{Chern classes} of \(V\) (with respect to the complex-orientation \(t\) of \(E\)), and the polynomial \(c(V) = \sum_{j=0}^n (-1)^{n-j} t^j c_{n-j}(V)\) is called the \index{Chern class!Chern polynomial}\textit{Chern polynomial}.
\end{definition}

The next major theorems concerning projectivization are the following:

\begin{corollary}
The sub-bundle of \(\pi^*(V)\) consisting of vectors \((v, (\ell, x))\) such that \(v\) lies along the line \(\ell\) splits off a canonical line bundle. \qed
\end{corollary}

\begin{corollary}[``Splitting principle'' / ``Complex-oriented descent'']\label{OriginalSplittingPrinciple}\index{splitting principle}
Associated to any \(n\)--dimensional complex vector bundle \(V\) over a base \(X\), there is a canonical map \(f_V\co Y_V \to X\) such that \((f_V)_E\co (Y_V)_E \to X_E\) is finite and faithfully flat, and there is a canonical splitting into complex line bundles:
\[\pushQED{\qed}
f_V^*(V) \cong \bigoplus_{i=1}^n \L_i. \qedhere
\popQED\]
\end{corollary}

This last Corollary is extremely important.  Its essential algebraic content is to say that any question about characteristic classes can be checked for sums of line bundles.  Specifically, because of the injectivity of \(f_V^*\), any relationship among the characteristic classes deduced in \(E^* Y_V\) must already be true in the ring \(E^* X\).  The following theorem is a consequence of this principle:

\begin{theorem}[{\cite[Theorem 16.2 and 16.10]{Switzer}}]\label{ChernClassesAreSymmInChernRoots}
Again take \(E\) to be complex-oriented.  The coset fibration \[U(n-1) \to U(n) \to S^{2n-1}\] deloops to a spherical fibration \[S^{2n-1} \to BU(n-1) \to BU(n).\]  The associated Serre spectral sequence \[E_2^{*, *} = H^*(BU(n); E^* S^{2n-1}) \Rightarrow E^* BU(n-1)\] degenerates at \(E_{2n}\) and induces an isomorphism \[E^* BU(n) \cong E^* \llbracket \sigma_1, \ldots, \sigma_n\rrbracket.\]  Now, let \(V\co X \to BU(n)\) classify a vector bundle \(V\).  Then the coefficient \(c_j\) in the polynomial \(c(V)\) is selected by \(\sigma_j\): \[c_j(V) = V^*(\sigma_j).\]
\end{theorem}
\begin{proof}[Proof sketch]
The first part is a standard calculation.  To prove the relation between the Chern classes and the \(\sigma_j\), the splitting principle states that we can complete the map \(V\co X \to BU(n)\) to a square
\begin{center}
\begin{tikzcd}
Y_V \arrow{d}{f_V} \arrow[densely dotted]{r}{\bigoplus_{i=1}^n \L_i} & BU(1)^{\times n} \arrow{d}{\oplus} \\
X \arrow{r}{V} & BU(n).
\end{tikzcd}
\end{center}
The equation \(c_j(f_V^* V) = V^*(\sigma_j)\) can be checked in \(E^* Y_V\).
\end{proof}

We now see that not only does \(\P(V)_E\) produce a point of \(\Div_n^+(\CP^\infty_E)\), but actually the scheme \(\Div_n^+(\CP^\infty_E)\) itself appears internally to topology:

\begin{corollary}\label{IdentificationOfBUnWithDivn}
For a complex orientable cohomology theory \(E\), there is an isomorphism \[BU(n)_E \cong \Div_n^+ \CP^\infty_E,\] so that maps \(V\co X \to BU(n)\) are transported to divisors \(\P(V)_E \subseteq \CP^\infty_E \times X_E\).  Selecting a particular complex orientation of \(E\) begets two isomorphisms
\begin{align*}
BU(n)_E & \cong \A^n, &
\Div_n^+ \CP^\infty_E & \cong \A^n,
\end{align*}
and these are compatible with the centered isomorphism above.\footnote{See \cite[Proposition 8.31]{StricklandFSFG} for a proof that recasts \Cref{ChernClassesAreSymmInChernRoots} itself in coordinate-free terms.} \qed
\end{corollary}

This description has two remarkable features.  One is its ``faithfulness'': this isomorphism of formal schemes means that the entire theory of characteristic classes is captured by the behavior of the divisor scheme.  The other aspect is its coherence with topological operations we find on \(BU(n)\).  For instance, the Whitney sum map translates as follows:

\begin{lemma}\label{WhitneySumOfDivisors}
There is a commuting square
\begin{center}
\begin{tikzcd}
BU(n)_E \times BU(m)_E \arrow{r}{\oplus} \arrow[equal]{d} & BU(n+m) \arrow[equal]{d} \\
\Div_n^+ \CP^\infty_E \times \Div_m^+ \CP^\infty_E \arrow{r}{\sqcup} & \Div_{n+m}^+ \CP^\infty_E.
\end{tikzcd}
\end{center}
\end{lemma}
\begin{proof}
The sum map \[BU(n) \times BU(m) \xrightarrow\oplus BU(n+m)\] induces on Chern polynomials the identity~\cite[Theorem 16.2.d]{Switzer} \[c(V_1 \oplus V_2) = c(V_1) \cdot c(V_2).\]  In terms of divisors, this means \[\P(V_1 \oplus V_2)_E = \P(V_1)_E \sqcup \P(V_2)_E. \qedhere\]
\end{proof}

The following is a consequence of combining this Lemma with the splitting principle:

\begin{corollary}
The map \(Y_E \xrightarrow{f_V} X_E\) of \Cref{OriginalSplittingPrinciple} pulls \(\P(V)_E\) back to give \[Y_E \times_{X_E} \P(V)_E \cong \bigsqcup_{i=1}^n \P(\L_i)_E.\]
\end{corollary}
\begin{proof}[Interpretation]
This says that the splitting principle is a topological enhancement of the claim that a divisor on a curve can be base-changed along a finite flat map where it splits as a sum of points.
\end{proof}

The other constructions from \Cref{CurveDivisorsSection} are also easily matched up with topological counterparts:

\begin{corollary}\label{ECohomBUIsFree}
There are natural isomorphisms \(BU_E \cong \Div_0 \CP^\infty_E\) and \((BU \times \Z)_E \cong \Div \CP^\infty_E\). Additionally, \((BU \times \Z)_E\) is the free formal group on the curve \(\CP^\infty_E\).\footnote{Similar methods to those of this section also show that the map \(\CP^{n-1}_E \to \Loops SU(n)_E\) presents \(\Loops SU(n)_E\) as the free formal group on \(\CP^{n-1}_E\).} \qed
\end{corollary}

\begin{figure}
\begin{center}
\begin{tabular}{@{}cccc@{}} \toprule
\multicolumn{2}{c}{Spaces} &
\multicolumn{2}{c}{Schemes}
\\
\cmidrule(r){1-2}
\cmidrule(r){3-4}
Object & Classifies & Object & Classifies \\ \midrule
\(BU(n)\) & rank \(n\) vector bundles & \(\Div_n^+ \CP^\infty_E\) & rank \(n\) effective divisors \\
\(\coprod_n BU(n)\) & all vector bundles & \(\Div^+ \CP^\infty_E\) & all effective divisors \\
\(BU \times \Z\) & stable virtual bundles & \(\Div \CP^\infty_E\) & stable Weil divisors \\
\(BU \times \{0\}\) & rank \(0\) stable virtual bundles & \(\Div_0 \CP^\infty_E\) & rank \(0\) stable divisors \\ \bottomrule
\end{tabular}
\end{center}
\caption{Different notions of vector bundles and their associated divisors}
\end{figure}

\begin{corollary}\label{ProductMapOfDivisorSchemes}
There is a commutative diagram
\begin{center}
\begin{tikzcd}
BU(n)_E \times BU(m)_E \arrow{r}{\otimes} \arrow[equal]{d} & BU(nm)_E \arrow[equal]{d} \\
\Div_n^+ \CP^\infty_E \times \Div_m^+ \CP^\infty_E \arrow{r}{\cdot} & \Div_{nm}^+ \CP^\infty_E,
\end{tikzcd}
\end{center}
where for a test algebra \(f \co \Spec T \to \Spec E_0\) the bottom map acts by \[(D_1, D_2 \subseteq f^* \CP^\infty_E) \mapsto (D_1 \times D_2 \subseteq f^* \CP^\infty_E \times f^* \CP^\infty_E \xrightarrow{\mu} f^* \CP^\infty_E),\] and \(\mu\) is the map induced by the \index{line bundle!tensor product}tensor product map \(\CP^\infty \times \CP^\infty \to \CP^\infty\).
\end{corollary}
\begin{proof}
By the splitting principle, it is enough to check this on sums of line bundles.  A sum of line bundles corresponds to a totally decomposed divisor, so we consider the case of a pair of such divisors \(\bigsqcup_{i=1}^n \{a_i\}\) and \(\bigsqcup_{j=1}^m \{b_j\}\).  Referring to \Cref{DefnFormalGps}, the map acts by \index{divisor!convolution}convolution of divisors: \[\left(\bigsqcup_{i=1}^n \{a_i\} \right) \left( \bigsqcup_{j=1}^m \{b_j\} \right) = \bigsqcup_{i, j} \{a_i +_{\CP^\infty_E} b_j\}. \qedhere\]
\end{proof}

\begin{corollary}\label{BdetCorollary}
The determinant map \[BU(n)_E \xrightarrow{B\det_E} BU(1)_E\] models the summation map \[\Div^+_n \CP^\infty_E \xrightarrow{\sigma} \CP^\infty_E.\]
\end{corollary}
\begin{proof}
This is a direct consequence of the splitting principle, the factorization
\begin{center}
\begin{tikzcd}
BU(n) \arrow["B\det"]{r} & BU(1) \\
BU(1)^{\times n} \arrow["\oplus"]{u} \arrow["\otimes"]{ru},
\end{tikzcd}
\end{center}
and \Cref{ProductMapOfDivisorSchemes}.
\end{proof}

Finally, we can connect our analysis of the divisors coming from topological vector bundles with the line bundles studied at the start of the section.
\begin{lemma}[{\cite[Section 19]{StricklandFPFP}, \cite[Definition 8.33]{StricklandFSFG}}]
Let \(\sheaf I(\P(V)_E)\) denote the \index{sheaf!ideal}ideal sheaf on \(X_E \times \CP^\infty_E\) associated to the divisor subscheme \(\P(V)_E\), and let \(\zeta: X_E \to X_E \times \CP^\infty_E\) denote the pointing of the formal curve \(\CP^\infty_E\).  There is a natural isomorphism of sheaves over \(X_E\): \[\zeta^* \sheaf I(\P(V)_E) \cong \ThomSheaf{V}.\]
\end{lemma}
\begin{proof}[Proof sketch]
As employed in \Cref{Pi2AndInvariantDiffls}, a model for $\zeta^* \sheaf I(D)$ is \[\zeta^* \sheaf I(D) = \sheaf I(D) / \sheaf I(D \sqcup 0).\]  The Thom space also has such a model: there is a cofiber sequence \[\P(V) \to \P(V \oplus \mathrm{triv}) \to T(V).\]  Applying $E$--cohomology yields the result.
\end{proof}

\begin{theorem}[{cf.\ \Cref{BUZTriumvirate}}]\label{ComplexOrientationsInTermsOfTrivs}
A trivialization \(t\co \ThomSheaf{\L} \cong \sheaf O_{\CP^\infty_E}\) of the Thom sheaf associated to the canonical bundle induces an \index{orientation}orientation \(\MUP \to E\).
\end{theorem}
\begin{proof}[Proof sketch]
Suppose that \(V\) is a rank \(n\) vector bundle over \(X\), and let \(f\co Y \to X\) be the space guaranteed by the splitting principle to provide an isomorphism \(f^* V \cong \bigoplus_{j=1}^n \L_j\).  The chosen trivialization \(t\) then pulls back to give a trivialization of \(\sheaf I(\P(f^* V)_E)\), and by finite flatness this descends to also give a trivialization of \(\sheaf I(\P(V)_E)\).  Pulling back along the zero section gives a trivialization of \(\ThomSheaf{V}\).  Then note that the system of trivializations produced this way is multiplicative, as a consequence of \(\P(V_1 \oplus V_2)_E \cong \P(V_1)_E \sqcup \P(V_2)_E\).  In the universal examples, this gives a sequence of compatible maps \(MU(n) \to E\) which assemble on the colimit \(n \to \infty\) to give the desired map of ring spectra.
\end{proof}










\section{Power operations for complex bordism}\label{QuillenPowerOpnsSection}

Our eventual goal, like in \Cref{UnorientedBordismChapter}, is to give an algebro-geometric description of \(MU_*(*)\) and of the cooperations \(MU_* MU\).  It is possible to approach this the same way as in \Cref{PiStarMOSection}, using the Adams spectral sequence~(\cite[Theorem 2]{QuillenAdamsSS}, \cite[Lecture 9]{LurieChromaticCourseNotes}).  However, \(MU_*(*)\) is an integral algebra and so we cannot make do with working out the mod--\(2\) Adams spectral sequence alone---we would at least have to work out the mod--\(p\) Adams spectral sequence for every \(p\).  At odd primes \(p\), there is the following unfortunate theorem:
\begin{theorem}
There is an isomorphism
\[\HFp_* \HFp \cong \F_p[\xi_1, \xi_2, \ldots] \otimes \Lambda[\tau_0, \tau_1, \tau_2, \ldots]\]
with \(|\xi_j| = 2p^j-2\) and \(|\tau_j| = 2p^j - 1\). \qed
\end{theorem}
\noindent There are odd--dimensional classes in this algebra, and the \emph{graded-commutativity} of the dual mod--\(p\)\index{Steenrod algebra!odd primes}\index{Bockstein} Steenrod algebra means that these classes anti-commute.  This prohibits us both from even writing ``\(\Spec(\HFp_* \HFp)\)'', and this is the first time we have encountered Hindrance~\#\ref{SkewCommutativeDeficiency} from \Cref{TheSteenrodAlgebraSection} in the wild.

Because of this, we will not feel any guilt for taking a completely alternative approach to this calculation.  This other method, due to Quillen, has as its keystone a completely different kind of cohomology operation called a \index{power operation}\textit{power operation}.  These operations are quite technical to describe, but at their core is taking the \(n\){\th} power of a cohomology class---and hence they have a frustrating lack of properties, including failures to be additive and to be stable.  Our goal in this Lecture is just to define these cohomology operations, specialized to the particular setting we will need for Quillen's proof.  Some remarks are in order:
\begin{itemize}
\item We will accomplish this as early as \Cref{PowerOpnsForSubgroupsToo}, and though we proceed from there to sketch how Quillen's methods belong to a general framework, the reader should not hesitate to continue on to \Cref{StabilizingTheMUSteenrodOps} when they grow bored, assured that the discussion of ``Tate power operations'' will not resurface.
\item Additionally, unlike with Steenrod squares, the algebro-geometric interpretation of these operations will not be immediately accessible to us, and indeed their eventual reinterpretation in these terms is one of the more hard-won pursuits in this area of mathematics---the curious reader will have to wait until \Cref{PowerOpnsSection}.
\item As before, because we are ultimately aiming to make a computation, we will phrase our discussion in terms of graded objects.
\end{itemize}

Power operations arise not just from taking the \(n\){\th} power of a cohomology class but from also remarking on the natural symmetry of that operation.  We record this symmetry using the following technical apparatus:
\begin{lemma}[{\cite[Theorem 1.6]{EKMM}, \cite[Corollary 2.2.4]{HSS}, \cite[Examples 6.1.4.2 and 6.1.6.2]{LurieHA}}]
Given a spectrum \(E\), its \(n\)--fold smash power forms a \index{equivariant spectrum}\textit{\(\Sigma_n\)--spectrum}, in the sense that there is a natural diagram \(* \mmod \Sigma_n \to \CatOf{Spectra}\) of \(\infty\)--categories which selects \(E^{\sm n}\) on objects.  In the case of $E = \Susp^\infty X$, this diagram agrees with the permutation diagram \(* \mmod \Sigma_n \to X^{\sm n}\) formed in the \(1\)--category \(\CatOf{Spaces}\). \qed
\end{lemma}

\noindent A cohomology class \(f\co \Susp^\infty_+ X \to E\) gives rise to a morphism of \(\Sigma_n\)--spectra \[f^{\sm n}\co (\Susp^\infty_+ X)^{\sm n} \to E^{\sm n}.\]  The homotopy colimit of such a diagram is called the \textit{homotopy orbits}\index{equivariant spectrum!homotopy orbits} of the spectrum, and this gives a natural diagram
\begin{center}
\begin{tikzcd}
(\Susp^\infty_+ X)^{\sm n} \arrow{d} \arrow["f^{\sm n}"]{r} & E^{\sm n} \arrow["\mu"]{r} \arrow{d} & E \\
(\Susp^\infty_+ X)^{\sm n}_{h\Sigma_n} \arrow["f^{\sm n}_{h\Sigma_n}"]{r} & E^{\sm n}_{h\Sigma_n} \arrow[densely dotted]{ur}[description]{\mu_n}.
\end{tikzcd}
\end{center}
A ring spectrum \(E\) equipped with a suite of factorizations \(\mu_n\) satisfying compatibility laws embodying term collection (i.e., \(x^a \cdot x^b = x^{a+b}\)) and iterated exponentiation (i.e., \((x^a)^b = x^{ab}\)) is called an \index{H infty ring spectrum@\(H_\infty\)--ring spectrum}\textit{\(H_\infty\)--ring spectrum}~\cite[Definition I.3.1]{BMMS}.  The composite \[P^{\Sigma_n}_{\mathrm{ext}}f\co (\Susp^\infty_+ X)^{\sm n}_{h\Sigma_n} \xrightarrow{f^{\sm n}_{h\Sigma_n}} E^{\sm n}_{h\Sigma_n} \xrightarrow{\mu_n} E\] is called the \textit{external (total) \(\Sigma_n\)--power operation} applied to \(f\), and the restriction to the diagonal subspace \[P^{\Sigma_n}f \co \Susp^\infty_+ X \sm \Susp^\infty_+ B\Sigma_n \simeq (\Susp^\infty_+ X)_{h\Sigma_n} \xrightarrow{\Delta_{h\Sigma_n}} (\Susp^\infty_+ X)^{\sm n}_{h\Sigma_n} \xrightarrow{P^{\Sigma_n}_{\mathrm{ext}}f} E\] is called the \textit{(internal total) \(\Sigma_n\)--power operation} applied to \(f\).  We can consider it as a cohomology class lying in \[P^{\Sigma_n} f \in E^0(X \times B\Sigma_n).\]

\begin{remark}[{\cite[Definition IV.7.1]{BMMS}}]\label{RestrictingSteenrodOpToBasepoint}
The optional adjective ``total'' is a reference to the following variant of this construction.  By choosing a class in \(E_* B\Sigma_n\), thought of as a functional on \(E^* B\Sigma_n\), the \index{Kronecker pairing}Kronecker pairing of the total power operation with this fixed class gives rise to a truly internal cohomology operation on \(E^* X\) as in the following composite:
\begin{center}
\begin{tikzcd}
\Susp^\infty_+ X \sm \S \arrow["1 \sm \sigma"]{r} & \Susp^\infty_+ X \sm \Susp^\infty_+ B\Sigma_n \sm E \arrow["P^{\Sigma_n} f \sm 1"]{r} & E \sm E \arrow["\mu"]{r} & E.
\end{tikzcd}
\end{center}
\end{remark}

Cohomology classes \(f \in E^{-q}(X)\) lying in degrees \(q \ne 0\) require more care.  Passing to the representative \(f\co \Susp^\infty_+ X \to \S^q \sm E\), the analogous diagram is
\begin{center}
\begin{tikzcd}
(\Susp^\infty_+ X)^{\sm n} \arrow{d} \arrow["f^{\sm n}"]{r} & (\S^q \sm E)^{\sm n} \arrow["\mu"]{r} \arrow{d} & \S^{nq} \sm E \\
(\Susp^\infty_+ X)^{\sm n}_{h\Sigma_n} \arrow["f^{\sm n}_{h\Sigma_n}"]{r} & (\S^q \sm E)^{\sm n}_{h\Sigma_n} \arrow{r} & \S^{q\rho}_{h\Sigma_n} \sm E^{\sm n}_{h\Sigma_n} \arrow[densely dotted]{u}[description]{\mu_{n,q}},
\end{tikzcd}
\end{center}
where \(\S^\rho\) is the \index{equivariant spectrum!representation sphere}representation sphere for the permutation representation of \(\Sigma_n\).  Such a system of factorizations is called an \(H_\infty^1\)--ring structure on \(E\)---but in practice these are very uncommon,\footnote{In fact, they can only appear on \(H\F_2\)--algebras~\cite[Section VII.6.1]{BMMS}.} and instead a subsystem of factorizations for every \(q \equiv 0 \pmod d\) is called an \(H_\infty^d\)--ring structure~\cite[Definition I.4.3]{BMMS}.  In the same way, these give rise to external and internal power operations on cohomology classes of those negative degrees which are divisible by \(d\).  We will mostly concern ourselves with the theory of \(H_\infty\)--ring spectra in this section, but we will eventually work with \(H_\infty^2\)--ring spectra in the intended application, and so we now describe a naturally occuring such structure on \(MU\)\index{H infty ring spectrum@\(H_\infty\)--ring spectrum!H infinity 2 on MU@\(H_\infty^2\)--structure on \(MU\)}.

\begin{definition}[{\cite[Definition VII.7.4]{Rudyak}}]\label{DefnAssociatedBundle}
Suppose that \(\xi\co X \to BU(k)\) presents a complex vector bundle of rank \(k\) on \(X\).  The \(n\)--fold product of this bundle gives a new bundle \[X^{\times n} \xrightarrow{\xi^{\times n}} BU(k)^{\times n} \xrightarrow{\oplus} BU(n \cdot k)\] of rank \(nk\) on which the symmetric group \(\Sigma_n\) acts.  By taking the (homotopy) \(\Sigma_n\)--quotient of the fiber, total, and base spaces, we produce a vector bundle \(\xi(\Sigma_n)\) on \(X^{\times n}_{h\Sigma_n}\) participating in the diagram
\begin{center}
\begin{tikzcd}
X^{\times n} \arrow["\xi^{\times n}"]{r} \arrow{d} & BU(k)^{\times n} \arrow["\oplus"]{r} & BU(nk) \arrow[leftarrow, "\xi(\Sigma_n)"' near end]{lld} \\
X^{\times n}_{h\Sigma_n} \arrow["\xi^{\times n}_{h\Sigma_n}"']{r} & BU(k)^{\times n}_{h\Sigma_n}. \arrow[densely dotted]{ru}[description]{\widetilde \mu_n} \arrow[crossing over, leftarrow]{u}
\end{tikzcd}
\end{center}
The universal case gives the map \(\widetilde \mu_n\).
\end{definition}

\begin{lemma}[{\cite[Equation VII.7.3]{Rudyak}}]\label{ThomSpectrumOfAssocBundle}
There is an isomorphism \[T(\xi(\Sigma_n)) \simeq (T\xi)^{\sm n}_{h\Sigma_n}.\]
\end{lemma}
\begin{proof}
This proof is mostly a matter of having had the idea to write down the Lemma to begin with.  From here, we string basic properties together:
\begin{align*}
T(\xi(\Sigma_n)) & = T(\xi^{\times n}_{h\Sigma_n}) \tag{definition} \\
& = T(\xi^{\times n})_{h\Sigma_n} \tag{colimits commute with colimits} \\
& = T(\xi)^{\sm n}_{h\Sigma_n}. \tag{\(T\) is monoidal: \Cref{ThomSpacesAreMonoidal}}
\end{align*}
\end{proof}

Applying the Lemma to the universal case produces a factorization \[MU(k)^{\sm n} \to MU(k)^{\sm n}_{h\Sigma_n} \to MU(nk)\] of the unstable multiplication map, and hence a stable factorization \[MU^{\sm n} \to MU^{\sm n}_{h\Sigma_n} \xrightarrow{\mu_n} MU.\]  We now enrich this to an \(H_\infty^2\) structure~(\cite[Section 3.2]{JohnsonNoel}, \cite[Corollary VIII.5.3]{BMMS}): by applying \Cref{ThomSpectrumOfAssocBundle} in reverse, the construction is
\begin{align*}
\mu_{n,2q}\co \S^{2qn}_{h\Sigma_n} \sm MU^{\sm n}_{h\Sigma_n} & \xrightarrow{1 \sm \mu_n} \S^{2qn}_{h\Sigma_n} \sm MU \\
& \simeq T(\text{\(\mathrm{triv}^{\oplus q}(\Sigma_n)\) over \(B\Sigma_n\)}) \sm MU \tag{\Cref{ThomSpectrumOfAssocBundle}} \\
& \simeq \Sigma^{2qn}_+ B\Sigma_n \sm MU \tag{Thom isomorphism} \\
& \to \Sigma^{2qn} MU. \tag{project to basepoint}
\end{align*}

\begin{remark}\label{PowerOpnsForSubgroupsToo}
The constructions above also go through with \(\Sigma_n\) replaced by a subgroup.  In our application, we will work with the cyclic subgroup \(C_n \le \Sigma_n\).
\end{remark}

We now return to the setting of a general \(H_\infty\)--ring spectrum \(E\).  Some properties of power operations are immediately visible---for instance, they are multiplicative: \[P^{\Sigma_n}(f \cdot g) = P^{\Sigma_n}(f) \cdot P^{\Sigma_n}(g),\] and restriction of \(P^{\Sigma_n}(f)\) to the basepoint in \(\Sigma^\infty_+ B\Sigma_n\) yields the cup power class \(f^n\).  In order to state any further properties, we will need to make some extraneous observations.  First, note that any map of groups \(\phi\co G \to \Sigma_n\) gives a variation on this construction by restriction of diagrams
\begin{center}
\begin{tikzcd}
* \mmod G \arrow["\phi"]{r} & * \mmod \Sigma_n \arrow[bend left, "(\Susp^\infty_+ X)^{\sm n}"]{r}[name=X, below]{} \arrow[bend right, "\hspace{-1.2em}E^{\sm n}"']{r}[name=E]{} & \CatOf{Spectra}.
\arrow[Rightarrow,to path=(X) -- (E), "f"]{}
\end{tikzcd}
\end{center}
This construction is useful when studying composites of power operations: the group \(\Sigma_k \wr \Sigma_n \subseteq \Sigma_{nk}\) acts naturally on \((E^{\sm k})^{\sm n} \simeq E^{\sm nk}\), and indeed there is an equivalence \[P^{\Sigma_n} \circ P^{\Sigma_k} = P^{\Sigma_k \wr \Sigma_n}.\]  In order to understand these modified power operations more generally, we are motivated to study such maps \(\phi\) more seriously.  Some basic constructions are summarized in the following definition:
\begin{definition}[{\cite[Sections XI.3 and XXV.3]{MayAlaskaNotes}, cf.\ also \cite[Section 6.1.6]{LurieHA}}]\label{EquivariantDefns}
Let \(\phi\co G \to G'\) be an inclusion of finite groups and let \(F\) be an \(G'\)--spectrum.  There is a natural map of homotopy colimits \(\phi_*\co F_{hG} \to F_{hG'}\) which induces a \index{restriction}\textit{restriction map} on cohomology: \[\Res_G^{G'}\co E^0 F_{hG'} \to E^0 F_{hG}.\]  The spectrum \(\bigvee_{G'/G} F\) considered as a \(G'\)--spectrum with the diagonal \(G'\)--action has the property \((\bigvee_{G'/G} F)_{hG'} = F_{hG}\), and the \(G'\)--equivariant averaging map \[F \xrightarrow{\bigvee_{G'/G} \operatorname{id}} \bigvee_{G'/G} F\] passes on homotopy orbits to the \index{norm!equivariant}\textit{additive norm map} \(N_G^{G'}\co F_{hG'} \to F_{hG}\), which again induces a map on cohomology classes \[\Tr_G^{G'}\co E^0 F_{hG} \to E^0 F_{hG'}\] called the \textit{transfer map}.
\end{definition}

The restriction and transfer maps appear prominently in the following formula, which measures the failure of the power operation construction to be additive:
\begin{lemma}[{\cite[Corollary II.1.6]{BMMS},~\cite[Proposition A.5, Equation 3.6]{AHSHinfty}}]
For cohomology classes \(f, g \in E^0 X\), there is a formula\footnote{This should be compared with the classical \index{norm!binomial formula}binomial formula \(\frac{1}{n!} (x + y)^n = \sum_{i+j = n} \frac{1}{i!j!} x^i y^j\).}
\[\pushQED{\qed}
P^{\Sigma_n}(f + g) = \sum_{i + j = n} \Tr_{\Sigma_i \times \Sigma_j}^{\Sigma_n}\left(P^{\Sigma_i}(f) \cdot P^{\Sigma_j}(g)\right). \qedhere
\popQED\]
\end{lemma}

\noindent To produce binomial formulas for the modified power operations, we use the following Lemma:

\begin{lemma}[{\cite[p.\ 109-110]{AdamsInfiniteLoopSpaces}, \cite[Section 6.5]{HKR}}]
Let \(G_1, G_2\) be subgroups of \(G'\), and consider the homotopy pullback diagram
\begin{center}
\begin{tikzcd}
P \arrow{r} \arrow{d} \arrow[dr, phantom, "\lrcorner", very near start] & * \mmod G_1 \arrow{d} \\
* \mmod G_2 \arrow{r} & * \mmod G'.
\end{tikzcd}
\end{center}
For any identification \(P \simeq \coprod_K (* \mmod K)\),\footnote{This decomposition into subgroups \(K\) is \emph{not} canonical.} there is a \index{push-pull}push-pull interchange formula
\[\pushQED{\qed}
\Res_{G_1}^{G'} \Tr_{G_2}^{G'} = \sum_K \Tr^{G_1}_K \Res_K^{G_2}. \qedhere
\popQED\]
\end{lemma}

\begin{corollary}
For a transitive subgroup \(G \le \Sigma_n\), there is a congruence \[P^G(f + g) \equiv P^G(f) + P^G(g) \pmod{\text{transfers from proper subgroups of \(G\)}}.\]
\end{corollary}
\begin{proof}[Proof sketch]
Note that \(P^G\) can be defined by means of restriction, as in \(P^G = \Res^{\Sigma_n}_G P^{\Sigma_n}\).  We can hence reuse the previous binomial formula: \[P^G(f + g) = \Res^{\Sigma_n}_G P^{\Sigma_n}(f + g) = \sum_{i+j = n} \Res^{\Sigma_n}_G \Tr_{\Sigma_i \times \Sigma_j}^{\Sigma_n}\left(P^{\Sigma_i}(f) \cdot P^{\Sigma_j}(g)\right).\]  In the cases \(i = 0\) or \(j = 0\), the transfer map is the identity operation, and we recover \(P^G(g)\) and \(P^G(f)\) respectively.  In all the other terms, the interchange lemma lets us pull the transfer to the outside, and transitivity of \(G\) guarantees this new transfer to be proper.
\end{proof}

Since the only operations we understand so far are stable operations, which are in particular additive, we are moved to find a target for the power operation \(P^G\) which kills the ideal generated by the proper transfers (i.e., the intermediate terms in the binomial formula) yet which remains computable.

\begin{definition}[{\cite[Section XXV.6, Section XXI.4]{MayAlaskaNotes}, \cite[Example 6.1.6.22, Definition 6.1.6.24]{LurieHA}}]\label{DefnTateConstruction}
Take \(G = C_p\), let \(F\) be an \(C_p\)--spectrum, and define its \index{equivariant spectrum!homotopy fixed points}\textit{homotopy fixed points} \(F^{hC_p}\) to be the homotopy limit spectrum.  It is possible to factor the norm map \(F_{hC_p} \to F\) of \Cref{EquivariantDefns} through the homotopy fixed points, as in \[F_{hC_p} \to F^{hC_p} \to F.\]  The cofiber of this first map is denoted \(F^{tC_p}\) and is called the \index{Tate construction}\textit{Tate spectrum}.  In turn, this gives rise to a notion of a Tate \(C_p\)--power operation via
\begin{center}
\begin{tikzcd}[row sep=0.2em]
\pi_0 E^{\Susp^\infty_+ X} \arrow["P^{C_p}"]{r} & \pi_0 E^{(\Susp^\infty_+ X)_{hC_p}} \arrow[equal]{r} & \pi_0 (E^{\Susp^\infty_+ X})^{hC_p} \arrow{r} & \pi_0 (E^{\Susp^\infty_+ X})^{tC_p}, \\
f \arrow[|->]{r} & P^{C_p} f \arrow[|->]{rr} & & P^{C_p}_{\Tate} f.
\end{tikzcd}
\end{center}
\end{definition}

\begin{example}\label{TateDestruction}
The formation of the Tate spectrum is quite destructive.\footnote{The construction in \Cref{DefnTateConstruction} goes through for any group \(G\), not just \(C_p\), but the destructiveness in this example does not hold for other groups---a sign that this is the ``wrong'' generalization.  For a general group \(G\), the definition of ``Tate construction'' can be further modified to kill all intermediate transfers, not just those which transfer up from the trivial subgroup---but this requires understanding \textit{families} of subgroups~\cite[Section XXI.4]{MayAlaskaNotes} and the corresponding generalized norm map~\cite[Section XXV.6]{MayAlaskaNotes}, which is more equivariant homotopy theory than we otherwise need.}  For example, take \(X\) to be a \emph{finite} trivial \(C_p\)--spectrum, and let \(\bigvee_{C_p} X\) be the \(C_p\)--indexed wedge with the free action.  The norm, transfer, and restriction maps belong to the following diagram:
\begin{center}
\begin{tikzcd}
(X^{\vee C_p})_{hC_p} \arrow{r} & (X^{\vee C_p})^{hC_p} \arrow{r} & X^{\vee C_p} \arrow{r} & (X^{\vee C_p})_{hC_p} \\
X \arrow[equal]{u} \arrow{r} & X \arrow[equal]{u} \arrow["\Delta"]{r} & X^{\vee C_p} \arrow[equal]{u} \arrow["\mu"]{r} & X \arrow[equal]{u},
\end{tikzcd}
\end{center}
where on the bottom row we have identified the homotopy orbits and fixed points of the wedge.  We claim that the norm map is an equivalence.  This diagram is natural in \(X\) and each node preserves cofiber sequences, so for finite \(X\) it suffices to check the claim on the spheres \(\S^n\).  Because we are norming up from the trivial group, the long horizontal composite is multiplication by \(|C_p| = p\).  We have thus factored \(p \in \pi_0 \End \S^n = \Z\) as some map followed by the diagonal and fold maps, which themselves compose to \(p\).  This forces the norm map itself to be an equivalence, and hence the cofiber sequence \[(X^{\vee C_p})_{hC_p} \xrightarrow{\simeq} (X^{\vee C_p})^{hC_p} \to (X^{\vee C_p})^{tC_p}\] causes \((X^{\vee C_p})^{tC_p}\) to be contractible.
\end{example}

As another example of the destructiveness of the Tate operation, we have the following:

\begin{lemma}
In the case \(G = C_p\), the Tate power operation is additive.
\end{lemma}
\begin{proof}
The image of the map \(\pi_0 (E^{\Susp^\infty_+ X})_{hC_p} \to \pi_0 (E^{\Susp^\infty_+ X})^{hC_p}\) is the kernel of the projection to the Tate object.  Since the only proper subgroup of \(C_p\) is the trivial subgroup, this image contains all transfers.
\end{proof}

The real miracle is that these cyclic Tate power operations are not only additive, but they are even \emph{completely} computable.

\begin{lemma}[{cf. \cite[Chapter IX]{LurieSAG}}]\label{TateConstructionIsACohomThy}
The assignment \(X \mapsto \pi_0 (E^{(\Susp^\infty_+ X)^{\sm p}})^{tC_p}\) is a cohomology theory on finite spectra \(X\), represented by \(E^{tC_p}\).
\end{lemma}
\begin{proof}
The Eilenberg--Steenrod axioms are clear except for the cofiber sequence axiom, which we will boil down to checking that the assignment \(X \mapsto (X^{\sm p})^{tC_p}\) preserves cofiber sequences of finite complexes.  A cofiber sequence \[X \to Y \to Y/X\] of pointed spaces is equivalent data to the diagram \[X \to Y,\] which has colimit \(Y\) and which admits a filtration by distance from the initial node with filtration quotients:
\begin{center}
\begin{tikzcd}[row sep=1em]
X \arrow{r} & Y \\
X \arrow[equal]{u} & Y/X \arrow[leftarrow]{u}.
\end{tikzcd}
\end{center}
By taking the \(p\)--fold Cartesian power of the diagram \(X \to Y\), we produce a diagram shaped like an \(p\)--dimensional hypercube with colimit \(Y^{\sm p}\) and which again admits a filtration by distance from the initial node.  The colimits of these partial diagrams give a \(C_p\)--equivariant filtration of \(Y^{\sm p}\) as:
\begin{center}
\begin{tikzcd}[column sep=1em, row sep=1em]
F_0 \arrow{r} & F_1 \arrow{r} & \cdots \arrow{r} & F_{p-1} \arrow{r} & Y^{\sm p} \\
X^{\sm p} \arrow[equal]{u} & \bigvee \left(X^{\sm (p-1)} \sm (Y/X)\right) \arrow[leftarrow]{u} & \cdots & \bigvee \left(X \sm (Y/X)^{\sm (p-1)}\right) \arrow[leftarrow]{u} & (Y/X)^{\sm p} \arrow[leftarrow]{u}.
\end{tikzcd}
\end{center}
We now apply \((-)^{tC_p}\) to this diagram.  The Tate construction carries cofiber sequences of \(C_p\)--spectra to cofiber sequences of spectra, so this is again a filtration diagram.  In the intermediate filtration quotients, the \(C_p\)--action is given by freely permuting wedge factors, from which it follows that the Tate construction vanishes on these nodes.  Hence, the diagram postcomposed with the Tate construction takes the form
\begin{center}
\begin{tikzcd}[row sep=1em]
F_0^{tC_p} \arrow[equiv]{r} & F_1^{tC_p} \arrow[equiv]{r} & \cdots \arrow[equiv]{r} & F_{p-1}^{tC_p} \arrow{r} & (Y^{\sm p})^{tC_p} \\
(X^{\sm p})^{tC_p} \arrow[equal]{u} & * \arrow[leftarrow]{u} & \cdots & * \arrow[leftarrow]{u} & ((Y/X)^{\sm p})^{tC_p}. \arrow[leftarrow]{u}
\end{tikzcd}
\end{center}
Eliminating the intermediate filtration stages with empty filtration quotients, we see that this filtration is equivalent data to a cofiber sequence \[(X^{\sm p})^{tC_p} \to (Y^{\sm p})^{tC_p} \to ((Y/X)^{\sm p})^{tC_p}.\]  Repeating this proof inside of \(E\)--module spectra gives the desired result.
\end{proof}

The effect of this Lemma is twofold.  For one, the Tate operation is not only additive, it is even \emph{stable}.  Secondly, it suffices to understand the behavior of passing to the Tate construction in the case of \(X = *\), i.e., the effect of the map \(E^{hC_p} \to E^{tC_p}\).  Since we are intending to make a particular computation, it will at this point be convenient to return to our case of interest, \(E = MU\).

\begin{theorem}\label{TateConstructionOnMU}
There is an isomorphism \[\pi_* MU^{tC_p} = MU^* BC_p[x^{-1}],\] where \(x\) is the restriction to \(MU^2 BC_p\) of the canonical class \(x \in MU^2(\CP^\infty)\).\footnote{Everything we say here will actually be valid for any number \(p\), not just a prime, as well as any ring spectrum \(E\) under \(MU\).}
\end{theorem}
\begin{proof}
Consider the \(C_p\)--equivariant cofiber sequence \[S(\C^m)_+ \to S^0 \to S^{\C^m},\] where \(S(\C^m)\) is the unit sphere inside of \(\C^m\) and \(S^{\C^m}\) is the one-point compactification of the \(C_p\)--representation \(\C^m\).  A key fact is that \(S(\C^m)\) admits a \(C_p\)--equivariant cell decomposition by free cells, natural with respect to the inclusions as \(m\) increases.  This buys us several consequences:
\begin{enumerate}
    \item The following Tate objects vanish: \((MU \sm S(\C^m)_+)^{tC_p} \simeq *\).  As in \Cref{TateDestruction}, this is because the Tate construction vanishes on free \(C_p\)--cells.
    \item We can use \(\colim_{m \to \infty} S(\C^m)_+\) as a model for \(EC_p\), so that \[MU_{hC_p} = \left( \colim_{m \to \infty} MU \sm S(\C^m)_+ \right)_{hC_p}.\]
    \item Coupling these two facts together, we get \[MU_{hC_p} = \left( \colim_{m \to \infty} MU \sm S(\C^m)_+ \right)_{hC_p} = \left( \colim_{m \to \infty} MU \sm S(\C^m)_+ \right)^{hC_p}.\]
    \item Pulling the fixed points functor out, this gives
    \begin{align*}
    MU^{tC_p} & = \cofib(MU_{hC_p} \to MU^{hC_p}) \\
    & = \cofib\left(\left( \colim_{m \to \infty} MU \sm S(\C^m)_+ \right)^{hC_p} \to MU^{hC_p}\right) \\
    & = \left(\cofib\left( \colim_{m \to \infty} MU \sm S(\C^m)_+ \right) \to MU \sm S^0\right)^{hC_p} \\
    & = \left( \colim_{m \to \infty} MU \sm S^{\C^m} \right)^{hC_p}.
    \end{align*}
\end{enumerate}
This last formula puts us in a position to calculate.  The Thom isomorphism for \(MU\) gives an identification \(MU \sm S^{\C} \simeq \Susp^2 MU\) as \(C_p\)--spectra, and the map \[(MU \sm S^0)^{hC_p} \to (MU \sm S^{\C})^{hC_p}\] can be identified with multiplication by the Thom class: \[MU^{\Susp^\infty BC_p} \xrightarrow{t \cdot -} (\Susp^2 MU)^{\Susp^\infty BC_p}.\]  In all, this gives
\begin{align*}
MU^{tC_p} & \simeq \colim_{m \to \infty} \left((MU \sm S^{\C^m})^{hC_p}\right) \\
& \simeq \colim_{m \to \infty} \left((\Susp^{2m} MU)^{BC_p}\right) \simeq MU^{BC_p}[t^{-1}]. \qedhere
\end{align*}
\end{proof}

\begin{corollary}
The non-additivity of the \(C_p\)--power operations associated to \(MU\) is governed by \(t\)--torsion classes. \qed
\end{corollary}

\begin{remark}[{\cite[Equations 3.10-11]{Quillen}}]
The picture Quillen paints of all this is considerably different from ours.  He begins by giving a different presentation of the complex cobordism groups of a manifold \(M\): a complex orientation of a smooth map \(Z \to M\) is a factorization \[Z \xrightarrow{i} E \xrightarrow{\pi} M\] through a complex vector bundle \(\pi\co E \to M\) by an embedding \(i\), as well as a complex structure on the normal bundle \(\nu_i\).  Up to suitable notions of stability (in the dimension of \(E\)) and homotopy equivalence (involving, in particular, isotopies of different embeddings \(i\)), these quotient to give cobordism classes of maps complex-oriented maps \(Z \to M\).  The collection of cobordism classes over \(M\) of codimension \(q\) is isomorphic to \(MU^q(M)\)~\cite[Proposition 1.2]{Quillen}.  Quillen's definition of the power operations is then given in terms of this geometric model: a representative \(f\co Z \to M\) of a cobordism class gives rise to another complex-oriented map \(f^{\times n}\co Z^{\times n} \to M^{\times n}\), and he defines \(P^{C_n}_{\mathrm{ext}}(f)\) to be the postcomposition with \(M^{\times n} \to M^{\times n}_{hC_n}\) (after taking care that the target isn't typically a manifold).  All the properties of his construction must therefore be explored through the lens of groups acting on manifolds.
\end{remark}

\begin{example}[{\cite{SteenrodCyclic,SteenrodSymmetric}}]
The chain model for \(\HFtwo\)--homology is actually also rigid enough to define power operations, and somewhat curiously these operations automatically turn out to be additive, without passing to the Tate construction.  This means that they are recognizable in terms of classical Steenrod operations: the \(C_2\)--construction \[\{\Susp^n \Susp^\infty_+ X \xrightarrow{f} \HFtwo\} \xrightarrow{P^{C_2}} \{\Susp^{2n} \Susp^\infty_+ X \sm \Susp^\infty_+ BC_2 \xrightarrow{P^{C_2}(f)} \HFtwo\}\] gives a class in \(\HFtwo^{2n-*}(X) \otimes \HFtwo^*(\RP^\infty)\), which decomposes as \[P^{C_2}(f) = \sum_{j=0}^{2n} \Sq^{2n-j}(f) \otimes x^j.\]  The Adem relations~\cite{Adem} can be extracted by studying the wreath product \(\Sigma_2 \wr \Sigma_2\) and the compositional identity for power operations.
\end{example}

\begin{remark}[{\cite[Theorems III.4.1-3, Remark III.4.4]{BMMS}}]\label{HinftyRingsModp}
Since the failure of the power operations to be additive was a consequence of the binomial formula, it is somewhat intuitive that modulo \(2\), where \((x + y)^2 \equiv x^2 + y^2\), that this operation becomes stable.  In fact, more than this is true: for instance, if an \(H_\infty\)--ring spectrum \(E\) has \(\pi_0 E = \F_p\), it must be the case that \(E\) receives a ring spectrum map from \(\HFp\).  This fact also gives an inexplicit means to recover \Cref{MOSplitsIntoHF2s}, after noting that the same methods used to endow \(MU\) with an \(H_\infty\)--ring spectrum structure do the same for \(MO\).
\end{remark}















\section{Explicitly stabilizing cyclic \texorpdfstring{\(MU\)}{MU}--power operations}\label{StabilizingTheMUSteenrodOps}

Having thus demonstrated that the Tate variant of the cyclic power operation decomposes as a sum of stable operations, we are motivated to understand the available such stable operations in complex bordism.  This follows quickly from our discussions in the previous few Lectures.  We learned in \Cref{IdentificationOfBUnWithDivn} that for any complex-oriented cohomology theory \(E\) we have the calculation \[E^* BU \cong E^*\llbracket \sigma_1, \sigma_2, \ldots, \sigma_j, \ldots\rrbracket,\] and we gave a rich interpretation of this in terms of divisor schemes: \[BU_E \cong \Div_0 \CP^\infty_E.\]

We would like to leverage the Thom isomorphism to gain a description of \(E^* MU\) generally and \(MU^* MU\) specifically.  However, the former is \emph{not} a ring, and although the latter is a ring its multiplication is exceedingly complicated\footnote{For a space \(X\), \(E^* X\) has a ring structure because \(X\) has a diagonal, and \(MU\) does not have a diagonal.  In the special case of \(E = MU\), there is a ring product coming from endomorphism composition.}, which means that our extremely compact algebraic description of \(E^* BU\) in \Cref{IdentificationOfBUnWithDivn} will be of limited use.  Instead, we will have to content ourselves with an \(E_*\)--module basis of \(E^* MU\).
\begin{definition}
Take \(\phi\co MU \to E\) to be a complex-oriented ring spectrum, which presents \(E^* BU\) as the subalgebra of symmetric functions inside of an infinite--dimensional polynomial algebra: \[E^* BU \subseteq E^* BU(1)^{\times \infty} \cong E^*\ps{x_1, x_2, \ldots}.\]  For any nonnegative multi-index \(\alpha = (\alpha_1, \alpha_2, \ldots)\) with finitely many entries nonzero, there is an associated \textit{symmetric monomial}\index{symmetric function!monomial} \(c_\alpha\), which is the sum of those monomials whose exponent lists contain exactly \(\alpha_j\) many instances of \(j\).\footnote{For example, \(\alpha = (1, 0, 2, 0, \ldots)\) corresponds to the sum \[b_\alpha = \sum_i \sum_{j \ne i} \sum_{\substack{k \ne i \\ k > j}} x_i x_j^3 x_k^3 = x_1 x_2^3 x_3^3 + x_1^3 x_2 x_3^3 + x_1^3 x_2^3 x_3 + \cdots.\]}  We then set \(s_\alpha \in E^* MU\) to be the image of \(c_\alpha\) under the Thom isomorphism of \(E_*\)--modules \[E^* MU \cong E^* BU.\]  It is called the \index{operations!Landweber Novikov@Landweber--Novikov}\textit{\(\alpha\){\th} Landweber--Novikov operation} with respect to \(\phi\).
\end{definition}

\begin{definition}
In the case of the identity orientation \(MU \xrightarrow{1} MU\), the resulting classes are called the \index{Chern class!Conner Floyd@Conner--Floyd}\textit{Conner--Floyd--Chern classes} and the associated cohomology operations are called the \textit{Landweber--Novikov operations} (without further qualification).
\end{definition}

\begin{remark}
For a vector bundle \(V\) and a complex-oriented cohomology theory \(E\), we define the \index{Chern class!total symmetric}\textit{total symmetric Chern class} of \(V\) by the sum \[c_{\t}(V) = \sum_\alpha c_\alpha(V) \t^{\alpha}.\]  In the case of a line bundle \(\L\) with first Chern class \(c_1(\L) = x\), this degenerates to the sum \[c_{\t}(\L) = \sum_{j=0}^\infty x^j t_j.\]  For a direct sum, \(c_{\t}(V \oplus W)\) satisfies a Cartan formula: \[c_{\t}(V \oplus W) = c_{\t}(V) \cdot c_{\t}(W).\]  Again specializing to line bundles \(\L\) and \(\H\) with first Chern classes \(c_1(\L) = x\) and \(c_1(\H) = y\), this gives
\begin{align*}
c_{\t}(\L \oplus \H) & = \left( \sum_{j=0}^\infty x^j t_j \right) \cdot \left( \sum_{k=0}^\infty y^k t_k \right) = \sum_{j = 0}^{\infty} \sum_{k=0}^\infty x^j y^k t_j t_k \\
& = 1 + (x + y)t_1 + (xy)t_1^2 + (x^2 + y^2)t_2 + (xy^2 + x^2y)t_1 t_2 + \cdots \\
& = 1 + c_1(U)t_1 + c_2(U)t_1^2 + \\
& \quad \quad + (c_1^2(U) - 2c_2(U))t_2 + (c_1(U) c_2(U))t_1t_2 + \cdots,
\end{align*}
where we have expanded out some of the pieces of the total symmetric Chern class in polynomials in the usual Chern classes.
\end{remark}

\begin{definition}[{\cite[Theorem I.5.1]{AdamsBlueBook}}]
Take \(MU \xrightarrow{1} MU\) as the orientation, so that we are considering \(MU^* MU\) and the Landweber--Novikov operations arising from the Conner--Floyd--Chern classes.  These account for the \emph{stable} operations in \(MU\)--cohomology, analogous to the Steenrod operations for \(\HFtwo\).  They satisfy the following properties:
\begin{itemize}
\item \(s_0\) is the identity.
\item \(s_\alpha\) is natural: \(s_\alpha(f^* x) = f^*(s_\alpha x)\).
\item \(s_\alpha\) is stable: \(s_\alpha(\sigma x) = \sigma(s_\alpha x)\).
\item \(s_\alpha\) is additive: \(s_\alpha(x + y) = s_\alpha(x) + s_\alpha(y)\).
\item \(s_\alpha\) satisfies a \index{Cartan formula}Cartan formula.  Define \[s_{\t}(x) := \sum_{\alpha} s_\alpha(x) \t^\alpha := \sum_\alpha s_\alpha(x) \cdot t_1^{\alpha_1} t_2^{\alpha_2} \cdots t_n^{\alpha_n} \cdots \in MU^*(X)\ps{t_1, t_2, \ldots}\] for an infinite sequence of indeterminates \(t_1\), \(t_2\), \ldots.  Then \[s_{\t}(x y) = s_{\t}(x) \cdot s_{\t}(y).\]
\item Let \(\xi\co X \to BU(n)\) classify a vector bundle and let \(\phi\) denote the Thom isomorphism \[\phi\co MU^* X \to MU^* T(\xi).\]  Then the Chern classes of \(\xi\) are related to the Landweber--Novikov operations on the Thom spectrum by the formula \[\phi(c_\alpha(\xi)) = s_\alpha(\phi(1)).\]
\end{itemize}
\end{definition}

Having set up an encompassing theory of stable operations, we now seek to give a formula for the cyclic Tate power operation in this framework.  In order to approach this, we initially set our sights on the too-lofty goal of computing the total power operation \(P^{C_n}(f)\) for \(f \in MU^{2q}(X)\) an \(MU\)--cohomology class on a finite complex \(X\), without necessarily passing to the Tate construction.  Because of the definition \(MU = \colim_k MU(k)\) and because \(P^{C_n}\) is natural under pullback, it will suffice for us to study the effect of \(P^{C_n}\) on the universal classes \[u_m\co MU(m) \to MU,\] after using the canonical \(MU\)--Thom isomorphism to reinterpret them as classes on a suspension spectrum.  We begin with the canonical orientation itself: \[u_1 = x \in h\CatOf{Spectra}(MU(1), MU) \cong MU^2 \CP^\infty.\]  In order to understand the effect \(P^{C_n}(x)\) of the power operation on \(x\), recall that \(x\) is also the \(1\)\textsuperscript{st} Conner--Floyd--Chern class of the \index{canonical bundle}\index{tautological bundle|see {canonical bundle}}tautological bundle \(\L\) on \(\CP^\infty\), i.e., \[x\co MU(1) \to MU\] is both the Thomification of the block inclusion \[\L\co BU(1) \to BU\] as well as its first Chern class in the canonical orientation for \(MU\)--theory.  The \(H_\infty^2\)--ring spectrum construction defining \(P^{C_n}_{\mathrm{ext}}(x)\) thus fits into the following diagram:
\begin{center}
\begin{tikzcd}
\Susp^\infty_+ BU(1) \arrow["\Delta"]{d} \arrow{r} & \Susp^\infty_+ BU(1) \sm \Susp^\infty_+ BC_n \arrow["\Delta_{hC_n}"]{d} \\
(\Susp^\infty_+ BU(1))^{\sm n} \arrow{r} \arrow[bend right=15, "\L^{\times n}"' near start]{rr} \arrow["c_1^{\sm n}"]{d} & \Susp^\infty_+ BU(1)^{\sm n}_{hC_n} \arrow["\L(C_n)"]{r} \arrow[crossing over]{d} \arrow["P^{C_n}_{\mathrm{ext}}" near end, crossing over]{rd} & \Susp^\infty_+ BU(n) \arrow["c_n"]{d} \\
(\Susp^2 MU)^{\sm n} \arrow{r} & (\Susp^2 MU)^{\sm n}_{hC_n} \arrow["\mu_{n,2}"]{r} & \Susp^{2n} MU.
\end{tikzcd}
\end{center}
The commutativity of the widest rectangle (i.e., the justification for the name ``\(c_n\)'' on the right-most vertical arrow) comes from the Cartan formula for Chern classes: because \(\L^{\oplus n}\) splits as the sum of \(n\) line bundles, \(c_n(\L^{\oplus n})\) is computed as the product of the \(1\)\textsuperscript{st} Chern classes of those line bundles.  Second, the commutativity of the right-most square is not trivial: it is a specific consequence of how the multiplicative structure on \(MU\) arises from the direct sum of vector bundles.\footnote{In general, any notion of first Chern class \(\Susp^\infty_+ BU(1) \to \Susp^2 E\) gives rise to a \emph{noncommuting} diagram of this same shape.  The two composites \(\Susp^\infty_+ BU(1)^{\sm n}_{hC_n} \to \Susp^{2n} E\) need not agree, since \(\L(C_n)\) has no \textit{a priori} reason to be compatible with the factorization appearing in the \(H_\infty^2\)--structure.  They turn out to be loosely related nonetheless, and their exact relation (as well as a procedure for making them agree in certain cases) is the subject of \Cref{PowerOpnsSection}.}  The commutativities of the other two squares comes from the natural transformation from a \(C_n\)--space to its homotopy orbit space.

Hence, the internal cyclic power operation \(P^{C_n}(x)\) applied to the canonical coordinate \(x\) is defined by the composite \[\Susp^\infty_+ BU(1)_{hC_n} \xrightarrow{\Delta_{hC_n}} (\Susp^\infty_+ BU(1))^{\sm n}_{hC_n} \to \Susp^{2n} MU^{\sm n}_{hC_n} \to \Susp^{2n} MU,\] which is to say \[P^{C_n}(x) = c_n(\Delta_{hC_n}^* \L(C_n)).\]  We have thus reduced to computing a particular Conner--Floyd--Chern class of a particular bundle.  Our next move is to transport more information from the \(C_n\)--equivariant bundle \(\Delta^* \L^{\times n}\) to the bundle \(\Delta_{hC_n}^* \L(C_n)\).

\begin{lemma}[{cf.\ \Cref{DefnAssociatedBundle}}]\label{GEquivBundlesVsBundlesOverBG}
Given a \(G\)--equivariant bundle \(V\) over a base \(X\) on which \(G\) acts trivially, the homotopy quotient bundle \(V_{hG}\) determines a vector bundle over \(X_{hG} \simeq X \times BG\).  This construction preserves direct sums, and it preserves tensor products when one of the factors has the trivial action. \pushQED\qed\qedhere\popQED
\end{lemma}

We thus proceed to analyze \(\Delta^*_{hC_n} \L(C_n)\) by first decomposing the \(C_n\)--equivariant bundle \(\Delta^* \L^{\times n}\).  The \(C_n\)--action is given by permutation of the factors, and hence we have an identification \[\Delta^* \L^{\times n} \cong \L \otimes \pi^* \rho,\] where \(\rho\) is the permutation representation of \(C_n\) (considered as a vector bundle over a point) and \(\pi\co BU(1) \to *\) is the constant map.  The permutation representation for the abelian group \(C_n\), also known as its regular representation, is accessible by character theory.  The generating character \(\chi\co U(1)[n] \to U(1)\) gives a decomposition \[\rho \cong \bigoplus_{j=0}^{n-1} \chi^{\otimes j}.\]  Applying this to our situation, we get a sequence of isomorphisms of \(C_n\)--equivariant vector bundles \[\Delta^* \L^{\times n} \cong \L \otimes \pi^* \rho \cong \L \otimes \bigoplus_{j=0}^{n-1} \pi^* \chi^{\otimes j} \cong \bigoplus_{j=0}^{n-1} \L \otimes \pi^* \chi^{\otimes j}.\]  Applying \Cref{GEquivBundlesVsBundlesOverBG}, we recast this as a calculation of the bundle \(\Delta_{hC_n}^* \L(C_n)\): \[\Delta_{hC_n}^* \L(C_n) = \bigoplus_{j=0}^{n-1} \pi_1^* \L \otimes \pi_2^* \eta^{\otimes j},\] where \(\eta\) is the bundle classified by \(\eta\co BU(1)[n] \to BU(1)\) and \(\pi_1\), \(\pi_2\) are the two projections off of \(BU(1) \times BC_n\).

We can use this to access \(c_n(\Delta_{hC_n}^* \L(C_n))\).  As the top Chern class of this \(n\)--dimensional vector bundle, we think of this as a calculation of its Euler class, which lets us lean on multiplicativity:
\begin{align*}
P^{C_n}(x) = c_n(\Delta_{hC_n}^* \L(C_n)) & = e\left( \bigoplus_{j=0}^{n-1} \pi_1^* \L \otimes \pi_2^* \eta^{\otimes j} \right)
= \prod_{j=0}^{n-1} e\left( \pi_1^* \L \otimes \pi_2^* \eta^{\otimes j} \right) \\
& = \prod_{j=0}^{n-1} c_1\left( \pi_1^* \L \otimes \pi_2^* \eta^{\otimes j} \right)
= \prod_{j=0}^{n-1} (x +_{MU} [j]_{MU}(t)).
\end{align*}
Here \(x\) is still the \(1\)\textsuperscript{st} Conner--Floyd--Chern class of \(\L\) and \(t\) is the Euler class of \(\eta\).  We now try to make sense of this product expression for \(c_n(\Delta_{hC_n}^* \L(C_n))\) by expanding it in powers of \(x\) and identifying its component pieces.
\begin{lemma}\label{AjAndBjAreInTheFGLSubring}
There is a series expansion in \(MU^* BU(1)_{hC_n}\): \[P^{C_n}(x) = \prod_{j=0}^{n-1} (x +_{MU} [j]_{MU}(t)) = w + \sum_{j=1}^\infty a_j(t) x^j,\] where \(+_{MU}\) is the natural formal group law on \(MU_*\), \([j]_{MU}(t)\) is the \(j\)--fold sum of \(t\) with itself using \(+_{MU}\), and \(a_j(t)\) is a series with coefficients in the subring \(C \subseteq MU_*\) spanned by the coefficients of \(+_{MU}\).  The leading term \[w = e(\overline \rho) = \prod_{j=0}^{n-1} e(\eta^{\otimes j}) = \prod_{j=0}^{n-1} [j]_{MU} (e(\eta)) = (n-1)! t^{n-1} + \sum_{j \ge n} b_j t^j\] is the Euler class of the reduced permutation representation, and, again, the elements \(b_j\) lie in the subring \(C\). \qed
\end{lemma}

We now turn from the \(1\)\textsuperscript{st} universal case to our original goal: understanding the action of \(P^{C_n}\) on each of the canonical classes \[u_m\co MU(m) \to MU.\]  We approach this via the splitting principle, first recasting the main formula in \Cref{AjAndBjAreInTheFGLSubring} so that it carries a Cartan formula:
\begin{align*}
P^{C_n}(c_1(\L)) & = \sum_{|\alpha| \le 1} w^{1 - |\alpha|} a_\alpha(t) c_\alpha(\L), &
a_\alpha(t) & = \prod_{j=0}^\infty a_j(t)^{\alpha_j}.
\end{align*}

\begin{corollary}
There is the universal formula \[P^{C_n}(u_m) = \sum_{|\alpha| \le m} w^{m - |\alpha|} a_\alpha(t) s_\alpha(u_m).\]
\end{corollary}
\begin{proof}
Writing \(\phi\) for the canonical Thom isomorphism associated to \(MU\)--cohomology, the main new observation is 
\begin{align*}
P^{C_n}(u_m) & = \phi(P^{C_n}(c_m(\xi_m))). \\
\intertext{From here, we apply the splitting principle, multiplicativity of power operations, and properties of Landweber--Novikov operations:}
& = \phi(P^{C_n}(c_1(\L_1) \cdots c_1(\L_m))) \\
& = \phi(P^{C_n}(c_1(\L_1)) \cdots P^{C_n}(c_1(\L_m))) \\
& = \phi\left(\left(\sum_{|\alpha| \le 1} w^{1 - |\alpha|} a_\alpha(t) c_\alpha(\L_1)\right) \cdots \left( \sum_{|\alpha| \le 1} w^{1 - |\alpha|} a_\alpha(t) c_\alpha(\L_m) \right)\right) \\
& = \phi\left(\sum_{|\alpha| \le m} w^{m - |\alpha|} a_\alpha(t) c_\alpha(\L_1 \oplus \cdots \oplus \L_m)\right) \\
& = \phi\left(\sum_{|\alpha| \le m} w^{m - |\alpha|} a_\alpha(t) c_\alpha(\xi_m) \right) = \sum_{|\alpha| \le m} w^{m - |\alpha|} a_\alpha(t) s_\alpha(u_m). \qedhere
\end{align*}
\end{proof}

We will use this to power the following conclusion about cohomology classes in general, starting with an observation about the fundamental class of a sphere:

\begin{lemma}[{cf.\ \cite[Corollary VII.7.14]{Rudyak}}]\label{PowerOpnsOnSuspensions}
For \(f \in MU^{2q}(X)\) a cohomology class in a finite complex \(X\), there is the \index{homology suspension}suspension relation \[P^{C_n}(\sigma^{2m} f) = w^m \sigma^{2m} P^{C_n}(f).\]
\end{lemma}
\begin{proof}
We calculate \(P^{C_n}\) applied to the fundamental class \[S^{2m} \xrightarrow{\iota_{2m}} T_m BU(m) \simeq \Susp^{2m} MU(m) \xrightarrow{\Susp^{2m} u_m} \Susp^{2m} MU\] by restricting the universal formula:
\begin{align*}
P^{C_n}(\iota_{2m}^* u_m) = \iota_{2m}^* P^{C_n}(u_m) & = \iota_{2m}^* \left( \sum_{|\alpha| \le m} w^{m - |\alpha|} a_\alpha(t) s_\alpha(u_m) \right) = w^m \iota_{2m},
\end{align*}
since \(s_\alpha(\iota_{2m}) = 0\) for any nonzero \(\alpha\), as the cohomology of \(S^{2m}\) is too sparse.  Because \(\sigma^{2m} f = \iota_{2m} \sm f\), we conclude the proof by multiplicativity of \(P^{C_n}\).
\end{proof}

\begin{theorem}[{cf.\ \cite[Proposition 3.17]{Quillen}, \cite[Corollary VII.7.14]{Rudyak}}]\label{QuillensKeyRelation}
Let \(X\) be a finite pointed space and let \(f\) be a cohomology class \[f \in \widetilde{MU}^{2q}(X).\]  For \(m \gg 0\), there is a formula \[w^m P^{C_n}(f) = \sum_{|\alpha| \le m+q} w^{q+m - |\alpha|} a_\alpha(t) s_\alpha(f),\] with \(t\), \(w\), and \(a_\alpha(t)\) as defined above.\footnote{The reader comparing with Quillen's paper will notice various apparent discrepancies between the statements of our Theorem and of his.  These are notational: he grades his cohomology functor homologically, which occasionally causes our \(q\) to match his \(-q\), so that his \(n\) is comparable to our \(m-q\).}
\end{theorem}
\begin{proof}
We take \(m\) large enough so that \(f\) is represented by an unstable map \[g\co \Susp^{2m} X \to T_{m+q} BU(m+q),\] in the sense that \(g\) intertwines \(f\) with the universal class \(u_{m+q}\) by the formula \[g^* u_{m+q} = \sigma^{2m} f.\]  We use \Cref{PowerOpnsOnSuspensions} and naturality to conclude
\begin{align*}
w^m \sigma^{2m} P^{C_n}(f) & = P^{C_n}(\sigma^{2m} f) = P^{C_n}(g^* u_{m+q}) = g^* P^{C_n}(u_{m+q}) \\
& = g^* \left(\sum_{|\alpha| \le m+q} w^{m+q - |\alpha|} a_\alpha(t) s_\alpha(u_{m+q}) \right) \\
& = \sum_{|\alpha| \le m+q} w^{m+q - |\alpha|} a_\alpha(t) \sigma^{2m} s_\alpha(f). \qedhere
\end{align*}
\end{proof}

Our conclusion, then, is that \(P^{C_n}\) is \emph{almost} naturally expressible in terms of the Landweber--Novikov operations, where the ``\emph{almost}'' is controlled by some \(w\)--torsion.  Our discussion of the Tate construction in the previous Lecture shows that this is, in some sense, a generic phenomenon, and indeed the above Theorem can be divided by \(w^m\) to recover a statement about the cyclic Tate power operation.  However, we have learned the additional information that the various factors in this statement---including \(w\) itself---are controlled by the formal group law ``\(+_{MU}\)'' associated to the tautological complex orientation of \(MU\) and the subring \(C\).  This is \emph{not} generic behavior.  In the next Lecture, we will discover the surprising fact that we only need to multiply by a \emph{single} \(w\)---also highly non-generic---and the equally surprising consequences this entails for \(MU_*\) itself.











\section{The complex bordism ring}\label{CalculationOfMUStarSection}

With \Cref{QuillensKeyRelation} in hand, we will now deduce Quillen's major structural theorem about \(MU_*\).  We will preserve the notation used in \Cref{AjAndBjAreInTheFGLSubring} and \Cref{QuillensKeyRelation}:
\begin{itemize}
\item \(\overline \rho\) is the reduced regular representation of \(C_n\), which coincides with its reduced permutation representation, and \(w = e(\overline \rho)\) is its Euler class.
\item \(\eta\co BC_n \to BU(1)\) is the line bundle associated to a choice of generating character \(C_n \to U(1)\), and \(t = e(\eta)\) is its Euler class.
\item \(C\) is the subring of \(MU_*\) generated by the coefficients of the formal group law associated to the identity complex-orientation.
\end{itemize}
In the course of working out the main Theorem, we will want to make use of the following property of the class \(t\).

\begin{definition}
For \(x\) a coordinate on a formal group, the \textit{divided \(n\)--series}\index{formal group!l series@\(\ell\)--series!divided} is determined by the formula \[x \cdot \<n\>(x) = [n](x).\]
\end{definition}

\begin{lemma}[{\cite[Proposition 4.4]{Quillen}}]\label{QuillensGysinFact}
Let \(X\) be a finite complex.  If a class \(\omega \in MU^*(X \times BC_n)\) satisfies \(t \cdot \omega = 0\), then there exists a class \(y \in MU^*(X)\) with \(\omega = y \cdot \<n\>_{MU}(t)\).
\end{lemma}
\begin{proof}[Proof sketch]
Consider the following piece of the Gysin sequence for the bundle \(\eta\) as viewed in the cohomology theory \(MU^*(X \times -)\): \[\widetilde{MU}^{*+1}(X \times S^1_{hC_n}) \xrightarrow{\partial} MU^*(X \times BC_n) \xrightarrow{t} MU^{*+2}(X \times BC_n).\]  Because the action of \(C_n\) on the total space \(S^1\) is free, we have \(S^1_{hC_n} \simeq S^1\).  The suspension isomorphism then rewrites the leftmost term as \[\widetilde{MU}^*(X) \xrightarrow{\<n\>_{MU}(t)} MU^*(X \times BC_n) \xrightarrow{t} MU^{*+2}(X \times BC_n).\]  In order to justify the image of \(\partial\), one needs to unwrap the definition of the Gysin sequence~\cite[Equation 4.7, Proposition 4.2]{Quillen}, which we elect not to do.  Given this, however, the Lemma statement follows from exactness.
\end{proof}

We now turn to the main Theorem.

\begin{theorem}[{\cite[Theorem 5.1]{Quillen}}]\label{CGenerationForFiniteCplx}
If \(X\) has the homotopy type of a finite complex, then
\begin{align*}
MU^*(X) & = C \cdot \sum_{q \ge 0} MU^q(X), &
\widetilde{MU}^*(X) & = C \cdot \sum_{q > 0} MU^q(X).
\end{align*}
\end{theorem}
\begin{remark}
In what follows, the reader should carefully remember the degree conventions stemming from the formula \[MU^* X = \pi_{-*} F(\Susp^\infty_+ X, MU).\]  The homotopy ring \(MU_*\) appears in the \emph{negative} degrees of \(MU^*(*)\), but the fundamental class of \(S^m\) appears in the \emph{positive} degree \(MU^m(S^m)\).
\end{remark}
\begin{proof}[{Proof of \Cref{CGenerationForFiniteCplx}}]
We can immediately reduce the claim in two ways.  First, it is true if and only if it is also true for reduced cohomology.  Second, we are free to restrict attention just to \(MU^{2*}(X)\), since we can handle the odd-degree parts of \(MU^*(X)\) by suspending \(X\) once.  Defining \[R^{2*} := C \cdot \sum_{q > 0} MU^{2q}(X),\] we can thus focus on the claim \[\widetilde{MU}^{2*}(X) \stackrel{?}{=} C \cdot \sum_{q > 0} MU^{2q}(X).\]  Noting that the claim is trivially true for all positive values of \(*\), we will show this by working \(p\)--locally and inducting on the value of ``\(-*\)''.

Suppose that \[R^{-2j}_{(p)} = \widetilde{MU}^{-2j}(X)_{(p)}\] for \(j < q\) and consider \(x \in \widetilde{MU}^{-2q}(X)_{(p)}\).  Then, for \(m \gg 0\), we have
\begin{align*}
w^m P^{C_p}(x) & = \sum_{|\alpha| \le m-q} w^{m-q - |\alpha|} a(t)^\alpha s_\alpha x \\
& = w^{m-q} x + \sum_{\substack{|\alpha| \le m-q \\ \alpha \ne 0}} w^{m-q - |\alpha|} a(t)^\alpha s_\alpha x.
\end{align*}
Recall that \(w\) is a power series in \(t\) with coefficients in \(C\) and leading term \((p-1)! \cdot t^{p-1}\), so that \(t^{p-1} = w \cdot \theta(t)\) for some multiplicatively invertible series \(\theta(t)\) with coefficients in \(C\).  Since \(s_\alpha\) raises degree, we have \(s_\alpha x \in R\) by the inductive hypothesis, and we may thus collect all those terms (as well as many factors of \(\theta(t)^{-1}\)) into a series \(\psi_x(t) \in R_{(p)}\llbracket t \rrbracket\) to write \[t^N(w^q P^{C_p}(x) - x) = \psi_x(t),\] where \(N = (m-q)(p-1)\).

Consider the set of possible integers \(N\) for which we can write such an equation with any power series on the right-hand side---we know that \(N = (m-q)(p - 1)\) works, but we now also consider values of \(N\) which are not multiples of \((p - 1)\).  We aim to conclude that \(N = 1\) is the minimum of this set, so suppose that \(N\) is the minimum such value.  Using \Cref{RestrictingSteenrodOpToBasepoint}, we find that restricting this equation along the inclusion \(i\co X \to X \times BU(1)[p]\) sets \(t = 0\) and yields \(\psi_x(0) = 0\).  It follows that \(\psi_x(t) = t \phi_x(t)\) is at least once \(t\)--divisible, and thus \[t (t^{N-1}(w^q P^{C_p}x - x) - \phi_x(t)) = 0.\]  Appealing to \Cref{QuillensGysinFact}, we produce a class \(y \in \widetilde{MU}^{-2q + 2(N-1)}(X)_{(p)}\) with \[t^{N-1}(w^q P^{C_p}(x) - x) = \phi_x(x) + y \<p\>(t).\]  If \(N > 1\), then \(y \in R_{(p)}\) for degree reasons and hence the right-hand side gives a series expansion, in contradiction of our minimality hypothesis.  So, we have \(N = 1\); the class \(y\) lies in the critical degree \(-2q\); and the outer factor of \(t^{N-1}\) is not present in the last expression.\footnote{One can interpret the proof thus far as giving a bound on the amount of \(w\)--torsion needed to get the stability relation described in \Cref{QuillensKeyRelation}.  Our answer is quite surprising: we have found that we need just a single \(w\) (indeed, a single \(t\)), which isn't much stability at all!}  Restricting along \(i\) again to set \(w = t = 0\) and \(P^{C_p}(x) = x^p\), we obtain the equation \[\left. \begin{array}{rr} -x & \text{if \(q > 0\)} \\ x^p - x & \text{if \(q = 0\)} \end{array} \right\} = \phi_x(0) + py.\]

In the first case, where \(q > 0\), it follows that \(MU^{-2q}(X)_{(p)}\) is contained in \(R^{-2q}_{(p)} + pMU^{-2q}(X)_{(p)}\), and since \(MU^{-2q}(X)_{(p)}\) is finitely generated\footnote{This is a consequence of \(X\) having finitely many cells, \(MU\) having finitely many cells in each degree, and each homotopy group of the stable sphere being finitely generated.} it follows that \(MU^{-2q}(X)_{(p)} = R^{-2q}_{(p)}\).  In the other case, \(x\) can be rewritten as a sum of elements in \(R^{0}_{(p)}\), elements in \(p \widetilde{MU}^{0}(X)_{(p)}\), and elements in \((\widetilde{MU}^0(X)_{(p)})^p\).  Since the ideal \(\widetilde{MU}^0(X)_{(p)}\) is nilpotent, it again follows that \(\widetilde{MU}^0(X)_{(p)} = R^0_{(p)}\), concluding the induction.
\end{proof}

\begin{corollary}[{\cite[Corollary 5.2]{Quillen}}]\label{QuillenSurjective}
The coefficients of the formal group law generate \(MU_*\).
\end{corollary}
\begin{proof}
This is the case of \Cref{CGenerationForFiniteCplx} where we set \(X = *\).
\end{proof}

\begin{remark}
This proof actually also goes through for \(MO\) as well.  In that case, it's even easier, since the equation \(2 = 0\) in \(\pi_0 MO\) causes much of the algebra to collapse.  The proof does not extend further to cases like \(M\SO\) or \(M\mathit{Sp}\): these bordism theories do not have associated formal group laws, and so we lose the control we earned in \Cref{StabilizingTheMUSteenrodOps}.
\end{remark}

Take \(\moduli{fgl}\) to be the moduli functor of formal group laws.  Since a formal group law is a power series satisfying some algebraic identities, this moduli object is an affine scheme with coordinate ring \(\sheaf O_{\moduli{fgl}}\).  A rephrasing of \Cref{QuillenSurjective} is that the natural map \[\sheaf O_{\moduli{fgl}} \to MU_*\] is \emph{surjective}.  This is reason enough to start studying \(\moduli{fgl}\) in earnest, which we take up in the next Case Study---but if we anachronistically assume one algebraic fact about \(\sheaf O_{\moduli{fgl}}\), we can prove that the natural map is an \emph{isomorphism}.  We begin with the following topological observation about mixing complex-orientations:

\begin{lemma}[{\cite[Lemma II.6.3 and Corollary II.6.5]{AdamsBlueBook}}]\label{OrientationsOnEAndMU}
Let \(\phi\co MU \to E\) be a complex-oriented ring spectrum and consider the two orientations on \(E \sm MU\) given by
\begin{align*}
\S \sm MU & \xrightarrow{\eta_E \sm 1} E \sm MU, &
MU \sm \S & \xrightarrow{\phi \sm \eta_{MU}} E \sm MU.
\end{align*}
The two induced coordinates \(x^E\) and \(x^{MU}\) on \(\CP^\infty_{E \sm MU}\) are related by the formulas
\begin{align*}
x^{MU} & = \sum_{j=0}^\infty b_j^E (x^E)^{j+1} =: g(x^E), \\
g^{-1}(x^{MU} +_{MU} y^{MU}) & = g^{-1}(x^{MU}) +_E g^{-1}(y^{MU}).
\end{align*}
where \(E_* MU \cong \frac{\Sym_{E_*} E_*\{\beta_0, \beta_1, \beta_2, \ldots\}}{\beta_0 = 1} \cong E_*[b_1, b_2, \ldots]\), as in \Cref{HF2BOIsSymAlg}, \Cref{HF2MOisFree}, and \Cref{CPinftyNiceCalculation}.
\end{lemma}
\begin{proof}
The second formula is a direct consequence of the first.  For the first formula, note that because \(\{\beta_j^E\}_j\) provides a homology basis for \((E \sm MU)_* \CP^\infty\), we can decompose \(x^{MU}\) by the Kronecker pairing: \[x^{MU} = \sum_{j=0}^\infty \<x^{MU}, \beta_j^E\> (x^E)^{j+1}.\]  Expending the definition of the pairing gives
\begin{center}
\begin{tikzcd}
\S^{2j} \arrow{r} \arrow[bend left=15]{rr}{\beta_j^E} \arrow["{\<x^{MU}, \beta_j^E\>}"']{rdd} & E \sm MU(1) \arrow["1 \sm \eta_{MU} \sm 1"']{r} \arrow["1 \sm u_1"]{dd} & E \sm MU \sm MU(1) \arrow["1 \sm 1 \sm u_1"]{d} \arrow[bend left=90]{dd}{1 \sm 1 \sm x^{MU}} \\
& & E \sm MU \sm MU \arrow["1 \sm 1 \sm \eta \sm 1"]{d} \\
& E \sm MU & (E \sm MU)^{\sm 2} \arrow{l}{\mu}.
\end{tikzcd}
\end{center}
By definition, the module generators \(\beta_{j+1} \in E_{2(j+1)} \CP^\infty = E_{2j} MU(1)\) push forward along \(u_1\co MU(1) \to MU\) to define the algebra generators \(b_j \in E_{2j} MU\), giving the desired identification.
\end{proof}

\begin{corollary}[{\cite[Corollary II.6.6]{AdamsBlueBook}}]\label{HZMUCarriesALog}
In particular, for the orientation \(MU \to H\Z\) we have \[x_1 +_{MU} x_2 = \exp^H(\log^H(x_1) + \log^H(x_2)),\] where \(\exp^H(x) = \sum_{j=0}^\infty b_j x^{j+1}\). \qed
\end{corollary}

However, one also notes that \(H\Z_* MU = \Z[b_1, b_2, \ldots]\) carries the universal example of a formal group law with a \index{logarithm}logarithm---this observation follows directly from the Thom isomorphism, and so is independent of any knowledge about the ring \(MU_*\).  This brings us one step away from understanding \(MU_*\):

\begin{theorem}[{To be proven as \Cref{LazardsTheorem}}]\label{DummyLazardsThm}
The ring \(\sheaf O_{\moduli{fgl}}\) carrying the universal formal group law is free: it is isomorphic to a polynomial ring over \(\Z\) in countably many generators. \qed
\end{theorem}

\begin{corollary}[Quillen's theorem]\label{QuillensTheorem}
The natural map \(\sheaf O_{\moduli{fgl}} \to MU_*\) classifying the formal group law on \(MU_*\) is an isomorphism.
\end{corollary}
\begin{proof}
We proved in \Cref{QuillenSurjective} that this map is surjective.  We also proved in \Cref{RationalFGLsHaveLogarithms} that every rational formal group law has a unique strict logarithm, i.e., the long composite on the second row
\begin{center}
\begin{tikzcd}
\sheaf O_{\moduli{fgl}} \arrow{r} \arrow{d} & MU_* \arrow{r} & (H\Z_* MU) \arrow{d} \\
\sheaf O_{\moduli{fgl}} \otimes \Q \arrow{r} \arrow[bend left=15, "\simeq" near end]{rr} & MU_* \otimes \Q \arrow{r} \arrow[leftarrow, crossing over]{u} & (H\Z_* MU) \otimes \Q
\end{tikzcd}
\end{center}
is an isomorphism.  It follows from \Cref{DummyLazardsThm} that the left-most vertical map is injective, hence the top-left horizontal map is injective, hence it is an isomorphism.
\end{proof}

\begin{corollary}
The ring \(\pi_*(MU \sm MU)\) carries the universal example of two \index{formal group!strict isomorphism}strictly isomorphic formal group laws.  Additionally, the ring \(\pi_0 (\MUP \sm \MUP)\) carries the universal example of two isomorphic formal group laws.
\end{corollary}
\begin{proof}
Combine \Cref{OrientationsOnEAndMU} and \Cref{QuillensTheorem}.
\end{proof}


















% \subsection*{Run off}








% % divisors and line bundles on \(1\)--dimensional objects?

% poincare duality for manifolds with oriented tangent bundle

% wrong-way maps: \(\zeta^* \zeta_* 1\) gives the Euler class of the bundle

% Explicit Thom isomorphism map for universal cohomology: \(\xi: X \to BU(n)\) Thomifies to \(T(\xi) \to MU(n) \to MU\), representing a class \(g \in MU^* T(\xi)\), and this gives a map
% \begin{align*}
% MU^*(X) & \to MU^* T(\xi) & \cong MU^*(E, E_0)\\
% x & \mapsto & g \smile p^*(x),
% \end{align*}
% where \(p: E \to X\) is the projection and \(E_0\) is the image of the zero section.

% The wrong--way maps come from conjugating by Poincar\'e duality: \[E^* X \cong E_{d_X-*} X \to E_{d_X-*} Y \cong E^*{*-d_X+d_Y} Y.\]  Poincar\'e duality comes from asserting that the stable normal bundle is oriented for the theory, and then Atiyah duality says \[D(X_+) \simeq \Susp^{-n} T(\nu) [\simeq T(\nu - n\eps) \simeq T(-\tau)]. \]

% -----

% Need to talk about Gysin pushforwards in complex bordism and in ordinary cohomology.  Compare these with the theory of Thom isomorphism in general.  They're equivalent, right?  A complex orientation makes proper maps induce shriek maps, and shriek maps can be used to deduce what Chern classes are by push-pull: if \(\zeta: X \to E\) is the zero section of a complex bundle \(\xi\), then \(e(\xi) = i^* i_*(1)\) I think.


