% -*- root: main.tex -*-

\chapter{Finite spectra}\label{ChapterFiniteSpectra}


\todo[inline]{Andy Senger correctly points out that ``stalkwise'' is the wrong word to use in all this (if we mean to be working in the Zariski topology, which surely we must).  The stalks are selected by maps from certain local rings; $E_\Gamma$ selects the formal neighborhood of the special point inside of this; and $K_\Gamma$ selects the special point itself.  Is ``fiberwise'' enough of a weasel word to get out of this?  In any case, make sure you clean up all instances of the word ``stalk''.}


Our goal in this Case Study is to thoroughly examine one of the techniques from \Cref{UnorientedBordismChapter} that has not yet resurfaced: the idea that $H\F_2$--homology takes values in quasicoherent sheaves over some algebro-geometric object encoding the coaction of the dual Steenrod Hopf algebra.  We will find that this situation is quite generic: associated to mildly nice ring spectra $E$, we will construct a very rich algebro-geometric object $\context{E}$, called its context, such that $E$--homology sends spaces $X$ to sheaves $\context{E}(X)$ over $\context{E}$.  In still nicer situations, the difference between the $E_*$--module $E_*(X)$ and the sheaf $\context{E}(X)$ tracks exactly the analogue of the action of the dual Steenrod algebra, called the \textit{Hopf algebroid of stable $E$--homology cooperations}.  From this perspective, we will reinterpret Quillen's \Cref{QuillensTheorem} as giving a presentation \[\context{MUP} \xrightarrow{\simeq} \moduli{fg},\] where $\moduli{fg}$ is the \textit{moduli of formal groups}.  This indicates a program for studying periodic complex bordism, which we will spend the rest of this introduction outlining.

Abstractly, one can hope to study any sheaf, including $\context{E}(X)$, by analyzing its stalks.  The main utility of Quillen's theorem is that it gives us access to a concrete model of the context $\context{MUP}$, so that we can determine where to even look for those stalks.  However, even this is not really enough to get off the ground: the stalks of some sheaf can exhibit nearly arbitrary behavior.  In particular, there is little reason to expect the stalks of $\context{E}(X)$ to vary nicely with $X$.  Accordingly, given a map $f$ in the diagram
\begin{center}
\begin{tikzcd}
\Spec R \arrow{r}{f} & \moduli{fgl} \arrow[equal]{r} \arrow{d} & \context{MUP}[0] \arrow{d} \arrow[equal]{r} & \Spec MUP_0 \\
& \moduli{fg} \arrow[equal]{r} \arrow[leftarrow]{lu} & \context{MUP},
\end{tikzcd}
\end{center}
life would be easiest if the $R$--module determined by $f^* \context{MUP}(X)$ were itself the value of a homology theory $R_0(X) = MUP_0 X \otimes_{MUP_0} R$---this is exactly what it would mean for $R_0(X)$ to ``vary nicely with $X$''.  Of course, this is unreasonable to expect in general: homology theories are functors which convert cofiber sequences of spectra to long exact sequences of groups, but base--change from $\moduli{fg}$ to $\Spec R$ preserves exact sequences exactly when the diagonal arrow is \textit{flat}.  However, if flatness is satisfied, this gives the following theorem:

\begin{theorem}[Landweber]\label{LandwebersStackyTheorem}
Given such a diagram where the diagonal arrow is flat, the functor \[R_0(X) := MUP_0(X) \otimes_{MUP_0} R\] is a homology theory.
\end{theorem}

\noindent In the course of proving this theorem, Landweber additionally devised a method to recognize flat maps.  Recall that a map $f\co Y \to X$ of schemes is flat exactly when for any closed subscheme $i\co A \to X$ with ideal sheaf $\mathcal I$ there is an exact sequence \[0 \to f^* \sheaf I \to f^* \sheaf O_X \to f^* i_* \sheaf O_A \to 0.\]  Landweber classified the closed subobjects of $\moduli{fg}$, thereby giving a precise list of conditions needed to check maps for flatness.

This appears to be a moot point, however, as it is unreasonable to expect this idea to apply to computing stalks: the inclusion of a geometric point is flat only in highly degenerate cases.  We will see that this can be repaired: the inclusion of the formal completion of a subobject is flat in friendly situations, and so we naturally become interested in the infinitesimal deformation spaces of the geometric points $\Gamma$ on $\moduli{fg}$.  If we can analyze those, then Landweber's theorem will produce homology theories called \textit{Morava $E_\Gamma$--theories}.  Moreover, if we find that these deformation spaces are \emph{smooth}, it will follow that their deformation rings support regular sequences.  In this excellent case, by taking the regular quotient we will be able to recover \textit{Morava $K_\Gamma$--theory}, a \emph{homology theory}, which plays the role\footnote{To be clear: $K_\Gamma(X)$ may not actually compute the literal stalk of $\context{MUP}(X)$ at $\Gamma$, since the homotopical operation of quotienting out the regular sequence is potentially sensitive to torsion sections of $\context{MUP}(X)$.} of computing the stalk of $\context{MUP}(X)$ at $\Gamma$.\footnote{Incidentally, this program has no content when applied to $\context{H\F_2}$, as $\Spec \F_2$ is simply too small.}

We have thus assembled a task list:
\begin{itemize}
\item Describe the open and closed subobjects of $\moduli{fg}$.
\item Describe the geometric points of $\moduli{fg}$.
\item Analyze their infinitesimal deformation spaces.
\end{itemize}
These will occupy our attention for the first half of this Case Study.  In the second half, we will exploit these homology theories $E_\Gamma$ and $K_\Gamma$, as well as their connection to $\moduli{fg}$ and to $MU$, to make various structural statements about the category $\CatOf{Spectra}$.  These homology theories are especially well-suited to understanding the subcategory $\CatOf{Spectra}^{\fin}$ of finite spectra, and we will recount several important statements in that setting.  Together with these homology theories, these celebrated results (collectively called the nilpotence and periodicity theorems) form the basis of \textit{chromatic homotopy theory}.  In fact, our \emph{real} goal in this Case Study is to give an introduction to the chromatic perspective that remains in line with our algebro-geometrically heavy narrative.








\section{Descent and the context of a spectrum}\label{StableContextLecture}

\todo{You are sloppy about $EP_0$ versus $E_*$ in this lecture.  Pretty sure you mean to choose $EP_0$ and be done with it.}

In \Cref{HopfAlgebraLecture} we took for granted the $H\F_2$--Adams spectral sequence, which had the form \[E_2^{*, *} = H^*_{\mathrm{gp}}(\InternalAut(\G_a); \widetilde{H\F_2P_0 X}) \Rightarrow \pi_* X^\wedge_2,\] where we had already established some yoga by which we could identify the dual Steenrod coaction on $H\F_2P_0 X$ with an action of $\InternalAut{\G_a}$ on its associated quasicoherent sheaf over $\Spec \F_2$.  Our goal in this Lecture is to revise this tool to work for other ring spectra $E$ and target spectra $X$, eventually arriving at a spectral sequence with signature \[E_2^{*, *} = H^*(\context{E}; \context{E}(X)) \Rightarrow \pi_* X.\]  In particular, we will encounter along the way the object ``$\context{E}$'' envisioned in the introduction to this Case Study.

At a maximum level of vagueness, we are seeking a process by which its homotopy $\pi_* X$ can be recovered from the $E$--homology groups $E_* X$.  Generally speaking, spectral sequences arise from taking homotopy groups of a topological version of this same recovery process---i.e., recovering the spectrum $X$ from the spectrum $E \sm X$.  Recognizing that $X$ can be thought of as an $\S$--module and $E \sm X$ can be thought of as its base change to an $E$--module, we are inspired to double back and consider as inspiration an algebraic analogue of the same situation.  Given a ring map $f\co R \to S$ and an $S$--module $N$, Grothendieck's framework of \textit{(faithfully flat) descent} addresses the following questions:
\begin{enumerate}
\item When is there an $R$--module $M$ such that $N \cong S \otimes_R M = f^* M$?
\item What extra data can be placed on $N$, called \textit{descent data}, so that the category of descent data for $N$ is equivalent to the category of $R$--modules under the map $f^*$?
\item What conditions can be placed on $f$ so that the category of descent data for any given module is always contractible, called \textit{effectivity}?
\end{enumerate}

Suppose that we begin with an $R$--module $M$ and set $N = f^* M$, so that we are certain \emph{a priori} that the answer to the first question is positive.  The $S$--module $N$ has a special property, arising from $f$ being a ring map: there is a canonical isomorphism of $(S \otimes_R S)$--modules\todo{Get this right.}
\begin{align*}
\phi\co S \otimes_R N = (f \otimes 1)^* N = ((f \otimes 1) \circ f)^* M & \cong ((1 \otimes f) \circ f)^* M = (1 \otimes f)^* N = N \otimes_R S, \\
s_1 \otimes (s_2 \otimes m) & \mapsto (s_1 \otimes m) \otimes s_2.
\end{align*}
In fact, this isomorphism is compatible with further shuffles, in the sense that the following diagram commutes:\footnote{The commutativity of this triangle shows that any number of shuffles also commutes.}
\begin{center}
\begin{tikzcd}
N \otimes_R S \otimes_R S \arrow["\phi_{13}", "\simeq"']{rr} \arrow["\phi_{12}", "\simeq"']{rd} & & S \otimes_R S \otimes_R N \\
& S \otimes_R N \otimes_R S \arrow["\phi_{23}"', "\simeq"]{ru},
\end{tikzcd}
\end{center}
where $\phi_{ij}$ denotes applying $\phi$ to the $i${\th} and $j${\th} coordinates.

\begin{definition}
An $S$--module $N$ equipped with such an isomorphism $\phi\co S \otimes_R N \to N \otimes_R S$ which causes the triangle to commute is called a \textit{descent datum for $f$}.
\end{definition}

Descent data admit two equivalent reformulations, both of which are useful to note.
\begin{remark}[{\cite{Amitsur}}]
The ring $C = S \otimes_R S$ admits the structure of an $S$--coring: we can use the map $f$ to produce a relative diagonal map \[\Delta\co S \otimes_R S \cong S \otimes_R R \otimes_R S \xrightarrow{1 \otimes f \otimes 1} S \otimes_R S \otimes_R S \cong (S \otimes_R S) \otimes_S (S \otimes_R S).\]  The descent datum $\phi$ on an $S$--module $N$ is equivalent to a $C$--coaction map.  The $S$--linearity of the coaction map is encoded by a square
\begin{center}
\begin{tikzcd}
S \otimes_R N \arrow["1 \otimes \psi"]{r} \arrow["\phi" near end]{rrd} \arrow{d} & S \otimes_R N \otimes_S (S \otimes_R S) \arrow[crossing over]{d} \\
N \arrow["\psi"]{r} & N \otimes_S (S \otimes_R S) \arrow[equal]{r} & N \otimes_R S,
\end{tikzcd}
\end{center}
and the long composite gives the descent datum $\phi$.  Conversely, given a descent datum $\phi$ we can restrict it to get a coaction map by \[\psi\co N = R \otimes_R N \xrightarrow{f \otimes 1} S \otimes_R N \xrightarrow{\psi} N \otimes_R S.\]  The coassociativity condition on the comodule is equivalent under this correspondence to the commutativity of the triangle associated to $\phi$.
\end{remark}

\begin{remark}[{\cite[Theorem A]{HoveyMoritaThy}}]\label{StrictCechDescentRemark}
Alternatively, descent data also arise naturally as sheaves on simplicial schemes.  Associated to the map $f\co \Spec S \to \Spec R$, we can form a {\Cech} complex
\[\mathcal D_f := \left\{
\begin{tikzcd}[ampersand replacement=\&]
\Spec S \arrow{r} \arrow[leftarrow,shift left=\baselineskip]{r} \arrow[leftarrow,shift right=\baselineskip]{r} \&
\begin{array}{c} \Spec S \\ \times_{\Spec R} \\ \Spec S \end{array} \arrow[leftarrow, shift left=(2*\baselineskip)]{r} \arrow[shift left=\baselineskip]{r} \arrow[leftarrow]{r} \arrow[shift right=\baselineskip]{r} \arrow[leftarrow, shift right=(2*\baselineskip)]{r} \&
\begin{array}{c} \Spec S \\ \times_{\Spec R} \\ \Spec S \\ \times_{\Spec R} \\ \Spec S \end{array} \arrow[leftarrow, shift left=(3*\baselineskip)]{r} \arrow[shift left=(2*\baselineskip)]{r} \arrow[leftarrow, shift left=\baselineskip]{r} \arrow{r} \arrow[leftarrow, shift right=\baselineskip]{r} \arrow[shift right=(2*\baselineskip)]{r} \arrow[leftarrow, shift right=(3*\baselineskip)]{r} \&
\cdots
\end{tikzcd}
\right\},\]
which factors the map $f$ as
\begin{center}
\begin{tikzcd}
\Spec S \arrow["\mathrm{sk}^0"]{r} \arrow[bend left, "f"]{rr} & \mathcal D_f \arrow["c"]{r} & \Spec R.
\end{tikzcd}
\end{center}
A quasicoherent (and Cartesian~\cite[Tag 09VK]{stacks-project}) sheaf $\sheaf F$ over a simplicial scheme $X$ is a sequence of quasicoherent sheaves $\sheaf F[n]$ on $X[n]$ as well as, for each map $\sigma\co [m] \to [n]$ in the simplicial indexing category inducing a map $X(\sigma)\co X[n] \to X[m]$, a natural choice of isomorphism of sheaves \[\sheaf F(\sigma)^*\co X(\sigma)^* \sheaf F[m] \to \sheaf F[n].\]  In particular, a pullback $c^* \widetilde{M}$ gives such a quasicoherent sheaf on $\mathcal D_f$.  By restricting attention to the first three levels we find exactly the structure of the descent datum described before.  Additionally, we have a natural \textit{Segal isomorphism}
\begin{align*}
\mathcal D_f[1]^{\times_{\mathcal D_f[0]}(n)} & \xrightarrow{\simeq} \mathcal D_f[n] &
\text{(cf.\ $S \otimes_R S \otimes_R S$} & \cong \text{$(S \otimes_R S) \otimes_S (S \otimes_R S)$ at $n = 2$)},
\end{align*}
which shows that any descent datum (including those not arising, a priori, from a pullback) can be naturally extended to a full quasicoherent sheaf on $\mathcal D_f$.
\end{remark}

The following Theorem is the culmination of a typical first investigation of descent:\footnote{For details and additional context, see Vistoli~\cite[Section 4.2.1]{Vistoli}.  The story in the context of Hopf algebroids is also spelled out in detail by Miller~\cite{MillerSheavesGradings}.}

\begin{theorem}[Grothendieck]\label{OriginalFFDescent}\citeme{Actually give an original citation for f.f.\ descent? Or just reference Vakil}
If $f\co R \to S$ is faithfully flat, then the natural assignments
\begin{center}
\begin{tikzcd}
\CatOf{QCoh}(\Spec R) \arrow[bend left,"c^*"]{r} & \CatOf{QCoh}(\mathcal D_f) \arrow[bend left,"\lim"]{l}
\end{tikzcd}
\end{center}
form an equivalence of categories.
\end{theorem}
\begin{proof}[Jumping off point]
The basic observation in this case is that $0 \to R \to S \to S \otimes_R S$ is an exact sequence of $R$--modules.\footnote{In the language of \Cref{HF2HomologyIsValuedInAutGaEquivarModules}, this says that $R$ itself appears as the cofixed points $S \cotensor_{S \otimes_R S} R$.}  This makes much of the homological algebra involved work out.
\end{proof}

Without the flatness hypothesis, this Theorem fails dramatically and immediately.  For instance, the inclusion of the closed point \[f\co \Spec \F_p \to \Spec \Z\] fails to distinguish the $\Z$--modules $\Z$ and $\Z/p$.  Remarkably, this can be to large extent repaired by reintroducing homotopy theory and passing to derived categories---for instance, the complexes $Lf^* \widetilde{\Z}$ and $Lf^* \widetilde{\Z/p}$ become distinct as objects of $D(\Spec \F_p)$.  Our preceding discussion of descent in \Cref{StrictCechDescentRemark} can be quickly revised for this new homotopical setting, provided we remember to derive not just the categories of sheaves but also their underlying geometric objects.  Our approach is informed by the following result:

\begin{lemma}[{\cite[Theorem IV.2.4]{EKMM}}]\todo{Does a correct statement need boundedness conditions?}
There is an equivalence of $\infty$--categories between $D(\Spec R)$ and $\CatOf{Modules}_{HR}$. \qed
\end{lemma}

Hence, given a map of rings $f\co R \to S$, we redefine the derived descent object to be the cosimplicial ring spectrum
\[\mathcal D_{Hf} := \left\{
\begin{tikzcd}[ampersand replacement=\&]
HS \arrow[leftarrow]{r} \arrow[shift left=\baselineskip]{r} \arrow[shift right=\baselineskip]{r} \&
\begin{array}{c} HS \\ \sm_{HR} \\ HS \end{array} \arrow[shift left=(2*\baselineskip)]{r} \arrow[leftarrow, shift left=\baselineskip]{r} \arrow{r} \arrow[leftarrow, shift right=\baselineskip]{r} \arrow[shift right=(2*\baselineskip)]{r} \&
\begin{array}{c} HS \\ \sm_{HR} \\ HS \\ \sm_{HR} \\ HS \end{array} \arrow[shift left=(3*\baselineskip)]{r} \arrow[leftarrow, shift left=(2*\baselineskip)]{r} \arrow[shift left=\baselineskip]{r} \arrow[leftarrow]{r} \arrow[shift right=\baselineskip]{r} \arrow[leftarrow, shift right=(2*\baselineskip)]{r} \arrow[shift right=(3*\baselineskip)]{r} \&
\cdots
\end{tikzcd}
\right\},\]
and note that an $R$--module $M$ gives rise to a cosimplicial left--$\mathcal D_{Hf}$--module which we denote $\mathcal D_{Hf}(HM)$.  The totalization of this cosimplicial module gives rise to an $HR$--module receiving a natural map from $M$, and we can ask for an analogue of \Cref{OriginalFFDescent}.

\begin{lemma}\label{DescentFromHFpToHZp}
For $f\co \Z \to \F_p$ and $M$ a connective complex of $\Z$--modules, the totalization $\Tot \mathcal D_{Hf}(HM)$ recovers the $p$--completion of $M$.
\end{lemma}
\begin{proof}[Proof sketch]
The Hurewicz map $H\Z \to H\F_p$ kills $(p) \subseteq \pi_0 H\Z$, and we further calculate \[H\F_p \sm_{H\Z} H\F_p \simeq H\F_p \vee \Susp H\F_p\] to be connective.  Combining these facts shows that the filtration of $\mathcal D_{Hf}(HM)$ gives the $p$--adic filtration of the homotopy groups $\pi_* HM$.  If $\pi_* HM$ is already $p$--complete, then the reassembly map $HM \to \Tot \mathcal D_{Hf}(HM)$ is a weak equivalence.
\end{proof}

We are now close enough to our original situation that we can make the last leap: rather than studying a map $Hf\co HR \to HS$, we instead have the unit map $\eta\co \S \to E$ associated to some ring spectrum $E$.  Fixing a target spectrum $X$, we define the analogue of the descent object:
\begin{definition}
The \textit{descent object} for $X$ along $\eta\co \S \to E$ is the cosimplicial spectrum
\[\mathcal{D}_E(X) := \left\{
\begin{tikzcd}
\begin{array}{c} E \\ \sm \\ X \end{array} \arrow[leftarrow, shift left=(\baselineskip)]{r}{\mu} \arrow[shift left=(2*\baselineskip)]{r}{\eta_L} \arrow{r}{\eta_R} &
\begin{array}{c} E \\ \sm \\ E \\ \sm \\ X \end{array} \arrow[shift left=(3*\baselineskip)]{r} \arrow[leftarrow, shift left=(2*\baselineskip)]{r} \arrow[shift left=(\baselineskip)]{r}{\Delta} \arrow[leftarrow]{r} \arrow[shift right=(\baselineskip)]{r} &
\begin{array}{c} E \\ \sm \\ E \\ \sm \\ E \\ \sm \\ X \end{array} \arrow[shift left=(4*\baselineskip)]{r} \arrow[leftarrow, shift left=(3*\baselineskip)]{r} \arrow[shift left=(2*\baselineskip)]{r} \arrow[leftarrow, shift left=(\baselineskip)]{r} \arrow{r} \arrow[leftarrow, shift right=(\baselineskip)]{r} \arrow[shift right=(2*\baselineskip)]{r} &
\cdots
\end{tikzcd}
\right\}.\]
\end{definition}

\begin{lemma}[{\cite[Theorem 4.4.2.8.ii]{LurieHA}}]
If $E$ is an $A_\infty$--ring spectrum, then $\mathcal D_E(X)$ can be considered as a cosimplicial object in the $\infty$--category of $\CatOf{Spectra}$. \qed
\end{lemma}

\begin{definition}\label{DefnOfNilpCompletionAndASS}
The \textit{$E$--nilpotent completion} of $X$ is the totalization of this cosimplicial spectrum: \[X^\wedge_E := \Tot \mathcal D_E(X).\]  It receives a natural map $X \to X^\wedge_E$, the analogue of the natural map of $R$--modules $M \to c_* c^* M$ considered in \Cref{OriginalFFDescent}.
\end{definition}

\begin{remark}[{\cite[Theorem 1.12]{RavenelLocalizationWRTPeriodic}, \cite{BousfieldLocalization}}]
Ravenel proves the following generalization of \Cref{DescentFromHFpToHZp}.  Let $E$ be a connective ring spectrum, let $J$ be the set of primes complementary to those primes $p$ for which $E_*$ is uniquely $p$--divisible, and let $X$ be a connective spectrum.  If each element of $E_*$ has finite order, then $X^\wedge_E = X^\wedge_J$ gives the arithmetic completion of $X$---which we reinterpret as $\S^\wedge_J \to E$ being of effective descent\todo{Is this right?}.  Otherwise, if $E_*$ has elements of infinite order, then $X^\wedge_E = X_{(J)}$ gives the arithmetic localization---which we reinterpret as saying that $\S_{(J)} \to E$ is of effective descent.  Sorting out more encompassing conditions on maps $f\co R \to S$ of $E_\infty$--rings for which descent holds is a subject of serious study~\cite[Appendix D]{LurieSAG}.
\end{remark}

\begin{remark}
Even for connective ring spectra $E$, the Bousfield localization $L_E X$ does \emph{not} have to recover an arithmetic localization of $X$ if $X$ is not connective.  Take $E = H\Z$ and $X = KU$, which Snaith's theorem\footnote{You can also work this example without knowing Snaith's theorem. All you really need to know is that $\Susp^\infty_+ \CP^\infty[\beta^{-1}] \to KU$ is a map of ring spectra, so that $\pi_0 \Susp^\infty_+ \CP^\infty[\beta^{-1}]$ can't be a rational group since $\pi_0 KU = \Z$.} presents as \[X = KU = \Susp^\infty_+ \CP^\infty[\beta^{-1}],\] where $\beta\co \CP^1 \to \CP^\infty$ is the Bott element.  This gives \[H\Z_* KU = H\Z_*(\CP^\infty[\beta^{-1}]) = (H\Z_* \CP^\infty)[b_1^{-1}].\]  We can identify the pieces in turn: \Cref{HZGivesGa} shows $\CP^\infty_{H\Z} = \G_a$, so the dual Hopf algebra $\sheaf O_{\G_a}^* = H\Z_* \CP^\infty$ is a divided polynomial algebra on the class $b_1$.  Inverting $b_1$ then gives \[(H\Z_* \CP^\infty)[b_1^{-1}] = \Gamma[b_1][b_1^{-1}] = \Q[b_1^\pm],\] so that, in particular, there is a weak equivalence $H\Z \sm KU \to H\Q \sm KU$.  The cofiber \[KU \to KU \otimes \Q \to KU \otimes \Q/\Z\] is thus a nonzero $H\Z$--acyclic spectrum.
\end{remark}

Finally, we can interrelate these algebraic and topological notions of descent by studying the coskeletal filtration spectral sequence for $\pi_* X^\wedge_E$, which we define to be the \textit{$E$--Adams spectral sequence} for $X$.  Applying the homotopy groups functor to the cosimplicial ring spectrum $\mathcal D_E$ gives a cosimplicial ring $\pi_* \mathcal D_E$, which we would like to connect with an algebraic descent object of the sort considered in \Cref{StrictCechDescentRemark}.  In order to make this happen, we need two niceness conditions on $E$:

\begin{definition}
A ring spectrum $E$ satisfies \CH, the \textbf Commutativity \textbf Hypothesis, when the ring $\pi_* E^{\sm j}$ is commutative for all $j \ge 1$.  In this case, we can form the simplicial scheme \[\context{E} = \Spec \pi_* \mathcal D_E,\] called the \textit{context} of $E$.
\end{definition}

\begin{definition}
A ring spectrum $E$ satisfies \FH, the \textbf Flatness \textbf Hypothesis, when the right-unit map $E_* \to E_* E$ is flat.\footnote{If $E$ is a commutative ring spectrum, then this is equivalent to asking that the left-unit map is a flat map of $E_*$--modules.}  In this case, the Segal map \[(E_* E)^{\otimes_{E_*} j} \otimes_{E_*} E_* X \to \pi_*(E^{\sm (j+1)} \sm X) = \pi_* \mathcal D_E(X)[j]\] is an isomorphism for all $X$.  In geometric language, this says that $\context{E}$ is valued in simplicial sets equivalent to nerves of groupoids and that \[\context{E}(X) := \widetilde{\pi_* \mathcal D_E(X)}\] forms a Cartesian quasicoherent sheaf over $\context{E}$.  In this sense, we have constructed a factorization
\begin{center}
\begin{tikzcd}
\CatOf{Spectra} \arrow["E_*(-)"]{rr} \arrow["\context{E}(-)"]{rd} & & \CatOf{Modules}_{E_*} \\
& \CatOf{QCoh}(\context{E}) \arrow["{(-)[0]}"]{ru} .
\end{tikzcd}
\end{center}
\end{definition}

While {\CH} and {\FH} are enough to guarantee that $\context{E}$ and $\context{E}(X)$ are well-behaved, they still do not exactly connect us with \Cref{StrictCechDescentRemark}.  The main difference is that the ring of homology cooperations for $E$ \[E_*E = \pi_*(E \sm E) = \pi_* \mathcal D_E[1]\] is only distantly related to the tensor product $E_* \otimes_{\pi_* \S} E_*$ (or even $\Tor^{\pi_* \S}_{*, *}(E_*, E_*)$).  This is a trade we are eager to make, as the latter groups are typically miserably behaved, whereas $E_* E$ is typically fairly nice.  In order to take advantage of this, we enlarge our definition to match:

\begin{definition}\label{FHGivesComodules}
Let $A$ and $\Gamma$ be commutative rings with associated affine schemes $X_0 = \Spec A$, $X_1 = \Spec \Gamma$.  A \textit{Hopf algebroid} consists of the pair $(A, \Gamma)$ together with structure maps\todo{Fix the spacing around the superscripts.}
\begin{align*}
\eta_L \co A & \to \Gamma, & s\co X_1 & \to X_0, \\
\eta_R \co A & \to \Gamma, & t\co X_1 & \to X_0, \\
\Delta \co \Gamma & \to \Gamma {}^{\eta_R} \otimes_A^{\eta_L} \Gamma, & \circ\co X_1 {}^{t}\times_{X_0}^{s} X_1 & \to X_1, \\
\chi \co \Gamma & \to \Gamma, & (-)^{-1}\co X_1 & \to X_1,
\end{align*}
such that $(X_0, X_1)$ forms a groupoid scheme.  An \textit{$(A, \Gamma)$--comodule} is an $A$--module equipped with a $\Gamma$--comodule structure, and such a comodule is equivalent to a Cartesian quasicoherent sheaf on the nerve of $(X_0, X_1)$.
\end{definition}

\begin{example}
A Hopf $k$--algebra $H$ gives a Hopf algebroid $(k, H)$.  The scheme of objects $\Spec k$ in the groupoid scheme is the constant scheme $0$.
\end{example}

\begin{lemma}\label{IdentifyingAdamsE2Page}
For $E$ an $A_\infty$--ring spectrum satisfying {\CH} and {\FH}, the $E_2$--page of its Adams spectral sequence can be identified as
\begin{align*}
E_2^{*, *} & = \Cotor_{E_* E}^{*, *}(E_*, E_* X) \\
& \cong H^*(\context{E}; \context{E}(X) \otimes \omega^{\otimes *}) \oplus H^*(\context{E}; \context{E}(\Susp X) \otimes \omega^{\otimes *})[1] \Rightarrow \pi_* X^\wedge_E.
\end{align*}
\end{lemma}
\begin{proof}[Proof sketch]
The homological algebra of Hopf algebras from \Cref{HopfAlgebraLecture} can be lifted almost verbatim, allowing us to define resolutions suitable for computing derived functors~\cite[Definition A1.2.3]{RavenelGreenBook}.  This includes the cobar resolution~\cite[Definition A1.2.11]{RavenelGreenBook}, which shows that the associated graded for the coskeletal filtration of $\mathcal D_E(X)$ is a complex computing the derived functors claimed in the Lemma statement.
\end{proof}

\begin{remark}\label{HurewiczRemark}
The sphere spectrum fails to satisfy {\CH}, so the above results do not apply to it, but the $\S$--Adams spectral sequence is particularly degenerate: it consists of $\pi_* X$, concentrated on the $0$--line.  For any other ring spectrum $E$, the unit map $\S \to E$ induces a map of Adams spectral sequences whose image on the $0$--line are those maps of comodules induced by applying $E$--homology to a homotopy element of $X$---i.e., the Hurewicz image of $E$.
\end{remark}

\begin{remark}\label{WarningAboutStacks}
In \Cref{DescentFromHFpToHZp}, we discussed translating from the algebra descent picture to a homotopical one, and a crucial point was how thorough we had to be: we transferred not just to the derived category $D(\Spec R)$ but we also replaced the base ring $R$ with its homotopical incarnation $HR$.  In \Cref{FHGivesComodules}, we have not been as thorough as possible: both $X_0$ and $X_1$ are schemes and hence satisfy a sheaf condition individually, but the functor $(X_0, X_1)$, thought of as valued in homotopy $1$--types, does not necessarily satisfy a homotopy sheaf condition.  Enforcing this descent condition results in the \textit{associated stack}~\cite[Definition 8.13]{HopkinsCOCTALOS}, denoted \[\Spec A \mmod \Spec \Gamma = X_0 \mmod X_1.\]  Remarkably, this does not change the category of Cartesian quasicoherent sheaves---it is still equivalent to the category of $(A, \Gamma)$--comodules~\cite[Proposition 11.6]{HopkinsCOCTALOS}.  However, several different Hopf algebroids (even those with maps between them inducing natural equivalences of groupoid schemes, as studied by Hovey~\cite[Theorem D]{HoveyMoritaThy}) can give the same associated stack, resulting in surprising equivalences of comodule categories.\footnote{We will employ one of these surprising equivalences in \Cref{OpenSubstacksOfMfg}.}  For the most part, it will not be especially relevant to us whether we are considering the groupoid scheme or its associated stack, so we will not draw much of a distinction.\todo{Maybe this is irresponsible and we should be careful not to be sloppy.}
\end{remark}

\begin{example}
Most of the homology theories we will discuss have these {\CH} and {\FH} properties.  For an easy example, $H\F_2P$ certainly has this property: there is only one possible algebraic map $\F_2 \to \mathcal A_*$, so {\FH} is necessarily satisfied.  This grants us access to a description of the context for $H\F_2$: \[\context{H\F_2P} = \Spec \F_2 \mmod \InternalAut{\G_a}.\]
\end{example}

\begin{example}\label{ContextOfMUPExample}
The context for $MUP$ is considerably more complicated, but Quillen's theorem can be equivalently stated as giving a description of it.  Quillen's theorem on its face gives an equivalence $\Spec MUP_0 \cong \moduli{fgl}$, but in \Cref{OrientationsOnEAndMU} we also gave a description of $\Spec MUP_0 MUP$: it is the moduli of pairs of formal group laws equipped with an invertible power series intertwining them.  Altogether, this presents $\context{MUP}$ as the moduli of formal groups: \[\context{MUP} \simeq \moduli{fg} := \moduli{fgl} \mmod \moduli{ps}^{\gpd},\] where $\moduli{ps} = \InternalEnd(\A^1)$ is the moduli of self-maps of the affine line (i.e., of power series) and $\moduli{ps}^{\gpd}$ is the multiplicative subgroup of invertible such maps.  We include a picture of the $p$--localized Adams $E_2$--page in \Cref{ANSS2Figure} and \Cref{ANSS3Figure}.  In view of \Cref{WarningAboutStacks}, there is an important subtlety about the stack $\moduli{fg}$: an $R$--point is a functor on affines over $\Spec R$ which is locally isomorphic to a formal group, but whose local isomorphism \emph{may not patch} to give a global isomorphism.  This does not agree, a priori, with the definition of formal group given in \Cref{DefnFormalGps}, where the isomorphism witnessing a functor as a formal variety was expected to be global.  We will address this further in \Cref{CoordinatizbleFGs} below.
\end{example}

\afterrectopage{
\begin{sidewaysfigure}
\centering
% -*- root: main.tex -*-

\begin{sseqpage}[
    degree={-1}{#1},
    differentials={-{>[width=4]}, target anchor=-35},
    classes={minimum width={0.3ex}},
    %edge labels=description,
    math nodes,    
    y range={0}{10},
    x range={0}{17},
    xscale=0.9,
    yscale=0.7,
    above left label distance={0em},
    label distance={0.2em},
]

\node[rectangle,fill] at (0,0) {};
\class(0,0)

\etaclass["{\alpha_1}" {above left=0.2em}](1,1)
% the eta tower on alpha_1
\etaclass["\alpha_1^2" {above left=0.2em}](2,2)
\etaclass["\alpha_1^3" {above left=0.2em}](3,3)
\etaclass["\alpha_1^4" {above left=0.2em}](4,4)
\etaclass["\alpha_1^5" {above left=0.2em}](5,5)
\etaclass["\alpha_1^6" {above left=0.2em}](6,6)
\etaclass(7,7)
\etaclass(8,8)
\etaclass(9,9)
\etaclass(10,10)
\etaclass(11,11)
\etaclass(12,12)
\etaclass(13,13)

\class["{\alpha_{2/2}}" {below=0.01em}, circlen=2] (3,1)

\class["\alpha_3" {below=0.2em}] (5,1)
% the eta tower on alpha_3
\etaclass(6,2)
\etaclass(7,3)
\etaclass(8,4)
\etaclass(9,5)
\etaclass(10,6)
\etaclass(11,7)
\etaclass(12,8)
\etaclass(13,9)
\etaclass(14,10)
\etaclass(15,11)

\class["{\alpha_{4/4}}" {below=0.1em}, circlen=4] (7,1)
% the eta tower on alpha_4
\etaclass(8,2)
\etaclass(9,3)
\etaclass(10,4)
\etaclass(11,5)
\etaclass(12,6)
\etaclass(13,7)
\etaclass(14,8)
\etaclass(15,9)
\etaclass(16,10)
\etaclass(17,11)
\etaclass(18,12)

\class["\alpha_5" {below=0.2em}] (9,1)
% the eta tower on alpha_5
\etaclass(10,2)
\etaclass(11,3)
\etaclass(12,4)
\etaclass(13,5)
\etaclass(14,6)
\etaclass(15,7)
\etaclass(16,8)
\etaclass(17,9)
\etaclass(18,10)
\etaclass(19,11)
\etaclass(20,12)

\class["\alpha_{6/3}" {below=0.2em}, circlen=3] (11,1)
% the eta tower on alpha_6
\etaclass(12,2)
\etaclass(13,3)
\etaclass(14,4)
\etaclass(15,5)
\etaclass(16,6)
\etaclass(17,7)
\etaclass(18,8)
\etaclass(19,9)

\class["\alpha_7" {below=0.2em}] (13,1)
% the eta tower on alpha_7
\etaclass(14,2)
\etaclass(15,3)
\etaclass(16,4)
\etaclass(17,5)
\etaclass(18,6)
\etaclass(19,7)
\etaclass(20,8)
\etaclass(21,9)

\class["\alpha_{8/5}" {below=0.2em}, circlen=5] (15,1)
% the eta tower on alpha_8
\etaclass(16,2)
\etaclass(17,3)
\etaclass(18,4)
\etaclass(19,5)
\etaclass(20,6)

\class["\alpha_9" {below=0.2em}] (17,1)
% the eta tower on alpha_9
\etaclass(18,2)
\etaclass(19,3)
\etaclass(20,4)





% the beta family
\class["{\beta_{2/2}}" {below=0.05em}](6,2)
\class["\beta_2" {below=0.2em}](8,2)
\class["\alpha_{2/2}^3" {below=0.2em}](9,3) \structline(8,2,-1)(9,3,-1)
\class["{\beta_{4/4}}" {below=0.2em}](14,2)
\class["\beta_3" {below right=0.2em}](14,2) \etaclass(15,3)
\class["{\beta_{4/3}}" {below=0.2em}](16,2) \etaclass(17,3)
\class["{\alpha_{2/2} \beta_3}" {below=0.2em}](17,3)


% the additive extensions
\structline[densely dotted](3,1)(3,3)
\structline[densely dotted](11,1)(11,3)

% differentials off of alpha_3
\d3(5,1)
\d3(6,2,1,1)
\d3(7,3)
\d3(8,4)
\d3(9,5)
\d3(10,6)
\d3(11,7)
\d3(12,8)
\d3(13,9)
\d3(14,10)

% differentials off of alpha_6
\d3(11,1)
\d3(12,2)
\d3(13,3)
\d3(14,4)
\d3(15,5)
\d3(16,6)
\d3(17,7)
\d3(18,8)

% differentials off of alpha_7
\d3(13,1)
\d3(14,2,1,1)
\d3(15,3,1,1)
\d3(16,4)
\d3(17,5)
\d3(18,6)

\end{sseqpage}

\caption{A small piece of the $MU_{(2)}$--Adams spectral sequence for the sphere, beginning at the second page~\cite[pg.\ 429]{RavenelNovicesGuide}.  North-east lines denote multiplication by $\eta = \alpha_1$, north-west lines denote $d_3$--differentials, and vertical dotted lines indicate additive extensions.  Elements are labeled according to the conventions of \Cref{GreekLetterElements}, and in particular $\alpha_{i/j}$ is $2^j$--torsion.}\label{ANSS2Figure}
\end{sidewaysfigure}
\begin{sidewaysfigure}
\centering
% -*- root: main.tex -*-

\begin{sseqpage}[
    Adams grading,
    differentials={-{>[width=4]}, target anchor=-60},
    y range={0}{10},
    x range={0}{45},
    x tick step=5,
    xscale=0.34,
    yscale=0.75,
    class labels={above left=0.2em},
]
\class[rectangle,fill,inner sep=3pt](0,0)

\class["\alpha_1"](3,1) \structline(0,0)(3,1)
\class["\alpha_2"](7,1)
\class[circlen=2,"\alpha_{3/2}"](11,1)
\class["\alpha_4"](15,1)
\class["\alpha_5"](19,1)
\class[circlen=2,"\alpha_{6/2}"](23,1)
\class["\alpha_7"](27,1)
\class["\alpha_8"](31,1)
\class[circlen=3,"\alpha_{9/3}"](35,1)
\class["\alpha_{10}"](39,1)
\class["\alpha_{11}"](43,1)

\class["\beta_1"](10,2) \structline(3,1)(10,2)
\class["\beta_2"](26,2)
\class["\beta_{3/3}"](34,2)
\class["\beta_{3/2}"](38,2)
\class["\beta_3"](42,2)

% tower definitions
\gdef\alphaclass(#1,#2){
    \class(#1,#2)
    \structline(#1-3,#2-1,-1)(#1,#2,-1)
}
\gdef\betaclass(#1,#2){
    \class(#1,#2)
    \structline(#1-7,#2-1,-1)(#1,#2,-1)
}

% tower off of beta_1
\alphaclass(13,3)
\betaclass(20,4)
\alphaclass(23,5)
\betaclass(30,6)
\alphaclass(33,7)
\betaclass(40,8)
\alphaclass(43,9)
\betaclass(50,10)

% tower off of beta_2
\alphaclass(29,3)
\betaclass(36,4)
\alphaclass(39,5)
\betaclass(46,6)
\alphaclass(49,7)


% tower off of beta_3/3
\alphaclass(37,3)
\betaclass(44,4)
\alphaclass(47,5)
\betaclass(54,6)

% tower off of beta_3
\class[fill,double=white](45,3) \structline(42,2,-1)(45,3,-1)
\betaclass(52,4)

% d5s
\d5(34,2)
\d5(44,4)
\end{sseqpage}

\caption{A small piece of the $MU_{(3)}$--Adams spectral sequence for the sphere, beginning at the second page~\cite[Figure 1.2.19]{RavenelGreenBook}.  North-east lines denote multiplication by $\alpha_1$ or by $\beta_1/\alpha_1$, and north-west lines denote $d_5$--differentials.  Elements are labeled according to the conventions of \Cref{GreekLetterElements}, and in particular $\alpha_{i/j}$ is $3^j$--torsion.}\label{ANSS3Figure}
\end{sidewaysfigure}
}

\begin{example}\label{ContextOfMOPExample}
The context for $MOP$, by contrast, is reasonably simple.  \Cref{CalculationOfPiStarMO} shows that the scheme $\Spec MOP_0$ classifies formal group laws over $\F_2$ which admit logarithms, so that $\context{MOP}$ consists of the groupoid of formal group laws with logarithms and isomorphisms between them.  This admits a natural deformation-retraction to the moduli consisting just of $\G_a$ and its automorphisms, expressing the redundancy in $MOP_0(X)$ encoded in the splitting of \Cref{MOSplitsIntoHF2s}.
\end{example}

\begin{remark}
The algebraic moduli $\context{MU} = (\Spec MU_*, \Spec MU_* MU)$ and the topological moduli $(MU, MU \sm MU)$ are quite different.  An orientation $MU \to E$ selects a coordinate on the formal group $\CP^\infty_E$, but $\CP^\infty_E$ itself exists independently of the orientation.  Hence, while $\context{MU}(E_*)$ can have many connected components corresponding to \emph{distinct formal groups} on the coefficient ring $E_*$, the groupoid $\CatOf{RingSpectra}(\sheaf D_{MU}, E)$ has only one connected component corresponding to the formal group $\CP^\infty_E$ intrinsic to $E$.\footnote{The reader ought to compare this with the situation in explicit local class field theory, where a local number field has a preferred formal group attached to it.}
\end{remark}

\begin{remark}\citeme{pg 5 of From Spectra To Stacks by Hopkins in the TMF volume}
If $E$ is a complex-oriented ring spectrum, then the simplicial sheaf $\context{MU}(E)$ has an extra degeneracy, which causes the $MU$--based Adams spectral sequence for $E$ to degenerate.  In this sense, the ``stackiness'' of $\context{MU}(E)$ is exactly a measure of the failure of $E$ to be orientable.
\end{remark}

\begin{remark}
It is also possible to construct an Adams spectral sequence by iteratively smashing with the fiber sequence $\overline E \to \S \to E$ to form the tower
\begin{center}
\begin{tikzcd}
\S \sm X \arrow{d} & \overline E \sm X \arrow{l} \arrow{d} & \overline E^{\sm 2} \sm X \arrow{l} \arrow{d} & \cdots \arrow{l} \\
E \sm X & E \sm \overline E \sm X & E \sm \overline E^{\sm 2} \sm X & \cdots.
\end{tikzcd}
\end{center}
This presentation makes the connection to descent much more opaque, but it does not require $E$ to be an $A_\infty$--ring spectrum.
\end{remark}




\todo{You can build an Adams resolution in the absence of an $A_\infty$ structure too, you just miss the descent picture.}
\todo{Think about what sorts of simplicial sheaves you really want. They seem like they should be valued in something like quasicategories: the stable operations are valued in space-like simplicial sets, the isogenies pile is a sheaf of \emph{categories}, then unstable operations generically have some other weird structure...}
\todo{We should expand the comparison of a Cartesian q.c.\ sheaf on an affinely presented stack with a module plus structural data.}
\citeme{Pridham's article \textit{Presenting higher stacks as simplicial schemes} seems like a good reference?  Maybe some Toen things are appropriate?  I don't really know where this simplicial scheme stuff is written down.}
\todo{Say what open, closed, flat maps of simplicial schemes are?}
\todo{Jon thinks that this picture can be instructively recast in terms of the cotangent complex.  I'm not sure how, but it's something to keep in mind for later.}
\todo{Hovey and Strickland have published a whole bunch of papers about establishing a reasonable (un/bounded) derived category for sheaves on various stacks important to topology, which go by names like $\operatorname{Stable}(BP_* BP)$.  It's probably worth putting in a remark about this here, admitting that the homotopical version of the stuff here isn't totally worked out, but this is the major foothold on it.}
\todo{Somewhere in here you should talk about how $DX_+$ is an $E_\infty$ ring pro-spectrum, the spectral cohomology ring $F(X_+, E)$ is the base-change of $DX_+$ from $\S$ to $E$, that the weak topology is a natural thing to consider, and that the subscript notation is motivated by this change-of-base observation.}









\section{The structure of \texorpdfstring{$\moduli{fg}$}{Mfg} I: The affine cover}\label{MfgI:AffineCover}

In \Cref{FHGivesComodules} we gave a factorization
\begin{center}
\begin{tikzcd}
\CatOf{Spectra} \arrow["MUP_0(-)"]{rr} \arrow["\context{MUP}(-)"]{rd} & & \CatOf{Modules}_{MUP_0} \\
& \CatOf{QCoh}(\context{MUP}) \arrow["{(-)[0]}"]{ru} ,
\end{tikzcd}
\end{center}
and in \Cref{ContextOfMUPExample} we established an equivalence \[\phi\co \context{MUP} \xrightarrow{\simeq} \moduli{fg}.\]  Our program, as outlined in the introduction, is to analyze this functor $\context{MUP}(-)$ by postcomposing it with $\phi^*$ and studying the resulting sheaf over $\moduli{fg}$.  In order to perform such an analysis, we will want a firm grip on the geometry of the stack $\moduli{fg}$, and in this Lecture we begin by studying the scheme $\moduli{fgl}$ as well as the natural covering map \[\moduli{fgl} \to \moduli{fg}.\]

\begin{definition}\label{MfglDefn}
There is an affine scheme $\moduli{fgl}$ classifying formal group laws.  Begin with the scheme classifying \emph{all} bivariate power series:
\begin{align*}
\Spec \Z[a_{ij} \mid i, j \ge 0] & \leftrightarrow \left\{ \text{bivariate power series} \right\}, \\
f \in \Spec\Z[a_{ij} \mid i, j \ge 0](R) & \leftrightarrow \sum_{i, j \ge 0} f(a_{ij}) x^i y^j.
\end{align*}
Then, $\moduli{fgl}$ is the closed subscheme selected by the formal group law axioms in \Cref{FGLDefinition}.
\end{definition}

This presentation of $\moduli{fgl}$ as a subscheme appears to be extremely complicated in that its ideal is generated by many hard-to-describe elements, but $\moduli{fgl}$ itself is actually not complicated at all.  We will prove the following:
\begin{theorem}[{\cite[Th\'eor\`eme II]{Lazard}}]\label{LazardsTheorem}
There is a noncanonical isomorphism
\[\pushQED{\qed}
\sheaf{O}_{\moduli{fgl}} \cong \Z[b_n \mid 1 \le n < \infty] =: L. \qedhere
\popQED{\qed}\]
\end{theorem}
\begin{proof}
Let $L = \Z[b_0, b_1, b_2, \ldots] / (b_0 - 1)$ be the universal ring supporting an exponential \[\exp(x) := \sum_{j=0}^\infty b_j x^{j+1}\] with compositional inverse \[\log(x) := \sum_{j=0}^\infty m_j x^{j+1}.\]  They induce a formal group law on $L$ by the conjugation formula \[x +_! y = \exp(\log(x) + \log(y)),\] which is in turn classified by a map $u\co \sheaf O_{\moduli{fgl}} \to L$.\footnote{This is \emph{not} the universal formal group law.  We will soon see that some formal group laws do not admit logarithms.}  Modulo decomposables, we compute
\begin{align*}
x & = \exp(\log(x)) \\
& = x + \sum_{n=1}^\infty m_n x^{n+1} + \sum_{n=1}^\infty b_n \left( x + \sum_{j=1}^\infty m_j x^{j+1} \right)^{n+1} \\
& \equiv x + \sum_{n=1}^\infty m_n x^{n+1} + \sum_{n=1}^\infty b_n x^{n+1} \pmod{\text{decomposables}},
\end{align*}
hence $b_n \equiv -m_n \pmod{\text{decomposables}}$.  Using this, we then compute
\begin{align*}
x +_! y & = \exp(\log(x) + \log(y)) \\
& = \left( (x + y) + \sum_{n=1}^\infty m_n (x^{n+1} + y^{n+1}) \right) + \sum_{n=1}^\infty b_n \left( (x + y) + \sum_{j=1}^\infty m_j (x^{j+1} + y^{j+1}) \right)^{n+1} \\
& \equiv x + y + \sum_{n=1}^\infty -b_n (x^{n+1} + y^{n+1}) + \sum_{n=1}^\infty b_n (x+y)^{n+1} \pmod{\text{decomposables}} \\
& = x + y + \sum_{n=1}^\infty b_n ((x+y)^{n+1} - x^{n+1} - y^{n+1}),
\end{align*}
hence \[u(a_{i(n-i)}) \equiv \binom{n}{i} b_{n-1} \pmod{\text{decomposables}}.\]  It follows that the map $Qu$ on degree $2n$ has image the subgroup $T_{2n}$ generated by $d_{n+1} b_n$, where $d_{n+1} = \gcd\left( \binom{n+1}{k} \middle| 0 < k < n + 1 \right)$.  \Cref{LazardSplittingLemma} below provides a canonical splitting of $Qu$, and we couple it to the freeness of $L$ to \emph{choose} an algebra splitting \[L \xrightarrow{v} \sheaf O_{\moduli{fgl}} \xrightarrow{u} L.\]  The map $uv$ is injective, so $v$ is injective.  Furthermore, $Qv$ is designed to be surjective, so $v$ itself is surjective and hence an isomorphism.
\end{proof}

Recall that we have yet to prove the following Lemma:

\begin{lemma}\label{LazardSplittingLemma}
There is a canonical splitting $T_{2n} \to (Q \mathcal O_{\moduli{fgl}})_{2n}$.
\end{lemma}

\begin{definition}\label{DefinitionSymmetric2Cocycle}
In order to prove the missing \Cref{LazardSplittingLemma}, it will be useful to study the series $+_\phi$ ``up to degree $n$'', i.e., modulo $(x, y)^{n+1}$.  Such a truncated series satisfying the analogues of the formal group law axioms is called a \textit{formal $n$--bud}.\footnote{A formal $n$--bud determines a ``multiplication'' $(\A^1 \times \A^1)^{(n)} \to \A^{1,(n)}$.  Note that this does \emph{not} belong to a group object, since $(\A^1 \times \A^1)^{(n)} \not\simeq \A^{1,(n)} \times \A^{1,(n)}$.  This is the observation that the ideals $(x, y)^{n+1}$ and $(x^{n+1}, y^{n+1})$ are distinct.}  We will additionally be moved to study the difference between a formal $n$--bud and a formal $(n+1)$--bud extending it.  The simplest case of this is when the formal $n$--bud is just the additive law $x +_\phi y = x + y$, in which case any extension to an $(n+1)$--bud has the form $x + y + f(x, y)$ for $f(x, y)$ a homogeneous polynomial of degree $n$.  Symmetry of the group law requires $f(x, y)$ to be symmetric, and associativity of the group law requires $f(x, y)$ to satisfy the equation \[f(x, y) - f(t + x, y) + f(t, x + y) - f(t, x) = 0.\]  Such a polynomial is called a \textit{symmetric $2$--cocycle} (of degree $n$).\footnote{We will justify the ``$2$--cocycle'' terminology in the course of the proof of \Cref{Symmetric2CocycleLemma}.}
\end{definition}

\begin{proof}[{Reduction of \Cref{LazardSplittingLemma} to \Cref{Symmetric2CocycleLemma}}]
We now show that the following conditions are equivalent:
\begin{enumerate}
\item (\Cref{Symmetric2CocycleLemma}) Symmetric $2$--cocycles that are homogeneous polynomials of degree $n$ are spanned by \[c_n = \frac{1}{d_n} \cdot ((x + y)^n - x^n - y^n),\] where $d_n = \gcd\left( \binom{n}{k} \middle| 0 < k < n \right)$.
\item For $F$ is an $n$--bud, the set of $(n+1)$--buds extending $F$ form a torsor under addition for $R \otimes c_n$.
\item Any homomorphism $(Q\mathcal O_{\moduli{fgl}})_{2n} \to A$ factors through the map $(Q \mathcal O_{\moduli{fgl}})_{2n} \to T_{2n}$.
\item (\Cref{LazardSplittingLemma}) There is a canonical splitting $T_{2n} \to (Q \mathcal O_{\moduli{fgl}})_{2n}$.
\end{enumerate}

To verify that Claims 1 and 2 are equivalent, suppose that $x +_\phi y$ is some $(n+1)$--bud and that $x +_\phi' y$ is some $(n+1)$--bud such that \[(x +_\phi' y) = (x +_\phi y) + f(x, y)\] where $f(x, y)$ is homogeneous of degree $(n+1)$.  Symmetry of $x +_\phi' y$ enforces symmetry of $f$, and from associativity we calculate
\begin{align*}
x +_\phi' (y +_\phi' z) & = x +_\phi' (y +_\phi z + f(y, z)) \\
& = x +_\phi (y +_\phi z + f(y, z)) + f(x, y +_\phi z + f(y, z)) \\
& \equiv x +_\phi (y +_\phi z) + f(y, z) + f(x, y + z) \pmod{(x, y)^{n+2}}, \\
(x +_\phi' y) +_\phi' z = & (x +_\phi y + f(x, y)) +_\phi' z \\
& = (x +_\phi y + f(x, y)) +_\phi z + f(x +_\phi y + f(x, y), z) \\
& \equiv (x +_\phi y) +_\phi z + f(x, y) + f(x + y, z) \pmod{(x, y)^{n+2}},
\end{align*}
resulting in the $2$--cocycle condition on $f$.  Conversely, given such a $2$--cocycle $f(x, y)$, the formal $(n+1)$--bud $+_\phi'$ formed by translating $+_\phi$ by $f$ is again a formal $(n+1)$--bud extending the same formal $n$--bud.

To see that Claim 2 is equivalent to Claim 3, note that a group map \[(Q\mathcal O_{\moduli{fgl}})_{2n} \to A\] is equivalent data to a ring map \[\mathcal O_{\moduli{fgl}} \to Z \oplus A\] with the prescribed behavior on $(Q\mathcal O_{\moduli{fgl}})_{2n}$ and which sends all other indecomposables to $0$..  This shows that such a homomorphism of groups determines an extension of the $n$--bud $\G_a$ to an $(n+1)$--bud, which takes the form of a $2$--cocycle with coefficients in $A$, and hence factors through $T_{2n}$.

Finally, Claim 4 is the universal case of Claim 3.
\end{proof}

We will now verify Claim 1 computationally, completing the proof of \Cref{LazardSplittingLemma} (and hence \Cref{LazardsTheorem}).

\begin{lemma}[{Symmetric $2$--cocycle lemma~\cite[Lemme 3]{Lazard}, cf.\ \cite[Theorem 3.1]{HopkinsCOCTALOS}}]\label{Symmetric2CocycleLemma}
Symmetric $2$--cocycles that are homogeneous polynomials of degree $n$ are spanned by \[c_n = \frac{1}{d_n} \cdot ((x + y)^n - x^n - y^n),\] where $d_n = \gcd\left( \binom{n}{k} \middle| 0 < k < n \right)$.
\end{lemma}
\begin{proof}
We begin with a reduction of the sorts of rings over which we must consider the possible symmetric $2$--cocycles.  First, notice that only the additive group structure of the ring matters: the symmetric $2$--cocycle condition does not involve any ring multiplication.  Second, it suffices to show the Lemma over a finitely generated abelian group, as a particular polynomial has finitely many terms and hence involves finitely many coefficients.  Noticing that the Lemma is true for $A \oplus B$ if and only if it's true for $A$ and for $B$, we couple these facts to the structure theorem for finitely generated abelian groups to reduce to the cases $\Z$ and $\Z/p^r$.  From here, we can reduce to the prime fields: if $A \le B$ is a subgroup and the Lemma is true for $B$, it's true for $A$, so we will be able to deduce the case of $\Z$ from the case of $\Q$.  Lastly, we can also reduce from $\Z/p^r$ to $\Z/p$ using an inductive Bockstein-style argument over the extensions \[(p^{r-1}) / (p^r) \to \Z/p^r \to \Z/p^{r-1}\] and noticing that $(p^{r-1}) / (p^r) \cong \Z/p$ as abelian groups.  Hence, we can now freely assume that our ground object is a prime field.

We now ground ourselves by fitting symmetric $2$--cocycles into a more general homological framework, hoping that we can use such a thing to power a computation.  For a formal group scheme $\G$, we can form a simplicial scheme $B\G$ in the usual way:
\[B\G := \left\{
\begin{tikzcd}[ampersand replacement=\&]
\begin{array}{c} * \\ \times \\ * \end{array} \arrow{r} \arrow[leftarrow,shift left=\baselineskip]{r} \arrow[leftarrow,shift right=\baselineskip]{r} \&
\begin{array}{c} * \\ \times \\ \G \\ \times \\ * \end{array} \arrow[leftarrow, shift left=(2*\baselineskip)]{r} \arrow[shift left=\baselineskip]{r} \arrow[leftarrow]{r} \arrow[shift right=\baselineskip]{r} \arrow[leftarrow, shift right=(2*\baselineskip)]{r} \&
\begin{array}{c} * \\ \times \\ \G \\ \times \\ \G \\ \times \\ *\end{array} \arrow[leftarrow, shift left=(3*\baselineskip)]{r} \arrow[shift left=(2*\baselineskip)]{r} \arrow[leftarrow, shift left=\baselineskip]{r} \arrow{r} \arrow[leftarrow, shift right=\baselineskip]{r} \arrow[shift right=(2*\baselineskip)]{r} \arrow[leftarrow, shift right=(3*\baselineskip)]{r} \&
\cdots
\end{tikzcd}
\right\}.\]
By applying the functor $\InternalHom{FormalSchemes}(-, \G_a)(k)$, we get a cosimplicial abelian group stemming from the group scheme structure on $\G_a$, and this gives a cochain complex of which we can take the cohomology.  In the case $\G = \G_a$, the $2$--cocycles in this cochain complex are \emph{precisely} the things we've been calling $2$--cocycles\footnote{They aren't obligated to be symmetric or of homogeneous degree, though.}, so we are interested in computing $H^2$.  First, we can quickly compute $B^2$, since $C^1$ is so small: \[d^1(x^k) = d_k c_k.\]  Secondly, one may think of this complex as a resolution computing various\footnote{Refer back to \Cref{ExtAndCotorAgree}.} derived functors \[\Cotor_{\sheaf O_{\G}}(k, k) \cong \Ext_{\sheaf O_{\G}}(k, k) \cong \Tor_{\sheaf O_{\G}^*}(k, k).\]\todo{Do you use ${}^*$ or ${}^\vee$ later for linear dual?}  We are now going to compute these last groups using a more efficient complex.

\begin{itemize}
\item[$\Q$:] There is a free $\Q[t]$--module resolution
\begin{center}
\begin{tikzcd}
& \Q \\
0 & \Q[t] \arrow{u} \arrow{l} & \Q[t] \arrow["\cdot t"]{l} & 0 \arrow{l},
\end{tikzcd}
\end{center}
to which we apply $(-) \otimes_{\Q[t]} \Q$ to calculate \[H^* \InternalHom{FormalSchemes}(B\G_a, \G_a)(\Q) = \begin{cases} \Q & \text{when $* = 0$}, \\ \Q & \text{when $* = 1$}, \\ 0 & \text{otherwise}. \end{cases}\] This means that every $2$--cocycle is a coboundary, symmetric or not.
\item[$\F_p$:] Now we are computing $\Ext$ over a free commutative $\F_p$--algebra on one generator with divided powers.  Such an algebra splits as a tensor of truncated polynomial algebras, and again computing a minimal free resolution results in the calculation
\[H^* \InternalHom{FormalSchemes}(B\G_a, \G_a)(\F_p) =
\begin{cases}
\frac{\F_p[\alpha_k \mid k \ge 0]}{\alpha_k^2 = 0} \otimes \F_p[\beta_k \mid k \ge 0] & \text{when $p > 2$}, \\
\F_2[\alpha_k \mid k \ge 0] & \text{when $p = 2$},
\end{cases}\] with $\alpha_k \in H^1$ and $\beta_k \in H^2$.  Now that we know what to look for, we can find representatives of each of these classes:
\begin{itemize}
\item The class $\alpha_k$ can be represented by $x^{p^k}$, as this is a minimally divisible monomial of degree $p^k$ satisfying the $1$--cocycle condition \[x^{p^k} - (x+y)^{p^k} + y^{p^k} = 0.\]
\item The $2$--cohomology is concentrated in degrees of the form $p^k$ and $p^j + p^k$, corresponding to $\beta_k$ and $\alpha_j \alpha_k$.  Since $c_{p^k}$ is a $2$--cocycle of the correct degree and not a $2$--coboundary (cf.\ $d^1(x^{p^k}) = d_{p^k} c_{p^k}$, and $p \mid d_{p^k}$), we can use it as a representative for $\beta_k$.  (Additionally, the asymmetric class $\alpha_k \alpha_j$ is represented by $x^{p^k} y^{p^j}$.)
\item Similarly, in the case $p = 2$ the exceptional class $\alpha_{k-1}^2$ is represented by $c_{2^k}(x, y)$, as this is a $2$--cocycle in the correct degree which is not a $2$--coboundary.
\end{itemize}
Given how few $2$--coboundaries and $2$--cohomology classes there are, we conclude that $c_n(x, y)$ and $x^{p^a} y^{p^b}$ give a basis for \emph{all} of the $2$--cocycles.  Of these it is easy to select the symmetric ones, which agrees with our expected conclusion. \qedhere
\end{itemize}
\end{proof}

The most important consequence of \Cref{LazardsTheorem} is \emph{smoothness}:
\begin{corollary}\label{MfglIsSmooth}
Given a formal group law $F$ over a ring $R$ and a surjective ring map $f\co S \to R$, there exists a formal group law $\widetilde F$ over $S$ with \[F = f^* \widetilde F.\]
\end{corollary}
\begin{proof}
Identify $F$ with the classifying map $\Spec R \to \moduli{fgl}$.  Employ an isomorphism \[\phi\co \moduli{fgl} \to \Spec L\] afforded by \Cref{LazardsTheorem}, so that $\phi \circ F$ is selected by a sequence of elements $r_n = \phi^* F^*(t_n) \in R$.  Each of these admit preimages $s_n$ through $f$, and we determine a map \[\widetilde{\phi \circ F}\co \Spec S \to \Spec L\] by the formula $\widetilde{\phi \circ F}^* (t_j) = s_j$ and freeness of $L$.  Since $\phi$ is an isomorphism, this determines a map $\widetilde F = \phi^{-1} \circ \widetilde{\phi \circ F}$ factoring $F$.
\end{proof}

\todo{This isn't super well stated, but it's at least here to be smoothed out later.}
In order to employ \Cref{MfglIsSmooth} effectively, we need to know when a map $\Spec R \to \moduli{fg}$ classifying a formal group can be lifted to a triangle
\begin{center}
\begin{tikzcd}
& \moduli{fgl} \arrow{d} \\
\Spec R \arrow{r} \arrow[densely dotted]{ru} & \moduli{fg},
\end{tikzcd}
\end{center}
so that a surjective map of rings $\Spec R \to \Spec S$ can then be completed to a second diagram
\begin{center}
\begin{tikzcd}
\Spec S \arrow[densely dotted]{r} \arrow[densely dotted]{rd} & \moduli{fgl} \arrow{d} \\
\Spec R \arrow{r} \arrow[crossing over]{ru} \arrow{u} & \moduli{fg}.
\end{tikzcd}
\end{center}

\begin{lemma}[{\cite[Proposition 11.7]{LurieChromaticCourseNotes}}]\label{CoordinatizbleFGs}
A map $\G\co \Spec R \to \moduli{fg}$ lifts to $\moduli{fgl}$ exactly when the Lie algebra $T_0 \G$ of $\G$ is isomorphic to $R$.
\end{lemma}
\begin{proof}
Certainly if $\G$ admits a global coordinate, then $T_0 \G \cong R$.  Conversely, the formal group $\G$ is certainly locally isomorphic to $\A^1$ by a covering $i_\alpha\co X_\alpha \to \Spec R$ and isomorphisms $\phi_\alpha$---but, \textit{a priori}, these isomorphisms may not glue, precisely corresponding to the nontriviality of the {\Cech} $1$--cocycle \[[\phi_\alpha] \in \check{H}^1(\Spec R; \moduli{ps}^{\gpd}).\]  The group scheme $\moduli{ps}^{\gpd}$ is populated by $T$--points of the form \[\moduli{ps}^{\gpd}(T) = \left\{ t_0 x + t_1 x^2 + t_2 x^3 + \cdots \mid t_j \in T, t_0 \in T^\times \right\},\] and it is admits a filtration by the closed subschemes \[\moduli{ps}^{\gpd, \ge N}(T) = \left\{ 1 \cdot x + t_N x^{N+1} + t_{N+1} x^{N+2} + \cdots \mid t_j \in T \right\}.\]  The associated graded of this filtration is $\Gm \times \mathbb G_a^{\times \infty}$, and hence the filtration spectral sequence shows \[\check{H}^1(\Spec R; \moduli{ps}^{\gpd}) \xrightarrow{\simeq} \check{H}^1(\Spec R; \Gm),\] as $\check{H}^1(\Spec R; \mathbb G_a) = 0$ for all affine schemes.  Finally, given a choice of trivialization $T_0 \G \cong R$, this induces compatible trivializations of $T_0 i_\alpha^* \G$, which we can use to rescale the isomorphisms $\phi_\alpha$ so that their image in $\check{H}^1(\Spec R; \Gm)$ vanishes, and hence $[\phi_\alpha]$ is induced from a class in \[\check{H}^1(\Spec R; \moduli{ps}^{\gpd, \ge 1}).\]  This obstruction group vanishes.
\end{proof}

\begin{remark}
Incidentally, a choice of trivialization of $T_0 \G$ exactly resolves the indeterminacy of $\log'(0)$ in \Cref{RationalFGLsHaveLogarithms}.
\end{remark}

\begin{remark}
The subgroup scheme $\moduli{ps}^{\gpd, \ge 1}$ is often referred to in the literature as the group of \textit{strict isomorphisms}.  There is an associated moduli of formal groups identified only up to strict isomorphism, which sits in a fiber sequence \[\Gm \to \moduli{fgl} \mmod \moduli{ps}^{\gpd, \ge 1} \to \moduli{fg}.\]  These appeared earlier in this Lecture as well: in the proof of \Cref{LazardsTheorem}, we constructed over $L$ the universal formal group law equipped with a \emph{strict} exponential map.\todo{Give a comparison with $\context{MU}$ vs $\context{MUP}$.}
\end{remark}











\section{The structure of \texorpdfstring{$\moduli{fg}$}{Mfg} II: Large scales}\label{MfgII:LargeScales}

We now turn to understanding the geometry of the quotient stack $\moduli{fg}$ itself, armed with two important tools: \Cref{RationalFGLsHaveLogarithms} and \Cref{MfglIsSmooth}.  We begin with a rephrasing of the former:

\begin{theorem}[{cf.\ \Cref{RationalFGLsHaveLogarithms}}]\label{RationalGeometricPointsOfMfg}
Let $k$ be any field of characteristic $0$.  Then $\G_a$ describes a unique map
\[\pushQED{\qed}
\Spec k \xrightarrow{\simeq} \moduli{fg}. \qedhere
\popQED\]
\end{theorem}

One of our overarching tasks from the introduction to this Case Study is to enhance this to a classification of \emph{all} of the geometric points of $\moduli{fg}$, including those where $k$ is a field of positive characteristic $p$: \[\G\co \Spec k \to \moduli{fg} \times \Spec \Z_{(p)}.\]  We proved this Theorem in the characteristic $0$ case by solving a certain differential equation, which necessitated integrating a power series, and integration is what we expect to fail in characteristic $p$.  The following definition tracks \emph{where} it fails:
\begin{definition}
Let $+_\phi$ be a formal group law over a $\Z_{(p)}$--algebra.  Let $n$ be the largest degree such that there exists a formal power series $\ell$ with \[\ell(x +_\phi y) = \ell(x) + \ell(y) \pmod{(x, y)^{n}},\] i.e., $\ell$ is a logarithm for the $(n-1)$--bud determined by $+_\phi$.  The \textit{$p$--height of $+_\phi$} is defined to be $\log_p(n)$.
\end{definition}

This turns out to be a crucial invariant of a formal group law, admitting many other interesting presentations.  In this Lecture, investigation of this definition will lead us to a classification of the closed substacks of $\moduli{fg}$, another of our overarching tasks.  As a first step, we would like to show that this value is well-behaved in various senses, including the following:
\begin{lemma}[{cf.\ \cite[Proposition 13.6]{LurieChromaticCourseNotes}}]\label{FGLHeightIsAnInteger}
Over a field of positive characteristic $p$, the $p$--height of a formal group law is always an integer (or $\infty$).  (That is, the radius of convergence of the logarithmic differential equation is either $\infty$ or $p^d$ for some nonnegative natural $d$.)
\end{lemma}

\noindent We will have to develop some machinery to get there.  First, we note that this definition really depends on the formal group rather than the formal group law.

\begin{lemma}\label{HeightIsAnIsomInvariant}
The height of a formal group law is an isomorphism invariant, i.e., it descends to give a function \[\height\co \pi_0 \moduli{fg}(T) \to \N \cup \{\infty\}\] for any test $\Z_{(p)}$--algebra $T$.
\end{lemma}
\begin{proof}
The series $\ell$ is a partial logarithm for the formal group law $\phi$, i.e., an isomorphism between the formal group defined by $\phi$ and the additive group.  Since isomorphisms compose, this statement follows.
\end{proof}

With this in mind, we look for a more standard form for formal group laws, where \Cref{FGLHeightIsAnInteger} will hopefully be obvious.  The most blindly optimistic standard form is as follows:
\begin{definition}[{cf.\ \cite[Proposition 15.2.4]{Hazewinkel}}]\label{DefnpTypicalLog}
Suppose that a formal group law $+_\phi$ does have a logarithm.  We say that its logarithm is \textit{$p$--typical} when it takes the form \[\log_\phi(x) = \sum_{j=0}^\infty \ell_j x^{p^j}.\]
\end{definition}

\begin{lemma}[{\cite[Theorem 15.2.9]{Hazewinkel}}]\label{EveryLogHaspTypification}
Every formal group law $+_\phi$ over a $\Z_{(p)}$--algebra with a logarithm $\log_\phi$ is naturally isomorphic to one whose logarithm is $p$--typical, called the \textit{$p$--typification} of $+_\phi$.
\end{lemma}
\begin{proof}
Let $\G$ be the formal group associated to $+_\phi$, and denote its inherited parameter by \[g_0\co \A^1 \xrightarrow{\cong} \G,\] so that the composite \[\A^1 \xrightarrow{g_0} \G \xrightarrow{\log} \G_a \xrightarrow{x} \A^1\] expresses $\log_\phi = \log \circ g_0$ as the power series \[\log_\phi(x) = \sum_{n=1}^\infty a_n x^n.\]  Our goal is to perturb this coordinate to a new coordinate $g_\infty$ which couples with the logarithm in the same way to give a series expansion of the form \[\log(g_\infty(x)) = \sum_{n=0}^\infty a_{p^n} x^{p^n}.\]  To do this, we introduce four operators on functions\footnote{Unfortunately, it is standard in the literature to call these operators on ``curves'', which does not fit well with our previous use of the term in \Cref{ComplexBordismChapter}.} $\A^1 \to \G$:
\begin{itemize}
\item Given $r \in R$, we can define a \textit{homothety} by rescaling the coordinate by $r$: \[\log(\theta_r g_0) = \log(g_0(rx)) = \sum_{n=1}^\infty (a_n r^n) x^n.\]
\item For $\ell \in \Z$, we can define a shift operator (or \textit{Verschiebung}) by \[\log(V_\ell g_0(x)) = \log(g(x^\ell)) = \sum_{n=1}^\infty a_n x^{n \ell}.\]
\item Given an $\ell \in \Z_{(p)}$, we define the \textit{$\ell$--series} by\footnote{Note that for $\ell \in \Z$, this agrees with $[\ell](g_0(x)) = \overset{\text{$\ell$ times}}{\overbrace{g_0(x) +_{\G} \cdots +_{\G} g_0(x)}}$.} \[\log([\ell](g_0(x))) = \ell \log(g_0(x)) = \sum_{n=1}^\infty \ell a_n x^n.\]
\item For $\ell \in \Z$, we can define a \textit{Frobenius operator}\footnote{There are other definitions of the Frobenius operator which are less mysterious but less explicit.  For instance, it also arises from applying the Verschiebung to the character group (or ``Cartier dual'') of $\G$.} by \[\log(F_\ell g_0(x)) = \log\left(\sum_{j=1}^\ell{}_{\G} g_0(\zeta_\ell^j x^{1/\ell}) \right),\] where $\zeta_\ell$ is a primitive $\ell${\th} root of unity.  Because this formula is Galois--invariant in choice of primitive root, it actually expands to a series which lies over the ground ring (without requiring an extension by $\zeta_\ell$).  But, by pulling the logarithm through and noting \[\sum_{j=1}^\ell \zeta_\ell^{jn} = \begin{cases}\ell & \text{if $\ell \mid n$}, \\ 0 & \text{otherwise}, \end{cases}\] we can explicitly compute the behavior of $F_\ell$: \[\log(F_\ell g_0(x)) = \sum_{n=1}^\infty \ell a_{n \ell} x^n.\]
\end{itemize}
Stringing these together, for $p \nmid \ell$ we have \[\log([1/\ell] V_\ell F_\ell g_0(x)) = \sum_{n=1}^\infty a_{n \ell} x^{n \ell}.\]  Hence, we can iterate over primes $\ell \ne p$, and for two adjacent such primes $\ell' > \ell$ we consider the perturbation \[g_{\ell'} = g_\ell -_{\G} [1/\ell] V_\ell F_\ell g_\ell.\]  Each of these differences gives a parameter according to \Cref{InverseFunctionTheoremForFVars}, and the first possible nonzero term appears in degree $\ell$, hence the coefficients stabilize linearly in $\ell$.  Passing to the limit thus gives a new parameter $g_\infty$ on the same formal group $\G$, but now with a $p$--typical logarithm.
\end{proof}

Of course, the whole idea of ``height'' is that not every formal group law supports a logarithm.  Because of this, we would like to re-express $p$--typicality in more general terms.  Our foothold for this is the following computation of the $p$--series of a formal group law with $p$--typical logarithm:

\begin{lemma}[{\cite[Section 4]{Araki}}]\label{pTypLogGivesNicePSeries}
For a formal group $+_\phi$ with a logarithm $\log_\phi$, the logarithm is $p$--typical if and only if there are elements $v_d$ with \[[p]_\phi(x) = px +_\phi v_1 x^p +_\phi v_2 x^{p^2} +_\phi \cdots +_\phi v_d x^{p^d} +_\phi \cdots.\]
\end{lemma}
\begin{proof}[Proof sketch]
Suppose first that $\log_\phi$ is $p$--typical.  We can then compare the two series
\begin{align*}
\log_\phi(px) & = px + \cdots, \\
\log_\phi([p]_\phi(x)) & = p \log_\phi(x) = px + \cdots.
\end{align*}
The difference is concentrated in degrees of the form $p^d$, beginning in degree $p$, so we can find an element $v_1$ such that \[p \log_\phi(x) - (\log_\phi(px) + \log_\phi(v_1 x^p))\] is also concentrated in degrees of the form $p^d$ but now starts in degree $p^2$.  Iterating this gives the equation
\begin{align*}
p \log_\phi(x) & = \log_\phi (px) + \log_\phi(v_1 x^p) + \log_\phi(v_2 x^{p^2}) + \cdots, \\
\intertext{at which point we can use formal properties of the logarithm to deduce}
\log_\phi [p]_\phi(x) & = \log_\phi \left(px +_\phi v_1 x^p +_\phi v_2 x^{p^2} +_\phi \cdots +_\phi v_n x^{p^n} +_\phi \cdots\right), \\
[p]_\phi(x) & = px +_\phi v_1 x^p +_\phi v_2 x^{p^2} +_\phi \cdots +_\phi v_n x^{p^n} +_\phi \cdots.
\end{align*}
In the other direction, the logarithm coefficients can be recursively recovered from the coefficients $v_d$ for a formal group law with $p$--typical $p$--series, using a similar manipulation:
\begin{align*}
p \log_\phi(x) & = \log_\phi\left([p]_\phi(x)\right) \\
p \sum_{n=0}^\infty m_n x^n & = \log_\phi \left(\sum_{d=0}^\infty{}_\phi v_d x^{p^d} \right) = \log_\phi(px) + \sum_{d=1}^\infty \log_\phi\left(v_d x^{p^d}\right), \\
\intertext{which is only soluable if $\log_\phi$ is concentrated in degrees of the form $p^d$.  In that case, we can push this slightly further:}
\sum_{d=0}^\infty p m_{p^d} x^{p^d} & = \sum_{d=0}^\infty \sum_{j=0}^\infty m_{p^j} v_d^{p^j} x^{p^{d+j}} = \sum_{n=0}^\infty \left( \sum_{k=0}^n m_{p^k} v_{n-k}^{p^k} \right) x^{p^n},
\end{align*}
implicitly taking $m_1 = 1$ and $v_0 = p$.
\end{proof}

This result portends much of what is to come.  We now set our definition of $p$--typical to correspond to the manipulations we were making in the course of proving \Cref{EveryLogHaspTypification}.

\begin{definition}\label{pTypicalityInGeneral}
A parameter $g\co \A^1 \to \G$ of a formal group is said to be \textit{$p$--typical} when $F_\ell g = 0$ for all $p \nmid \ell$.
\end{definition}

\begin{corollary}[{cf.\ \Cref{EveryLogHaspTypification}}]\label{EveryFGLIsPTypical}
Every formal group law $+_\phi$ is naturally isomorphic to a $p$--typical one. \qed
\end{corollary}

\begin{lemma}[{\cite[Section 4]{Araki}, cf.\ \Cref{pTypLogGivesNicePSeries}}]\label{pTypLawsHaveNicePSeries}
If $+_\phi$ is a $p$--typical formal group law, then there are elements $v_d$ with \[[p]_\phi(x) = px +_\phi v_1 x^p +_\phi v_2 x^{p^2} +_\phi \cdots +_\phi v_d x^{p^d} +_\phi \cdots.\]
\end{lemma}
\begin{proof}
As before, let $\G$ denote the formal group associated to $+_\phi$ and let $g\co \A^1 \to G$ denote the induced $p$--typical coordinate.  Any auxiliary function $h\co \A^1 \to \G$ can be expressed in the form \[h = \sum_{m = 0}^\infty {}_{\G} V_m \theta_{a_m} g.\]  We will show that if $h$ is $p$--typical (i.e., $F_\ell h = 0$ for $p \nmid \ell$) then $a_m = 0$ for every $m \ne p^d$.\footnote{The converse so this claim also holds: since $F_\ell F_p = F_p F_\ell$ for $p \nmid \ell$, we can commute $F_\ell$ through the sum expression (which is absent any non-commuting terms by hypothesis), where it then kills $g$ to give $F_\ell h = 0$.}  Suppose instead that we can find a smallest index $m = rp^d$ with $p \nmid r$, $r \ne 1$, and $a_m \ne 0$.  We can then write
\begin{align*}
F_\ell \left( h -_{\G} \sum_{j=0}^{d} {}_{\G} V_{p^j} \theta_{a_{p^j}} g \right) & = F_\ell(V_m \theta_{a_m} g + \cdots) \\
& = r V_{p^d} \theta_{a_m} g + \cdots \ne 0.
\end{align*}
Since $p$--typical curves are closed under difference, $h$ could not have been $p$--typical.

Finally, we specialize to the case $h = [p]_{\G}(g)$.  Since $F_\ell$ and $[p]$ commute, $[p]$ is $p$--typical, hence has an expression of the desired form.
\end{proof}

\begin{proof}[{Proof of \Cref{FGLHeightIsAnInteger}}]
Replace the formal group law by its $p$--typification.  Using the formulas from \Cref{pTypLogGivesNicePSeries}, we see that the height of a $p$--typical formal group law over a field of characteristic $p$ coincides with the appearance of the first nonzero coefficient in its $p$--series.
\end{proof}

\Cref{pTypLogGivesNicePSeries} shows that the $p$--series of a formal group law with $p$--typical logarithm contains exactly as much information as the logarithm itself (and hence fully determines the formal group law).  We would again like to show that ``all'' of the data of a $p$--typical group law is found in its $p$--series, even if it does not have a logarithm to mediate the two.  The following important theorem makes this thought precise.

\begin{theorem}[{cf.\ \cite[Proposition 5.1]{MillerNotesCobordism}, \cite[Theorem A2.2.3]{RavenelGreenBook}, and the proof of \cite[Proposition 19.10]{HopkinsCOCTALOS}}]\label{KudoArakiIsomorphism}
The \textit{Kudo--Araki map} determined by \Cref{pTypLawsHaveNicePSeries} \[\Z_{(p)}[v_1, v_2, \ldots, v_d, \ldots] \xrightarrow{v} \sheaf O_{\moduli{fgl}^{\ptyp}}\] is an isomorphism.
\end{theorem}
\begin{proof}
\todo{Is there a version of this proof that doesn't make a rational argument?}
Begin with a universal group law over the ring $\sheaf O_{\moduli{fgl}}$.  This group law $p$--typifies by \Cref{EveryFGLIsPTypical} to a second group law which is selected by a map $\eps\co \sheaf O_{\moduli{fgl}} \to \sheaf O_{\moduli{fgl}}$.  The following diagram includes the image factorization of $\eps$, as well as its rationalization and the map $v$:
\begin{center}
\begin{tikzcd}[column sep=0.2em]
& \sheaf O_{\moduli{fgl}} \arrow["s"]{rdd} \arrow["\eps"]{rrd} \arrow{rrr} & & & \sheaf O_{\moduli{fgl}} \otimes \Q \arrow["\eps"]{rrd} \arrow["s" near end]{rdd} \\
& & & \sheaf O_{\moduli{fgl}} \arrow[crossing over]{rrr} & & & \sheaf O_{\moduli{fgl}} \otimes \Q \\
\Z_{(p)}[v_1, \ldots, v_d, \ldots] \arrow["v"]{rr} & & \sheaf O_{\moduli{fgl}^{\ptyp}} \arrow["i"]{ru} \arrow{rrr} & & & \sheaf O_{\moduli{fgl}^{\ptyp}} \otimes \Q \arrow["i"]{ru}.
\end{tikzcd}
\end{center}
We immediately deduce that all the horizontal arrows are injections: in \Cref{LazardsTheorem} we calculated $\sheaf O_{\moduli{fgl}}$ to be torsion-free; $\sheaf O_{\moduli{fgl}^{\ptyp}}$ is a subring of $\sheaf O_{\moduli{fgl}}$, hence it is also torsion-free; and \Cref{pTypLogGivesNicePSeries} shows that $(i \circ v)(v_n)$ agrees with $pm_{p^n}$ in the module of indecomposables $Q(\sheaf O_{\moduli{fgl}} \otimes \Q)$.

\todo{This is muddy.}
To complete the proof, we need to show that $v$ is surjective, which will follow from calculating the indecomposables in $\sheaf O_{\moduli{fgl}^{\ptyp}}$ and checking that $Qv$ is surjective.  Since $s$ is surjective, the map $Qs$ on indecomposables is surjective as well, and its effect can largely be calculated rationally.  Since $(Q\eps)(m_n) = 0$ for $n \ne p^d$, we have that $Q(\sheaf O_{\moduli{fgl}^{\ptyp}})$ is generated by $s(b_{p^d-1})$ under an isomorphism as in \Cref{LazardsTheorem}.  It follows that $Qi$ injects, hence $Qv$ must surject by the calculation of $Q(i \circ v)(v_n)$ above.
\end{proof}

\begin{corollary}\label{PSeriesDetermines}
If $[p]_\phi(x) = [p]_\phi(x)$ for two $p$--typical formal group laws $+_\phi$ and $+_\psi$, then $+_\phi$ and $+_\psi$ are themselves equal. \qed
\end{corollary}

\begin{corollary}\label{EveryPSeriesArises}
For any sequence of coefficients $v_j \in R$ in a $\Z_{(p)}$--algebra $R$, there is a unique $p$--typical formal group law $+_\phi$ with
\[\pushQED{\qed}
[p]_\phi = px +_\phi v_1 x^p +_\phi v_2 x^{p^2} +_\phi \cdots +_\phi v_d x^{p^d} +_\phi \cdots. \qedhere
\popQED\]
\end{corollary}

Finally, we exploit these results to make deductions about the geometry of $\moduli{fg} \times \Spec \Z_{(p)}$.  There is an inclusion of groupoid--valued sheaves from $p$--typical formal group laws with $p$--typical isomorphisms to all formal group laws with all isomorphisms.  \Cref{EveryFGLIsPTypical} can be viewed as presenting this inclusion as a deformation retraction, witnessing a natural \emph{equivalence} of groupoids.  It follows from \Cref{WarningAboutStacks} that they both present the same stack.  The central utility of this equivalence is that the Kudo--Araki moduli of $p$--typical formal group laws is a considerably smaller algebra than $\sheaf O_{\moduli{fgl}}$, resulting in a less noisy picture of the Hopf algebroid.

Our final goal in this Lecture is to exploit this refined presentation in the study of invariant functions.
\begin{definition}[{\cite[Lemma 2.28]{GoerssQCohOnMfg}}]\label{DefnClosedSubstack}
Let $(X_0, X_1)$ be the groupoid scheme associated to a Hopf algebroid $(A, \Gamma)$.  A function $f\co X_0 \to \mathbb A^1$ is said to be \textit{invariant} when it is stable under isomorphism, i.e., when there is a diagram
\begin{center}
\begin{tikzcd}
X_1 \arrow[shift left=0.3em, "t"]{d} \arrow[shift right=0.3em, "s"']{d} \arrow["s^* f = t^* f"]{rd} \\
X_0 \arrow["f"]{r} & \mathbb A^1.
\end{tikzcd}
\end{center}
(In terms of Hopf algebroids, the corresponding element $a \in A$ satisfies $\eta_L(a) = \eta_R(a)$.)  Correspondingly, a closed subscheme $A \subseteq X_0$ determined by the simultaneous vanishing of functions $f_\alpha$ is said to be \textit{invariant} when the vanishing condition is invariant---i.e., a point lies in the simultaneous vanishing locus if and only if its entire orbit under $X_1$ also lies in the simultaneous vanishing locus.  (In terms of Hopf algebroids, the corresponding ideal $I \subseteq A$ satisfies $\eta_L(I) = \eta_R(I)$.)  Finally, a \textit{closed substack} is a substack determined by an invariant ideal of $X_0$.
\todo{At some point you really should talk about open complements (of affine schemes and then of affinely presented stacks).}
\end{definition}

We are in a good position to discern all of the closed substacks of $\moduli{fg} \times \Spec \Z_{(p)}$---or, equivalently, to discern all of the invariant ideals of $\sheaf O_{\moduli{fgl}^{\ptyp}}$.

\begin{corollary}[{\cite[Theorem 4.6 and Lemmas 4.7-8]{Wilson}}]\label{IdIsAnInvariantIdeal}
The ideal $I_d = (p, v_1, \ldots, v_{d-1})$ is invariant under the action of \emph{strict} formal group law isomorphisms for all $d$.  It determines the closed substack $\moduli{fg}^{\ge d}$ of formal group laws of $p$--height at least $d$.
\end{corollary}
\begin{proof}
\todo{\textit{Put a dictionary entry here defining the submoduli of height $\ge d$ formal group laws.}}
Recall from \Cref{KudoArakiIsomorphism} the Kudo--Araki isomorphism \[\moduli{fgl}^{\ptyp} \xrightarrow{\simeq} \Spec \Z_{(p)}[v_1, v_2, \ldots, v_d, \ldots] =: \Spec V,\] and let $+_L$ denote the associated universal $p$--typical formal group law with $p$--series \[[p]_L(x) = px +_L v_1 x^p +_L v_2 x^{p^2} +_L \cdots +_L v_d x^{p^d} +_L \cdots.\]  Over $\Spec V[t_1, t_2, \ldots]$, we can form a second group law $+_R$ by conjugating $+_L$ by the universal $p$--typical coordinate transformation $g(x) = \sum_{j=0}^\infty {}_L t_j x^{p^j}$.  The corresponding $p$--series \[[p]_R(x) = \sum_{d=0}^\infty {}_R \eta_R(v_d) x^{p^d}\] determines the $\eta_R$ map of the Hopf algebroid $(V, V[t_1, t_2, \ldots])$ presenting the moduli of $p$--typical formal group laws and $p$--typical isomorphisms.  We cannot hope to compute $\eta_R(v_d)$ explicitly, but modulo $p$ we can apply Freshman's Dream to the expansion of \[[p]_L(g(x)) = g([p]_R(x))\] to discern some information: \[\sum_{\substack{i \ge 0 \\ j > 0}}{}_L t_i \eta_R(v_j)^{p^i} \equiv \sum_{\substack{i > 0 \\ j \ge 0}}{}_L v_i t_j^{p^i} \pmod p.\]  This is still inexplicit, since $+_L$ is a very complicated operation, but we can see $\eta_R(v_d) \equiv v_d \pmod{I_d}$.  It follows that $I_d$ is invariant for each $d$.  Additionally, the closed substack this determines are those formal groups admitting local $p$--typical coordinates for which $v_{\le d} = 0$, guaranteeing that the height of the associated formal group is at least $d$.
\end{proof}

\noindent What is \emph{much} harder to prove is the following:

\begin{theorem}[{\cite[Corollary 2.4 and Proposition 2.5]{LandweberInvariantRegIdeals}, cf.\ \cite[Theorem 4.9]{Wilson}}]\label{LandwebersClassificationOfClosedSubstacks}
The unique closed reduced substack of $\moduli{fg} \times \Spec \Z_{(p)}$ of codimension $d$ is selected by the invariant prime ideal $I_d \subseteq \sheaf O_{\moduli{fgl}^{\ptyp}}$.
\end{theorem}
\begin{proof}[Proof sketch]
We want to show that if $I$ is an invariant prime ideal, then $I = I_d$ for some $d$.  To begin, note that $v_0 = p$ is the only invariant function on $\moduli{fgl}^{\ptyp}$, hence $I$ must either be trivial or contain $p$.  Then, inductively assume that $I_d \subseteq I$.  If this is not an equality, we want to show that $I_{d+1} \subseteq I$ is forced.  Take $y \in I \setminus I_d$; if we could show \[\eta_R(y) = a v_d^j t^K + \text{higher order terms}\] for nonzero $a \in \Z_{(p)}$, we could proceed by primality to show that $v_d \in I$ and hence $I_{d+1} \subseteq I$.  This is possible (and, indeed, this is how the full proof goes), but it requires serious bookkeeping.
\end{proof}

\begin{remark}\label{OpenSubstacksOfMfg}
The complementary open substack of dimension $d$ is harder to describe.  From first principles, we can say only that it is the locus where the coordinate functions $p$, $v_1$, \ldots, $v_d$ do not \emph{all simultaneously vanish}.  It turns out that:
\begin{enumerate}
\item On a cover, at least one of these coordinates can be taken to be invertible.\item Once one of them is invertible, a coordinate change on the formal group law can be used to make $v_d$ (and perhaps others in the list) invertible.  Hence, we can use $v_d^{-1} \sheaf O_{\moduli{fgl}^{\ptyp}}$ as a coordinate chart.
\item Over a further base extension and a further coordinate change, the higher coefficients $v_{d+k}$ can be taken to be zero.  Hence, we can also use $v_d^{-1} \Z_{(p)}[v_1, \ldots, v_d]$ as a coordinate chart.
\end{enumerate}
\end{remark}

\begin{remark}[{cf.\ \cite[Section 12]{StricklandFGNotes} and \cite[Remark 13.9]{LurieChromaticCourseNotes}}]
Specialize now to the case of a field $k$ of characteristic $p$.  Since the additive group law has vanishing $p$--series and is $p$--typical, a consequence of \Cref{PSeriesDetermines} is that \emph{every} $p$--typical group law with vanishing $p$--series is exactly equal to $\G_a$, and in fact any formal group law with vanishing $p$--series $p$--typifies exactly to $\G_a$.  This connects several ideas we have seen so far: the presentation of formal group laws with logarithms in \Cref{CalculationOfAutGaActionOnMO}, the presentation of the context $\context{MOP}$ in \Cref{ContextOfMOPExample}, and the Hurewicz image of $MU_*$ in $H\F_p{}_* MU$ in \Cref{HZMUCarriesALog}.
\end{remark}

\begin{remark}
It's worth pointing out how strange all of this is. In Euclidean geometry, open subspaces are always top-dimensional, and closed subspaces can drop dimension.  Here, proper open substacks of every dimension appear, and every nonempty closed substack is $\infty$--dimensional (albeit of positive codimension).
\end{remark}

\begin{remark}
The results of this section have several alternative forms in the literature.  For instance, $[p]_\phi(x)$ can also be expressed as \[[p]_\phi(x) = px + v_1 x^p + v_2 x^{p^2} + \cdots + v_d x^{p^d} + \cdots,\] and this also determines a presentation of $\sheaf O_{\moduli{fgl}^{\ptyp}}$.  These other elements $v_d$, called \textit{Hazewinkel coordinates}, differ substantially from the Kudo--Araki coordinates favored here, although they are equally ``canonical''.  Different coordinate patches are useful for accomplishing different tasks, and the reader would be wise to remain flexible.
\todo{The end of this remark could be beefed up by mentioning some of the stuff around Question 100 in the $E$--theory seminar notes.}
\end{remark}

\begin{remark}[{\cite[Section 17.5]{Hazewinkel}}]\label{ArtinHasseExponential}
The $p$--typification operation often gives ``unusual'' results.  For instance, we will examine the standard multiplicative formal group law of \Cref{GmAndItsLogExample}, its rational logarithm, and its rational exponential:
\begin{align*}
x +_{\G_m^{\mathrm{std}}} y & = x + y - xy, &
\log_{\G_m^{\mathrm{std}}}(x) & = -\log(1 - x), &
\exp_{\G_m^{\mathrm{std}}}(x) & = 1 - \exp(-x).
\end{align*}
By \Cref{EveryLogHaspTypification}, we see that the $p$--typification of this rational logarithm takes the form \[\log_{\G_m^{\ptyp}}(x) = \sum_{j=0}^\infty \frac{x^{p^j}}{p^j}.\]  We can couple this to the standard exponential of the rational multiplicative group
\begin{center}
\begin{tikzcd}
\A^1 \arrow["\eps x"]{r} \arrow[bend left, "\log^{\ptyp}"]{rr} & \G_m \arrow["\log"]{r} & \G_a \arrow["\exp"]{r} \arrow[bend left,"\exp^{\mathrm{std}}"]{rr} & \G_m \arrow["x"]{r} & \A^1
\end{tikzcd}
\end{center}
to produce the coordinate change from \Cref{EveryFGLIsPTypical}: \[1 - \exp \left( -\sum_{j=0}^\infty \frac{x^{p^j}}{p^j} \right) = 1 - E_p(-x).\]  This series $E_p(x)$ is known as the \textit{Artin--Hasse exponential}, and it has the miraculous property that it is a series lying in $\Z_{(p)}\ps{x} \subseteq \Q\ps{x}$, as it is a change of coordinate series on $\G_m$ over $\Spec \Z_{(p)}$.
\end{remark}

\todo{I'm not sure where this goes---maybe it goes right here---but Allen told me how to give a nice proof of this fact.  He didn't give me a citation for this, but the main point was that flatness can be checked locally, so the general formula for the pullback $\mathcal M_{(A_1, \Gamma_1)} \to \mathcal M_{(L, W)} \from \mathcal M_{(A_2, \Gamma_2)}$ being $\mathcal M_{(A_1 \otimes W \otimes A_2, \Gamma_1 \otimes W \otimes \Gamma_2)}$ specializes to compute the pullback of $\Spec A \to \mathcal M_{(L, W)} \from \Spec E(d)_*$ to be $\mathcal M_{(A \otimes W \otimes E(d)_*, A \otimes W \otimes E(d)_*)} = \Spec (A \otimes W \otimes E(d)_*)$.  But the map $\Spec A \to \mathcal M_{(L, W)}$ factors through $\Spec L$, so it suffices just to check flatness for $\Spec L \otimes_L W \otimes_L E(d)_* \cong \Spec W \otimes_L E(d)_* \to \Spec E(d)_*$, which you finally do by hand.}
\todo{Also, people seem to say things about the Mischenko logarithm rather than the invariant differential, but I wonder if we should phrase things in those terms.}






\section{The structure of \texorpdfstring{$\moduli{fg}$}{Mfg} III: Small scales}\label{SectionMfgSmallScales}

In the previous two Lectures, we analyzed the structure of $\moduli{fg}$ as a whole: first we studied the cover \[\moduli{fgl} \to \moduli{fg},\] and then we turned to the stratification described by the height function \[\height\co \pi_0 \moduli{fg}(\text{$T$ a $\Z_{(p)}$--algebra}) \to \N \cup \{\infty\}.\]  In this Lecture, we will concern ourselves with the small scale behaviors of $\moduli{fg}$: its geometric points and their local neighborhoods.  To begin, we have all the tools in place to perform an outright classification of the geometric points.

\begin{theorem}[{\cite[Th\'eor\`eme IV]{Lazard}}]\label{FGpsOverAlgClosedFields}
Let $\bar k$ be an algebraically closed field of positive characteristic $p$.  The height map \[\height\co \pi_0 \moduli{fg}(\bar k) \to \N_{> 0} \cup \{\infty\}\] is a bijection.
\end{theorem}
\begin{proof}
Surjectivity follows from \Cref{EveryPSeriesArises}.  Namely, the \textit{$d${\th} Honda formal group law} is the $p$--typical formal group law over $k$ determined by \[[p]_{\phi_d}(x) = x^{p^d},\] and it gives a preimage for $d$.  To show injectivity, we must show that every $p$--typical formal group law $\phi$ over $\bar k$ is isomorphic to the appropriate Honda group law.  Suppose that the $p$--series for $\phi$ begins \[[p]_\phi(x) = x^{p^d} +_\phi a x^{p^{d+k}} + \cdots.\]  Then, we will construct a coordinate transformation $g(x) = \sum_{j=0}^\infty{}_\phi b_j x^{p^j}$ satisfying
\begin{align*}
g(x^{p^d}) & \equiv [p]_\phi(g(x)) & \pmod{x^{p^{d+k} + 1}} \\
\sum_{j=0}^\infty {}_\phi b_j x^{p^{d+j}} & \equiv \left( \sum_{j=0}^\infty {}_\phi b_j x^{p^j} \right)^{p^d} +_\phi a \left( \sum_{j=0}^\infty {}_\phi b_j x^{p^j} \right)^{p^{d+k}} & \pmod{x^{p^{d+k} + 1}} \\
\sum_{j=0}^\infty {}_\phi b_j x^{p^{d+j}} & \equiv \left( \sum_{j=0}^\infty {}_{(\operatorname{Frob}^d)^* \phi} b_j^{p^d} x^{p^{d+j}} \right) + a x^{p^{d+k}} & \pmod{x^{p^{d+k} + 1}}.
\end{align*}
For $g$ to be a coordinate transformation, we must have $b_0 = 1$, and because $\bar k$ is algebraically closed we can induct on $j$ to solve for the other coefficients in the series.  The coordinate for $\phi$ can thus be perturbed so that the term $x^{p^{d+k}}$ does not appear in the $p$--series, and inducting on $d$ gives the result.
\end{proof}

\begin{remark}[{\cite[Remark 11.2]{StricklandFGNotes}}]
We can now see see that $\pi_0 \moduli{fg}$, sometimes called the \textit{coarse moduli of formal groups}, is not representable by a scheme.  From \Cref{FGpsOverAlgClosedFields}, we see that there are infinitely many points in $\pi_0 \moduli{fg}(\F_p)$.  From \Cref{MfglIsSmooth}, we see that these lift along the surjection $\Z \to \F_p$ to give infinitely many distinct points in $\pi_0 \moduli{fg}(\Z)$.  On the other hand, by \Cref{RationalGeometricPointsOfMfg} there is a single $\Q$--point of the coarse moduli, whereas the $\Z$--points of a representable functor would inject into its $\Q$--points.
\end{remark}

We now turn to understanding the infinitesimal neighborhoods of these geometric points.  In general, for $p\co \Spec k \to X$ a closed $k$--point of a scheme, we defined in \Cref{JetSpacesDefn} and \Cref{DefnCompletion} an infinitesimal neighborhood object $X^\wedge_p$ with a lifting property
\begin{center}
\begin{tikzcd}
\Spec k \arrow["p"]{r} \arrow{d} & X^\wedge_p \arrow{d} \\
\Spf R \arrow{r} \arrow[densely dotted]{ru} & X
\end{tikzcd}
\end{center}
for any infinitesimal thickening $\Spf R$ of $\Spec k$.  Thinking of $X$ as representing a moduli problem, a typical choice for $\Spf R$ is $\A^1_k$, and a map $\A^1_k \to X$ extending $p$ gives a series solution to the moduli problem which specializes at the origin to $p$.  In turn, $X^\wedge_p$ is the smallest object through which all such maps factor, and so we think of it as classifying Taylor expansions of solutions passing through $p$.

For a formal group $\Gamma\co \Spec k \to \moduli{fg}$, the definition is formally similar, but actually writing it out is made complicated by \Cref{WarningAboutStacks}.  In particular, $p\co \Spec k \to X$ may not lift directly through $\Spf R \to X$, but instead $\Spec R/\m \to X$ may present $p$ on a cover $i\co \Spec R/\m \to \Spec k$.

\begin{definition}[{\cite[Section 2.4]{RezkFelixKlein}, cf.\ \cite[Section 6]{StricklandFiniteSubgps}}]\label{LubinTateDefn}
Define $(\moduli{fg})^\wedge_\Gamma$, the \textit{Lubin--Tate stack}, to be the groupoid-valued functor which on an infinitesimal thickening $R$ of $k$ has objects
\begin{center}
\begin{tikzcd}[row sep=4em, column sep=3em]
& \moduli{fg} \\
\Spec k
\arrow["\Gamma", bend left]{ru}
&
\Spec R/\m
\arrow[bend left]{u}[name=L,label=left:$i^* \Gamma$]{}
\arrow[bend right]{u}[name=R,label=right:$\pi^* \G$]{}
\arrow[shorten <=6pt, Rightarrow, to path={(L) -- node[label=above:$\alpha$] {} (R)}]{}
\arrow["i"']{l}
\arrow["\pi"]{r}
&
\Spf R
\arrow["\G"', bend right]{lu}
,
\end{tikzcd}
\end{center}
where $i$ is an inclusion of $k$ into the residue field $R/\m$ and $\alpha\co i^* \Gamma \to \pi^* \G$ is an isomorphism of formal groups.  The morphisms in the groupoid are maps $f\co \G \to \G'$ of formal groups over $\Spf R$ covering the identity on $i^* \Gamma$, called \textit{$\star$--isomorphisms}.
\end{definition}

\begin{remark}[{cf.\ \cite[Section 4.1]{RezkNotesOnHMThm}}]\label{LubinTateStackInFGLTerms}
The local formal group $\Gamma\co \Spec k \to \moduli{fg}$ always has trivializable Lie algebra, hence \Cref{CoordinatizbleFGs} shows that it always admits a presentation by a formal group law.  In fact, any deformation $\G\co \Spf R \to \moduli{fg}$ of $\Gamma$ also has a trivializable Lie algebra, since projective modules (such as $T_0 \G$) over local rings like $R$ are automatically free (i.e., trivializable).  It follows that the groupoid $(\moduli{fg})^\wedge_\Gamma(R)$ admits a presentation in terms of formal group \emph{laws}.  Starting with the pullback square of groupoids
\begin{center}
\begin{tikzcd}
(\moduli{fg})^\wedge_\Gamma(R) \arrow{d} \arrow{rr} & & \moduli{fg}(B) \arrow{d} \\
\displaystyle\coprod_{i\co \Spec R/\m \to \Spec k} \{\Gamma\} \arrow{r} & \displaystyle\coprod_{i\co \Spec R/\m \to \Spec k} \moduli{fg}(k) \arrow{r} & \moduli{fg}(R/\m)
\end{tikzcd}
\end{center}
and selecting formal group laws everywhere, the objects of the groupoid $(\moduli{fg})^\wedge_\Gamma(R)$ are given by diagrams
\begin{center}
\begin{tikzcd}
{(\A^1_k, +_\Gamma)} \arrow{d} & {(\A^1_{R/\m}, +_{i^* \Gamma})} \arrow{rd} \arrow[equal]{rr} \arrow{l} & & {(\A^1_{R/\m}, +_{\pi^* \G})} \arrow{ld} \arrow{r} & {(\A^1_R, +_{\G})} \arrow{d} \\
\Spec k & & \Spec R/\m \arrow["i"']{ll} \arrow["\pi"]{rr} & & \Spf R,
\end{tikzcd}
\end{center}
where we have required an \emph{equality} of formal group laws over the common pullback.  A morphism in this groupoid is a formal group law isomorphism $f$ over $\Spf R$ which reduces to the identity over $\Spec R/\m$.
\end{remark}

The main result about this infinitesimal space $(\moduli{fg})^\wedge_\Gamma$ is due to Lubin and Tate:
\begin{theorem}[{\cite[Theorem 3.1]{LubinTate}}]\label{LubinTateModuliThm}
Suppose that $\height \Gamma < \infty$ for $\Gamma$ a formal group over $k$ a perfect field of positive characteristic $p$.  The functor $(\moduli{fg})^\wedge_\Gamma$ is valued in essentially discrete groupoids, and it is naturally equivalent to a smooth formal scheme over $\W_p(k)$ of dimension $(\height(\Gamma) - 1)$.
\end{theorem}

\begin{remark}[{\cite[Theorem 4.35]{Zink}}]
The presence of the \textit{$p$--local Witt ring} $\W_p(k)$ is explained by its universal property: for $k$ as above and $R$ an infinitesimal thickening of $k$, $\W_p(k)$ has the lifting property\footnote{Rings with such lifting properties are generally called \textit{Cohen rings.}  In the case that $k$ is a perfect field of positive characteristic $p$, the Witt ring $\W_p(k)$ happens to model a Cohen ring for $k$.}
\begin{center}
\begin{tikzcd}
\W_p(k) \arrow{d} \arrow[densely dotted, "\exists!"]{r} & R \arrow{d} \\
k \arrow["i"]{r} & R/\m
\end{tikzcd}
\end{center}
For the finite perfect fields $k = \F_{p^d} = \F_p(\zeta_{p^d-1})$, the Witt ring can be computed to be $\W_p(\F_{p^d}) = \Z_p(\zeta_{p^d-1})$.
\end{remark}

\begin{remark}\label{LubinTateModuliThmInFGLTerms}
In light of \Cref{LubinTateStackInFGLTerms}, we can also state \Cref{LubinTateModuliThm} in terms of formal group laws and their $\star$--isomorphisms.  For a group law $+_\Gamma$ over a perfect field $k$ of positive characteristic, it claims that there exists a ring $X$, noncanonically isomorphic to $\W_p(k)\ps{u_1, \ldots, u_{d-1}}$, as well as a certain group law $+_{\widetilde \Gamma}$ on this ring.  The group law $+_{\widetilde \Gamma}$ has the following property: if $+_\G$ is a formal group law on an infinitesimal thickening $\Spf R$ of $\Spec k$ which reduces along $\pi\co \Spec R/\m \to \Spf R$ to $+_\Gamma$, then there is a unique ring map $f\co X \to R$ such that $f^* (+_{\widetilde \Gamma})$ is $\star$--isomorphic to $\pi^* (+_{\G})$.  Moreover, this $\star$--isomorphism is unique.
\end{remark}

We will spend the rest of this Lecture working towards a proof of \Cref{LubinTateModuliThm}.  We first consider a very particular sort of infinitesimal thickening: the square-zero extension $R = k[\eps] / \eps^2$ with pointing $\eps = 0$.  We are interested in two kinds of data over $R$: formal group laws $+_\Delta$ over $R$ reducing to $+_\Gamma$ at the pointing, and formal group law automorphisms $\phi$ of $+_\Gamma$ which reduce to the identity automorphism at the pointing.
\begin{lemma}
Define
\begin{align*}
\Gamma_1 & = \frac{\partial(x +_\Gamma y)}{\partial x}, &
\Gamma_2 & = \frac{\partial(x +_\Gamma y)}{\partial y}.
\end{align*}
Such automorphisms $\phi$ are determined by series $\psi$ satisfying \[0 = \Gamma_1(x, y) \psi(x) - \psi(x +_\Gamma y) + \psi(y) \Gamma_2(x, y).\]  Such formal group laws $+_\Delta$ are determined by bivariate series $\delta(x, y)$ satisfying \[0 = \Gamma_1(x +_\Gamma y, z) \delta(x, y) - \delta(x, y +_\Gamma z) + \delta(x +_\Gamma y, z) - \delta(y, z) \Gamma_2(x, y +_\Gamma z).\]
\end{lemma}
\begin{proof}
Such an automorphism $\phi$ admits a series expansion
\begin{align*}
\phi(x) & = x + \eps \cdot \psi(x). \\
\intertext{Then, we take the homomorphism property}
\phi(x +_\Gamma y) & = \phi(x) +_\Gamma \phi(y) \\
(x +_\Gamma y) + \eps \cdot \psi(x +_\Gamma y) & = (x + \eps \cdot \psi(x)) +_\Gamma (y + \eps \cdot \psi(y)) \\
\intertext{and apply $\left. \frac{\partial}{\partial\eps} \right|_{\eps = 0}$ to get}
\psi(x +_\Gamma y) & = \Gamma_1(x, y) \cdot \psi(x) + \Gamma_2(x, y) \cdot \psi(y).
\end{align*}
Similarly, such a formal group law $+_\Delta$ admits a series expansion \[x +_\Delta y = (x +_\Gamma y) + \eps \cdot \delta(x, y).\]  Beginning with the associativity property \[(x +_\Delta y) +_\Delta z = x +_\Delta (y +_\Delta z),\] we compute $\left. \frac{\partial}{\partial \eps} \right|_{\eps = 0}$ applied to both sides:
\begin{align*}
\left. \frac{\partial}{\partial \eps} \right|_{\eps = 0} \left((x +_\Delta y) +_\Delta z \right) & = \left. \frac{\partial}{\partial \eps} \right|_{\eps = 0} \left((((x +_\Gamma y) + \eps \cdot \delta(x, y)) +_\Gamma z) + \eps \cdot \delta(x +_\Gamma y, z)\right) \\
& = \Gamma_1(x +_\Gamma y,z) \cdot \delta(x, y) + \delta(x +_\Gamma y, z), \\
\intertext{and similarly}
\left. \frac{\partial}{\partial \eps} \right|_{\eps = 0} \left( x +_\Delta (y +_\Delta z) \right) & = \left. \frac{\partial}{\partial \eps} \right|_{\eps = 0} \left( (x +_\Gamma ((y +_\Gamma z) + \eps \cdot \delta(y, z))) + \eps \cdot \delta(x, y +_\Gamma z) \right) \\
& = \Gamma_2(x, y +_\Gamma z) \cdot \delta(y, z) + \delta(x, y +_\Gamma z).
\end{align*}
Equating these gives the condition in the Lemma statement.
\end{proof}

The key observation is that these two conditions appear as cocycle conditions for the first two levels of a natural cochain complex.
\begin{definition}[{\cite[Section 3]{LazarevDeformations}}]
\todo{Can this be phrased geometrically?}
The deformation complex $\widehat C^*(+_\Gamma; k)$ is defined by \[k \to k\ps{x_1} \to k\ps{x_1, x_2} \to k\ps{x_1, x_2, x_3} \to \cdots\] with differential
\begin{align*}
(df)(x_1, \ldots, x_{n+1}) & = \Gamma_1\left(\sum_{i=1}^n {}_\Gamma x_i, x_{n+1} \right) \cdot f(x_1, \ldots, x_n) \\
& \quad + \sum (-1)^i f(x_1, \ldots, x_i +_\Gamma x_{i+1}, \ldots, x_{n+1}) \\
& \quad + (-1)^{n+1} \left( \phi_2\left(x_1, \sum_{i=2}^{n+1} {}_\Gamma x_i \right) \cdot f(x_2, \ldots, x_{n+1}) \right),
\end{align*}
where we have again written
\begin{align*}
\Gamma_1(x, y) & = \frac{\partial(x +_\Gamma y)}{\partial x}, &
\Gamma_2(x, y) & = \frac{\partial(x +_\Gamma y)}{\partial y}.
\end{align*}
\end{definition}

The complex even knits the information together intelligently:

\begin{corollary}[{\cite[p.\ 1320]{LazarevDeformations}}]\label{InterpretLTCoboundaries}
\todo{Consider proving this.}
Two extensions $+_\Delta$ and $+_{\Delta'}$ of $+_\Gamma$ to $k[\eps] / \eps^2$ are isomorphic if their corresponding $2$--cocycles in $\widehat{Z}^2(+_\Gamma; k)$ differ by an element in $\widehat B^2(+_\Gamma; k)$. \qed
\end{corollary}

Remarkably, we have already encountered this complex before:

\begin{lemma}[{\cite[p.\ 1320]{LazarevDeformations}}]\label{LazarevComparisonOfCplxes}
Write $\G$ for the formal group associated to the group law $+_\Gamma$.  The cochain complex $\widehat C^*(+_\Gamma; k)$ is quasi-isomorphic to the cohomology cochain complex considered in the proof of \Cref{Symmetric2CocycleLemma}:
\begin{align*}
\widehat C^*(+_\Gamma; k) & \to \InternalHom{FormalSchemes}(B\G, \G_a)(k) \\
f & \mapsto \Gamma_1\left(0, \sum_{i=1}^n{}_\Gamma x_i \right)^{-1} f(x_1, \ldots, x_n). \qed
\end{align*}
\end{lemma}

Two Lectures ago while proving \Cref{Symmetric2CocycleLemma}, we computed the cohomology of this complex in the specific case of $\G = \G_a$.  This is the one case where Lubin and Tate's theorem does \emph{not} apply, since it requires $\height \G < \infty$.  Nonetheless, by filtering the multiplication on $\G$ by degree, we can use this specific calculation to get up to the general one we now seek.

\begin{lemma}\label{CalculationOfLTTangentSpace}
Let $\G$ be a formal group of finite height $d$ over a field $k$.  Then $H^1(\G; \G_a) = 0$ and $H^2(\G; \G_a)$ is a free $k$--vector space of dimension $(d - 1)$.
\todo{$d$ or $(d-1)$?  There's $\beta_0$ through $\beta_{d-1}$\ldots.  Also compare this with \Cref{Symmetric2CocycleLemma}.}
\end{lemma}
\begin{proof}[Proof (after Hopkins)]
We select a $p$--typical coordinate on $\G$ of the form \[x +_\phi y = x + y + \mathrm{unit} \cdot c_{p^d}(x, y) + \cdots,\] where $c_{p^d}(x, y)$ is as in one of Lazard's symmetric $2$--cocycles, as in \Cref{Symmetric2CocycleLemma}.  Filtering $\G$ by degree, the multiplication projects to $x +_\phi y = x + y$ in the associated graded, and the resulting filtration spectral sequence has signature \[[H^*(\G_a; \G_a)]_* \Rightarrow H^*(\G; \G_a),\] where the second grading comes from the degree of the homogeneous polynomial representatives of classes in $H^*(\G_a; \G_a)$.

Because \Cref{Symmetric2CocycleLemma} gives different calculations of $H^*(\G_a; \G_a)$ for $p = 2$ and $p > 2$, we specialize to $p > 2$ for the remainder of the proof and leave the similar $p = 2$ case to the reader.  For $p > 2$, \Cref{Symmetric2CocycleLemma} gives \[[H^*(\G_a; \G_a)]_* = \left[\frac{k[\alpha_j \mid j \ge 0]}{\alpha_j^2 = 0} \otimes k[\beta_j \mid j \ge 0]\right]_*,\] where $\alpha_j$ is represented by $x^{p^j}$ and $\beta_j$ is represented by $c_{p^j}(x, y)$.  To compute the differentials in this spectral sequence generally, one computes by hand the formula for the differential in the bar complex, working up to lowest nonzero degree.  For instance, to compute $d(\alpha_j)$ we examine the series \[(x +_\phi y)^{p^j} - (x^{p^j} + y^{p^j}) = (\text{unit}) \cdot c_{p^{d + j}}(x, y) + \cdots,\] where we used $c_{p^d}^{p^j} = c_{p^{j+d}}$.  So, we see that nothing in the $1$--column of the spectral sequence is a permanent cocycle and that there are $d - 1$ things at the bottom of the $2$--column of the spectral sequence which are not coboundaries.  To conclude the Lemma statement, we need only to check that they are indeed permanent cocycles.  To do this, we note that they are indeed realized as deformations, by noting \[x +_{\mathrm{univ}} y \cong x + y + v_j c_{p^j}(x, y) \pmod{v_1, \ldots, v_{j-1}, (x, y)^{p^j+1}}\] where $+_{\mathrm{univ}}$ is the Kudo--Araki universal $p$--typical law (cf.\ \cite[Proposition 1.1]{LubinTate}).
\end{proof}

\begin{proof}[Proof of \Cref{LubinTateModuliThm} using \Cref{LubinTateModuliThmInFGLTerms}]
We will prove this inductively on the order of the infinitesimal neighborhood of $\Spec k = \Spec R / \m$ in $\Spf R$: \[\Spec R / \m \xrightarrow{j_r} \Spec R/\m^r \xrightarrow{i_r} \Spf R.\]  Suppose that we have demonstrated the Theorem for $+_{\G_{r-1}} = i_{r-1}^* (+_{\G})$, so that there is a map $\alpha_{r-1}\co \W_p(k)\ps{u_1, \ldots, u_{d-1}} \to R/\m^{r-1}$ and a strict isomorphism $g_{r-1}\co +_{\G_{r-1}} \to \alpha_{r-1}^* +_{\tilde \Gamma}$ of formal group laws.  The exact sequence \[0 \to \m^{r-1} / \m^r \to R/\m^r \to R/\m^{r-1} \to 0\] exhibits $R/\m^r$ as a square--zero extension of $R/\m^{r-1}$ by $M = \m^{r-1} / \m^r$.  Then, let $\beta$ be \emph{any} lift of $\alpha_{r-1}$ and $h$ be \emph{any} lift of $g_{r-1}$ to $R/I^r$, and let $A$ and $B$ be the induced group laws
\begin{align*}
x +_A y & = \beta^* \tilde \phi, &
x +_B y & = h\left( h^{-1}(x) +_{\G_r} h^{-1}(y) \right).
\end{align*}
Since these both deform the group law $+_{\G_{r-1}}$, by \Cref{InterpretLTCoboundaries} and \Cref{CalculationOfLTTangentSpace} there exist $m_j \in M$ and $f(x) \in M\ps{x}$ satisfying \[(x +_B y) - (x +_A y) = (df)(x, y) + \sum_{j=1}^{d-1} m_j c_{p^j}(x, y),\] where $c_{p^j}(x, y)$ is the $2$--cocycle associated to the cohomology $2$--class $\beta_j$.  The following definitions complete the induction:
\begin{align*}
g_r(x) & = h(x) - f(x), &
\alpha_r(u_j) & = \beta(u_j) + m_j. \qedhere
\end{align*}
\end{proof}

\begin{remark}\label{ActionBySnLiftsToLTn}
Our calculation $H^1(\G_\phi; \G_a) = 0$ shows that the automorphisms $\alpha\co \Gamma \to \Gamma$ of the special fiber induce automorphisms of the entire Lubin--Tate stack by universality.  Namely, for $\Gamma \to \widetilde \Gamma$ the universal deformation, the precomposite \[\Gamma \xrightarrow{\alpha} \Gamma \to \widetilde \Gamma\] presents $\widetilde \Gamma$ as a deformation of $\Gamma$ in a different way, hence induces a map $\widetilde \alpha\co \widetilde \Gamma \to \widetilde \Gamma$, which by \Cref{LubinTateModuliThm} is in turn induced by a map $\widetilde \alpha\co (\moduli{fg})^\wedge_\Gamma \to (\moduli{fg})^\wedge_\Gamma$.  The action is \emph{highly} nontrivial in all but the most degenerate cases, and its study is of serious interest to homotopy theorists (cf.\ \Cref{ChromaticLocalizationSection}) and to arithmetic geometers (cf.\ \Cref{ThePeriodMapSection}).
\end{remark}

\begin{remark}
We also see that our analysis fails wildly for the case $\Gamma = \G_a$.  The differential calculation in \Cref{CalculationOfLTTangentSpace} is meant to give us an upper bound on the dimensions of $H^1(\Gamma; \G_a)$ and $H^2(\Gamma; \G_a)$, but this family of differentials is zero in the additive case.  Accordingly, both of these vector spaces are infinite dimensional, completely prohibiting us from making any further assessment.
\end{remark}

Having accomplished all our major goals, we close our algebraic analysis of $\moduli{fg}$ with \Cref{MfgPicture}, a diagram summarizing our results.
\begin{sidewaysfigure}
\centering
PICTURE GOES HERE.

\vspace{\baselineskip}
Draw a picture of $\Spec \Z_{(p)}$ for the base object: a generic point $(0)$ and a point $(p)$ with a $1$--dimensional arithmetic neighborhood.
Draw some of the geometric points of $\moduli{fg} \times \Spec \Z_{(p)}$: $\G_a \otimes \Q$, $\G_m$, something at height $2$, something at height $3$, \ldots, and $\G_a \otimes \F_p$ at height $\infty$.
Draw formal neighborhoods of each finite height point: include the arithmetic deformation direction covering the arithmetic deformation of the base, plus the geometric deformation directions where available.
Draw a crazy cloud around the infinite height point, indicating a poorly understood deformation theory.
Draw braces around parts of the picture to indicate a closed and open substack.  Make the braces right-align to indicate that dropping height is ``normal'' and raising height is ``exceptional''.  Label the substacks with some basic properties: their co/dimensions, for example.
Label the heights of the formal group, and indicate the behavior of the height function.
Draw a ``zoomed in'' version of the height $1$ geometric point, indicating the existence of many non-closed field points covering it, or ``Forms of $\G_m$''.
Draw ``attaching data'' between the different formal neighborhoods, indicating that they are nontrivially connected to one another.  The idea should be something like projective space, where the open height-dropping condition determines an ``around the edges'' map for the closed height-raising condition.
Indicate (perhaps in a different color) the topological analogues of everything in the picture: $H\Z_{(p)}$, $H\F_p$, $H\Q$, $K_\Gamma$, $E(d)$, $P(d)$, $E_\Gamma$, $BP_{(p)}$, \ldots.
Include a legend: dots for geometric points, fuzz for deformation neighborhoods, \ldots.

\vspace{\baselineskip}
I drew an approximation to this picture by hand in \texttt{other resources}.  I didn't get everything right, but I'd definitely like to use it as a template.
\caption{Portrait of $\moduli{fg} \times \Spec \Z_{(p)}$.}\label{MfgPicture}
\end{sidewaysfigure}


% \begin{lemma}[{\cite[Reduction to Theorem 21.5]{LurieChromaticCourseNotes}}]
% \Cref{LubinTateModuliThm} is true in general if and only if it holds for the square-zero thickening $R = k[\eps]/\eps^2$.
% \end{lemma}
% \begin{proof}
% \todo{I don't \emph{really} understand this proof. I feel that if I did, I wouldn't phrase it quite like this.  //  It's also not so clear to me how much this depends upon having already shown that $(\moduli{fg})^\wedge_\Gamma$ is discrete.}
% Of course, if \Cref{LubinTateModuliThm} is true in general, then it holds for the special case of $R = k[\eps]/\eps^2$.  Our task is to reduce the general case to this special case, and we proceed by inducting on the length of the Artinian ring $R$.  If $R$ has length $1$, then $R = k$ and we are done.  Otherwise, we can choose an element $r \in R$ which is annihilated by $\m$, and we consider the relationship between $(\moduli{fg})^\wedge_\Gamma(R)$ and $(\moduli{fg})^\wedge_\Gamma(R/r)$.  There is a pullback diagram
% \begin{center}
% \begin{tikzcd}
% (\moduli{fg})^\wedge_\Gamma(R \times_{R/r} R) \arrow["m"]{r} \arrow{d} & (\moduli{fg})^\wedge_\Gamma(R) \arrow["\pi"]{d} \\
% (\moduli{fg})^\wedge_\Gamma(R) \arrow{r} & (\moduli{fg})^\wedge_\Gamma(R/r).
% \end{tikzcd}
% \end{center}
% First observing that $R \times_{R/r} R \cong k[\eps] / \eps^2 \times_k R$, we note that $G = (\moduli{fg})^\wedge_\Gamma(k[\eps]/\eps^2)$ forms a group, that the map $G$ acts on $(\moduli{fg})^\wedge_\Gamma(R)$ through the map $m$, that $\pi$ factors to give an embedding \[\pi/G\co (\moduli{fg})^\wedge_\Gamma(R) / G \to (\moduli{fg})^\wedge_\Gamma(R/r),\] and that \Cref{MfglIsSmooth} shows that $\pi/G$ is additionally surjective.  From all this, we see that $\pi$ presents $(\moduli{fg})^\wedge_\Gamma(R)$ as a principal homogeneous space for $G$ over $(\moduli{fg})^\wedge_\Gamma(R/r)$.  An identical argument shows that smooth formal schemes also have this property, hence an isomorphism \[G \cong \CatOf{Algebras}_{k/}(\W_p(k)\ps{u_1, \ldots, u_{n-1}}, k[\eps] / \eps^2)\] would finish the induction.
% \end{proof}

\todo{\cite[Section 4.A]{Drinfeld} also contains a Lubin--Tate cohomology theory, this time for formal $A$--modules. You should also cite \cite{LubinTate}.}
\citeme{Neil's FG notes in the first half of section 18 talk about additive extensions and their relation to infinitesimal deformations.  In the second half, he (more or less) talks about the de Rham crystal and shows that $\Ext_{\mathrm{rigid}}(G, \G_a) \cong \operatorname{Prim}(H^1_{dR}(G/X))$ in 18.37.}
\todo{I still have some confusion about the formal similarity between deforming formal group laws over square-zero extensions of the base and deforming formal $n$--buds over the finite order nilpotent neighborhoods of a point.  This would be a good place to sort that out.}









\section{Nilpotence and periodicity in finite spectra}\label{NilpotenceAndPeriodicity}

\todo{The bottom of COCTALOS page 68 has a better interpretation of what picking a formal group law lifting a flat map to $\moduli{fg}$ has to do with anything.  It probably belongs in this Lecture.}
\todo{Much of this section is written graded-ly.  I guess we still haven't decided whether this is the right presentation.}
\todo{Danny has been nervous lately about the LEFT yielding module spectra and ring spectra. It would be good to write out how LEFT applied to a sheaf of rings gives a ring spectrum.}
\todo{Akhil says some nice things about nilpotence and periodicity here: http://mathoverflow.net/questions/116663/connection-of-xn-spectra-to-formal-group-laws . The discussion of vanishing lines could be included here. The moral indication that nilpotence is the geometric link between stable homotopy and the moduli of formal groups also seems like an important point to be made here.}


With our analysis of $\moduli{fg}$ complete, our first goal in this Lecture is to finish the program sketched in the introduction to this Case Study by manufacturing those interesting homology theories connected to the functor $\context{MU}(-)$.  We begin by rephrasing our main tool, \Cref{LandwebersStackyTheorem}, in terms of algebraic conditions.
\begin{theorem}[{\cite[Corollary 2.7]{LandweberHomologicalComodules} and \cite[Theorem 21.4 and Proposition 21.5]{HopkinsCOCTALOS}, cf.\ \Cref{LandwebersStackyTheorem}}]\label{LEFTRealTheoremWithProof}
Let $\sheaf F$ a quasicoherent sheaf over $\moduli{fg} \times \Spec \Z_{(p)}$, thought of as a comodule $M$ for the Kudo--Araki Hopf algebroid (cf.\ \Cref{KudoArakiIsomorphism}) \[(A, \Gamma) = (\sheaf O_{\moduli{fgl}^{\ptyp}}, \sheaf O_{\moduli{fgl}^{\ptyp}}[t_1, t_2, \ldots]).\]  If $(p, v_1, \ldots, v_d, \ldots)$ forms an infinite regular sequence on $M$, then \[X \mapsto M \otimes_{\sheaf O_{\moduli{fgl}^{\ptyp}}} MU_0(X)\] determines a homology theory on \emph{finite} spectra $X$.  Moreover, if $M/I_d = 0$ for some $d \gg 0$, then the same formula determines a homology theory on \emph{all} spectra $X$.
\end{theorem}
\begin{proof}
Following the discussion in the introduction, we note that a cofiber sequence \[X' \to X \to X''\] of spectra gives rise to an exact sequence
\begin{center}
\begin{tikzcd}
\cdots \arrow{r} & \context{MU}(X') \arrow{r} & \context{MU}(X) \arrow{r} & \context{MU}(X'') \arrow{r} & \cdots \\
\cdots \arrow{r} & \sheaf N' \arrow[equal]{u} \arrow{r} & \sheaf N \arrow[equal]{u} \arrow{r} & \sheaf N'' \arrow[equal]{u} \arrow{r} & \cdots.
\end{tikzcd}
\end{center}
We thus see that we are essentially tasked with showing that $\sheaf F$ is flat, so that tensoring with $\sheaf F$ does not disturb the exactness of this sequence.  In that case, we can then apply Brown representability to the composite functor $\sheaf F \otimes \context{MU}(X)$.

Flatness of $\sheaf F$ is equivalent to $\Tor_1(\sheaf F, \sheaf N) = 0$ for an arbitrary auxiliary quasicoherent sheaf $\sheaf N$ (soon to be thought of as $\context{MU}(X)$).  By our regularity hypothesis, there is an exact sequence of sheaves \[0 \to \sheaf F \xrightarrow{p} \sheaf F \to \sheaf F / (p) \to 0,\] so applying $\Tor_*(-, \sheaf N)$ gives an exact sequence \[\Tor_2(\sheaf F / (p), \sheaf N) \xrightarrow{} \Tor_1(\sheaf F, \sheaf N) \xrightarrow{p} \Tor_1(\sheaf F, \sheaf N)\] of $\Tor$ groups.  The sequence gives the following sufficiency condition: \[[\Tor_1(p^{-1} \sheaf F, \sheaf N) = 0 \quad \text{and} \quad \Tor_2(\sheaf F/(p), \sheaf N) = 0] \quad \Rightarrow \quad \Tor_1(\sheaf F, \sheaf N) = 0.\]  Similarly, the $v_1$--multiplication sequence gives another sufficiency condition: \[[\Tor_2(v_1^{-1} \sheaf F/(p), \sheaf N) = 0 \quad \text{and} \quad \Tor_3(\sheaf F/I_2, \sheaf N) = 0] \quad \Rightarrow \quad \Tor_2(\sheaf F/(p), \sheaf N) = 0.\]  Continuing in this fashion, for some $D \gg 0$ we would like to show
\begin{align*}
\Tor_{d+1}(v_d^{-1} \sheaf F/I_d, \sheaf N) & = 0 & \text{(for each $d < D$),} \\
\Tor_{D+1}(\sheaf F/I_{D+1}, \sheaf N) & = 0.
\end{align*}
The second condition is satisfied one of two ways, corresponding to our two auxiliary hypotheses and two conclusions in the Theorem statement:
\begin{itemize}
\item If $\sheaf F$ itself satisfies $\sheaf F / I_{D+1} = 0$, we are done.
\item Writing $j_{D+1}\co \moduli{fg}^{\ge D+1} \to \moduli{fg}$ for the inclusion of the prime closed substack, we can identify $\sheaf N / I_{D+1}$ with $j_{D+1}{}_* j_{D+1}^* \sheaf N$.  If $\sheaf N$ is coherent (for instance, in the case that $\sheaf N = \context{MU}(X)$ for a \emph{finite} complex $X$), then $j_{D+1}^* \sheaf N$ is free for large $D$ and hence has vanishing $\Tor$ groups.
\end{itemize}

We then turn to the first collection of conditions.  They are \emph{always} satisfied, but this requires an argument.  We write $i_d\co \moduli{fg}^{= d} \to \moduli{fg}$ for the inclusion of the substack of formal groups of height exactly $d$, which (following \Cref{OpenSubstacksOfMfg}) has a presentation by the Hopf algebroid \[(v_d^{-1} A / I_d, \Gamma \otimes v_d^{-1} A / I_d).\]  We are trying to study the derived functors of \[\sheaf N \mapsto (i_d{}_* i_d^* \sheaf F) \otimes \sheaf N \cong i_d{}_* (i_d^* \sheaf F \otimes i_d^* \sheaf N).\]  Since $i_d{}_*$ is exact, we are moved to study the composite functor spectral sequence for \[\CatOf{QCoh}_{\moduli{fg}} \xrightarrow{i_d^*} \CatOf{QCoh}_{\moduli{fg}^{=d}} \xrightarrow{i_d^* \sheaf F \otimes -} \CatOf{QCoh}_{\moduli{fg}^{=d}}.\]  The second functor is actually exact: the geometric map \[\Gamma_d\co \Spec k \to \moduli{fg}^{=d}\] is a faithfully flat cover, and $k$--modules have no nontrivial $\Tor$.  Meanwhile, the first functor has at most $d$ derived functors: $i_d^*$ is modeled by tensoring with $v_d^{-1} A / I_d$, but $A / I_d$ admits a Koszul resolution with $d$ stages and $A / I_d \to v_d^{-1} A / I_d$ is exact.  As $\Tor_{d+1}$ is beyond the length of this resolution, it is always zero.
\end{proof}

\begin{definition}\label{DefnChromaticHomologyThys}
Coupling \Cref{LEFTRealTheoremWithProof} to our understanding of $\moduli{fg}$, we produce many interesting homology theories, collectively referred to as \textit{chromatic\footnote{The elements of \Cref{ANSS2Figure} and \Cref{ANSS3Figure} are related to each other by ``$v_d$--multiplication'' (cf.\ \Cref{GreekLetterElements}), and families of such elements can be selected for by inverting $v_d$, i.e., by passing to the open substack $\moduli{fg}^{\le d}$.  The word ``chromatic'' here thus refers to an analogy: this localization selects certain periodic families of elements, like a bandpass filter selects certain frequencies out of a complicated radio signal.} homology theories}:
\begin{itemize}
\item Recall that the moduli of $p$--typical group laws is affine, presented in \Cref{KudoArakiIsomorphism} by \[\sheaf O_{\moduli{fgl}^{\ptyp}} \cong \Z_{(p)}[v_1, v_2, \ldots, v_d, \ldots].\]  Since the inclusion of $p$--typical group laws into all group laws induces an equivalence of stacks, it is in particular flat, and hence this formula determines a homology theory on finite spectra, called \textit{Brown--Peterson homology}: \[BPP_0(X) := MUP_0(X) \otimes_{MUP_0} BPP_0.\]
\item A chart for the open substack $\moduli{fg}^{\le d}$ in terms of $\moduli{fgl}^{\ptyp}$ was given in \Cref{OpenSubstacksOfMfg} by $\Spec \Z_{(p)}[v_1, v_2, \ldots, v_d^\pm]$.  Since open maps are in particular flat, it follows that there is a homology theory $E(d)P$, called \textit{the $d${\th} Johnson--Wilson homology}, defined on all spectra by \[E(d)P_0(X) := MUP_0(X) \otimes_{MUP_0} \Z_{(p)}[v_1, v_2, \ldots, v_d^\pm].\]
\item Similarly, for a formal group $\Gamma$ of height $d < \infty$, we produced in \Cref{LubinTateModuliThm} a chart $\Spf \Z_p\ps{u_1, \ldots, u_{d-1}}$ for its deformation neighborhood.  Since inclusions of deformation neighborhoods of substacks of Noetherian stacks are flat\citeme{Consider citing this}, there is a corresponding homology theory $E_\Gamma{}$, called \textit{the (discontinuous) Morava $E$--theory for $\Gamma$}, determined by \[E_\Gamma{}_0(X) := MUP_0(X) \otimes_{MUP_0} \Z_p\ps{u_1, \ldots, u_{d-1}}[u^\pm].\]  In the case that $\Gamma = \Gamma_d$ is the Honda formal group of height $d$, the notation is often abbreviated from $E_{\Gamma_d}$ to merely $E_d$.
\item Since $(p, u_1, \ldots, u_{d-1})$ forms a regular sequence on $E_\Gamma{}_*$, we can form the regular quotient at the level of spectra, using cofiber sequences
\begin{align*}
E_\Gamma \xrightarrow{p} E_\Gamma & \to E_\Gamma / (p), \\
E_\Gamma / (p) \xrightarrow{u_1} E_\Gamma / (p) & \to E_\Gamma / (p, u_1), \\
& \vdots \\
E_\Gamma / I_{d-1} \xrightarrow{u_{d-1}} E_\Gamma / I_{d-1} & \to E_\Gamma / I_d.
\end{align*}
This determines a spectrum $K_\Gamma = E_\Gamma / I_d$, and hence determines a homology theory called \textit{the Morava $K$--theory for $\Gamma$}.  In the case where $\Gamma$ comes from the Honda $p$--typical formal group law (of height $d$), this spectrum is often written as $K(d)$.  As an edge case, we also set $K(0) = H\Q$ and $K(\infty) = H\F_p$.\footnote{By \Cref{FGpsOverAlgClosedFields} and \Cref{FieldSpectraAreKTheories} to follow, it often suffices to consider just these spectra $K(d)$ to make statements about all $K_\Gamma$.  With more care, it even often suffices to consider $d \ne \infty$.}
\item More delicately, there is a version of Morava $E$--theory which takes into account the formal topology on $(\moduli{fg})^\wedge_\Gamma$, called \textit{continuous Morava $E$--theory}.  It is defined by the pro-system $\{E_\Gamma(X) / u^I\}$, where $I$ ranges over multi-indices and the quotient is again given by cofiber sequences.
\item There is also a homology theory associated to the closed substack $\moduli{fg}^{\ge d}$.  Since $I_d = (p, v_1, \ldots, v_{d-1})$ is generated by a regular sequence on $BPP_0$, we can directly define the spectrum $P(d)P$ by a regular quotient: \[P(d)P = BP / (p, v_1, \ldots, v_{d-1}).\]  This spectrum does have the property $P(d)P_0 = BPP_0 / I_d$ on coefficient rings, but $P(d)P_0(X) = BPP_0(X) / I_d$ \emph{only} when $I_d$ forms a regular sequence on $BPP_0(X)$---which is reasonably rare among the cases of interest.
\end{itemize}
\end{definition}

\begin{remark}
The trailing ``$P$'' in these names is to disambiguate them from similar less-periodic objects in the literature.  Namely, $BP$ is often taken to be a minimal wedge summand of $MU_{(p)}$, whereas $E(d)$, $E_\Gamma$, and $K(d)$ can all be taken to be $2(p^d-1)$--periodic (for heights $0 < d < \infty$).  The one exception to this minimality convention is $E_\Gamma$, which is \emph{usually} taken to be $2$--periodic already, so we do not attach a ``$P$'' to its name.
\end{remark}

\begin{example}[{cf.\ \Cref{CPinftyKUExample}}]\label{ExampleOfMoravasTheoriesAtGm}
In the case $\Gamma = \G_m$, the resulting spectra are connected to complex $K$--theory:
\begin{align*}
E_{\G_m} & \cong KU^\wedge_p, &
K_{\G_m} & \cong KU / p, &
E(1)P & \cong KU_{(p)}.
\end{align*}
\end{example}

\begin{remark}[{\cite[Section 5.2]{KLW}, \cite[Corollaries 2.14 and 2.16]{RavenelLocalizationWRTPeriodic}, \cite[Theorem 2.13]{StricklandProductsOnModules}}]\label{MoravaKIsNotCommutative}
In general, the quotient of a ring spectrum by a homotopy element does not give another ring spectrum.  The most typical example of this phenomenon is that $\S/2$ is not a ring spectrum, since its homotopy is not $2$--torsion.  Most of our constructions above do not suffer from this deficiency, with one exception: Morava $K$--theories at $p = 2$ are not commutative.  Instead, there is a derivation $Q_d: K(d) \to \Susp K(d)$ which tracks the commutativity by the relation \[ab - ba = u Q_d(a) Q_d(b).\]  In particular, we find that $K(d)^* X$ is a commutative ring whenever $K(d)^1 X = 0$, which is often the case.
\end{remark}

Having constructed these chromatic homology theories, for the rest of this Lecture we pursue an example of a ``fiberwise'' analysis of a phenomenon in homotopy theory.  First, recall the following classical theorem:

\begin{theorem}[{\cite{Nishida}, \cite[Section II.2]{BMMS}}]
Every homotopy class $\alpha \in \pi_{\ge 1} \S$ is nilpotent. \qed
\end{theorem}

\noindent People studying $K$--theory in the '$70$s discovered the following related phenomenon:

\begin{theorem}[{\cite[Theorem 12.1]{AdamsJXIV}}]\label{AdamsSelfMapThm}
Let $M_{2n}(p)$ denote the mod--$p$ Moore spectrum with bottom cell in degree $2n$.  Then there is an index $n$ and a map $v: M_{2n}(p) \to M_0(p)$ such that $KU_* v$ acts by multiplication by the $n$\textsuperscript{th}\, power of the Bott class.  The minimal such $n$ is given by the formula \[n = \begin{cases} p-1 & \text{when $p \ge 3$}, \\ 4 & \text{when $p = 2$}. \qed \end{cases}\]
\end{theorem}

\noindent In particular, the map $v$ cannot be nilpotent, since a null-homotopic map induces the zero map in any homology theory.  Just as we took the non-nilpotent endomorphism $p \in \pi_0 \End \S$ and coned it off, we can take the endomorphism $v \in \pi_{2p-2} \End M_0(p)$ and cone it off to form a new spectrum called $V(1)$.\footnote{The spectrum $V(1)$ is actually defined to be a finite spectrum with $BP_* V(1) \cong BP_* / (p, v_1)$. At $p = 2$ this spectrum doesn't exist and this is a misnomer.  More generally, at odd primes $p$ Nave shows that $V((p+1)/2)$ doesn't exist~\cite[Theorem 1.3]{Nave}.}  One can ask, then, whether the pattern continues: does $V(1)$ have a non-nilpotent self-map, and can we cone it off to form a new such spectrum with a new such map?  Can we then do that again, indefinitely?  In order to study this question, we are motivated to find spectra satisfying the following condition:

\begin{definition}[{\cite[Definition 4]{HopkinsSmith}, cf.\ \cite[Theorem 1]{DHS}}]
A ring spectrum $E$ \textit{detects nilpotence} if for any ring spectrum $R$ the kernel of the Hurewicz homomorphism \[R_*\eta_E\co \pi_* R \to E_* R\] consists of nilpotent elements.  (In particular, such an $E$ cannot send such a nontrivial self-map to zero.)
\end{definition}

This question and surrounding issues formed the basis of Ravenel's nilpotence conjectures~\cite[Section 10]{RavenelLocalizationWRTPeriodic}, which were resolved by by Devinatz, Hopkins, and Smith~\cite{DHS,HopkinsSmith}.  One of their two main technical achievements was to demonstrate that we already have access to a nice homology theory which detects nilpotence:

\begin{theorem}[{\cite[Theorem 1.i]{DHS}}]\label{DevinatzHopkinsSmith}
The spectrum $MU$ detects nilpotence. \qed
\end{theorem}

\noindent This is a very hard theorem, and we will not attempt to prove it.
% \todo{Someday, I want to include a formal-geometric interpretation of the proof of this theorem.  I would be thrilled to dedicate another 20 pages to that task here, if only I knew how to do it.}
However, taking this as input, they are easily able to show several other interesting structural results about finite spectra.  For instance, they also show that the $MU$ is the universal object which detects nilpotence, in the sense that any other ring spectrum can have this property checked stalkwise on $\context{MU}$.

\begin{corollary}[{\cite[Theorem 3]{HopkinsSmith}}]\label{LocalNilpotenceDetection}
A ring spectrum $E$ detects nilpotence if and only if for all $0 \le d \le \infty$ and for all primes $p$, $K(d)_* E \ne 0$ (i.e., the support of $\context{MU}(E)$ is not a proper substack of $\context{MU}$).
\end{corollary}
\begin{proof}
If $K(d)_* E = 0$ for some $d$, then the non-nilpotent unit map $\S \to K(d)$ lies in the kernel of the Hurewicz homomorphism for $E$, so $E$ fails to detect nilpotence.

In the other direction, suppose that for every $d$ we have $K(d)_* E \ne 0$.  Because $K(d)_*$ is a field, it follows by picking a basis of $K(d)_* E$ that $K(d) \sm E$ is a nonempty wedge of suspensions of $K(d)$.  So, for $\alpha \in \pi_* R$, if $E_* \alpha = 0$ then $(K(d) \sm E)_* \alpha = 0$ and hence $K(d)_* \alpha = 0$.  So, we need to show that if $K(d)_* \alpha = 0$ for all $n$ and all $p$ then $\alpha$ is nilpotent.  Taking \Cref{DevinatzHopkinsSmith} as given, it would suffice to show merely that $MU_* \alpha$ is nilpotent.  This is equivalent to showing that the ring spectrum $MU \sm R[\alpha^{-1}]$ is contractible or that the unit map is null: \[\S \to MU \sm R[\alpha^{-1}].\]

A nontrivial result of Johnson and Wilson shows that if $MU_* X = 0$, then for any $d$ we have $K([0, d])_* X = 0$ and $P(d+1)_* X = 0$.\footnote{Specifically, it is immediate that $MU_* X = 0$ forces $P(d+1)_* X = 0$ and $v_{d'}^{-1} P(d')_*(X) = 0$ for all $d' < d$.  What's nontrivial is showing that $v_{d'}^{-1} P(d')_*(X) = 0$ if and only if $K(d')_*(X) = 0$~\cite[Theorem 2.1.a]{RavenelLocalizationWRTPeriodic},~\cite[Section 3]{JohnsonWilson}.}  Taking $X = R[\alpha^{-1}]$, we have assumed all of these are zero except for $P(d+1)$.  But $\colim_d P(d+1) \simeq H\F_p \simeq K(\infty)$, and $\S \to K(\infty) \sm R[\alpha^{-1}]$ is assumed to be null as well.  By compactness of $\S$, that null-homotopy factors through some finite stage $P(d+1) \sm R[\alpha^{-1}]$ with $d \gg 0$.
\end{proof}

\Cref{LocalNilpotenceDetection} has the following consequence, which speaks to the primacy of both the chromatic program and these results.

\begin{definition}
A ring spectrum $R$ is a \textit{field spectrum} when every $R$--module (in the homotopy category) splits as a wedge of suspensions of $R$.  (Equivalently, $R$ is a field spectrum when it has K\"unneth isomorphisms.)
\end{definition}

\begin{corollary}\label{FieldSpectraAreKTheories}
Every field spectrum $R$ splits as a wedge of Morava's $K(d)$ theories.
\end{corollary}
\begin{proof}
It is easy to check (as mentioned in the proof of \Cref{LocalNilpotenceDetection}) that $K(d)$ is a field spectrum.

Now, consider an arbitrary field spectrum $R$.  Set $E = \bigvee_{\text{primes $p$}} \bigvee_{d \in [0, \infty]} K(d)$, so that $E$ detects nilpotence.  The class $1$ in the field spectrum $R$ is non-nilpotent, so it survives when paired with some $K$--theory $K(d)$, and hence $R \sm K(d)$ is not contractible.  Because both $R$ and $K(d)$ are field spectra, the smash product of the two simultaneously decomposes into a wedge of $K(d)$s and a wedge of $R$s.  So, $R$ is a retract of a wedge of $K(d)$s, and picking a basis for its image on homotopy shows that it is a sub-wedge of $K(d)$s.
\end{proof}

\begin{remark}
In the $2$--periodic setting we've become accustomed to, the analogue of \Cref{FieldSpectraAreKTheories} is that every $2$--periodic field spectrum splits as a wedge of suspensions of $K(d)P$.
\end{remark}

\begin{remark}
In service of \Cref{ExampleOfMoravasTheoriesAtGm}, the geometric definition of $MU$ given in \Cref{OriginalDefnOfBordism}, the edge cases of $K(0) = H\Q$ and $K(\infty) = H\F_p$, and the claimed primacy of these methods, we might wonder if there is any geometric interpretation of the field theories $K(d)$ for $0 < d < \infty$.  To date, this is a completely open question and the subject of intense research.
\end{remark}

We're now well-situated to address Ravenel's question about finite spectra and periodic self-maps.  The key observation is that spectra admitting such self-maps are closed under some natural operations, leading to the following definition:

\begin{definition}
A full subcategory of a triangulated category (e.g., the homotopy category of $p$--local finite spectra) is \textit{thick} if\ldots
\begin{itemize}
\item \ldots it is closed under isomorphisms and retracts.
\item \ldots it has a $2$-out-of-$3$ property for cofiber sequences.
\end{itemize}
\end{definition}

\noindent Examples of thick subcategories include:
\begin{itemize}
\item The category $\CatOf{C}_d$ of $p$--local finite spectra which are $K(d-1)$--acyclic.  (For instance, if $d = 1$, the condition of $K(0)$--acyclicity is that the spectrum have purely torsion homotopy groups.)  These are called ``finite spectra of type at least $d$''.
\item The category $\CatOf{D}_d$ of $p$--local finite spectra $F$ for which there is a self-map $v: \Susp^N F \to F$, $N \gg 0$ which induces multiplication by a unit in $K(d)$--homology and which is nilpotent in $K(\ne d)$--homology.  These are called ``finite spectra admitting $v_d$--self--maps''.
\todo{Do you really need the nilpotence condition?}
\end{itemize}

\noindent The categories $\CatOf C_d$ are the ones we are interested in analyzing, and we hope to identify these putative spaces $V(d)$ inside of them.  Ravenel shows the following foothold interrelating the $\CatOf C_d$:

\begin{lemma}[{\cite[Theorem 2.11]{RavenelLocalizationWRTPeriodic}}]\label{CdCategoriesNest}
For $X$ a finite complex, there is a bound \[\dim K(d-2)_* X \le \dim K(d-1)_* X.\]  In particular, there is an inclusion $\CatOf C_{d-1} \subseteq \CatOf C_d$.
\end{lemma}
\begin{proof}[Proof sketch]
One should compare this with the statement that the stalk dimension of a coherent sheaf is upper semi-continuous.  In fact, this analogy gives the essentials of Ravenel's proof: one considers the ring spectrum $v_d^{-1} BP / I_{d-1}$, which admits two maps
\begin{center}
\begin{tikzcd}[row sep=1em]
& v_{d-1}^{-1} (v_d^{-1} BP / I_{d-1}) \\
v_d^{-1} BP / I_{d-1} \arrow{ru} \arrow{rd} \\
& (v_d^{-1} BP / I_{d-1}) / v_{d-1}.
\end{tikzcd}
\end{center}
Studying the relevant $\Tor$ spectral sequences gives the result.
\end{proof}

Hopkins and Smith are able to use their local nilpotence detection result, \Cref{LocalNilpotenceDetection}, to completely understand the behavior not only of the thick subcategories $\CatOf C_d$ but of \emph{all} thick subcategories of $\CatOf{Spectra}_{(p)}^{\fin}$.  In particular, this connects the $\CatOf C_d$ with the $\CatOf D_d$, as we will see.

\begin{theorem}[{\cite[Theorem 7]{HopkinsSmith}}]\label{ThickSubcatClassification}
Any thick subcategory $\CatOf C$ of the category of $p$--local finite spectra must be $\CatOf C_d$ for some $d$.
\end{theorem}
\begin{proof}
Since $\CatOf C_d$ are nested by \Cref{CdCategoriesNest} and they form an exhaustive filtration (i.e., $C_\infty$ = 0), it is thus sufficient to show that any object $X \in \CatOf C$ with $X \in \CatOf C_d$ induces an inclusion $\CatOf C_d \subseteq \CatOf C$.  Write $R$ for the endomorphism ring spectrum $R = F(X, X)$, and write $F$ for the fiber of its unit map: \[F \xrightarrow{f} \S \xrightarrow{\eta_R} R.\]  Finally, let $Y \in \CatOf C_d$ be \emph{any} finite spectrum of type at least $d$.  Our goal is to demonstrate $Y \in \CatOf C$.  

Now consider applying $K(n)$--homology (for \emph{arbitrary} $n$) to the map \[1 \sm f\co Y \sm F \to Y \sm \S.\]  The induced map is always zero:
\begin{itemize}
\item In the case that $K(n)_* X$ is nonzero, then $K(n)_* \eta_R$ is injective because $K(n)_*$ is a graded field, and so $K(n)_* f$ is zero.
\item In the case that $K(n)_* X$ is zero, then $n \le d$ and, because of the bound on type, $K(n)_* Y$ is zero as well.
\end{itemize}
By a small variant of local nilpotence detection (\Cref{LocalNilpotenceDetection}, \cite[Corollary 2.5]{HopkinsSmith}), it follows for $j \gg 0$ that \[Y \sm F^{\sm j} \xrightarrow{1 \sm f^{\sm j}} Y \sm \S^{\sm j}\] is null-homotopic.  Hence, one can calculate the cofiber to be \[\cofib\left( Y \sm F^{\sm j} \xrightarrow{1 \sm f^{\sm j}} Y \sm \S^{\sm j} \right) \simeq Y \sm \cofib f^{\sm j} \simeq Y \vee (Y \sm \Susp F^{\sm j}),\] so that $Y$ is a retract of this cofiber.

We now work to show that this smash product lies in the thick subcategory $\CatOf C$ of interest.  First, note that it suffices to show that $\cofib f^{\sm j}$ on its own lies in $\CatOf C$: a finite spectrum (such as $Y$ or $F$) can be expressed as a finite gluing diagram of spheres, and smashing this through with $\cofib f^{\sm j}$ then expresses $Y \sm \cofib f^{\sm j}$ as the iterated cofiber of maps with source and target in $\CatOf C$.  With that in mind, we will in fact show that $\cofib f^{\sm k}$ lies in $\CatOf C$ for all $k \ge 1$.  Consider the following smash version of the octahedral axiom: the factorization \[F \sm F^{\sm (k-1)} \xrightarrow{1 \sm f^{\sm (k-1)}} F \sm \S^{\sm (k-1)} \xrightarrow{f \sm 1} \S \sm \S^{\sm (k-1)}\] begets a cofiber sequence \[F \sm \cofib f^{\sm (k-1)} \to \cofib f^{\sm k} \to \cofib f \sm \S^{\sm (k-1)}.\]  Noting that the base case $\cofib(f) = R = X \sm DX$ lies in $\CatOf C$, we can inductively use the $2$-out-of-$3$ property on the octahedral cofiber sequence to see that $\cofib(f^{\sm k})$ lies in $\CatOf C$ for all $k$.  It follows in particular that $Y \sm \cofib(f^{\sm j})$ lies in $\CatOf C$, and using the retraction $Y$ belongs to $\CatOf C$ as well.
\end{proof}

\begin{theorem}[{\cite[Theorem 9]{HopkinsSmith}}]\label{CdEqualsDd}
A $p$--local finite spectrum is $K(d-1)$--acyclic exactly when it admits a $v_d$--self--map.  Additionally, the inclusion $\CatOf C_d \subsetneq \CatOf C_{d-1}$ is proper.
\end{theorem}
\begin{proof}[Executive summary of proof]
Given the classification of thick subcategories, if a property is closed under thickness then one need only exhibit a single spectrum with the property to know that all the spectra in the thick subcategory it generates also all have that property.  Inductively, they manually construct finite spectra $M_0(p^{i_0}, v_1^{i_1}, \ldots, v_{d-1}^{i_{d-1}})$ for sufficiently large\footnote{We ran into the asymptotic condition $I \gg 0$ earlier, when we asserted that there is no root of the $2$--local $v_1$--self--map $v\co M_8(2) \to M_0(2)$.} indices $i_*$ which admit a self-map $v$ governed by a commuting square
\begin{center}
\begin{tikzcd}
BP_* M_{|v_d| i_d}(p^{i_0}, v_1^{i_1}, \ldots, v_{d-1}^{i_{d-1}}) \arrow{r}{v} \arrow[equal]{d} & BP_* M_0(p^{i_0}, v_1^{i_1}, \ldots, v_{d-1}^{i_{d-1}}) \arrow[equal]{d} \\
\Susp^{|v_d| i_d} BP_* / (p^{i_0}, v_1^{i_1}, \ldots, v_{d-1}^{i_{d-1}}) \arrow{r}{- \cdot v_d^{i_d}} & BP_* / (p^{i_0}, v_1^{i_1}, \ldots, v_{d-1}^{i_{d-1}}).
\end{tikzcd}
\end{center}
These maps are guaranteed by very careful study of Adams spectral sequences.
\todo{This spectrum you've described is the one from section 5 of Nilpotence II paper as an example of a spectrum with few self maps, right?  Hopkins and Smith construct a (possibly?) different spectrum $X_n$ with $v_n$ self maps in section 4 -- or at least assert their existence anyway (theorem 4.11).  I don't know whether it is easy to see that the generalized Moore spectra you've described have $v_n$ self maps without already knowing the result in the statement of this theorem.}
\end{proof}

They therefore conclude the strongest possible positive response to Ravenel's conjectures.  Not only can we continue the sequence \[\S, \; \S/p, \; \S/(p, v), \; \ldots,\] but in fact \emph{any} finite spectrum admits an (essentially unique) interesting periodic self-map.  This is maybe the most remarkable of the statements: although Nishida's theorem initially led us to think of periodic self-maps as rare, they are in fact ubiquitous.  Additionally, we learned that the shift\footnote{This is sometimes referred to as the ``wavelength'' in the chromatic analogy.} of this self-map is determined by the first nonvanishing $K(d)$--homology, giving an effective detection tool.  Finally, all such periodicity shifts arise: for any $d$, there is a spectrum admitting a $v_d$--self--map but not a $v_{d-1}$--self--map.

\todo{In either this section, near the definition of chromatic homology theories, or in the next section, where the acyclicity of $K(d) \sm K(d')$ is stated, we should discuss $H\F_p{}_* BP$ (perhaps in terms of $H\F_p{}_* MU$ but certainly in terms of a quotient algebra of the dual Steenrod algebra), as well as the utility of the Adams spectral sequence in computing extraordinary homology theories from ordinary ones, and the role that Bocksteins play in those calculations (and, in particular, where all the $\tau_j$ operators went in the deformation from $H\F_p$ to $BP$).}









\section{Chromatic dissembly}\label{ChromaticLocalizationSection}

In this Lecture, we will couple the ideas of \Cref{StableContextLecture} to the homology theories and structure theorems described in \Cref{NilpotenceAndPeriodicity}.  In particular, we have not yet exhausted \Cref{ThickSubcatClassification}, and for inspiration about how to utilize it, we will begin with an algebraic analogue of the situation considered thus far.

For a ring $R$, the full derived category $D(\Spec R)$ and the derived category of perfect complexes $D^{\perf}(\Spec R)$ form examples of triangulated categories analogous to $\CatOf{Spectra}$ and $\CatOf{Spectra}^{\fin}$.  By interpreting an $R$--module as a quasicoherent sheaf over $\Spec R$, we can use them to probe for structure of $\Spec R$---for instance, we can test whether $\widetilde M$ is supported over some closed subscheme $\Spec R/I$ by restricting the sheaf, which amounts algebraically to asking whether $M$ is annihilated by $I$.  In the reverse, we can also try to discern what ``closed subscheme'' should mean in some arbitrary triangulated category by codifying the properties of the subcategory of $D(\Spec R)$ supported away from $\Spec R$.  The key observation is this subcategory is closed under tensoring modules: if $M$ is annihilated by $I$, then $M \otimes_R N$ is also annihilated by $I$.

\begin{definition}[{\cite[Definition 1.3]{Balmer}}]
Let $\CatOf C$ be a triangulated $\otimes$--category $\CatOf C$.  A thick subcategory $\CatOf C' \subseteq \CatOf C$ is\ldots
\begin{itemize}
\item \ldots a \textit{$\otimes$--ideal} when $x \in \CatOf C'$ forces $x \otimes y \in \CatOf C'$ for any $y \in \CatOf C$.
\item \ldots a \textit{prime $\otimes$--ideal} when $x \otimes y \in \CatOf C'$ also forces at least one of $x \in \CatOf C'$ or $y \in \CatOf C'$.
\end{itemize}
Finally, define the \textit{spectrum} of $\CatOf C$ to be its collection of prime $\otimes$--ideals.  For any $x \in \CatOf C$ we define a basic open $U(x) = \{\CatOf C' \mid x \in \CatOf C'\}$, which altogether give a basis for a topology on the spectrum.\todo{Double check that you have the directionality of this right.  Is $U$ a basic open or a basic closed?  It it full of things that contain $x$ or that don't contain in $x$?}
\end{definition}

The basic result about this definition is that it does not miss any further conditions:

\begin{theorem}[{\cite[Proposition 8.1]{Balmer}}]
The spectrum of $D^{\perf}(\Spec R)$ is naturally homeomorphic to the Zariski spectrum of $R$. \qed
\end{theorem}

\noindent Satisfied, we apply the definition to the more difficult case of $\CatOf{Spectra}$.

\begin{theorem}[{\cite[Corollary 9.5]{Balmer}}]
The spectrum of $\CatOf{Spectra}_{(p)}^{\mathrm{fin}}$ consists of the thick subcategories $\CatOf C_d$, and $\{\CatOf C_n\}_{n=0}^d$ are its open sets.
\end{theorem}
\begin{proof}
Using \Cref{CdEqualsDd}, we can characterize $\CatOf C_d$ as the kernel of $K(d-1)_*$.  This shows it to be a prime $\otimes$--ideal: \[K(d-1)_*(X \sm Y) \cong K(d-1)_* X \otimes_{K(d-1)_*} K(d-1)_* Y\] is zero exactly when at least one of $X$ and $Y$ is $K(d-1)$--acyclic.
\end{proof}

\begin{corollary}[{cf.\ \Cref{ThickSubcatClassification} and \Cref{CdEqualsDd}}]
The functor \[\context{MU}(-)\co \CatOf{Spectra}^{\fin} \to \CatOf{Coh}(\context{MU})\] induces\footnote{This has to be interpreted delicately, as the functor $\context{MU}(-)$ is not (quite) a functor of triangulated categories~\cite[2.4.2]{MoravaCplxBordismInAT}.} a homeomorphism of the spectrum of $\CatOf{Spectra}^{\fin}$ to that of $\moduli{fg}$. \qed
\end{corollary}

The construction as we have described it falls short of completely recovering $\Spec R$, as we have constructed only a topological space rather than a locally ringed space (or anything otherwise equipped locally with algebraic data, as in our functor of points perspective).  The approach taken by Balmer~\cite[Section 6]{Balmer} is to use Tannakian reconstruction to extract a structure sheaf of local rings from the prime $\otimes$--ideal subcategories.  We, however, are at least as interested in finite spectra as we are the ring spectrum $\S$, so we will take an approach that emphasizes module categories rather than local rings.  Specifically, Bousfield's theory of homological localization allows us to lift the localization structure among open substacks of $\moduli{fg}$ to the category $\CatOf{Spectra}$ as follows:

\begin{theorem}[{\cite{BousfieldLocalization}, \cite[Theorem 7.7]{Margolis}}]
Let $j\co \Spec R \to \moduli{fg}$ be a flat map, and let $R_*$ denote the homology theory associated to it by \Cref{LandwebersStackyTheorem}.  There is then a diagram
\begin{center}
\begin{tikzcd}[column sep=2.2cm,row sep=2cm]
\CatOf{Spectra}_R \arrow["\context{R}(-)", "\mathrm{conservative}"', red]{r} \arrow[leftarrow, shift right=0.20cm, red, "L_R"']{d} & \CatOf{QCoh}(\context{R}) \arrow[shift right=0.20cm, red, leftarrow, "j^*"']{d} \\
\CatOf{Spectra} \arrow[leftarrow,shift right=0.20cm, "\dashv", "i"']{u} \arrow[red]{ru}[description]{\context{R}(-)} \arrow[red]{r}{\context{MU}(-)} & \CatOf{QCoh}(\context{MU}), \arrow[leftarrow, shift right=0.20cm, "\dashv", "j_*"']{u}
\end{tikzcd}
\end{center}
such that $L_R$ is left-adjoint to $i$, $j^*$ is left-adjoint to $j_*$, $i$ and $j_*$ are inclusions of full subcategories, $L_R$ and $j^*$ are idempotent, the red composites are all equal, and $R_*$ is conservative on $\CatOf{Spectra}_R$.\footnote{The meat of this theorem is in overcoming set-theoretic difficulties in the construction of $\CatOf{Spectra}_R$.  Bousfield accomplished this by describing a model structure on $\CatOf{Spectra}$ for which $R$--equivalences create the weak--equivalences.} \qed
\end{theorem}

The idea, then, is that $\CatOf{Spectra}_R$ plays the topological role of the derived category of those sheaves supported on the image of the map $j$.  In \Cref{DefnChromaticHomologyThys}, we identified several classes of interesting such maps $j$ tied to the geometry of $\moduli{fg}$.  We record these special cases now:
\begin{definition}
In the case that $R = E_\Gamma$ models the inclusion of the deformation space around the point $\Gamma$, we will denote the localizer by $L_\Gamma$.  In the special case that $\Gamma = \Gamma_d$ is taken to be the Honda formal group, we further abbreviate the localizer by \[\CatOf{Spectra} \xrightarrow{\widehat L_d} \CatOf{Spectra}_{\Gamma_d}.\]  In the case when $R = E(d)$ models the inclusion of the open complement of the unique closed substack of codimension $d$, we will denote the localizer by \[\CatOf{Spectra} \xrightarrow{L_d} \CatOf{Spectra}_d = \CatOf{Spectra}_{\moduli{fg}^{\le d}}.\]
\end{definition}

These localizers have a number of nice properties linking them to algebraic models.

\begin{lemma}
There are natural factorizations
\begin{align*}
\operatorname{id} \to L_d \to L_{d-1}, & & \operatorname{id} \to L_d \to \widehat L_d.
\end{align*}
In particular, $L_d X = 0$ implies both $L_{d-1} X = 0$ and $\widehat L_d X = 0$.
\end{lemma}
\begin{proof}[Analogy to $j_* \vdash j^*$]
The open substack of dimension $d$ properly contains both the open substack of dimension $(d-1)$ and the infinitesimal deformation neighborhood of the geometric point of height $d$.  The factorization is inclusions gives a factorization of pullback functors.
\end{proof}

\begin{lemma}[{\cite[Theorem 7.5.6]{RavenelOrangeBook}, \cite[Proof of Lemma 2.3]{HoveyCSC}}]\label{FormulaForKnLocalization}
There are equivalences
\begin{align*}
L_d X & \simeq (L_d \S) \sm X, &
\widehat L_d X & \simeq \lim_I \left( M_0(v^I) \sm L_d X \right).
\end{align*}
\end{lemma}
\begin{proof}[Analogy to $j_* \vdash j^*$]
The first formula stems from $j$ an open inclusion, which has $j^* M \simeq R \otimes M$ in the algebraic setting.  The second formula can be compared to the inclusion $j$ of the formal infinitesimal neighborhood of a closed subscheme, which has algebraic model $j^* M = \lim_j (R/I^j \otimes M)$.\footnote{In keeping with our discussion of continuous Morava $E$--theory, it is also possible to consider the object $\{\left( M_0(v^I) \sm L_d X \right)\}_I$ itself as a pro-spectrum.  This is an interesting thing to do: Davis and Lawson have shown that setting $X = \S$ gives an $E_\infty$ pro-spectrum, even though none of the individual objects are $E_\infty$ ring spectra themselves~\cite{DavisLawson}.}
\end{proof}

\begin{lemma}\label{StableMixedKthyCoopnsVanish}
Let $k$ be a field of positive characteristic $p$, and let $\Gamma$ and $\Gamma'$ be two formal groups over $k$ of differing heights $0 \le d, d', \le \infty$.  Then $K_\Gamma \sm K_{\Gamma'} \simeq 0$.
\end{lemma}
\begin{proof}[Analogy to $j_* \vdash j^*$]
The map classifying the formal group $\CP^\infty_{K_{\Gamma} \sm K_{\Gamma'}}$ simultaneously factors through the maps classifying the formal groups $\CP^\infty_{K_\Gamma} = \Gamma$ and $\CP^\infty_{K_{\Gamma'}} = \Gamma'$.  By \Cref{HeightIsAnIsomInvariant}, such a formal group must simultaneously have heights $d$ and $d'$, which forces the homotopy ring to be the zero ring.\footnote{Alternatively, \Cref{FieldSpectraAreKTheories} shows that $K_\Gamma \sm K_{\Gamma'}$ simultaneously decomposes as a wedge of $K_\Gamma$s and of $K_{\Gamma'}$, which forces both wedges to be empty.}
\end{proof}

\begin{lemma}[{\cite[Lemma 23.6]{LurieChromaticCourseNotes}}]\label{ChromaticFractureInput}
For $d > \height \Gamma$, $\widehat L_{\Gamma} L_d \simeq 0$.
\end{lemma}
\begin{proof}[Proof sketch]
After a nontrivial reduction argument, this comes down to an identical fact: the formal group associated to $E(d) \sm K_\Gamma$ must simultaneously be of heights at most $d$ and exactly $\height \Gamma > d$, which forces the spectrum to vanish.
\end{proof}

\begin{corollary}
$L_\Gamma E = 0$ for any coconnective $E$, and hence $L_\Gamma E = L_\Gamma(E[n, \infty))$ for \emph{any} spectrum $E$ and \emph{any} index $n$.\footnote{This property has the memorable slogan that Morava $K$--theories remember the ``germ at $\infty$'' of $E$.}
\end{corollary}
\begin{proof}
Any coconnective spectrum can be expressed as the colimit of its truncations
\begin{center}
\begin{tikzcd}
E[n, n] \arrow{r} \arrow[equals]{d} & E[n-1, n] \arrow{r} \arrow{d} & E[n-2, n] \arrow{d} \arrow{r} & \cdots \arrow["\colim"]{r} & E(-\infty, n] \\
\Susp^n H\pi_nE & \Susp^{n-1} H\pi_{n-1} E & \Susp^{n-2} H\pi_{n-2} E & \cdots.
\end{tikzcd}
\end{center}
Applying $L_\Gamma$ preserves this colimit diagram, but the above argument shows that $HA$ is $L_\Gamma$--acyclic for any abelian group $A$.  This gives the statement about coconnective spectra, from which the general statement follows by considering the cofiber sequence \[E[n, \infty) \to E \to E(-\infty, n). \qedhere\]
\end{proof}

\begin{corollary}[{\cite[Proposition 23.5]{LurieChromaticCourseNotes}}]\label{ChromaticFractureSquares}
There are a homotopy pullback squares
\begin{center}
\begin{tikzcd}
L_d X \arrow{r} \arrow{d} \arrow[dr, phantom, "\lrcorner", very near start] & \widehat L_d X \arrow{d} & X \arrow{r} \arrow{d} \arrow[dr, phantom, "\lrcorner", very near start] & \prod_p X^\wedge_p \arrow{d} \\
L_{d-1} X \arrow{r} & L_{d-1} \widehat L_d X, & X_{\Q} \arrow{r} & \left( \prod_p X^\wedge_p \right)_{\Q}.
\end{tikzcd}
\end{center}
\end{corollary}
\begin{proof}[Analogy to $j_* \vdash j^*$]
For the left-hand square, the inclusion of the open substack of dimension $d-1$ into the one of dimension $d$ has relatively closed complement the point of height $d$.  Algebraically, this gives a Mayer-Vietoris sequence with analogous terms.  The right-hand square is analogous to the ad\`elic decomposition of abelian groups.\footnote{Whenever $L_B L_A = 0$, $L_{A \vee B}$ appears as the homotopy pullback of the cospan $L_A \to L_A L_B \from L_B$.  Hence, this follows from \Cref{ChromaticFractureInput}, as well as the identification $L_{E(d-1) \vee K(d)} \simeq L_{E(d)}$.}
\end{proof}

\begin{remark}
\Cref{ChromaticFractureSquares} is maybe the most useful result discussed in this Lecture.  It shows that a map to an $L_d$--local spectrum can be understood as a system of compatible maps to its $\widehat L_j$--localizations, $j \le d$.  In turn, any map into an $\widehat L_j$--local object factors through the $\widehat L_j$--localization of the source.  Thus, if the source itself has chromatic properties, this often puts \emph{very} strong restrictions on how maps to the original target can behave.
\end{remark}

These functors and their properties listed thus far give a tight analogy between certain local categories of spectra and sheaves supported on particular submoduli of formal groups, in a way that lifts the six-functors formalism of $j_* \vdash j^*$ to the level of spectra.  With this analogy in hand, however, one is led to ask considerably more complicated questions whose proofs are not at all straightforward.  For instance, a useful fact about coherent sheaves on $\moduli{fg}$ is that they are completely determined by their restrictions to all of the open submoduli.  The analogous fact about finite spectra is referred to as \textit{chromatic convergence}:

\begin{theorem}[{\cite[Theorem 7.5.7]{RavenelOrangeBook}}]\label{ChromaticConvergence}
The homotopy limit of the tower \[\cdots \to L_d F \to L_{d-1} F \to \cdots \to L_1 F \to L_0 F\] recovers the $p$--local homotopy type of any finite spectrum $F$.\footnote{Spectra satisfying this limit property are said to be \textit{chromatically complete}, which is closely related to being \textit{harmonic}, i.e., being local with respect to $\bigvee_{d=0}^\infty K(d)$.  (I believe this a joke about ``music of the spheres''.)  It is known that nice Thom spectra are harmonic~\cite{Kriz} (so, in particular, every suspension and finite spectrum), that every finite spectrum is chromatically complete, and that there exist some harmonic spectra which are not chromatically complete~\cite[Section 5.1]{Barthel}.} \qed
\end{theorem}

In addition to furthering the analogy, \Cref{ChromaticConvergence} suggests a method for analyzing the homotopy groups of spheres: we could study the homotopy groups of each $L_d \S$ and perform the reassembly process encoded by this inverse limit.  Additionally, \Cref{ChromaticFractureSquares} shows that this process is inductive: $L_d \S$ can be understood in terms of the spectrum $L_{d-1} \S$, the spectrum $\widehat L_d \S$, and some gluing data in the form of $L_{d-1} \widehat L_d \S$.  Hence, we become interested in the homotopy of $\widehat L_d \S$, which is the target of the $E_d$--Adams spectral sequence considered in \Cref{StableContextLecture}.

\begin{theorem}[{\Cref{IdentifyingAdamsE2Page}, see also \Cref{Pi2AndInvariantDiffls}, \Cref{DefnOfNilpCompletionAndASS}, and \Cref{FHGivesComodules}}]
The $E_d$--based\footnote{Although the $K(d)$--Adams spectral sequence more obviously targets $\widehat L_d \S$, we have chosen to analyze the $E_d$--Adams spectral sequence above because $K(d)$ fails to satisfy {\CH}.  Starting with $BPP_0 BPP \cong BPP_0[t_0^{\pm}, t_1, t_2, \ldots]$ from \Cref{DefnChromaticHomologyThys} and \Cref{IdIsAnInvariantIdeal}, we can calculate $E(d)P_0 E(d)P$ by base-changing this Hopf algebroid: $E(d)P_0 E(d)P = E(d)P_0 \otimes_{BPP_0} BPP_0 BPP \otimes_{BPP_0} E(d)P_0$, which is again free over $E(d)P_0$.  Since $K(d)P$ is formed from $E(d)P$ by quotienting by a regular sequence, we calculate that $K(d)P_0 E(d)P$ is free over $K(d)P_0$, generated by the same summands.  However, when quotienting by the regular sequence \emph{again} to form $K(d)P_* K(d)P$, the maps in the quotient sequences act by elements in $I_d = 0$, hence introduce Bocksteins.  The end result is \[K(d)P_* K(d)P = \left(K(d)P_* \otimes_{BPP_*} BPP_* BPP \otimes_{BPP_*} K(d)P_*\right) \otimes \Lambda[\tau_0, \ldots, \tau_{d-1}],\] where $\tau_j$ in degree $1$ controls the cofiber of $E(d)P \xrightarrow{v_j} E(d)P$.} Adams spectral sequence for the sphere converges strongly to $\pi_* \widehat L_d \S$.  Writing $\omega$ for the line bundle on $\context{E_d}$ of invariant differentials, we have
\[\pushQED{\qed}
E_2^{*, *} = H^*(\context{E_d}; \omega^{\otimes *}) \Rightarrow \pi_* \widehat L_d \S. \qedhere
\popQED\]
\end{theorem}

The utility of this Theorem is in the identification of the stack $\context{E_d} \cong (\moduli{fg})^\wedge_{\Gamma_d}$ from \Cref{DefnChromaticHomologyThys}.  Our algebraic analysis from \Cref{LubinTateModuliThm} and \Cref{LubinTateModuliThmInFGLTerms} shows a further identification \[\context{E_{\Gamma_d}} = \left( \moduli{fg} \right)^\wedge_{\Gamma_d} \simeq \widehat{\mathbb A}^{d-1}_{\mathbb W(k)} \mmod \InternalAut(\Gamma_d).\]  This computation is thus boiled down to a calculation of the cohomology of the $\Aut(\Gamma_d)$--representations arising via \Cref{ActionBySnLiftsToLTn} as the global sections of the sheaves $\omega^{\otimes *}$ (cf.\ the discussion in \Cref{HopfAlgebrasFromFiniteGroups} and \Cref{HF2HomologyIsValuedInAutGaEquivarModules}).\footnote{In fact, the stable \emph{operations} of $E_d$ take the form of the twisted group-ring $E_d^0 E_d = E_d^0\<\!\<\Aut(\Gamma_d)\>\!\>$.}  We will later deduce the following polite description of $\Aut \Gamma_d$:
\begin{theorem}[{cf.\ \Cref{FormOfStabilizerGroup}}]
For $\Gamma_d$ the Honda formal group law of height $d$ over a perfect field $k$ of positive characteristic $p$, we compute \[\Aut \Gamma_d \cong \left( \W_p(k) \<S\> \middle/ \left( \begin{array}{c} Sw = w^\phi S, \\ S^d = p \end{array} \right) \right)^\times,\] where $\phi$ denotes a lift of the Frobenius from $k$ to $\W_p(k)$. \qed
\end{theorem}

\begin{remark}\label{StablizerRepIsComplicated}
As a matter of emphasis, this Theorem does not give any description of the \emph{representation} of $\Aut \Gamma_d$, which is very complicated (cf.\ \Cref{ThePeriodMapSection}).  Nonetheless, the arithmetically-minded reader might recognize this description of $\Aut \Gamma_d$ as the group of units of a maximal order $\mathbf o_D$ in the division algebra $D$ of Brauer--Hasse invariant $1/d$ over $k$---another glimpse of arithmetic geometry poking through to affect stable homotopy theory.\footnote{This finally explains our preference for using the letter ``$d$'' to represent the height of a formal group---the ``$d$'' (or, rather the ``$D$'') stands for ``division algebra''.}
\end{remark}

\begin{example}[Adams]\label{piLK1SExample}\citeme{Find a citation}
In the case $d = 1$, the objects involved are small enough that we can compute them by hand.  To begin, we have an isomorphism $\operatorname{Aut}(\Gamma_1) = \Z_p^\times$, and the action of this group on $\pi_* E_1 = \Z_p[u^\pm]$ is by $\gamma \cdot u^n \mapsto \gamma^n u^n$.  At odd primes $p$, one computes\footnote{At odd primes, $p$ is coprime to the order of the torsion part of $\Z_p^\times$.  At $p = 2$, this is not true, so the representation has infinite cohomological dimension and there is plenty of room for differentials in the ensuing $E_{\G_m}$--Adams spectral sequence..} \[H^s(\operatorname{Aut}(\Gamma_1); \pi_* E_1) = \begin{cases}\Z_p & \text{when $s = 0$}, \\ \bigoplus_{j = 2(p-1)k} \Z_p\{u^j\} / (pk u^j) & \text{when $s = 1$}, \\ 0 & \text{otherwise}. \end{cases}\]  This, in turn, gives the calculation\footnote{The groups $\pi_* \widehat L_1 \S$ are familiar to homotopy theorists: the Adams conjecture~\cite{AdamsJXIV} (and its solution) implies that the $J$--homomorphism $J_{\C}\co BU \to BGL_1 \S$ described in \Cref{DefnRealJHomomorphism} and \Cref{ComplexJHomomorphism} selects exactly these elements for nonnegative $t$.}
\[
\pi_t \widehat L_1 \S^0 = \begin{cases} \Z_p & \text{when $t = 0$}, \\ \Z_p / (pk) & \text{when $t = k|v_1| - 1$}, \\ 0 & \text{otherwise}. \end{cases}
\]
With this in hand, we can compute the homotopy of the rest of the fracture square:
\begin{center}
\begin{tikzcd}
\pi_* L_1 \S \arrow{r} \arrow{d} & \Z_p \oplus \bigoplus_{t = k|v_1| - 1} \Susp^t \Z_p/(pk) \arrow{d} \\
\Q \arrow{r} & \Q_p \oplus \Susp^{-1} \Q_p,
\end{tikzcd}
\end{center}
from which we deduce
\begin{align*}
\pi_t L_1 \S^0 & = \begin{cases} \Z_{(p)} & \text{when $t = 0$}, \\ \Z_p / (pk) & \text{when $t = k|v_1| - 1$ and $t \ne 0$}, \\ \Z/p^\infty & \text{when $t = (0 \cdot |v_1| - 1) - 1 = -2$}, \\ 0 & \text{otherwise}. \end{cases}
\end{align*}
\end{example}

\begin{example}[{\cite[Example 7.18]{Rezk512Notes}}]
We can also give an explicit chromatic analysis of the homotopy element $\eta \in \pi_1 \S$ studied in \Cref{HopfAlgebraLecture}.  As before, consider the complex $\CP^2 = \Susp^2 C(\eta)$.  We now consider the possibility that $\CP^2$ splits as $\S^2 \vee \S^4$, in which case there would be a dotted retraction in the cofiber sequence
\begin{center}
\begin{tikzcd}
\S^2 \arrow{r} & \CP^2 \arrow{r} \arrow[densely dotted, bend right, "i"']{l} & \S^4.
\end{tikzcd}
\end{center}
If this were possible, we would also be able to detect the retraction after chromatic localization---so, for instance, we could consider the cohomology theory $E_{\G_m} = KU^\wedge_p$ from \Cref{ExampleOfMoravasTheoriesAtGm} and test this hypothesis in $\G_m$--local homotopy.  Writing $t$ for a coordinate on $\CP^\infty_{KU^\wedge_p}$, this cofiber sequence gives a short exact sequence on $KU^\wedge_p$--cohomology:
\begin{center}
\begin{tikzcd}
0 & (t) / (t)^2 \arrow{l} \arrow[bend left, densely dotted, "i^*"]{r} & (t) / (t)^3 \arrow{l} & (t)^2 / (t)^3 \arrow{l} & 0. \arrow{l}
\end{tikzcd}
\end{center}
Because $i$ is taken to be a retraction, the map $i^*$ would satisfy $i^*(t) = t \pmod{t^2}$, so that $i^*(t) = t + at^2$ for some $a$.  Additionally, $i^*$ would be natural with respect to all cohomology operations on $KU^\wedge_p$.  In particular, the element $(-1) \in \Z_p^\times \cong \Aut \G_m$ gives rise to an operation $\psi^{-1}$, which acts by the $(-1)$--series on the coordinate $t$.  In the case that $t$ is the coordinate considered in \Cref{CPinftyKUExample}, this gives \[[-1](t) = -\sum_{j=1}^\infty t^j = -t - t^2 \pmod{t^3}.\]  We thus compute:
\begin{align*}
\psi^{-1} \circ i(t) & = i \circ \psi^{-1}(t) \\
\psi^{-1}(t + at^2) & = i(-t) \\
(-t - t^2) + a(-t - t^2)^2 & = -(t + at^2) \\
-t + (a - 1) t^2 & = -t - at^2,
\end{align*}
so that we would arrive at a contradiction if the equation $2a = 1$ were insoluable.  Note that this has no solution in $\Z_2$, so that the attaching map $\eta$ in $\CP^2$ is nontrivial in $\G_m$--local homotopy at the prime $2$ (hence also in the global homotopy group $\pi_1 \S$).  For $p$ odd, this equation \emph{does} have a solution in $\Z_p$, and it furthermore turns out that $\eta = 0$ at odd primes.  This problem also disappears if we require $i(t) = 2t + at^2$ instead, so that the above argument does not obstruct the triviality of $2 \eta$ (and, indeed, \Cref{HF2ASSFigure} shows that the relation $2 \eta = 0$ holds in $2$--adic homotopy).
\end{example}

\begin{example}[{\cite[Example 7.17 and Corollary 5.12]{Rezk512Notes}}]
Take $k$ to be a perfect field of positive characteristic $p$, and take $\Gamma$ over $\Spec k$ to be a finite height formal group with associated Morava $E$--theory $E_\Gamma$.  By smashing the unit map $\S \to E_\Gamma$ with the mod--$p$ Moore spectrum, we get an induced map of homotopy groups \[h_{2n}\co \pi_{2n} M_0(p) \to \pi_{2n} E_\Gamma.\]  We concluded as a consequence of \Cref{IdIsAnInvariantIdeal} that there is an invariant section $v_1$ of $\omega^{\otimes(p-1)}$ on $\moduli{fg}^{\ge 1} \to \A^1$, and hence a preferred element of $\pi_{2(p-1)} E_\Gamma$ which is natural in the choice of $\Gamma$.  One might hope that these elements are the image of an element in $\pi_{2(p-1)} M_0(p)$ under the Hurewicz map $h$, and this turns out to be true: this element is called $\alpha_{1/1}$.  This element furthermore turns out to be $p$--torsion, meaning it extends to a map
\begin{center}
\begin{tikzcd}
\S^{2(p-1)} \arrow["p"]{r} \arrow["0"]{rd} & \S^{2(p-1)} \arrow["\alpha_{1/1}"]{d} \arrow["\cofib"]{r} & M_{2(p-1)}(p) \arrow[densely dotted, "v"]{ld} \\
& M_0(p).
\end{tikzcd}
\end{center}
At odd primes, this turns out to be the $v_1$--self-map $v\co M_{2(p-1)}(p) \to M_0(p)$ announced in \Cref{AdamsSelfMapThm} (cf.\ also \cite[Proposition 12.7]{AdamsJXIV}).

More generally, different powers $v_1^j$ of the section $v_1$ also give rise to homotopy elements $\alpha_{j/1} \in \pi_{2(p-1)j} M_0(p)$.  These have varying orders of divisibility, and we write $\alpha_{j/k}$ for the element satisfying $p^{k-1} \alpha_{j/k} = \alpha_{j/1}$.  Compositionally, these maps satisfy the useful relation $\alpha_{p^{j-1}/j-1}^p = \alpha_{p^j/j}$.
\todo{Is this relation right?}
The other invariant functions described in \Cref{IdIsAnInvariantIdeal} (e.g., $v_d$ modulo $I_d$) also give rise to elements in $H^*(\moduli{fg}^{\ge d}; \omega^{\otimes *})$, which map to the $BP$--Adams $E_2$--term and which sometimes survive the spectral sequence to give to homotopy elements of the generalized Moore spectra $M_0(v^I)$.  Homotopy elements arising in this way are collectively referred to as \textit{Greek letter families}~\cite[Section 3]{MRW}.
\end{example}

\begin{remark}\label{GreekLetterElements}
In the broader literature, the phrase ``Greek letter elements'' typically refers to the pushforward of the above elements to the homotopy groups of $\S$ by pinching to the top cell.  This is somewhat obscuring: for instance, this significantly entangles how multiplication by $\alpha_{j/k}$ behaves.  Finally, the incarnation of these element in $\G_m$--local homotopy are exactly the elements witnessed by the invariant function $u^{2(p-1)k}$ in \Cref{piLK1SExample}.
\end{remark}

\todo{Another fun fact is that because $p$--local finite spectra $F$ are $\bigvee_{d < \infty} K(d)$--local, there are no nontrivial maps $H\F_p \to F$ (and, in particular, the Spanier--Whitehead dual of $H\F_p$ is null).  (In fact, the Hovey paper \textit{Bousfield Localization Functors and Hopkins' Chromatic Splitting Conjecture} shows that they are local for any properly infinite collection of Morava $K$--theories.)}
\todo{Danny was asking some interesting questions about the relationships between the $t_i$ and the $h_j$ elements of the Adams spectral sequence.  You could try to explain some of this comparison and how you get the Hopf invariant one theorem out of it.}
\todo{Danny was also asking about the role of the $T(j)$ spectra, which are pretty cool.  They (and their utility as described in \textit{From Spectra to Stacks}) might live in a Remark somewhere.}









