% -*- root: main.tex -*-

\chapter{Power operations}

\todo{I wish this had a better title.}

\todo{Write an introduction for me.}


\todo[inline]{There should be a context-based presentation of this chapter's material too.  What do contexts for structured ring spectra look like?  Why would you consider them --- what object are you trying to approximate?  How do you guess that the algebraic model is reasonable until you're aware of something like Strickland's theorem?}



\section{Isogenies, level structures}

\todo[inline]{Something that these notes routinely fail to do is to lead into the algebraic geometry in a believable way.  ``Today we're going to talk about isogenies'' --- and then, lo' and behold, isogenies appear the next day in algebraic topology.  This book would read much better if it showed how these structures were guessed to exist to begin with.}

Here's a definition of an isogeny.  Weierstrass preparation can be phrased as saying that a Weierstrass map is a coordinate change and a standard isogeny.\todo{Where does this come from?  It's Strickland's, I know that.}
\begin{definition}
Take $C$ and $D$ to be formal curves over $X$.  A map $f: C \to D$ is an \textit{isogeny} when the induced map $C \to C \times_X D$ exhibits $C$ as a divisor on $C \times_X D$ as $D$--schemes.
\end{definition}


In fact, every map in positive characteristic can be factored as a coordinate change and an isogeny, which is a weak form of preparation.


Lubin's finite quotients of formal groups. (Interaction with the Lubin--Tate moduli problem?  Or does this belong in the next day?)


Write out isogenies of the additive formal group, note that you just get the unstable Steenrod algebra again.  This is a remarkable accident.


Push and pull maps for divisor schemes


Moduli of subgroup divisors


The Drinfel'd moduli ring, level structures



\begin{lemma}\citeme{Prop 6.2 of HKR}
The following conditions on a homomorphism \[\phi: \Lambda_r^* \to F[p^r](R)\] are equivalent:
\begin{enumerate}
\item For all $\alpha \ne 0$ in $\Lambda_r^*$, $\phi(\alpha)$ is a unit (resp., not a zero-divisor).
\item The Hopf algebra homomorphism \[R\ps{x} / [p^r](x) \to R^{\Lambda_r^*}\] is an isomorphism (resp., a monomorphism). \qed
\end{enumerate}
\end{lemma}

\begin{lemma}
Let $\L_r(R)$ be the set of all group homomorphism \[\phi: \Lambda_r^* \to F[p^r](R)\] satisfying either of the conditions 1 or 2 above.  This functor is representable by a ring \[L_r(E^*) := S^{-1} E^*(B\Lambda_r)\] that is finite and faithfully flat over $p^{-1} E^*$.  (Here $S$ is generated by the $\phi(\alpha)$ with $\alpha \ne 0$, $\phi: \Lambda_r^* \to F[p^r](E^* B\Lambda_r)$ the canonical map.)
\end{lemma}






\section{Character theory for Lubin--Tate spectra}

There's a sufficient amount of reliance on character theory in Matt's thesis that we should talk about it.  You should write that action and then backtrack here to see what you need for it.

See Morava's \textit{Local fields} paper

\begin{remark}
Theorem 2.6 of Greenlees--Strickland for a nice transchromatic perspective.  See also work of Stapleton and Schlank--Stapleton, of course.\todo{Flesh this out.}
\end{remark}


------

\begin{theorem}\citeme{Theorem A}
Let $E$ be any complex-oriented cohomology theory.  Take $G$ to be a finite group and let $\CatOf{Ab}_G$ be the full subcategory of the orbit category of $G$ built out of abelian subgroups of $G$.  Finally, let $X$ be a finite $G$--CW complex.  Then, each of the natural maps \[E^*(EG \times_G X) \to \lim_{A \in \CatOf{Ab}_G} E^*(EG \times_A X) \to \int_{A \in \CatOf{Ab}_G} E^*(BA \times X^A)\] becomes an isomorphism after inverting the order of $G$.  In particular, there is an isomorphism \[\frac{1}{|G|} E^* BG \to \lim_{A \in \CatOf{Ab}_G} \frac{1}{|G|} E^* BA. \qed\]
\end{theorem}

This is an analogue of Artin's theorem:
\begin{theorem}
There is an isomorphism \[\frac{1}{|G|} R(G) \to \lim_{C \in \CatOf{Cyclic}_G} \frac{1}{|G|} R(C). \qed\]
\end{theorem}


------

HKR intro material connecting Theorem A to character theory:

Recall that classical characters for finite groups are defined in the following situation: take $L = \Q^{\mathrm{ab}}$ to be the smallest characteristic $0$ field containing all roots of unity, and for a finite group $G$ let $Cl(G; L)$ be the ring of class functions on $G$ with values in $L$.  The units in the profinite integers $\widehat{\Z}$ act on $L$ as the Galois group over $\Q$, and since $G = \CatOf{Groups}(\widehat{\Z}, G)$ they also act naturally on $G$.  Together, this gives a conjugation action on $Cl(G; L)$: for $\phi \in \widehat{\Z}$, $g \in G$, and $\chi \in Cl(G; L)$, one sets \[(\phi \cdot \chi)(g) = \phi(\chi(\phi^{-1}(g))).\]  The character map is a ring homomorphism \[\chi: R(G) \to Cl(G; L)^{\widehat{\Z}},\] and this induces isomorphisms \[\chi: L \otimes R(G) \xrightarrow{\simeq} Cl(G; L)\] and even \[\chi: \Q \otimes R(G) \xrightarrow{\simeq} Cl(G; L)^{\widehat{\Z}}.\]

Now take $E = E_\Gamma$ to be a Morava $E$--theory of finite height $d = \height(\Gamma)$.  Take $E^*(B\Z_p^d)$ to be topologized by $B(\Z/p^j)^d$.  A character $\alpha: \Z_p^d \to S^1$ will induce a map $\alpha^*: E^* \CP^\infty \to E^* B\Z_p^d$.  We define $L(E^*) = S^{-1} E^*(B\Z_p^d)$, where $S$ is the set of images of a coordinate on $\CP^\infty_E$ under $\alpha^*$ for nonzero characters $\alpha$.  Note that this ring inherits an $\operatorname{Aut}(\Z_p^d)$ action by $E^*$--algebra maps.

The analogue of $Cl(G; L)$ will be $Cl_{d,p}(G; L(E^*))$, defined to be the ring of functions $\chi: G_{d, p} \to L(E^*)$ stable under $G$--orbits.  Noting that \[G_{d,p} = \operatorname{Hom}(\Z_p^d, G),\] one sees that $\operatorname{Aut}(\Z_p^d)$ acts on $G_{d,p}$ and thus on $Cl_{d,p}(G; L(E^*))$ as a ring of $E^*$--algebra maps: given $\phi \in \operatorname{Aut}(\Z_p^d)$, $\alpha \in G_{d,p}$, and $\chi \in Cl_{d,p}(G; L(E^*))$ one lets \[(\phi \cdot \chi)(\alpha) = \phi(\chi(\phi^{-1}(\alpha))).\]

Now we introduce a finite $G$--CW complex $X$.  Let \[\operatorname{Fix}_{d, p}(G, X) = \coprod_{\alpha \in \operatorname{Hom}(\Z_p^d, G)} X^{\operatorname{im} \alpha}.\]  This space has commuting actions of $G$ and $\operatorname{Aut}(\Z_p^d)$.  We set \[Cl_{d, p}(G, X; L(E^*)) = L(E^*) \otimes_{E^*} E^*(\operatorname{Fix}_{d,p}(G, X))^G,\] which is again an $E^*$--algebra acted on by $\operatorname{Aut}(Z_p^d)$.  We define the character map ``componentwise'': a homomorphism $\alpha \in \operatorname{Hom}(\Z_p^d, G)$ induces \[E^*(EG \times_G X) \to E^*(B\Z_p^d) \otimes_{E^*} E^*(X^{\operatorname{im} \alpha}) \to L(E^*) \otimes_{E^*} E^*(X^{\operatorname{im} \alpha}).\]  Taking the direct sum over $\alpha$, this assembles into a map \[\chi_{d,p}^G: E^*(EG \times_G X) \to Cl_{d,p}(G, X; L(E^*))^{\operatorname{Aut}(Z_p^d)}.\]
\todo{Nat taught you how to say all these things with $p$--adic tori, which was \emph{much} clearer.}
\begin{theorem}\citeme{Theorem C}
The invariant ring is $L(E^*)^{\operatorname{Aut}(\Z_p^d)} = p^{-1} E^*$, and $L(E^*)$ is faithfully flat over $p^{-1} E^*$.\todo{Checking this invariant ring claim is easiest done by comparing the functors the two things corepresent.}  The character map $\chi_{d,p}^G$ induces isomorphisms
\begin{align*}
\chi_{d,p}^G \co L(E^*) \otimes_{E^*} E^*(EG \times_G X) & \xrightarrow{\simeq} Cl_{d,p}(G, X; L(E^*)), \\
\chi_{d,p}^G \co p^{-1} E^*(EG \times_G X) & \xrightarrow{\simeq} Cl_{d,p}(G, X; L(E^*))^{\operatorname{Aut}(\Z_p^d)}.
\end{align*}
In particular, when $X = *$, there are isomorphisms
\begin{align*}
\chi_{d,p}^G \co L(E^*) \otimes_{E^*} E^*(BG) & \xrightarrow{\simeq} Cl_{d,p}(G; L(E^*)), \\
\chi_{d,p}^G \co p^{-1} E^*(BG) & \xrightarrow{\simeq} Cl_{d,p}(G; L(E^*))^{\operatorname{Aut}(\Z_p^d)}. \qed
\end{align*}
\end{theorem}

------

Jack gives an interpretation of this in terms of formal $\sheaf{O}_L$--modules.

------

I also have this summary of Nat's of the classical case:

It's not easy to decipher if you weren't there for the conversation, but here's my take on it. First, the map we wrote down today was the non-equivariant chern character: it mapped non-equivariant $KU \otimes \Q$ to non-equivariant $H\Q$, periodified. The first line on Nat's board points out that if you use this map on Borel-equivariant cohomology, you get nothing interesting: $K^0(BG)$ is interesting, but $H\Q^*(BG) = H\Q^*(*)$ collapses for finite $G$. So, you have to do something more impressive than just directly marry these two constructions to get something interesting.

That bottom row is Nat's suggestion of what ``more interesting'' could mean. (Not really his, of course, but I don't know who did this first. Chern, I suppose.) For an integer $n$, there's an evaluation map of (forgive me) topological stacks \[* \mmod (\Z/n) \times \operatorname{Hom}(* \mmod (\Z/n), * \mmod G) \xrightarrow{\mathrm{ev}} * \mmod G\] which upon applying a global-equivariant theory like $K_G$ gives \[K_{\Z/n}(*) \otimes K_G(\coprod_{\text{conjugacy classes of $g$ in $G$}} *) \xleftarrow{ev^*} K_G(*).\]

Now, apply the genuine $G$-equivariant Chern character to the $K_G$ factor to get \[K_{\Z/n}(*) \otimes H\Q_G(\coprod *) \from K_{\Z/n}(*) \otimes K_G(\coprod *),\] where the coproduct is again taken over conjugacy classes in G. Now, compute $K_{\Z/n}(*) = R(\Z/n) = \Z[x] / (x^n - 1)$, and insert this calculation to get \[K_{\Z/n}(*) \otimes H\Q_G(\coprod *) = \Q(\zeta_n) \otimes (\bigoplus_{\text{conjugacy classes}} \Q),\] where $\zeta_n$ is an $n${\th} root of unity.  As $n$ grows large, this selects sort of the part of the complex numbers $\C$ that the character theory of finite groups cares about, and so following all the composites we've built a map \[K_G(*) \to \C \otimes (\bigoplus_{\text{conjugacy classes}} \C).\]  The claim, finally, is that this map sends a $G$-representation (thought of as a point in $K_G(*)$) to its class function decomposition.







\section{Descending coordinates along level structures}

Ando's Theorem 3.4.4: Let $D_j$ be the ring extension of $E_n$ which trivializes the $p^j$--torsion subgroup of $\G_{E_n}$.  Let $H$ be a finite subgroup of $\G_{E_n}(D_k)$.  There is an unstable transformation of ring-valued functors \[E_n X \xrightarrow{\Psi^H} D_j \otimes E_n X,\] and if $F$ is an Ando coordinate then for any line bundle $\L \to X$ there is a formula \[\psi^H(e\L) = \prod_{h \in H} (h +_F e\L) \in D_j \otimes E_n(X).\]

$D_j$ is Galois over $E_n$ with Galois group $\GL_n(\Z/p^j)$.  If $\rho$ is a collection of finite subgroups weighted by elements of $E_n$ which is stable under the action of the Galois group, then $\Psi^\rho$ descends to take values in just $E_n$.  (For example, the entire subgroup has this property.)

This is built by a character map.  Take $H \subseteq F(D_j)[p^j]$ to be a finite subgroup again; then there is a map \[\chi^H: E_n(D_{H^*} X) \to D_j \otimes E_n(X),\] where $D_{H^*}$ denotes the extended power construction on $X$ using the Pontryagin dual of $H$.  This composes to give an operation \[Q^H: MU^{2*}(X) \xrightarrow{P_{H^*}} MU^{2|H|*}(D_{H^*} X) \to E_n^{2|H|*}(D_{H^*} X) \xrightarrow{\chi^H} D_j \otimes E_n^{2|H|*}(X).\]  Then $Q^H$ is a ring homomorphism with effects
\begin{align*}
Q^H F^{MU} & = F/H, &
Q^H(e_{MU} \L) & = \prod_{h \in H} h +_F e\L.
\end{align*}

Then we need to factor $Q^H: MU(X) \to D_j \otimes E_n(X)$ across the orienting map $MU \to E_n$.  Since $E_n$ is Landweber flat and $Q^H$ is a ring map, it suffices to do this for the one--point space, i.e., to construct a ring homomorphism \[\Psi^H: E_n \to D_j\] so that $\Psi^H = \Psi^H(*) \otimes Q^H$.  The first condition above then translates to $\Psi^H F^{MU} = F/H$.

Matt claims that 3.2.10 is the beating heart of the paper.  Certainly it deserves mention here, since we saw the version in Quillen's paper weeks ago.

Section 3.3 is the part that uses HKR character theory.  The beginning only uses the familiar description of $BA_{E_n}$ for finite abelian $A$, but then it gets serious.






\section{The moduli of subgroup divisors}

Following... the original? Following Nat?

Continuing on from the above, if we expected $E_n$ to be $E_\infty$ (or even $H_\infty$) so that it had power operations, then we would want to understand $E_n B\Sigma_{p^j}$ and match that with the operations we see.

---

There are union maps \[B\Sigma_j \times B\Sigma_k \to B\Sigma_{j+k},\] stable transfer maps \[B\Sigma_{j+k} \to B\Sigma_j \times B\Sigma_k,\] and diagonal maps \[B\Sigma_j \to B\Sigma_j \times B\Sigma_j.\]  These induce a coproduct $\psi$ as well as products $\times$ and $\bullet$ on $E^0 \P \S^0$, where $\P\S^0 = \coprod_{j=0}^\infty B\Sigma_j$ is the free $E_\infty$--ring on $\S^0$.  This is a Hopf ring, and under $\times$ alone it is a formal power series ring.  The $\times$--indecomposables (which, I guess, are analogues of considering additive unstable cooperations) are \[Q^\times E^0 \P\S^0 = \prod_{k \ge 0} \left( E^0 B\Sigma_{p^k} / \operatorname{tr} E^0 B\Sigma_{p^{k-1}}^p \right),\] where the $k${\th} factor in the product is naturally isomorphic to $\sheaf{O}_{\Sub_{p^k}(\G)}$.  The primitives are also accessible as the kernel of the dual restriction map.

Theorem 3.2 shows that $E^0 B\Sigma_k$ is free over $E^0$, Noetherian, and of rank controlled by generalized binomial coefficients.  Prop 3.4 is the only place where work gets done, and it's all in terms of $K$--theory and HKR characters.

There's actually an extra coproduct, coming from applying $D$ to the fold map $S^0 \vee S^0 \to S^0$.

The main content of Prop 5.1 (due to Kashiwabara) is that $K_0 \P \S^0$ injects into $K_0 \OS{BP}{0}$.  Grading $K_0 \P \S^0$ using the $k$--index in $B\Sigma_k$, you can see that it's of graded finite type, so we need only know it has no nilpotent elements to see that $K_0 \P \S^0$ is $\ast$--polynomial.  This follows from our computation that $K_0 \OS{BP}{0}$ is a tensor of power series and Laurent series rings.  Corollary 5.2 is about $K_0 Q S^0$, which is the group completion of $K_0 \P \S^0$, so it's the tensor of $K_0 \P \S^0$ with a graded field.

Prop 5.6, using a double bar spectral sequence method, shows that $K^0 Q S^2$ is a formal power series algebra.  Tracking the spectral sequences through, you'll find that $Q^\times K^0 Q S^0$ agrees with $P K^0 Q S^2$.  (You'll also notice that $K^0 Q S^2$ only has one product on it, cf.\ Remark 5.4.)

Snaith's theorem says $\Sigma^\infty QX = \Sigma^\infty \P X$ for connected spaces $X$.  You can also see (just after Theorem 6.2) the nice equivalences \[\P_k S^2 \simeq B\Sigma_k^{V_k} \simeq \P_k(S^0)^{V_k},\] where superscript denotes Thom complex.  So, for a complex-orientable cohomology theory, you can learn about $\P_k S^0$ from $\P_k S^2$.  In particular, we finally learn that $E^0 \P S^0$ is a formal power series $\times$--algebra (once checking that the Thom isomorphism is a ring map).  (We already knew the homological version of this claim.)

Section 8 has a nice discussion about indecomposables and primitives, to help move back and forth between homology and cohomology.  It probably helps most with the dimension count argument below that we aren't going to get into.

Start again with $D_{p^k} S^2 \simeq B\Sigma_{p^k}^{V_{p^k}}$.  We can associate to this a divisor $\D(V_{p^k})$ on $(B\Sigma_{p^k})_E$, which we know little about, but it is classified by a map to $\Div_{p^k} \CP^\infty_E$.  This receives a closed inclusion from $\Sub_{p^k} \CP^\infty_E$, so their pullback $Z_k$ is the largest subscheme of $(B\Sigma_{p^k})_E$ over which $\D(V_{p^k})$ is a subgroup divisor.
\begin{center}
\begin{tikzcd}
H_k \arrow{rr} \arrow{dd} & & \D(V_{p^k}) \\
& Z_k \arrow{rr} \arrow{rd} & & \Sub_{p^k} \CP^\infty_E \arrow{rd} \\
\Spf E^0 B\Sigma_{p^k} / \mathrm{tr} \arrow{rr} \arrow[densely dotted]{ru} & & (B\Sigma_{p^k})_E \arrow{rr} \arrow[crossing over,leftarrow]{uu} & & \Div_{p^k} \CP^\infty_E
\end{tikzcd}
\end{center}
We will show the existence of the dashed map, implying that the restricted divisor $H_k$ is a subgroup divisor on $Y_k = \Spf E^0 B\Sigma_{p^k} / \mathrm{tr}$.

(Prop 9.1:) This proof falls into two parts: first we construct a family of maps to $(B\Sigma_{p^k})_E$ on whose image $\D(V_{p^k})$ restricts to a subgroup divisor, and then we show that the union of their images is exactly $Y_k$.  Let $A$ be an abelian $p$--subgroup of $\Sigma_{p^k}$ that acts transitively on $\{1, \ldots, p^k\}$ (i.e., it is not boosted from some transfer).  The restriction of $V_{p^k}$ to $A$ is the regular representation, which splits as a sum of characters $V_{p^k}|_A = \bigoplus_{\L \in A^*} \L$.  Identifying $BA_E = \InternalHom{FormalGroups}(A^*, \CP^\infty_E)$, $\D(V_{p^k})$ restricts all the way to $\sum_{\L \in A^*} [\phi(\L)]$, with $\phi: A^* \to $``$\Gamma(\operatorname{Hom}(A^*, \G), \G)$''.  In Finite Subgroups of Formal Groups (see Props 22 and 32), we learned that the restriction of $\D(V_{p^k})$ further to $\Level(A^*, \CP^\infty_E)$ is a subgroup divisor.  So, our collection of maps are those of the form \[\Level(A^*, \CP^\infty_E) \to \InternalHom{FormalGroups}(A^*, \CP^\infty_E) = BA_E \to (B\Sigma_{p^k})_E.\]  Here, finally, is where we have to do some real work involving Chern classes and commutative algebra, so I'm inclined to skip it in the lectures.  Finally, you do a dimension count to see that $Z_k$ and $\Spf E^0 B\Sigma_{p^k} / \mathrm{tr}$ have the same dimension (which requires checking enough commutative algebra to see that ``dimension'' even makes sense), and so you show the map is injective and you're done.


-----

Here's Neil's proof of the joint images claim.  It seems like a clear enough use of character theory that we should include it, if we can make character theory itself clear.

Recall from [18, Theorem 23] that $\Level(A^*,\G)$ is a smooth scheme, and thus that $D(A) = \sheaf O_{\Level(A^*,\G)}$ is an integral domain. Using [18, Proposition 26], we see that when $\L \in A^*$ is nontrivial, we have $\phi(\L) \ne 0$ as sections of $\G$ over $\Level(A^*, \G)$, and thus $e(\L) = x(\phi(\L)) \ne 0$ in $D(A)$. It follows that that $c_{p^k} = \prod_{\L \ne 1} e(\L)$ is not a zero-divisor in $D(A)$. On the other hand, if $A'$ is an Abelian $p$-subgroup of $\Sigma_{p^k}$ which does not act transitively on $\{1, \ldots, p^k\}$, then the restriction of $V_{p^k} − 1$ to $A'$ has a trivial summand, and thus $c_{p^k}$ maps to zero in $D(A')$. Next, we recall the version of generalised character theory described in [8, Appendix A].
\[p^{-1} E^0 BG = \left(\prod_A p^{-1} D(A)\right)^G\]
where $A$ runs over all Abelian $p$-subgroups of $G$. As $\overline R_k = E^0(B\Sigma_{p^k} )/ ann(c_{p^k} )$ and everything in sight is torsion-free, we see that $p^{−1} \overline R_k$ is the quotient of $p^{−1}E^0B\Sigma_{p^k}$ by the annihilator of the image of $c_{p^k}$ . Using our analysis of the images of $c_{p^k}$ in the rings $D(A)$, we conclude that
\[p^{-1} \overline R_k = \left(\prod_A p^{−1}D(A)\right)^{\Sigma_{p^k}},\]
where the product is now over all transitive Abelian $p$-subgroups. This implies that for such $A$, the map $E^0B\Sigma_{p^k} \to D(A)$ factors through $\overline R_k$, and that the resulting maps $\overline R_k \to D(A)$ are jointly injective. This means that $Y_k = \Spf \overline R_k$ is the union of the images of the corresponding schemes $\Level(A^*,\G)$, as required.








\section{Interaction with $\Theta$--structures}

The Ando--Hopkins--Strickland result that the $\sigma$--orientation is an $H_\infty$--map

The main classical point is that an $MU\<0\>$--orientation is $H_\infty$ when the following diagram commutes for every choice of $A$:
\begin{center}
\begin{tikzcd}
(BA^* \times \CP^\infty)^{V_{reg} \otimes \L} \arrow{r} & D_n MU\<0\> \arrow{d} \arrow{r} & D_n E \arrow{d} \\
& MU\<0\> \arrow{r} & E
\end{tikzcd}
\end{center}
(This is equivalent to the condition given in the section on Matt's thesis.  In fact, maybe I should try writing this so that Matt's thesis uses the same language?)  If you write out what this means, you'll see that a given coordinate on $E$ pulls back to give two elements in the $E$--cohomology of that Thom spectrum (or: sections of the Thom sheaf), and the orientation is $H_\infty$ when they coincide.

Similarly, an $MU\<6\>$--orientation corresponds to a section of the sheaf of cubical structures on a certain Thom sheaf.  Using the $H_\infty$ structures on $MU\<6\>$ and on $E$ give two sections of the pulled back sheaf of cubical structures, and the $H_\infty$ condition is that they agree for all choices of group $A$.




\todo{Section 12.4 compares doing $H_\infty$ descent with doing $E_\infty$ descent and shows that they're the same (in the case of interest?).}




\subsection*{Other stuff that goes in this chapter}

Neil's \textit{Finite Subgroups of Formal Groups} has (in addition to lots of results) a section 14 where he talks about the action of a generalized Hecke algebra on the $E$--theory of a space.

Dyer--Lashof operations, the Steenrod operations, and isogenies of the formal additive group \citeme{See Neil's \textit{Steenrod algebra} note, maybe? Talk to Mike?}

Another augmentation to the notion of a context: working not just with $E_* X$ but with $E_*(X \times BG)$ for finite $G$.
