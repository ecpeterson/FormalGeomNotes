% -*- root: main.tex -*-

\section{Conventions}

Throughout this book, we use the following conventions:

\begin{itemize}
\item $\CatOf{C}(X, Y)$ will denote the mapping object of arrows $X \to Y$ in a category $\CatOf{C}$.  If $\CatOf{C}$ is an $\infty$--category, this will often be interpreted as a mapping \emph{space}.  If $\CatOf{C}$ has a self-enrichment, we will often write $\underline{\CatOf{C}}(X, Y)$ (or, e.g., $\InternalAut(X)$) to distinguish the internal mapping object from $\CatOf{C}(X, Y)$ the classical mapping set.
\item Following Lurie, for an object $X \in \CatOf C$ we will write $\CatOf C_{/X}$ for the slice category of objects \emph{over} $X$ and $\CatOf C_{X/}$ for the slice category of objects \emph{under} $X$.
\item For a ring spectrum $E$, we will write $E_* = \pi_* E$ for its coefficient ring, $E^* = \pi_{-*} E$ for its coefficient ring with the opposite grading, and $E_0 = E^0 = \pi_0 E$ for the $0${\th} degree component of its coefficient ring.  In particular, this allows us to make sense of expressions like ``$E^*\ps{x}$'', which we interpret as \[E^*\ps{x} = (E^*)\ps{x} = (\pi_{-*} E)\ps{x}.\]
\item For a space or spectrum, we will write $X[n, \infty) \to X$ for the upward $n${\th} Postnikov truncation over $X$ and $X \to X(-\infty, n)$ for the downward $n${\th} Postnikov truncation under $X$.  There is thus a natural fiber sequence \[X[n, \infty) \to X \to X(-\infty, n).\]  This notation extends naturally to objects like $X(a, b)$ or $X[a, b]$, where $-\infty \le a \le b \le \infty$ denote the (closed or open) endpoints of any interval.\todo{This notation is visible in Greenlees--May, but probably doesn't originate there.}
\end{itemize}
