% -*- root: main.tex -*-

\section{Conventions}

Throughout this book, we use the following conventions:

\begin{itemize}
\item Categories will be consistently typeset as in the examples \[\CatOf{Spaces}, \quad \CatOf{FormalGroups}, \quad \CatOf{GradedHopfAlgebras}.\]
\item $\CatOf{C}(X, Y)$ will denote the mapping set of arrows $X \to Y$ in a category $\CatOf{C}$.  If $\CatOf{C}$ is an $\infty$--category, this will be interpreted instead as a mapping \emph{space}.  If $\CatOf{C}$ has a self-enrichment, we will often write $\underline{\CatOf{C}}(X, Y)$ (or, e.g., $\InternalAut(X)$) to distinguish the internal mapping object from the classical mapping set $\CatOf{C}(X, Y)$.  As a first exception to this uniform notation, we will sometimes abbreviate $\underline{\CatOf{Spaces}}(X, Y)$ to $F(X, Y)$, and similarly we will sometimes abbreviate $\underline{\CatOf{Spectra}}(X, Y)$ to $F(X, Y)$, with ``$F$'' short for ``function''.  As a second exception, for two formal groups $\G$ and $\widehat{\mathbb H}$, we denote the function scheme by \[\InternalHom{FormalGroups}(\G, \widehat{\mathbb H}),\] even though this is a \emph{scheme} rather than a \emph{formal scheme}.
\item Following Lurie, for an object $X \in \CatOf C$ we will write $\CatOf C_{/X}$ for the slice category of objects \emph{over} $X$ and $\CatOf C_{X/}$ for the slice category of objects \emph{under} $X$.
\item For a spectrum $E$, we will write $E^*(X)$ for the unreduced $E$--cohomology of a space $X$ and $E_*(X)$ for the unreduced $E$--homology of $X$.  We denote the reduced $E$--cohomology of a pointed space $X$ by $\widetilde E^*(X)$ and the reduced $E$--homology by $\widetilde E_*(X)$.  Finally, for $F$ another spectrum, we write $E^*(F)$ and $E_*(F)$ for the $E$--cohomology and $E$--homology respectively of $F$.  Altogether, these satisfy the relations \[E_*(X) = E_*(\Susp^\infty_+ X) = E_*(\Susp^\infty X) \oplus E_* = \widetilde E_*(X) \oplus E_*,\] and similarly for cohomology.
\item For a spectrum $E$, we will write $\OS{E}{n}$ for the $n${\th} space in the $\Omega$--spectrum representing $E$.  The homotopy type of this space is determined by the formula \[h\CatOf{Spaces}(X, \OS{E}{n}) = h\CatOf{Spaces}(X, \Loops^\infty \Susp^n E) = h\CatOf{Spaces}(\Susp^\infty X, \Susp^n E) = \widetilde E^n(X).\]
\item For a ring spectrum $E$, we will write $E_* = \pi_* E$ for its coefficient ring, $E^* = \pi_{-*} E$ for its coefficient ring with the opposite grading, and $E_0 = E^0 = \pi_0 E$ for the $0${\th} degree component of its coefficient ring.  This allows us to make sense of expressions like ``$E^*\ps{x}$'', which we interpret as \[E^*\ps{x} = (E^*)\ps{x} = (\pi_{-*} E)\ps{x} = \left\{ \sum_{j=0}^\infty a_j x^j \middle| \begin{array}{c} \text{$a_j$ is of degree $* -j|x|$} \\ \text{for some fixed degree $*$} \end{array} \right\}.\]
\item For a space or spectrum, we will write $X[n, \infty) \to X$ for the upward $n${\th} \index{Postnikov tower}Postnikov truncation over $X$ and $X \to X(-\infty, n)$ for the downward $n${\th} Postnikov truncation under $X$.  There is thus a natural fiber sequence \[X[n, \infty) \to X \to X(-\infty, n).\]  This notation extends naturally to objects like $X(a, b)$ or $X[a, b]$, where $-\infty \le a \le b \le \infty$ denote the (closed or open) endpoints of any interval.
\item We will write $S^n$ for the $n${\th} sphere when considered as a space and $\S^n$ for its suspension spectrum.  We will often abbreviate $\S^0$ to simply $\S$.
\item We prefer the notation $\sheaf O_X$ for the ring of functions on a scheme $X$ and $\sheaf I_D$ for ideal sheaf determined by a subscheme $D$, but we will also denote these by the synonyms $\sheaf O(X)$ and $\sheaf I(D)$ when the subscripts reach sufficient complexity.
\item We write $KO$ and $KU$ for periodic real and complex $K$--theory, and we write $kO$ and $kU$ for their respective connective variants.  (Other authors write $ko$ and $ku$, or $bo$ and $bu$, or even the ill-advised $BO$ and $BU$ for these spectra.)
\item We primarily treat \(2\)--periodic spectra, though ``in the wild'' many of the spectra we consider here are taken to have lower periodicity (e.g., \(E(d)\) is typically taken to have periodicity \(2(p^d-1)\)) or no periodicity at all (e.g., the ordinary homology spectrum \(\HFtwo\)).  Where confusion might otherwise arise, we have done our best to insert a ``\(P\)'' into the names of our standard spectra as clear indication that we are speaking about the \(2\)--periodic version.
\end{itemize}

\noindent In all these cases, I have done my best to be absolutely consistent in these regards, and I apologize profusely for any erratic typesetting that might have slipped through.
