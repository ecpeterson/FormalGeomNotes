% -*- root: main.tex -*-

\chapter{The $\sigma$-orientation}


\todo[inline]{Write an introduction for me.  Use unstable cooperations from Morava's theories to classical complex and real $K$--theory.}

\todo[inline]{Part of the theme of this chapter should be to use the homomorphism from topological vector bundles to algebraic line bundles --- Neil's $\mathbb L$ construction --- as inspiration for what to do, given suitable algebraic background.}





\section{The complex Atiyah--Bott--Shapiro orientation}

The analysis of $BSU_K$, purely in terms of special divisors.

The phrasing of $MUP$ and $MU$ orientations in terms of $C^0$ and $C^1$--structures


I'm not sure what story I mean to tell on this day.  I want to get some low-order version of the story out of the way, to use as a template for the topological side of things.





\section{Elliptic curves and $\theta$--functions}

-------

This is just a list of theorems and ideas, not a coherent lecture.  I guess the point of this lecture is that this material is somewhat ``standard training'' for an algebraic geometer (or, perhaps, any algebraically-inclined graduate student), and so it's not strange for these constructions to be in the front of someone's mind.  This will matter when we start to analyze $kU^{2n}(\CP^\infty)^{\times n}$ and come across familiar constructions all over again.  So, the goal should be to introduce the players in this story in its own context, but to put emphasis on the ``shapes'' of things.

\begin{theorem}[Theorem of the cube]\label{TheoremOfTheCube}\citeme{Milne's Abelian Varieties chapter, Theorem 6.1}\todo{This is stated far more generally than we care.}
Let $U$, $V$, and $W$ be complete varieties over $k$ with base points $u_0 \in U(k)$, $v_0 \in V(k)$, $w_0 \in W(k)$.  An invertible sheaf $\L$ on $U \times V \times W$ is trivial if its restrictions to $\{u_0\} \times V \times W$, $U \times \{v_0\} \times W$, and $U \times V \times \{w_0\}$ are all trivial. \qed
\end{theorem}

\begin{corollary}\label{Theta3IsTrivial}\citeme{Milne's Abelian Varieties chapter, Corollary 6.4}
Let $A$ be an abelian variety, let $p_i: A \times A \times A \to A$ be the projection onto the $i${\th} factor, and let $p_{ij} = p_i +_A p_j$, $p_{ijk} = p_i +_A p_j +_A p_k$.  Then for any invertible sheaf $\L$ on $A$, the sheaf \[\Theta^3(\L) := \frac{p_{123}^* \L \otimes p_1^* \L \otimes p_2^* \L \otimes p_3^* \L}{p_{12}^* \L \otimes p_{23}^* \L \otimes p_{31}^* \L}\] on $A \times A \times A$ is trivial.  If $\L$ is rigid, then $\Theta^3(\L)$ is canonically trivialized. \qed
\end{corollary}

\begin{theorem}[Theorem of the square]\label{TheoremOfTheSquare}
For all invertible sheaves $\L$ on an abelian variety $A$ and points $a, b \in A(k)$, \[\L \otimes t_{a+b}^* \L \cong t_a^* \L \otimes t_b^* \L.\]
\end{theorem}
\begin{proof}
Pull back the sheaf in \Cref{Theta3IsTrivial} along the map \[A \xrightarrow{(\id, a, b)} A \times A \times A\] to get the canonically trivialized sheaf \[\frac{t_{a+b}^* \L \otimes \L \otimes a^* \L \otimes b^* \L}{t_a^* \L \otimes (a+b)^* \L \otimes t_b^* \L}.\]  Cancelling the constant factors gives the theorem.
\end{proof}

\begin{remark}\citeme{Remarks 6.8-9 of Milne's Abelian Varities chapter}
Fixing $\L$, the assignment \[a \in A(k) \mapsto t_a^* \L \otimes \L^{-1} \in \Pic(A)\] is a homomorphism.  To see this, tensor both sides of \Cref{TheoremOfTheSquare} with $\L^{-2}$.  If we write $D \simeq D'$ for linear equivalence of divisors (i.e., for $\L(D) \cong \L(D')$), then we have \[t_{a+b} D + D = t_a D + t_b D\] and hence for $\sum_{i=1}^n{}^A a_i = 0$ it follows that \[\sum_{i=1}^n t_{a_i} D = nD.\]
\end{remark}

\begin{theorem}\citeme{Strip out the easy parts of Milne's Theorem 7.1}
Every elliptic curve is projective.
\end{theorem}
\begin{proof}
First, assume that the ground field $k$ is algebraically closed.  We aim to construct a very ample linear system, meaning that it separates points (for every pair $a, b$ of distinct closed points on the variety, there is a $D$ in the system with $a \in D$ but $b \not\in D$) and it separates tangent vectors (for every closed point $a$ and tangent vector $t$ at $a$, there is a $D$ in the system with $a \in D$ but $t \not\in T_a D$).  Consider just the condition that a linear system separates the origin point from all other closed points in $A$; the case of a curve is too low-dimensional for this is to be interesting, as we can pick our linear system to consist of only the point divisor $D = \{0\}$.  If $a$ and $b$ are any two points in $A(k)$, then one can always find a third point $c$ so that $D_a + D_c + D_{-c-a}$ misses $b$ --- that is, the duplicated and translated system of divisors separates $a$ from $b$.  The theorem of the cube shows that $D_a + D_c + D_{-c-a} = 3D$, regardless of $a$ and $c$, so we take $3D$ as the basic unit of our system of divisors.  Finally, looking around II.7.8.2 of Hartshorne\todo{Expand this out.} shows that this gives rise to the appropriate projective embedding.

Finally, one can show that the projectivity of $A_{\bar k}$ entails the projectivity of $A_k$.\todo{Expand me?}
\end{proof}

\begin{remark}
Let $D = 3\{0\}$ and choose a suitable basis $\{1, x, y\}$ of $H^0(A; \L(3D))$. Then the map $A \to \P^2$ defined by the basis identifies $A$ with the cubic projective curve \[Y^2 Z + a_1 XYZ + a_3 YZ^2 = X^3 + a_2 X^2 Z + a^4 X Z^2 + a_6 Z^3.\] \todo[inline]{Actually, I think that the Weierstrass p function gives this presentation? The $\theta$--function presentation writes an elliptic curve as the intersection of two quadraics in $\P^3$ rather than in $\P^2$.}
\end{remark}

Every abelian variety is projective (Theorem 7.1 of Milne's \textit{Abelian varieties (short)})

Elliptic $\theta$--functions: every elliptic curve embeds into the \emph{same} projective space

$\Theta$--structures and $\theta$--functions for general abelian varieties: continuously embedding the universal continuous family of abelian varieties into a single projective space.

Corollary 7.2 of Milne's chapter is that every abelian variety has a \emph{symmetric} ample invertible sheaf.  (Remarks after that say that any ample sheaf becomes very ample after cubing it. That's really what the theorem of the cube's use in the above proof is saying.)

Entire elliptic functions are constant.  The $\theta$--function is a quasiperiodic function (so, section of a nontrivial line bundle) with a unique simple zero and no poles (so, suitable for building functions with more complicated divisor sets).  The field of elliptic functions (so, allowing poles) is generated by Weierstrass p and its derivative.


Section--preserving maps of abelian schemes are automatically homomorphisms: Olsson's Prop 1.11.

The theorem of the cube: $\Theta^3 \L$ is canonically trivial for any rigidified line bundle $\L$ on an abelian scheme $A$.  You can find this in Olsson Thm 2.2.  Or, Akhil Mathew has notes from an algebraic geometry class ( https://math.berkeley.edu/{\textasciitilde}amathew/232b.pdf ) where lectures 3--5 address the theorem of the cube.

$\Theta$--structures and symmetric biextensions

Weil pairings: checks how far some extension is from being commutative. You can see this in Olsson 3.11 or Mumford's \textit{Abelian varieties} p.\ 222.

-----

Some special function stuff:

\[\wp_L(z) = \frac{1}{z^2} + \sum_{\omega \in L \setminus \{0\}} \frac{1}{(z - \omega)^2} - \frac{1}{\omega^2}.\]
\[\zeta_L(z) = \frac{1}{z} + \sum_{\omega \in L \setminus \{0\}} \frac{1}{z - \omega} + \frac{1}{\omega} + \frac{z}{\omega^2}.\]
\[\wp_L'(z) = -2 \sum_{\omega \in L \setminus \{0\}} \frac{1}{(z - \omega)^3}.\]
$\zeta' = -\wp$.  $\wp$ is elliptic, and in fact the field of elliptic functions on $\C / L$ is spanned by $\wp$ (the even elliptic function) and $\wp'$ (the odd elliptic function) over the constants.

There is a function $\sigma$ with \[\zeta = \frac{\sigma'}{\sigma} = \frac{d}{dz} \log \sigma.\]  Any elliptic function can be written as $c \cdot \prod_{i=1}^n \frac{\sigma(z - a_i)}{\sigma(z - b_i)}$.

There is a differential equation \[\wp_L'(z)^2 = 4 \wp_L(z)^3 - g_2(L) \wp_L(z) - g_3(L)\] for constants $g_2(L)$ and $g_3(L)$.  Accordingly, let $E$ be the elliptic curve with homogeneous equation $wy^2 = 4x^3 - g_2(L) w^2 x - g_3(L) w^3$; then there is an analytic group isomorphism
\begin{align*}
\C / L & \to E(\C), \\
z \mod L & \mapsto (1, \wp_L(z), \wp_L'(z)).
\end{align*}

Now exponentiate, so that $\C / L$ is replaced by $\C^\times / q^{\Z}$ for some $0 < |q| < 1$ and $q = e^{2 \pi i \tau}$.  Define $\theta$ as \[\theta_\tau(z) = \sum_{n \in \Z} e^{2 \pi i n z + \pi i n^2 \tau}.\]  Then $\theta_\tau(z) = \theta_\tau(-z)$ and \[\theta_\tau(z + a \tau + b) = e^{-\pi i a^2 \tau - 2 \pi i a z} \theta_\tau(z),\] so that $\theta_\tau$ is periodic against $1$ and quasiperiodic against $\tau$.  It has no poles and its lone zero is at $1/2 + \tau/2$, the center of the fundamental parallelogram.  Using this, we define auxiliary $\theta$ functions for each choice of rationals $a$ and $b$: \[\theta_\tau^{a,b}(z) = e^{\pi i a^2 \tau + 2 \pi i a (z + b)} \theta_\tau(z + a \tau + b).\]  For any $N > 0$, we define $V_\tau[N]$ to be the space of entire functions $f$ with $f(z + N) = f(z)$ and $f(z + \tau) = e^{-2 \pi i N z - \pi i N^2 \tau} f(z)$.  Then $V_\tau[N]$ has $\C$--dimension $N^2$ and $\theta_\tau^{a, b}$ give a basis by picking representatives $(a_i, b_i)$ of the classes in $((1/N)\Z / \Z)^2$.  We build a map
\begin{align*}
\C / N(\Z \tau + \Z) & \to \P^{N^2-1}(\C), \\
z & \mapsto [\cdots : \theta_\tau^{a_i, b_i}(z) : \cdots].
\end{align*}
For $N > 1$, it is an embedding.

One can work out how it goes for $N = 2$: the four functions there are
\begin{align*}
\theta_\tau^{0, 0}(z) & = \theta_\tau(z), & \mathrm{zeroes} & = \frac{\tau + 1}{2} + L_\tau, \\
\theta_\tau^{0, 1/2}(z) & = \theta_\tau(z + 1/2), & \mathrm{zeroes} & = \frac{\tau}{2} + L_\tau, \\
\theta_\tau^{1/2, 0}(z) & = e^{\frac{1}{4} \pi i \tau + \pi i z} \theta_\tau(z + \tau/2), & \mathrm{zeroes} & = \frac{1}{2} + L_\tau, \\
\theta_\tau^{1/2, 1/2}(z) & = e^{\frac{1}{4} \pi i \tau + \pi i (z + \frac{1}{4})} \theta_\tau\left(z + \frac{\tau + 1}{2}\right), & \mathrm{zeroes} & = L_\tau.
\end{align*}
The image of $f_{(2)}$ in $\P^{2^2 - 1}(\C)$ is cut out by the equations
\begin{align*}
A^2 x_0^2 & = B^2 x_1^2 + C^2 x_2^2, \\
A^2 x_3^2 & = C^2 x_1^2 - B^2 x_2^2,
\end{align*}
where
\begin{align*}
x_0 & = \theta_\tau^{0, 0}(2z), &
x_1 & = \theta_\tau^{0, 1/2}(2z), \\
x_2 & = \theta_\tau^{1/2, 0}(2z), &
x_3 & = \theta_\tau^{1/2, 1/2}(2z)
\end{align*}
and
\begin{align*}
A & = \theta_\tau^{0, 0}(0) = \sum_n q^{n^2}, \\
B & = \theta_\tau^{0, 1/2}(0) = \sum_n (-1)^n q^{n^2}, \\
C & = \theta_\tau^{1/2, 0}(0) = \sum_n q^{(n + 1/2)^2}
\end{align*}
upon which there is the additional ``Jacobi'' relation \[A^4 = B^4 + C^4.\]

In fact, the second derivative of the logarithm of $\theta_\tau^{1/2, 1/2}$ gives $\wp$, so the logarithmic derivative of $\theta_\tau^{1/2,1/2}$ is the Weierstrass $\zeta$--function.  It follows that $\theta_\tau^{1/2,1/2}$ is ``essentially'' the same as $\sigma$.

What's left to copy over is in sections 10.5-6 of Husemoller's \textit{Elliptic curves}.

-------

Here's what AHS has to say about things (in Section 2.6):

Start by thinking complex-analytically, so that $D'$ is the punctured unit-radius complex disk and $C'_{an}$ is the Tate curve over it: \[C'_{an} = \C^\times \times D' / (u, q) \sim (qu, q).\]  Meromorphic functions on $C'_{an}$ are meromorphic functions $f$ on $C^\times \times D'$ satisfying $f(u, q) = f(qu, q)$.  The pullback of the ideal sheaf $\sheaf I'(0)$ of functions vanishing at the origin on $C'_{an}$ to $\C^\times \times D'$ is the line bundle whose holomorphic sections are functions vanishing at the points $(q^{\Z}, q)$.  We can naively define such a function with simple zeroes on this locus: \[\tilde \theta(u, q) = \prod_{n \in \Z}(1 - q^n u).\]  This gives a trivialization of the pulled-back line bundle, but it doesn't descend to $C'_{an}$, since it doesn't satisfy the right functional equation --- instead, it satisfies \[\tilde \theta(qu, q) = -u^{-1} \tilde\theta(u, q).\]  However, they claim it's easy to check that $\delta^3 \tilde\theta(u, q)$ does descend to a rigid trivialization of $\Theta^3(\sheaf I(0))$, so gives the unique cubical structure on $C'_{an}$.

Next, we can provide an actual equation for the Tate curve, which will show us how to compactify it at the puncture point.  Make the definitions
\begin{align*}
\sigma_k(n) & = \sum_{d \mid n} d^k, &
\alpha_k & = \sum_{n > 0} \sigma_k(n) q^n,
\end{align*}
so that the Weierstrass cubic \[y^2 + xy = x^3 + -5\alpha_3 x + -\frac{5\alpha_3 + 7\alpha_5}{12}\] presents a projective plane curve.  This transforms isomorphically to the Tate curve by the coordinates
\begin{align*}
x & = \frac{u}{(1 - u)^2} + \sum_{n > 0} q^n \sum_{d \mid n} d(u^d - 2 + u^{-d}), \\
y & = \frac{u^2}{(1 - u)^3} + \sum_{n > 0} q^n \sum_{d \mid n} \frac{d}{2} ((d - 1) u^d + 2 - (d+1)u^{-d}).
\end{align*}
The Weierstrass version of the equation makes sense at $q = 0$ (although it becomes singular), so we get a family $C_{an}$ of generalized elliptic curves over the open unit disk $D$.  (The fiber over $q = 0$ is the twisted cubic $y^2 + xy = x^3$.)  The invariant differential is $dx / (2y + x) = du / u$, and the cubical structure is given by the unique extension of $\delta^3 \tilde\theta(u, q)$.

Now let $A \subseteq \Z\ps{q}$ by the subring of power series which converge absolutely on the open unit disk; then the coefficients of the Weierstrass cubic lie in $A$, so it determines a generalized elliptic curve $C$ over $\Spec A$, and $C_{an}$ is the curve given by base-change from $A$ to the ring of holomorphic functions on $D$.  The Tate curve $C_{\Tate} $is defined to be the generalized elliptic curve over the intermediate object $D_{\Tate} = \Spec \Z\ps{q}$ as base-changed from $A$.  Since meromorphic sections of $\Theta^3 \sheaf I(0)$ on $C^3$ inject into such over $C_{an}^3$, the (transcendental) equation $s(C_{an}^3) = \delta^3 \tilde\theta(u, q)$ nonetheless determines the cubical structure on $\sheaf I(0)$ over $C$ and hence over $C_{\Tate}$ as well.

Finally, we turn to the associated formal groups.  First, the quotient map \[\C^\times \times D' = \Gm \times D' \to C_{an}'\] is a local analytic isomorphism, so it restricts to an isomorphism \[\G_m \times D' \to \widehat C_{an}'\] and extends across the origin to an isomorphism \[\G_m \times D \to \widehat C_{an}.\]  Additionally, even though $\tilde\theta$ does not descend to $C_{an}$, it does descend to $\widehat C_{an}$ --- and there it gives a coordinate.  Hence, $s(\widehat C_{an}) = \delta^3 \tilde\theta(u, q)$, where $\tilde\theta$ is treated directly as a coordinate on $\widehat C_{an}$.

Lastly, we compare the three coordinates we have available on $\widehat C_{an}'$:
\begin{align*}
t & = x/y, &
\tilde\theta(u, q) & = \tilde\theta(t), &
1 - u & = 1 - u(t).
\end{align*}
Of these, only $t$ gives an algebraic coordinate on $C_{an}'$ (and in fact on $C_{an}$).  As power series in $t$,
\begin{align*}
\tilde\theta(t) & = t + O(t^2), &
1 - u(t) & = t + O(t^2).
\end{align*}
The coefficients of the powers of $t$ in these series are holomorphic functions on the punctured disk $D'$.  In fact, they extend to $D$ and have integer coefficients (which you see by working over the completion of $\Z[u^pm]\ps{q}$ at $(1 - u)$).  Thus $\tilde\theta(t)$ and $u(t)$ actually lie in $A\ps{t}$, so give functions on $\widehat C$ and $\widehat C_{\Tate}$.  It's supposed to follow, finally, that $s(\widehat C/A) \in C^3(\widehat C; \sheaf I(0))$ is $\delta^3 \tilde\theta(t)$, using all the definitions above.








\section{The Ando--Strickland analysis}

\todo[inline]{Hopefully by this point we've gone through the comparison of $BSU_K$ with $\SDiv_0 \G$. Since we're moving to $C_*$ schemes, we should include a description of the comparison map between $\SDiv_0$ and $C_2$.}

\begin{lemma}\label{EasyCompatibilityWithdn}\citeme{Lemma 4.5 of AS}
There is a unique $\lambda \in \Z/n$ such that the following triangle commutes:
\begin{center}
\begin{tikzcd}
(B\Z/p^j)^{\times 2} \arrow{d}{\lambda \beta \mu} \arrow{rd}{d_n(\L_1, \L_2)} \\
K(\Z, 3) \arrow{r}{i} & BU[6, \infty).
\end{tikzcd}
\end{center}
\end{lemma}
\begin{proof}
Recall the definition of $d_n(\L_1, \L_2)$:
\begin{align*}
d_n(\L_1, \L_2) & := \sum_{k=1}^{p^j-1} \left( (\L_1 - 1)(\L_1^k - 1)(\L_2 - 1) - (\L_1 - 1)(\L_2^k - 1)(\L_2 - 1)\right). \\
\intertext{Forgetting down to $BU$ and working with just virtual bundles rather than lifts of virtual bundles to elements of $kU^6$, this gives:}
& = (\L_1 - 1)(\L_2^{p^j} - 1) - (\L_1^{p^k} - 1)(\L_2 - 1).
\end{align*}
Since we have restricted to $(B\Z/p^j)^{\times 2}$, $\L_1^{p^k} = 1$ and $\L_2^{p^k} = 1$, so this formula collapses and the composite map \[(B\Z/p^j)^{\times 2} \xrightarrow{d_n} BU[6, \infty) \xrightarrow{\pi_2} BSU \xrightarrow{\pi_2} BU\] is null.  We now study the lifting problem across the two maps:
\begin{enumerate}
\item[$\pi_1$:] Since the long composite is null, it follows that the shorter composite $\pi_2 \circ d_n$ factors through the fiber of $\pi_2$.  But, the fiber of $\pi_2$ is $S^1$, and there are no maps \[\{(B\Z/p^j)^{\times 2} \to S^1\} \cong H^1((B\Z/p^j)^{\times 2}; \Z) = 0.\]  It follows that $\pi_2 \circ d_n$ is itself null.
\item[$\pi_2$:] Since $\pi_2 \circ d_n$ is null, $d_n$ must factor through the fiber of $\pi_2$, which is $K(\Z, 3)$.  With identical motives, one considers $H^3((B\Z/p^j)^{\times 2}; \Z)$, which is cyclic of order $n$ and generated by $\beta \circ \mu$. \qedhere
\end{enumerate}
\end{proof}

\begin{theorem}\citeme{AS Theorem 4.2, proved in Section 5}
The value $\lambda$ can be taken to be $\pm 1$ in \Cref{EasyCompatibilityWithdn}. \qed
\end{theorem}

\begin{corollary}\citeme{Corollary 4.4 of AS}
The following square commutes (up to sign):
\begin{center}
\begin{tikzcd}
BU[6, \infty)^E \arrow{r}{\gamma^E} \arrow{d}{\Pi_3} & K(\Z, 3)^E \arrow{d}{b_*} \\
C^3(\CP^\infty_E; \mathbb G_m) \arrow{r}{e} & \Weil(\CP^\infty_E).
\end{tikzcd}
\end{center}
\end{corollary}
\begin{proof}
We will check compatibility on $\Weil_n(\CP^\infty_E)$ for arbitrary $n$.  (Note: the sign can't bounce with $n$ because $\CP^\infty_E$ is $p$--divisible.)  Since $\Weil_n(\CP^\infty_E)$ is a subscheme of $\InternalHom{FormalSchemes}(\CP^\infty_E[n]^{\times 2}, \Gm)$, we can push forward to checking equality here, i.e., of the two maps \[\CP^\infty_E[n]^{\times 2} \times BU[6, \infty)^E \to \Gm.\]

The construction of adjoint elements from maps of spectra converts sums to products and is natural in the source spectrum.  Writing $z = \prod_{i=1}^3 (1 - \L_i) \in kU^6 (\CP^\infty)^{\times 3}$, it follows that the adjoint map $\hat z$ is given by the composite \[\hat z \co (\CP^\infty_E)^{\times 3} \times BU[6, \infty)^E \to (\CP^\infty_E)^{\times 3} \times C^3(\CP^\infty_E; \Gm) \xrightarrow{\operatorname{eval}} \Gm.\]  It follows by naturality that if $z = (1 - \L_1)(1 - \L_1^k)(1 - \L_2) \in kU^6 (\CP^\infty)^{\times 2}$, then $\hat z$ corresponds to the map \[\hat z \co (g_1, g_2, f) \mapsto f(g_1, kg_1, g_2),\] and continuing along these lines we see \[d_n(\L_1, \L_2)\hat{} \co (g_1, g_2, f) \mapsto \prod_{k=1}^{n-1} \frac{f(g_1, kg_1, g_2)}{f(g_1, kg_2, g_2)} = e_n(f)(g_0, g_1).\]

That described the bottom-left arm of the square.  For the other arm, take $w = \beta \circ \mu \in H^3(B\Z/p^j)^{\times 2}$ with adjoint $\hat w \co \CP^\infty_E[p^j]^{\times 2} \times K(\Z, 3)^E \to \Gm$.  Naturality shows $\hat w \circ \gamma^E = \widehat{\gamma_* w}$, and the theorem shows $\gamma_* w = \pm d_n(\L_1, \L_2)$, hence $\widehat{\gamma_* w} = (\pm d_n(\L_1, \L_2))\hat{}$, which is adjoint to $(e_n \Pi_3)^\pm$.
\end{proof}

\begin{lemma}\citeme{AS Lemma 6.1}
There is a short exact sequence \[BSU^E \from BU^E \from (\CP^\infty)^E.\]
\end{lemma}
\begin{proof}
There is a topological splitting $BU \simeq BSU \times \CP^\infty$ using the determinant map (although not as $H$--spaces, of course), and hence a short exact sequence of formal group schemes \[0 \to BSU_E \to BU_E \to \CP^\infty_E \to 0.\]  Since Cartier duality is an exact functor, \[0 \from BSU^E \from BU^E \from (\CP^\infty)^E \from 0\] is also a short exact sequence.\footnote{Although, you'll notice that Cartier duality is not functorial for maps of formal schemes which are not homomorphisms, so the map $BU(1)_E \to BU_E$ is not necessarily sent to anything useful.}
\end{proof}

\todo[inline]{There's a nice construction of the $\Pi_k$ maps in the AHS preprint where the ``$C_k$'' constructions are performed directly on the spectra, then applying $(-)_E$ carries those constructions to relevant constructions on group schemes, and finally Cartier duality gives the maps $\Pi_k$ of the sort described above.  This is superior to saying ``adjoint'', in my opinion, though it could be remarked that these are equivalent.}

\begin{lemma}\citeme{Lemma 6.2 of AS}
The adjoint of the map $E_0 \CP^\infty \to E_0 BU$ induces a map $\Pi_1: BU^E \to C^1(\CP^\infty_E; \mathbb G_m)$ which is an isomorphism.  In fact, the Cartier duality isomorphism $(\CP^\infty)^E \cong \InternalHom{FormalGroups}(\CP^\infty_E, \mathbb G_m)$ fits into a commuting square
\begin{center}
\begin{tikzcd}
(\CP^\infty)^E \arrow{r} \arrow{d} & \InternalHom{FormalGroups}(\CP^\infty_E, \Gm) \arrow{d}{\begin{array}{c} \text{natural} \\ \text{inclusion} \end{array}} \\
BU^E \arrow{r}{\Pi_1} & C^1(\CP^\infty_E; \Gm).
\end{tikzcd}
\end{center}
\end{lemma}
\begin{proof}
\todo{Include this.}
\end{proof}

\begin{theorem}\label{BUBSUandC1C2Commute}
The following square commutes:
\begin{center}
\begin{tikzcd}
BU^E \arrow{d}{\Pi_1} \arrow{r} & BSU^E \arrow{d}{\Pi_2} \\
C^1(\CP^\infty_E; \Gm) \arrow{r}{\delta} & C^2(\CP^\infty_E; \Gm).
\end{tikzcd}
\end{center}
\end{theorem}
\begin{proof}
\citeme{Lemma 6.4 of AS}
This is a matter of expanding definitions and using \[(\L_1 - 1)(\L_2 - 1) = \mu^*(\L - 1) - \pi_1^*(\L - 1) - \pi_2(\L - 1).\]
\end{proof}

\begin{theorem}
The following is a map of short exact sequences:
\begin{center}
\begin{tikzcd}
0 \arrow{r} & (\CP^\infty)^E \arrow{d} \arrow{r} & BU^E \arrow{r} \arrow{d} & BSU^E \arrow{r} \arrow{d} & 0 \\
0 \arrow{r} & \InternalHom{FormalGroups}(\CP^\infty_E; \Gm) \arrow{r} & C^1(\CP^\infty; \Gm) \arrow{r} & C^2(\CP^\infty; \Gm) \arrow{r} & 0.
\end{tikzcd}
\end{center}
\end{theorem}
\begin{proof}
\citeme{Prop 6.5 of AS. They reference a superior alternate argument in the AHS preprint though\ldots}
\end{proof}

\begin{lemma}
The same holds as in \Cref{BUBSUandC1C2Commute} with $BSU$, $BU[6, \infty)$, $C^2$, and $C^3$.
\end{lemma}
\begin{proof}
\citeme{AS Lemma 7.1}
\end{proof}

\begin{lemma}
The map $C^2 \to C^3$ is injective for $\CP^\infty_E$ a $p$--divisible group.
\end{lemma}
\begin{proof}
The kernel of this map consists of maps alternating, biexponential maps $(\CP^\infty_E)^{\times 2} \to \Gm$.  We can restrict such a map to get a map \[f \co \CP^\infty_E[p^j] \times \CP^\infty_E \to \Gm,\] where we can calculate \[f(x, p^j y) = f(p^j x, y) = f(0, y) = 1.\]  But since $p^j$ is surjective on $\CP^\infty_E$, every point on the right-hand side can be so written, so at every left-hand stage the map is trivial.  Finally, $\CP^\infty_E = \colim_j \CP^\infty_E[p^j]$, so this filtration is exhaustive and we conclude that the kernel is trivial.\citeme{Lemma 7.2 of AS}
\end{proof}

\begin{lemma}
In fact, the following sequence is exact\todo{This is not a typo, we don't get right-exactness yet.} \[0 \to C^2(\CP^\infty_E; \Gm) \xrightarrow\delta C^3(\CP^\infty_E; \Gm) \to \Weil(\CP^\infty_E).\]
\end{lemma}
\begin{proof}\citeme{Lemma 7.3 of AS}
This is hard work.  Breen's idea is to show that picking a preimage under $\delta$ is the same as picking a trivialization of the underlying symmetric biextension of the cubical structure.  Then (following Mumford), one shows that the underlying symmetric biextension is trivial exactly if the Weil pairing is trivial.
\end{proof}

These together culminate in a map of exact sequences with marked isomorphisms:
\begin{center}
\begin{tikzcd}
0 \arrow{r} & BSU^E \arrow{r} \arrow{d}{\simeq} & BU[6, \infty)^E \arrow{r} \arrow{d} & K(\Z, 3)^E \arrow{d}{\simeq} \arrow{r} & 0 \\
0 \arrow{r} & C^2(\CP^\infty_E; \Gm) \arrow{r} & C^3(\CP^\infty_E; \Gm) \arrow{r} & \Weil(\CP^\infty_E).
\end{tikzcd}
\end{center}

\begin{corollary}
The map \[BU[6, \infty)^E \to C^3(\CP^\infty_E; \Gm)\] is an isomorphism.  Also, the map \[C^3(\CP^\infty_E; \Gm) \to \Weil(\CP^\infty_E)\] is a surjection. \qed
\end{corollary}

-------------

Moving from $BU$ to $MU$: $MU\<6\>^E$ is a $\G_m$--torsor over $BU\<6\>^E$, so if you can produce another torsor and any map between them, that automatically gives you an isomorphism and hence a description.  This is pretty easy to read about in section 2.4 of the AHS preprint.  The big theorem is Theorem 2.42 in Section 2.4.4.\todo[inline]{Or, maybe this belongs in the next section.  The requisite chatter about Thom spectra and line bundles is somewhat afield.  I'm not sure where it belongs in the narrative --- it definitely belongs somewhere.}









\section{Elliptic spectra}

\begin{definition}
An \textit{elliptic spectrum} consists of:
\begin{enumerate}
\item An even-periodic ring spectrum $E$.
\item A (generalized) elliptic curve $C$ over $S_E$.\todo{I'm not sure if this is worth explaining. I guess we just mean elliptic curves with certain singularities allowed far away from the origin. Maybe it is worth explaining: you don't get examples like $H\Z P$ or $K^{\Tate}$ without allowing degeneracies.}
\item An isomorphism $\phi: C^\wedge_0 \cong \CP^\infty_E$.
\end{enumerate}
A \textit{map of elliptic spectra} consists of
\begin{enumerate}
\item A map of ring spectra $f: E \to E'$.
\item An \emph{isomorphism} of elliptic curves $f^* C \to C'$.
\end{enumerate}
\end{definition}

\todo{In particular, isogenies of elliptic curves are \emph{not} allowed. This is the realm of power operations.}

\begin{example}
Cohomology with complex coefficients and a selected lattice in the plane: $HP_\Lambda$.  The required isomorphism of formal groups comes from the logarithm map inverse to formally expanding $\C \to \C/\Lambda$ at the origin.
\end{example}

\begin{example}
Integral cohomology with the curve $zy^2 = x^3$.
\end{example}

\begin{example}
Ordinary $K$--theory with the curve $zy^2 + zxy = x^3$.
\end{example}

\begin{example}
Tate $K$--theory
\end{example}

\todo[inline]{I want to sketch the reduction for even-periodic elliptic cohomology theories to the case of $MUP$, then from there to $HkP$ for the prime fields $K$, then from there to questions about additive cocycles.  We certainly don't need to recall any of these calculations, but I think it's a nice example of the philosophy that the additive formal group is such a knotted point of $\moduli{fg}$ that it suffices to check something there to learn it for the rest of the stack.  This survives in the published form of AHS, but it's stated pretty clearly as Prop 3.4 in the unpublished verison.  See also 5.12 of the unpublished version.}

\todo[inline]{From the intro to the AHS preprint: For any lattice $\Lambda \subseteq \C$, we get a map $\Phi: MU[6, \infty) \to HP_\Lambda$ which sends $(2n)$--dimensional bordism classes $M$ to numbers $\Phi(M; \Lambda) \cdot u_{\Lambda}^n$. Suppose $\Lambda$ and $\Lambda'$ are two lattices with $\lambda \cdot \Lambda = \Lambda'$. This induces a map $HP_{\Lambda'} \to HP_\Lambda$ which intertwines the maps $\Phi$ by $\Phi(M; \lambda \cdot \Lambda) = \lambda^{-n} \Phi(M; \Lambda)$.  The usual appearance of a modular form (via $SL_2$ invariance) can be extracted from the top of page 5, if you want.  You can also show that this ``functional equation for a modular form'' is actually realized by a function by considering the elliptic cohomology theory built out of the bundle of elliptic curves $\mathfrak h \times \C/(1, \tau) \to \mathfrak h$ and the ordinary coefficient ring $\sheaf O[u^\pm]$, $\sheaf O$ the ring of holomorphic functions on $\mathfrak h$.}

-----------

Define the classical $\theta$--function on the Tate curve by \[\tilde \theta_q(u) = (1 - u) \prod_{n > 0} (1 - q^n u)(1 - q^nu^{-1}) \in \Z[u^{\pm}]\llbracket q \rrbracket.\]  Write $t = 1-u$ for the usual coordinate on the formal multiplicative group; then we can think of $\tilde \theta_q(u)$ as an element of $\Z\llbracket q\rrbracket\llbracket t\rrbracket$ and thus as a function on $\G_m \times D_{\Tate}$, $D_{\Tate} = \Spec \Z\llbracket t \rrbracket$ the Tate domain.  In fact, $\tilde\theta_q(u)$ is even a coordinate on this formal group over $D_{\Tate}$, which one can identify with $\widehat C_{\Tate}$.

By formal rearrangements one can produce the familiar functional equations
\begin{align*}
\tilde \theta_q(qu) & = -u^{-1} \tilde\theta_q(u), \\
\tilde \theta_q(q^k u) & = q^{-k(k-1)/2} (-u)^{-k} \tilde\theta_q(u).
\end{align*}
\todo{This is actually kind of hard to do algebraically. It's discussed in Appendix A of the AHS preprint.}
Regarding $\tilde\theta$ as an element of $C^0(\widehat C_{\Tate}; \L)$, this gives a cubical structure \[\delta^3(\tilde\theta) \in C^3(\widehat C_{\Tate}; \L),\] and one computes $\delta(\tilde\theta) = dt/\theta$ for \[\theta_q(u) = (1 - u) \prod_{n > 0} \frac{(1 - q^nu)(1 - q^n u^{-1})}{(1 - q^n)^2} \in \Z[u^\pm]\llbracket q \rrbracket,\] so you can also apply $\delta^2$ to this expression.\todo{Why would someone find this more familiar?  Also, in what sense is $\delta$ anything like differentiation?}  The functional equation has something to say about this cubical structure: \[(\delta^3 \tilde\theta_q)(u, v, w) = \begin{cases} (\delta^3 \tilde\theta_q)(qu, v, w), \\ (\delta^3 \tilde\theta_q)(u, qv, w), \\ (\delta^3 \tilde\theta_q)(u, v, qw). \end{cases}\]\todo{Mike has a nice remark about this: the exponents in the iterated functional equation for $\tilde\theta_q$ are quadratic in $k$ and so killed by $\delta^3$, which is another differentiation-type claim.}

\begin{theorem}\citeme{AHS preprint Prop 2.49}
The cubical structure $\delta^3(\tilde \theta)$ is the restriction of $s(C_{\Tate} / D_{\Tate})$ to $\widehat C_{\Tate}$.
\end{theorem}
\begin{proof}
The ratio $s(\widehat C_{\Tate} / D_{\Tate}) / \delta^3(\tilde\theta)$ is a power series $g \in \Z\llbracket q, t_0, t_1, t_2\rrbracket$ and we need to show that $g = 1$.  This will hold if we can show that there is a neighborhood of $0$ in $\C^4$ on which $g$ converges to $1$, so we can employ complex analytic techniques.  Fix $q \in \C$ with $0 < |q| < 1$ and let $C_q$ be the $\C$--analytic elliptic curve fibering over this point in $D_{\Tate} \times \Spec \C$.  The product expansion of $\tilde\theta_q(u)$ converges locally uniformly to an analytic function on $\C^\times$ vanishing only on $q^{\Z}$ and there only to first order.  It should suffice to show that $s(C_q/\C) = \delta^3(\tilde\theta_q)$ as analytic functions on $(\C^\times)^{\times 3}$.  The consequence for $\delta^3 \tilde\theta$ of the functional equation for $\tilde\theta$ recalled above shows that $\delta^3\tilde\theta_q$ descends to give a meromorphic $1$--form $\phi$ on $C_q^{\times 3}$.  Then, because $\tilde\theta_q$ has only simple poles on $q^{\Z}$ and none elsewhere, we deduce that $\phi$ is actually a cubical structure, and unicity then finally forces $\phi = s(C_q / \C)$.
\end{proof}

\todo[inline]{See also the bottom of page 22 in the AHS preprint for the relevant theory of integration (esp. Prop 2.54), and see Proposition 2.56 for a comparison theorem between the integration theory and $\sigma_{\Tate}$.}

\begin{definition}
Let $\gamma$ denote the element $K_{\Tate}(\Z \times BU)$ determined by the vector bundle operation \[\gamma: -V \mapsto \prod_{n > 0} \sum_{k \ge 0} q^{nk} \Sym^k(V),\] and let $\bar\gamma$ denote its complex conjuguate.  Since $K_{\Tate}^0(MUP)$ is a module over $K_{\Tate}(\Z \times BU)$, we can define an element \[\sigma_{\Tate} := \gamma \cdot \bar\gamma \cdot \alpha,\] where $\alpha$ is the usual orientation $MP \to KU$ corresponding to the coordinate $1 - t$ on the formal group $\G_m$.
\end{definition}

\todo{``Modularity'' of the $K_{\Tate}$ orientation?}


------------

Criteria for the existence of symmetric cocycle schemes.

AHS: they exist
\todo[inline]{The technical condition guaranteeing the existence of symmetric power schemes is that the symmetric cocycle schemes are coalgebraic formal schemes, since then we have an involutive Cartier duality functor.  This comparison essentially comes out of saying that $C_k$ can be defined by a strong colimit, so if we can check that this strong colimit exists\ldots (cf. Prop 3.3 of the AHS preprint).}

------------

Here's what the published version of AHS has to say about the $\sigma$--orientation of $K_{\Tate}$.  (See Section 2.7.)

$K_{\Tate}$ has multiplicative cohomology theory $K\ps{q}$, formal group multiplicative (as induced by $K$--theory), and isomorphism to $\widehat C_{\Tate}$ given by $1 - u(t)$ as in Lecture 5.2.  Since the cubical structure on $\widehat C_{\Tate}$ is given as $\delta^3$ of something, it follows that the $\sigma$--orientation factors as
\begin{center}
\begin{tikzcd}
MU[6, \infty) \arrow{d} \arrow{drr} \\
MU \arrow{r} & MUP \arrow{r}{\tilde \theta} & K\ps{q}.
\end{tikzcd}
\end{center}
Our goal is to express in terms of characteristic classes the induced map on homotopy by the horizontal composite.

To start with, the topological restriction $MU \to MUP \to E$ sends the coordinate $f$ on $\G_E$ to the rigid section $\delta f$ of $\Theta^1(\sheaf I(0)) = \sheaf I(0)_{0} \otimes \sheaf I(0)^{-1}$.  The most straightforward formula for $\delta f$ is $f(0) / f$, which is confusing, since $f(0) = 0$ usually but not as a section of $\sheaf I(0)_0$.  It's probably clearer to express $\delta f$ in terms of the isomorphism \[\sheaf I(0)_0 \otimes \sheaf I(0)^{-1} \cong \omega \otimes \sheaf I(0)^{-1},\] where $\delta f$ is given by the formula \[\delta f = \frac{f'(0) Dx}{f(x)},\] $Dx$ the invariant differential with value $dx$ at $0$.

In the example of $K$--theory, the usual complex Atiyah--Bott--Shapiro map $MP \to K$ corresponds to the coordinate $1 - u$ on the formal completion of $\Gm = \Spec \Z[u^\pm]$.  The invariant differential is $D(1 - u) = -du/u$, and the restriction to $MU$ classifies the $\Theta^1$--struucture \[\delta(1 - u) = \frac{1}{1 - u}\left(-\frac{du}{u}\right).\]  In the more complex example of Tate $K$--theory, the map \[MU \to MUP \xrightarrow{\tilde\theta} K_{\Tate}\] factors by the coordinate change map \[MU \to MU \sm MU \simeq MU \sm BU_+ \xrightarrow{\delta(1 - u) \sm \theta'} K_{\Tate},\] where $\theta'$ is the element of $BU^{K_{\Tate}} \cong C^1(\widehat C_{\Tate}; \Gm)$ given by the formula \[\theta' = \prod_{n \ge 1} \frac{(1 - q^n)^2}{(1 - q^n u)(1 - q^n u^{-1})}.\]

\needproof{The Todd genus}
In geometric terms, the homotopy groups $\pi_* MU \sm BU_+$ are the bordism groups of pairs $(M, V)$ consisting of a stably almost complex manifold $M$ and a virtual complex vector bundle $V$ over $M$ of virtual dimension $0$.  The map \[\pi_* MU \to \pi_* (MU \sm BU_+)\] sends a manifold $M$ to the pair $(M, \nu)$ where $\nu$ is the reduced stable normal bundle.  Next, the map $\pi_* \delta(1 - u)$ sends a manifold $M$ of dimension $2n$ to $p_!(1) \in K^{-2n}(*)$ where $p: M \to *$ is the unique map.  One has \[p_!(1) = \operatorname{Td}(M) \left( -\frac{du}{u} \right)^n,\] where $\operatorname{Td}(M)$ is the Todd genus of $M$ (and it is customary to suppress the grading and write $p_!(1) = \operatorname{Td}(M)$).  The map $\theta'$ is where the real work is: it is the stable exponential characteristic class taking the value \[\prod_{n \ge 1} \frac{(1 - q^n)^2}{(1 - q^n \L)(1 - q^n \L^{-1})}\] on the reduced class $(1 - \L)$ of a line bundle $\L$.  This stable exponential characteristic class can easily be identified with \[V \mapsto \bigotimes_{n \ge 1} \Sym_{q^n}(- \bar V_{\C}),\] where $V_{\C} = V \otimes_{\R} \C$, $\bar V_{\C} = V_{\C} - \eps^{\oplus \dim V}$, and $\Sym_t(W)$ is defined for (complex) vector bundles $W$ by \[\Sym_t(W) = \bigoplus_{m \ge 0} \Sym^m(V) t^m \in K(M)\ps{t}\] and extended to virtual bundles using the exponential rule $\Sym_t(W_1 - W_2) = \Sym_t(W_1) / \Sym_t(W_2)$.  Altogether, the effect of the $\sigma$--orientation therefore sends an almost complex manifold $M$ of dimension $2n$ to
\begin{align*}
\pi_* \sigma_{K_{\Tate}}(M) & = f_! \left( \bigotimes_{n \ge 1} \Sym_{q^n}(\bar T_{\C}) \right) \\
& = \operatorname{Td}\left(M; \bigotimes_{n \ge 1} \Sym_{q^n}(\bar T_{\C}) \right) \left( -\frac{du}{u} \right)^n \\
& \in \widetilde K\ps{q}^0(S^{2n}).
\end{align*}

This is basically Witten's formula for his genus.  There is a small caveat: Witten's genus is defined for Spin manifolds.  With some care, perhaps we could construct a homotopical square\todo{Is this possible?  Or do we really need the $\String$--orientation to make this happen?  Is this exactly the topic for the next day?!!}
\begin{center}
\begin{tikzcd}
MSU \arrow{r} \arrow{d} & MU \arrow{d} \\
M\Spin \arrow[densely dotted]{r} & K_{\Tate},
\end{tikzcd}
\end{center}
but for the moment we content ourselves with a square of homotopy groups
\begin{center}
\begin{tikzcd}
\pi_* MSU \arrow{r} \arrow{d} & \pi_* MU \arrow{d} \\
\pi_* M\Spin \arrow[densely dotted]{r} & \pi_* K_{\Tate}.
\end{tikzcd}
\end{center}
Let $M$ be a $\Spin$--manifold of dimension $2n$, and use the splitting principle to write \[TM = \L_1 \oplus \cdots \oplus \L_n\] for complex line bundles $\L_i$.  The $\Spin$--structure gives a square root of $\prod \L_i$, which is equivalent to picking a square root for each $\L_i$.\todo{Is it?}  Since, for each $i$, the $O(2)$--bundles underlying $\L_i^{1/2}$ and $\L_i^{-1/2}$ are isomorphic, we can write \[TM = \sum \L_i + \L_i^{-1/2} - \L_i^{1/2},\] which is now a sum of $SU$--bundles.  Using this, one easily checks that the $\sigma$--orientation of $M$ gives \[\widehat A \left(M; \bigotimes_{n \ge 1} \Sym_{q^n}(\bar T_{\C}) \right) \left(-\frac{du}{u}\right)^n,\] where the $\widehat A$--genus is the push-forward in $KO$--theory associated to the unique orientation $M\Spin \to KO$ fitting into the commutative diagram
\begin{center}
\begin{tikzcd}
MSU \arrow{r} \arrow{d} & MU \arrow{d} \\
M\Spin \arrow{rd} \arrow{r} & K \\
& KO \arrow{u}{\widehat A}.
\end{tikzcd}
\end{center}

Finally, we want to see that for $[M] \in \pi_{2n} MU[6, \infty)$, \[\Phi(M) := (\pi_{2n} \sigma_{K_{\Tate}})(M) \left( -\frac{du}{u} \right)^{-n} \in \pi_0 K_{\Tate} = Z\ps{q}\] is a modular form.  We've at least seen that $\Phi(M)$ is a holomorphic function on $D$, with integral $q$--expansion coefficients.  It suffices to show that if $\pi: \h \to D$ is the map $\pi(\tau) = e^{2 \pi i \tau}$, then $\pi^* \Phi(M)$ transforms correctly under the action of $SL_2(\Z)$.  This is supposed to follow from stuff in the introduction (pp.\ 600-1, also Example 2.3).


\todo[inline]{Now that you have an extra few days, you could actually go through the calculation of $H\F_{p*} \OS{ku}{2k}$ and $C^k(\G_a; \Gm)$.}






\section{The $\String$ orientation}

Kitchloo--Laures--Wilson's results on Restriction A Hopf algebras and the bar spectral sequence

Formal schemes for certain real $K$--theory spaces

The Atiyah--Bott--Shapiro orientation and the fibration $BSU \to BSpin$ \citeme{Theorem 2.3.5.iv in KLW}

The $\String$ orientation and $\Sigma$--structures



\begin{theorem}
There is a bi-Cartesian square
\begin{center}
\begin{tikzcd}
& \Div_0 \overline{\G}[2] \arrow{rr} \arrow{ld} & & \Div_0 \G[2] \arrow{ld} \\
\Div_0 \overline{\G} \arrow{rr} & & BO_K.
\end{tikzcd}
\end{center}
\end{theorem}
\begin{proof}
\todo{Analyze the Atiyah--Hirzebruch spectral sequence}
\end{proof}

\begin{lemma}
Consider the cube constructed by taking pointwise fibers of the composite to $X$.
\begin{center}
\begin{tikzcd}
& A' \arrow{rr} \arrow{ld} \arrow{dd} & & B' \arrow{ld} \arrow{dd} \\
C' \arrow{rr} \arrow{dd} & & D' \arrow[crossing over]{dd} \\
& A \arrow{rr} \arrow{ld} & & B \arrow{ld} \\
C \arrow{rr} & & D \arrow{rr} & & X.
\end{tikzcd}
\end{center}
If the bottom face is bi-Cartesian, then so is the top.
\end{lemma}
\begin{proof}
\todo{Prove this? It's valid in an arbitrary abelian category.}
\end{proof}

\begin{corollary}
There is a bi-Cartesian square
\begin{center}
\begin{tikzcd}
& \Div_0 \overline{\G}[2] \arrow{rr} \arrow{ld} & & \SDiv_0 \G[2] \arrow{ld} \\
\Div_0 \overline{\G} \arrow{rr} & & BSO_K. & & \qed
\end{tikzcd}
\end{center}
\end{corollary}
\begin{proof}
\todo{Write this for real.}
The fibration $BSO \to BO \to BO(1)$ gives a short exact sequence of Hopf algebras.  The composite $\Div \overline{\G} \to \G[2]$ acts by zero and the composite $\Div \G[2] \to \G[2]$ acts by summation.  The summation one you can probably prove by comparing with the determinant (or Postnikov) section for $BU$.
\end{proof}

\begin{corollary}
\todo{Write this for real. Even the statement is bad: see ``$\ker \omega$''.}
There is a bi-Cartesian square
\begin{center}
\begin{tikzcd}
& \Div_0 \overline{\G}[2] \arrow{rr} \arrow{ld} & & \ker \omega \arrow{ld} \\
\Div_0 \overline{\G} \arrow{rr} & & B\Spin_K. & & \qed
\end{tikzcd}
\end{center}
\end{corollary}
\begin{proof}
This goes similarly to the one above.  You can compute that the composite $\Div \overline{\G} \to \G[2]^{\wedge 2}$ is zero using an identical technique.  To compute the action on the other factor, KLW show that there's a diagram of exact sequences
\begin{center}
\begin{tikzcd}
& & & K_* \arrow{d} \\
& & & K_* K(\Z, 3) \arrow{d} \\
K_* \arrow{r} & K_* B\Spin \arrow{r} \arrow{d} & K_* BSU \arrow{r}{\tau} \arrow[-,double]{d} & K_* BU[6, \infty) \arrow{d}{\delta} \\
K_* \arrow{r} & K_* BSO \arrow{r}{i} & K_* BSU \arrow{r}{1 - \xi} & K_* BSU \arrow{d} \\
& & & K_* .
\end{tikzcd}
\end{center}
Since $(1 - \xi) \circ i = 0$, we have that $\delta \circ \tau \circ i = 0$ and hence that $\tau \circ i$ lifts to $K_* K(\Z, 3)$.  Identifying $\SDiv_0 \G[2]$ with $C_2 \G[2]$, we check that the composites \[C_2 \G[2] \xrightarrow{\omega} \G[2]^{\wedge 2} \xrightarrow{\eps} C_3 \G\] and \[C_2 \G[2] \to C_2 \G \xrightarrow{\tau} C_3 \G\] agree.  For a point $[a, b] \in C_2 \G$, this is the claim
\begin{align*}
0 & = \eps(a \wedge b) - \tau[a, b] \\
& = [a, a, b] - [b, a, b] - [-a - b, a, b] \\
& = [a, a, b] - [b + a, a, b] + [b, a + a, b] - [b, a, b],
\end{align*}
and this is the expression called $R(b, a, a, b)$, which is forced null in $C_3 \G$.

\todo[inline]{I'm a little fuzzy on the coherence of this with the Bockstein: this computes the lift of $\tau \circ f$ into $K(\Z, 3)_K$, and it does happen to factor through the subscheme $K(\Z/2, 2)_K$ determined by the Bockstein. However, I don't immediately see why this agrees with the bottom Postnikov section of $BSO$: that's a map off of $BSO$ and this is a rotated map into $BU[6, \infty)$, so it's not an immediate consequence of naturality.}
\end{proof}



\todo{What follows is the analysis for $M\String$.  Is the one for $M\Spin$ analogous and do-able?  Does it involve $CK_2$ and maybe a clever choice of $MSU$--orientation?}


The sequence $\Spin/SU \to BU[6, \infty) \to B\String$ is exact and right-exact.  The kernel of the map $\Spin/SU \to BU[6, \infty)$ is ``$CK_3$'' \citeme{Theorem 2.3.5.vi of KLW}, where \[CK_j = \bigoplus_{k=j}^\infty K_* K(\Z/2, k).\]  More than that, KLW even say where the polynomial and nonpolynomial parts of $K_* \Spin/SU$ land inside of $K_* BU[6, \infty)$.  I think\todo{But I have not checked!} that this means that $K_* BU[6, \infty)$ is a flat $K_* \Spin/SU$--module at heights $d \le 2$.

Anyway, there's always a $\Tor$--spectral sequence owing to the pushout diagram
\begin{center}
\begin{tikzcd}
\Susp^\infty_+ \Spin/SU \arrow{r} \arrow{d} & MU[6, \infty) \arrow{d} \\
* \arrow{r} & M\String
\end{tikzcd}
\end{center}
of signature \[\Tor^{K_* \Spin/SU}_{*, *}(K_* MU[6, \infty), K_*) \Rightarrow K_* M\String.\]  So, under the flatness hypothesis above, there are no higher $\Tor$ terms so the spectral sequence collapses to give \[K_* M\String \cong K_* MU[6, \infty) \mmod K_* \Spin/SU.\]  So, what remains to be shown is that $K_* \Spin/SU$ picks out the correct extra relation for $\Sigma$--structures.  Then, we need a density argument to show that this handles all of the at-a-point cases of elliptic cohomology.






\section{Ando--French--Ganter}

Iterated $\Theta$ structures

The ``two--variable Jacobi genus''




-------

Some other things that might belong in this chapter:

The cubical structure on a singular (generalized) elliptic curve is not unique, but (published) AHS has an argument showing that the unicity of the choice on the nonsingular ``bulk'' extends to a unique choice on the ``boundary'' of the compactified moduli too.












