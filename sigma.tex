% -*- root: main.tex -*-

\chapter{The $\sigma$-orientation}


\todo[inline]{Write an introduction for me.  Use unstable cooperations from Morava's theories to classical complex and real $K$--theory.}





\section{The complex Atiyah--Bott--Shapiro orientation}





\section{Elliptic curves and $\theta$--functions}

Every elliptic curve is projective

Every abelian variety is projective (Theorem 7.1 of Milne's \textit{Abelian varieties (short)})

Elliptic $\theta$--functions: every elliptic curve embeds into the \emph{same} projective space

$\Theta$--structures and $\theta$--functions for general abelian varieties: continuously embedding the universal continuous family of abelian varieties into a single projective space.


Section--preserving maps of abelian schemes are automatically homomorphisms: Olsson's Prop 1.11.

The theorem of the cube: $\Theta^3 \L$ is canonically trivial for any rigidified line bundle $\L$ on an abelian scheme $A$.  You can find this in Olsson Thm 2.2.  Or, Akhil Mathew has notes from an algebraic geometry class ( https://math.berkeley.edu/{\textasciitilde}amathew/232b.pdf ) where lectures 3--5 address the theorem of the cube.

$\Theta$--structures and symmetric biextensions

Weil pairings: checks how far some extension is from being commutative. You can see this in Olsson 3.11 or Mumford's \textit{Abelian varieties} p.\ 222.





\section{The Ando--Strickland analysis}

\todo[inline]{Hopefully by this point we've gone through the comparison of $BSU_K$ with $\SDiv_0 \G$. Since we're moving to $C_*$ schemes, we should include a description of the comparison map between $\SDiv_0$ and $C_2$.}

\begin{lemma}\label{EasyCompatibilityWithdn}\citeme{Lemma 4.5 of AS}
There is a unique $\lambda \in \Z/n$ such that the following triangle commutes:
\begin{center}
\begin{tikzcd}
(B\Z/p^j)^{\times 2} \arrow{d}{\lambda \beta \mu} \arrow{rd}{d_n(\L_1, \L_2)} \\
K(\Z, 3) \arrow{r}{i} & BU[6, \infty).
\end{tikzcd}
\end{center}
\end{lemma}
\begin{proof}
Recall the definition of $d_n(\L_1, \L_2)$:
\begin{align*}
d_n(\L_1, \L_2) & := \sum_{k=1}^{p^j-1} \left( (\L_1 - 1)(\L_1^k - 1)(\L_2 - 1) - (\L_1 - 1)(\L_2^k - 1)(\L_2 - 1)\right). \\
\intertext{Forgetting down to $BU$ and working with just virtual bundles rather than lifts of virtual bundles to elements of $kU^6$, this gives:}
& = (\L_1 - 1)(\L_2^{p^j} - 1) - (\L_1^{p^k} - 1)(\L_2 - 1).
\end{align*}
Since we have restricted to $(B\Z/p^j)^{\times 2}$, $\L_1^{p^k} = 1$ and $\L_2^{p^k} = 1$, so this formula collapses and the composite map \[(B\Z/p^j)^{\times 2} \xrightarrow{d_n} BU[6, \infty) \xrightarrow{\pi_2} BSU \xrightarrow{\pi_2} BU\] is null.  We now study the lifting problem across the two maps:
\begin{enumerate}
\item[$\pi_1$:] Since the long composite is null, it follows that the shorter composite $\pi_2 \circ d_n$ factors through the fiber of $\pi_2$.  But, the fiber of $\pi_2$ is $S^1$, and there are no maps \[\{(B\Z/p^j)^{\times 2} \to S^1\} \cong H^1((B\Z/p^j)^{\times 2}; \Z) = 0.\]  It follows that $\pi_2 \circ d_n$ is itself null.
\item[$\pi_2$:] Since $\pi_2 \circ d_n$ is null, $d_n$ must factor through the fiber of $\pi_2$, which is $K(\Z, 3)$.  With identical motives, one considers $H^3((B\Z/p^j)^{\times 2}; \Z)$, which is cyclic of order $n$ and generated by $\beta \circ \mu$. \qedhere
\end{enumerate}
\end{proof}

\begin{theorem}\citeme{AS Theorem 4.2, proved in Section 5}
The value $\lambda$ can be taken to be $\pm 1$ in \Cref{EasyCompatibilityWithdn}. \qed
\end{theorem}

\begin{corollary}\citeme{Corollary 4.4 of AS}
The following square commutes (up to sign):
\begin{center}
\begin{tikzcd}
BU[6, \infty)^E \arrow{r}{\gamma^E} \arrow{d}{\Pi_3} & K(\Z, 3)^E \arrow{d}{b_*} \\
C^3(\CP^\infty_E; \mathbb G_m) \arrow{r}{e} & \Weil(\CP^\infty_E).
\end{tikzcd}
\end{center}
\end{corollary}
\begin{proof}
We will check compatibility on $\Weil_n(\CP^\infty_E)$ for arbitrary $n$.  (Note: the sign can't bounce with $n$ because $\CP^\infty_E$ is $p$--divisible.)  Since $\Weil_n(\CP^\infty_E)$ is a subscheme of $\InternalHom{FormalSchemes}(\CP^\infty_E[n]^{\times 2}, \Gm)$, we can push forward to checking equality here, i.e., of the two maps \[\CP^\infty_E[n]^{\times 2} \times BU[6, \infty)^E \to \Gm.\]

The construction of adjoint elements from maps of spectra converts sums to products and is natural in the source spectrum.  Writing $z = \prod_{i=1}^3 (1 - \L_i) \in kU^6 (\CP^\infty)^{\times 3}$, it follows that the adjoint map $\hat z$ is given by the composite \[\hat z \co (\CP^\infty_E)^{\times 3} \times BU[6, \infty)^E \to (\CP^\infty_E)^{\times 3} \times C^3(\CP^\infty_E; \Gm) \xrightarrow{\operatorname{eval}} \Gm.\]  It follows by naturality that if $z = (1 - \L_1)(1 - \L_1^k)(1 - \L_2) \in kU^6 (\CP^\infty)^{\times 2}$, then $\hat z$ corresponds to the map \[\hat z \co (g_1, g_2, f) \mapsto f(g_1, kg_1, g_2),\] and continuing along these lines we see \[d_n(\L_1, \L_2)\hat{} \co (g_1, g_2, f) \mapsto \prod_{k=1}^{n-1} \frac{f(g_1, kg_1, g_2)}{f(g_1, kg_2, g_2)} = e_n(f)(g_0, g_1).\]

That described the bottom-left arm of the square.  For the other arm, take $w = \beta \circ \mu \in H^3(B\Z/p^j)^{\times 2}$ with adjoint $\hat w \co \CP^\infty_E[p^j]^{\times 2} \times K(\Z, 3)^E \to \Gm$.  Naturality shows $\hat w \circ \gamma^E = \widehat{\gamma_* w}$, and the theorem shows $\gamma_* w = \pm d_n(\L_1, \L_2)$, hence $\widehat{\gamma_* w} = (\pm d_n(\L_1, \L_2))\hat{}$, which is adjoint to $(e_n \Pi_3)^\pm$.
\end{proof}

\begin{lemma}\citeme{AS Lemma 6.1}
There is a short exact sequence \[BSU^E \from BU^E \from (\CP^\infty)^E.\]
\end{lemma}
\begin{proof}
There is a topological splitting $BU \simeq BSU \times \CP^\infty$ using the determinant map (although not as $H$--spaces, of course), and hence a short exact sequence of formal group schemes \[0 \to BSU_E \to BU_E \to \CP^\infty_E \to 0.\]  Since Cartier duality is an exact functor, \[0 \from BSU^E \from BU^E \from (\CP^\infty)^E \from 0\] is also a short exact sequence.\footnote{Although, you'll notice that Cartier duality is not functorial for maps of formal schemes which are not homomorphisms, so the map $BU(1)_E \to BU_E$ is not necessarily sent to anything useful.}
\end{proof}

\todo[inline]{There's a nice construction of the $\Pi_k$ maps in the AHS preprint where the ``$C_k$'' constructions are performed directly on the spectra, then applying $(-)_E$ carries those constructions to relevant constructions on group schemes, and finally Cartier duality gives the maps $\Pi_k$ of the sort described above.  This is superior to saying ``adjoint'', in my opinion, though it could be remarked that these are equivalent.}

\begin{lemma}\citeme{Lemma 6.2 of AS}
The adjoint of the map $E_0 \CP^\infty \to E_0 BU$ induces a map $\Pi_1: BU^E \to C^1(\CP^\infty_E; \mathbb G_m)$ which is an isomorphism.  In fact, the Cartier duality isomorphism $(\CP^\infty)^E \cong \InternalHom{FormalGroups}(\CP^\infty_E, \mathbb G_m)$ fits into a commuting square
\begin{center}
\begin{tikzcd}
(\CP^\infty)^E \arrow{r} \arrow{d} & \InternalHom{FormalGroups}(\CP^\infty_E, \Gm) \arrow{d}{\begin{array}{c} \text{natural} \\ \text{inclusion} \end{array}} \\
BU^E \arrow{r}{\Pi_1} & C^1(\CP^\infty_E; \Gm).
\end{tikzcd}
\end{center}
\end{lemma}
\begin{proof}
\todo{Include this.}
\end{proof}

\begin{theorem}\label{BUBSUandC1C2Commute}
The following square commutes:
\begin{center}
\begin{tikzcd}
BU^E \arrow{d}{\Pi_1} \arrow{r} & BSU^E \arrow{d}{\Pi_2} \\
C^1(\CP^\infty_E; \Gm) \arrow{r}{\delta} & C^2(\CP^\infty_E; \Gm).
\end{tikzcd}
\end{center}
\end{theorem}
\begin{proof}
\citeme{Lemma 6.4 of AS}
This is a matter of expanding definitions and using \[(\L_1 - 1)(\L_2 - 1) = \mu^*(\L - 1) - \pi_1^*(\L - 1) - \pi_2(\L - 1).\]
\end{proof}

\begin{theorem}
The following is a map of short exact sequences:
\begin{center}
\begin{tikzcd}
0 \arrow{r} & (\CP^\infty)^E \arrow{d} \arrow{r} & BU^E \arrow{r} \arrow{d} & BSU^E \arrow{r} \arrow{d} & 0 \\
0 \arrow{r} & \InternalHom{FormalGroups}(\CP^\infty_E; \Gm) \arrow{r} & C^1(\CP^\infty; \Gm) \arrow{r} & C^2(\CP^\infty; \Gm) \arrow{r} & 0.
\end{tikzcd}
\end{center}
\end{theorem}
\begin{proof}
\citeme{Prop 6.5 of AS. They reference a superior alternate argument in the AHS preprint though\ldots}
\end{proof}

\begin{lemma}
The same holds as in \Cref{BUBSUandC1C2Commute} with $BSU$, $BU[6, \infty)$, $C^2$, and $C^3$.
\end{lemma}
\begin{proof}
\citeme{AS Lemma 7.1}
\end{proof}

\begin{lemma}
The map $C^2 \to C^3$ is injective for $\CP^\infty_E$ a $p$--divisible group.
\end{lemma}
\begin{proof}
The kernel of this map consists of maps alternating, biexponential maps $(\CP^\infty_E)^{\times 2} \to \Gm$.  We can restrict such a map to get a map \[f \co \CP^\infty_E[p^j] \times \CP^\infty_E \to \Gm,\] where we can calculate \[f(x, p^j y) = f(p^j x, y) = f(0, y) = 1.\]  But since $p^j$ is surjective on $\CP^\infty_E$, every point on the right-hand side can be so written, so at every left-hand stage the map is trivial.  Finally, $\CP^\infty_E = \colim_j \CP^\infty_E[p^j]$, so this filtration is exhaustive and we conclude that the kernel is trivial.\citeme{Lemma 7.2 of AS}
\end{proof}

\begin{lemma}
In fact, the following sequence is exact\todo{This is not a typo, we don't get right-exactness yet.} \[0 \to C^2(\CP^\infty_E; \Gm) \xrightarrow\delta C^3(\CP^\infty_E; \Gm) \to \Weil(\CP^\infty_E).\]
\end{lemma}
\begin{proof}\citeme{Lemma 7.3 of AS}
This is hard work.  Breen's idea is to show that picking a preimage under $\delta$ is the same as picking a trivialization of the underlying symmetric biextension of the cubical structure.  Then (following Mumford), one shows that the underlying symmetric biextension is trivial exactly if the Weil pairing is trivial.
\end{proof}

These together culminate in a map of exact sequences with marked isomorphisms:
\begin{center}
\begin{tikzcd}
0 \arrow{r} & BSU^E \arrow{r} \arrow{d}{\simeq} & BU[6, \infty)^E \arrow{r} \arrow{d} & K(\Z, 3)^E \arrow{d}{\simeq} \arrow{r} & 0 \\
0 \arrow{r} & C^2(\CP^\infty_E; \Gm) \arrow{r} & C^3(\CP^\infty_E; \Gm) \arrow{r} & \Weil(\CP^\infty_E).
\end{tikzcd}
\end{center}

\begin{corollary}
The map \[BU[6, \infty)^E \to C^3(\CP^\infty_E; \Gm)\] is an isomorphism.  Also, the map \[C^3(\CP^\infty_E; \Gm) \to \Weil(\CP^\infty_E)\] is a surjection. \qed
\end{corollary}

-------------

Moving from $BU$ to $MU$: $MU\<6\>^E$ is a $\G_m$--torsor over $BU\<6\>^E$, so if you can produce another torsor and any map between them, that automatically gives you an isomorphism and hence a description.  This is pretty easy to read about in section 2.4 of the AHS preprint.  The big theorem is Theorem 2.42 in Section 2.4.4.\todo[inline]{Or, maybe this belongs in the next section.  The requisite chatter about Thom spectra and line bundles is somewhat afield.  I'm not sure where it belongs in the narrative --- it definitely belongs somewhere.}









\section{Elliptic spectra}

\begin{definition}
An \textit{elliptic spectrum} consists of:
\begin{enumerate}
\item An even-periodic ring spectrum $E$.
\item A (generalized) elliptic curve $C$ over $S_E$.\todo{I'm not sure if this is worth explaining. I guess we just mean elliptic curves with certain singularities allowed far away from the origin.}
\item An isomorphism $\phi: C^\wedge_0 \cong \CP^\infty_E$.
\end{enumerate}
A \textit{map of elliptic spectra} consists of
\begin{enumerate}
\item A map of ring spectra $f: E \to E'$.
\item An \emph{isomorphism} of elliptic curves $f^* C \to C'$.
\end{enumerate}
\end{definition}

\todo{In particular, isogenies of elliptic curves are \emph{not} allowed. This is the realm of power operations.}

\begin{example}
Cohomology with complex coefficients and a selected lattice in the plane: $HP_\Lambda$.  The required isomorphism of formal groups comes from the logarithm map inverse to formally expanding $\C \to \C/\Lambda$ at the origin.
\end{example}

\begin{example}
Integral cohomology with the curve $zy^2 = x^3$.
\end{example}

\begin{example}
Ordinary $K$--theory with the curve $zy^2 + zxy = x^3$.
\end{example}

\begin{example}
Tate $K$--theory
\end{example}

\todo[inline]{I want to sketch the reduction for even-periodic elliptic cohomology theories to the case of $MUP$, then from there to $HkP$ for the prime fields $K$, then from there to questions about additive cocycles.  We certainly don't need to recall any of these calculations, but I think it's a nice example of the philosophy that the additive formal group is such a knotted point of $\moduli{fg}$ that it suffices to check something there to learn it for the rest of the stack.  This survives in the published form of AHS, but it's stated pretty clearly as Prop 3.4 in the unpublished verison.  See also 5.12 of the unpublished version.}

\todo[inline]{From the intro to the AHS preprint: For any lattice $\Lambda \subseteq \C$, we get a map $\Phi: MU[6, \infty) \to HP_\Lambda$ which sends $(2n)$--dimensional bordism classes $M$ to numbers $\Phi(M; \Lambda) \cdot u_{\Lambda}^n$. Suppose $\Lambda$ and $\Lambda'$ are two lattices with $\lambda \cdot \Lambda = \Lambda'$. This induces a map $HP_{\Lambda'} \to HP_\Lambda$ which intertwines the maps $\Phi$ by $\Phi(M; \lambda \cdot \Lambda) = \lambda^{-n} \Phi(M; \Lambda)$.  The usual appearance of a modular form (via $SL_2$ invariance) can be extracted from the top of page 5, if you want.  You can also show that this ``functional equation for a modular form'' is actually realized by a function by considering the elliptic cohomology theory built out of the bundle of elliptic curves $\mathfrak h \times \C/(1, \tau) \to \mathfrak h$ and the ordinary coefficient ring $\sheaf O[u^\pm]$, $\sheaf O$ the ring of holomorphic functions on $\mathfrak h$.}

-----------

Define the classical $\theta$--function on the Tate curve by \[\tilde \theta_q(u) = (1 - u) \prod_{n > 0} (1 - q^n u)(1 - q^nu^{-1}) \in \Z[u^{\pm}]\llbracket q \rrbracket.\]  Write $t = 1-u$ for the usual coordinate on the formal multiplicative group; then we can think of $\tilde \theta_q(u)$ as an element of $\Z\llbracket q\rrbracket\llbracket t\rrbracket$ and thus as a function on $\G_m \times D_{\Tate}$, $D_{\Tate} = \Spec \Z\llbracket t \rrbracket$ the Tate domain.  In fact, $\tilde\theta_q(u)$ is even a coordinate on this formal group over $D_{\Tate}$, which one can identify with $\widehat C_{\Tate}$.

By formal rearrangements one can produce the familiar functional equations
\begin{align*}
\tilde \theta_q(qu) & = -u^{-1} \tilde\theta_q(u), \\
\tilde \theta_q(q^k u) & = q^{-k(k-1)/2} (-u)^{-k} \tilde\theta_q(u).
\end{align*}
\todo{This is actually kind of hard to do algebraically. It's discussed in Appendix A of the AHS preprint.}
Regarding $\tilde\theta$ as an element of $C^0(\widehat C_{\Tate}; \L)$, this gives a cubical structure \[\delta^3(\tilde\theta) \in C^3(\widehat C_{\Tate}; \L),\] and one computes $\delta(\tilde\theta) = dt/\theta$ for \[\theta_q(u) = (1 - u) \prod_{n > 0} \frac{(1 - q^nu)(1 - q^n u^{-1})}{(1 - q^n)^2} \in \Z[u^\pm]\llbracket q \rrbracket,\] so you can also apply $\delta^2$ to this expression.\todo{Why would someone find this more familiar?  Also, in what sense is $\delta$ anything like differentiation?}  The functional equation has something to say about this cubical structure: \[(\delta^3 \tilde\theta_q)(u, v, w) = \begin{cases} (\delta^3 \tilde\theta_q)(qu, v, w), \\ (\delta^3 \tilde\theta_q)(u, qv, w), \\ (\delta^3 \tilde\theta_q)(u, v, qw). \end{cases}\]\todo{Mike has a nice remark about this: the exponents in the iterated functional equation for $\tilde\theta_q$ are quadratic in $k$ and so killed by $\delta^3$, which is another differentiation-type claim.}

\begin{theorem}\citeme{AHS preprint Prop 2.49}
The cubical structure $\delta^3(\tilde \theta)$ is the restriction of $s(C_{\Tate} / D_{\Tate})$ to $\widehat C_{\Tate}$.
\end{theorem}
\begin{proof}
The ratio $s(\widehat C_{\Tate} / D_{\Tate}) / \delta^3(\tilde\theta)$ is a power series $g \in \Z\llbracket q, t_0, t_1, t_2\rrbracket$ and we need to show that $g = 1$.  This will hold if we can show that there is a neighborhood of $0$ in $\C^4$ on which $g$ converges to $1$, so we can employ complex analytic techniques.  Fix $q \in \C$ with $0 < |q| < 1$ and let $C_q$ be the $\C$--analytic elliptic curve fibering over this point in $D_{\Tate} \times \Spec \C$.  The product expansion of $\tilde\theta_q(u)$ converges locally uniformly to an analytic function on $\C^\times$ vanishing only on $q^{\Z}$ and there only to first order.  It should suffice to show that $s(C_q/\C) = \delta^3(\tilde\theta_q)$ as analytic functions on $(\C^\times)^{\times 3}$.  The consequence for $\delta^3 \tilde\theta$ of the functional equation for $\tilde\theta$ recalled above shows that $\delta^3\tilde\theta_q$ descends to give a meromorphic $1$--form $\phi$ on $C_q^{\times 3}$.  Then, because $\tilde\theta_q$ has only simple poles on $q^{\Z}$ and none elsewhere, we deduce that $\phi$ is actually a cubical structure, and unicity then finally forces $\phi = s(C_q / \C)$.
\end{proof}

\todo[inline]{See also the bottom of page 22 in the AHS preprint for the relevant theory of integration (esp. Prop 2.54), and see Proposition 2.56 for a comparison theorem between the integration theory and $\sigma_{\Tate}$.}

\begin{definition}
Let $\gamma$ denote the element $K_{\Tate}(\Z \times BU)$ determined by the vector bundle operation \[\gamma: -V \mapsto \prod_{n > 0} \sum_{k \ge 0} q^{nk} \Sym^k(V),\] and let $\bar\gamma$ denote its complex conjuguate.  Since $K_{\Tate}^0(MUP)$ is a module over $K_{\Tate}(\Z \times BU)$, we can define an element \[\sigma_{\Tate} := \gamma \cdot \bar\gamma \cdot \alpha,\] where $\alpha$ is the usual orientation $MP \to KU$ corresponding to the coordinate $1 - t$ on the formal group $\G_m$.
\end{definition}

\todo{``Modularity'' of the $K_{\Tate}$ orientation?}


------------

Criteria for the existence of symmetric cocycle schemes.

AHS: they exist
\todo[inline]{The technical condition guaranteeing the existence of symmetric power schemes is that the symmetric cocycle schemes are coalgebraic formal schemes, since then we have an involutive Cartier duality functor.  This comparison essentially comes out of saying that $C_k$ can be defined by a strong colimit, so if we can check that this strong colimit exists\ldots (cf. Prop 3.3 of the AHS preprint).}







\section{The $\String$ orientation}

Kitchloo--Laures--Wilson's results on Restriction A Hopf algebras and the bar spectral sequence

Formal schemes for certain real $K$--theory spaces

The Atiyah--Bott--Shapiro orientation and the fibration $BSU \to BSpin$ \citeme{Theorem 2.3.5.iv in KLW}

The $\String$ orientation and $\Sigma$--structures














