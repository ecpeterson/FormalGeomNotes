% -*- root: main.tex -*-

\subsection*{The \(E_{\Gamma}\)-cohomology of finite abelian groups}

%Ask Eric to change 3.5.2 to have an arbitrary k.

%Introduce \(E = E_{\Gamma}\), the universal deformation, and borel equivariant cohomology theories. Use 2.6.1 to identify \(E^*(BC_p)\) (look at how gradings have been discussed in the book ie. graded formal schemes?). Use Kunneth to identify \(E^*(BA)\). End by introducing the ``approximation problem", approximating \(E^0(X)\) be rational cohomology - good for finite complexes, bad for big complexes.

Let \(\Gamma\) be a height \(d\) formal group over \(k\), a perfect field of characteristic \(p\). By \Cref{LubinTateModuliThm}, there is a noncanonical isomorphism 
\[
\sheaf O_{(\moduli{fg})^\wedge_\Gamma} \cong W(k)\ps{u_1, \ldots, u_{d-1}}
\] 
and this ring carries the universal deformation of \(\Gamma\), which we denote as \(\G\). By \Cref{DefnChromaticHomologyThys}, there is an associated chromatic cohomology theory
\[
E = E_{\Gamma}
\]
called height \(d\) Lubin--Tate theory or Morava \(E\)--theory such that
\[
BS^1_E = \Spf(E^0(BS^1)) = \G.
\]
Fixing a coordinate on \(\G\) provides us with a formal group law
\[
x +_{\G} y.
\]

\begin{proposition}[{\cite[Proposition 5.2 and Lemma 5.7]{HKR}, cf.\ \Cref{KtheoryOfClassifyingSpace}}] \label{app:cyclic}
Let \(C_{m} = S^1[m]\). There is an isomorphism (depending on the chosen coordinate) 
\[
E^*(BC_{m}) \cong E^*\ps{x}/([m]_{\G}(x))
\]
of \(E^*\)-algebras. Moreover, \(E^*(BC_{m})\) is free as an \(E^*\)-module of rank \(p^{kd}\) where \(p^k|m\) and \(p^{k+1} \nmid m\).\footnote{To prove this result it is most natural to set \(|x| = 2\), but for the purposes of this Lecture (and, indeed, this textbook) it is most natural to set \(|x| = 0\). We can move back and forth between these choices by the change of coordinates \(x \leftrightarrow ux\), where \(u\) is a generator of \(\pi_2 E\).} \pushQED\qed \qedhere \popQED
\end{proposition}

\begin{corollary}[{\cite[Corollary 5.10]{HKR}}] \label{app:ab}
Let 
\[
A \cong C_{m_1} \times \cdots \times C_{m_i}
\]
be a finite abelian group.  There is then an isomorphism of \(E^*\)-algebras
\[
E^*(BA) \cong E^*\ps{x_1, \ldots, x_{i}}/([m_1]_{\G}(x_1), \ldots, [m_i]_{\G}(x_i)). \pushQED\qed \qedhere \popQED
\]
\end{corollary}


If \(A \cong C_{p^{k_1}} \times \cdots \times C_{p^{k_i}}\) is an abelian \(p\)-group, it follows that \(E^*(BA)\) is a free \(E^*\)-module of rank \(p^{k_1d+\cdots + k_id}\). If \(m\) is prime to \(p\), then \([m]_{\G}(x)/x\) is a unit in \(E^*\ps{x}\). This implies that the rank of \(E^*(BA)\) only depends on the Sylow \(p\)-subgroup of \(A\).

The most computable of chromatic cohomology theories is periodic rational cohomology. For a \(\Q\)-algebra \(R\), the formal group associated to \(HRP\) is the (height \(0\)) additive formal group. It is natural to want to gain information regarding \(E^*(X)\) (for any chromatic cohomology theory \(E\)) by comparing \(E\)-cohomology with periodic rational cohomology.\footnote{We have employed a similar-sounding strategy elsewhere in this text, where we have approximated an arbitrary complex-orientable theory by \(\HFp P\), using the fact that the infinite-height group \(\G_a \otimes \F_p\) is a dense point in \(\moduli{fg} \times \Spec \Z_{(p)}\).  However, the height \(0\) point \(\G_a \otimes \Q\) is \emph{not} dense, and so our intention here is much more delicate.}  For instance, this is the purpose of the classical Chern character: for \(X\) a space, the Chern character is a map of commutative rings
\[
K^0(X) \to H\Q P^0(X)
\] 
induced by the map of spectra\footnote{We could replace \(\Q\) with \(\R\) or \(\C\) and the result would still hold.} \[K \to H\Q \wedge K \simeq H\Q P.\] For \(X\) a finite \(CW\)-complex, it induces an isomorphism
\[
\Q \otimes K^0(X) \cong H\Q P^0(X).
\]

When \(X\) is a finite CW complex, there is an isomorphism
\begin{equation*}
\Q \otimes E^*(X) \cong H(\Q \otimes E^0)P^*(X).
\end{equation*}
This follows from the fact that \(\Q \otimes E^*\) is a flat \(E^*\)-algebra. When \(X\) is not a finite CW complex, Proposition \ref{app:ab} gives a non-example. It implies that
\[
\Q \otimes E^*(BA)
\]
is free as a \(\Q \otimes E^*\)-module of rank \(p^{k_1d + \ldots + k_id}\). However, \[H(\Q \otimes E^0)P^*(BA) \cong \Q \otimes E^*\] is free of rank \(1\).  This is essentially a consequence of Machke's theorem: using the facts that the order of \(A\) is invertible in \(\Q \otimes E^0\) and that the action of \(A\) is trivial, we have isomorphisms
\begin{align*}
\pi_*((H(\Q \otimes E^0)P)^{BA}) &\cong \pi_*((H(\Q \otimes E^0)P)^{hA}) \\ &\cong (\pi_*(H(\Q \otimes E^0)P))^A \\ &\cong \pi_*(H(\Q \otimes E^0)P).
\end{align*}

To explain one of the inputs into the main theorem of this Lecture, we need one more notion. For \(G\) a finite group, a \(G\)--CW complex \(X\) is a \(G\)--space built inductively by attaching ``cells'' of the form \((G/H) \times D^n\), as in the following pushout diagram:
\begin{center}
\begin{tikzcd}
\coprod_{i \in I} (G/H_i \times S^{n-1}) \arrow{r} \arrow{d} & \coprod_{i \in I} (G/H_i \times D^n) \arrow{d} \\
X^{(n-1)} \arrow{r} & X^{(n)}.
\end{tikzcd}
\end{center}
We say that a \(G\)--CW complex is finite if it can be built out of finitely many cells of the form \(G/H \times D^n\).

\begin{theorem}[{Equivariant CW Approximation, \cite[Chapter I.1]{MayAlaskaNotes}}] \label{app:CWapprox}
Every \(G\)--space is weakly equivalent to a \(G\)--CW complex. \qed
\end{theorem}

Given a cohomology theory \(E\), an equivariant version of the cohomology theory can be formed by setting
\[
E^{*}_{G}(X) = E^*(X_{hG}),
\]
where \(X\) is a \(G\)-space and
\[
X_{hG} \simeq EG \times_G X = (EG \times X)/G
\]
is the homotopy orbits for the \(G\)-action on \(X\). The resulting equivariant cohomology theory is called \textit{Borel equivariant \(E\)--theory}.

The purpose of this appendix is to explain, in the case of \(E = E_{\Gamma}\), how to extend the above formula \[\Q \otimes E^*(X) \cong H(\Q \otimes E^0)P^*(X),\] originally stated for finite complexes \(X\), to an analogous formula for finite \(G\)--CW complexes. Said another way, for each finite group, we want to approximate Borel equivariant \(E\)--cohomology (restricted to finite \(G\)--CW complexes) by some version of Borel equivariant rational cohomology.  In particular, this links back up to our original goal of studying \(E^*(B\Sigma_n)\) by the formula \[E^*(B\Sigma_n) = E_{\Sigma_n}^*((G/G) \times *).\]

\subsection*{Formal Geometry}

%p-divisible groups from formal groups (reference 4.4-5.1). Algebro-geometric description of \(\Spf E^0(BA)\). Fix \(S^1\), build p-divisible group from \(p^k\)-torsion in \(S^1\). Note that all of this makes sense over \(\Spec E^0\). Return to approximation problem for \(E^0(BC_{p^k})\), phrase it in an algebro geometric way and work out the need to build \(C_0\).

Recall from \Cref{DefnPDivGp} that associated to a finite height formal group is a connected \(p\)--divisible group built out of the \(p^k\)--torsion of the formal group as \(k\) varies. Associated to the universal deformation \(\G\) is the following \(p\)--divisible group of height \(d\)
\[
\G[p] \hookrightarrow \G[p^2] \hookrightarrow \ldots,
\]
where the maps are the inclusions of the \(p^k\)--torsion into the \(p^{k+1}\)--torsion.  \Cref{app:cyclic} implies first 
\[
B(C_{p^k})_{E} \cong \G[p^k],
\]
and that the \(p\)--divisible group associated to \(\G\) can be formed using the inclusions \(C_{p^k} = S^1[p^k] \hookrightarrow C_{p^{k+1}} = S^1[p^{k+1}]\). 
%as the system of schemes
%\[
%BS^1[p]_{E} \hookrightarrow BS^1[p^2]_{E} \hookrightarrow \ldots.
%\]
\Cref{app:cyclic} also implies that all of the schemes in this system are finite flat commutative groups schemes over \(\Spf E^0\). The \(p\)--divisible group is called ``height \(d\)'' because \(\G[p]\) is a finite group scheme of order \(p^d\) or, equivalently, since \(E^0(BC_p)\) has rank \(p^d\) as a free module over \(E^0\).

There are \(p\)--divisible groups that are \emph{not} built from formal groups, and the simplest of these plays a role in this story. We define the \textit{constant \(p\)--divisible group of height \(1\)} to be 
\[
C_{p^{\infty}} = S^1[p^\infty] = \colim \big ( C_p \hookrightarrow C_{p^2} \hookrightarrow \ldots \big ).
\]
It is called \textit{constant} because it is made up of constant group schemes, and it is called height \(1\) because \(|S^1[p]| = p^1\).  The constant \(p\)--divisible group of height \(d\) is then the \(d\)--fold product \(C_{p^\infty}^{\times d} = (S^1)^{\times d}[p^\infty]\).

More generally, a similar description of \(BA_{E}\) can be given using formal geometry when \(A\) is a finite abelian group. Let \(A^*\) be the Pontryagin dual of \(A\) and let
\[
\InternalHom{FormalGroups}(A^*, \G)
\]
be the formal scheme that associates to a complete local \(E^*\)--algebra \(R\)
\[
\InternalHom{FormalGroups}(A^*, \G)(R) = \CatOf{AbelianGroups}(A^*, \G(R)).
\]
For large enough \(k\), we have \(A = A[p^k]\), and from this it follows that any map from \(A^*\) to \(\G(R)\) must land in the \(p^k\)--torsion \(\G[p^k](R)\).  There is thus an isomorphism of formal schemes
\[
\InternalHom{FormalGroups}(A^*, \G) \cong \InternalHom{FormalGroups}(A^*, \G[p^k]).
\]
\begin{proposition}[{\cite[Proposition 5.12]{HKR}}] \label{app:abeliangroupdualhom}
There is a canonical isomorphism of formal schemes
\[
\Spf(E^0(BA)) \cong \InternalHom{FormalGroups}(A^*, \G),
\]
natural in maps of finite abelian groups. \qed
\end{proposition}


The formal scheme \(\InternalHom{FormalGroups}(A^*,\G)\) is finite and flat over \(\Spf(E^0)\). Because of this, this formal scheme can be viewed as a scheme---really a scheme, and not a formal scheme!---over \(\Spec(E^0)\). Without going into the technical details of formal to informal constructions, the main idea is that the topology on a complete local \(E^0\)-algebra is not important to \(\G[p^k]\): for a continuous \(E^0\)-algebra \(R\), there is a bijection
\[
\CatOf{Algebras}_{E^0}^{\mathrm{cts}}(E^0(BC_p),R) \cong \CatOf{Algebras}_{E^0}(E^0(BC_p),R).
\]
The same cannot be said for \(E^0(BS^1)\).  In the case of \(E^0(BC_p)\), though, we may thus restrict our attention to the scheme
\[
\InternalHom{FormalGroups}(A^*,\G[p^k]) \co \CatOf{Algebras}_{E^0} \rightarrow \CatOf{Sets}
\]
sending an \(E^0\)--algebra \(R\) to
\[
\CatOf{AbelianGroups}(A^*, \G[p^k](R)) \cong \CatOf{GroupSchemes}(A^*, \Spec(R) \underset{{\Spec(E^0)}}{\times} \G[p^k]).
\]
Since the ring map \(E^0 \rightarrow \Q \otimes E^0\) is not continuous, this discussion is important to phrasing an algebro-geometric version of the appoximation problem explained at the end of the previous section.

When \(G = A\) and \(X = *\), the approximation problem can now be given an algebro-geometric description: we would like to find a \((\Q \otimes E^0)\)--algebra \(R\) such that 
\[
\Spec(R) \times \InternalHom{FormalGroups}(A^*, \G[p^k]) \cong  \InternalHom{FormalGroups}(A^*, \Spec(R) \times \G[p^k])
\]
is recognizable as \(\Spec(-)\) of the \(HRP\)-cohomology of a space. This would only solve the approximation problem for \(E^0(BA)\)---but, astoundingly, it will turn out that this is enough. There is a surprisingly simple answer to the approximation problem in this case. 

Fix\footnote{This choice of \(\Lambda\) is the only choice made in this Lecture, and last subsection of this Lecture addresses how natural the character map is in automorphisms of \(\Lambda\).  From a certain perspective, the results of Barthel--Stapleton~\cite{BarthelStapleton} can be viewed as an analysis of how natural the character map is among finite index maps \(\Lambda \hookrightarrow \Lambda\).} \(\Lambda = \Z_{p}^{d}\) and let \(\Lk = \Lambda/p^k\Lambda\), so that there are non-canonical isomorphisms
\[
\Lambda^* \cong (\Q/\Z_{(p)})^d \cong (S^1[p^{\infty}])^d.
\] 
as well as canonical isomorphisms
\[
\Lambda^*[p^k] \cong (\Lambda_{k})^* \cong \CatOf{TopologicalGroups}(\Lambda_k, C_{p^k}).
\]
%Let \(\Lk = (C_{p^k}^{\times d})^*\) so that there are canonical isomorphisms
%\[
%\Lk^* \cong (S^1)^{\times d}[p^k] = C_{p^k}^{\times d}.
%\]
If we could find an \(R\) such that
\[
\Spec(R) \times_{\Spec(E^0)} \G[p^k] \cong \Lk^*
\]
as abelian group schemes, then we would be done, since
\[
\sheaf O_{\Lk^*} \cong \prod_{\Lk^*} R \cong HRP^0\left(\coprod_{\Lk^*} \ast\right)
\]
and
\begin{align*}
\Spec(R) \times_{\Spec(E^0)} \InternalHom{FormalGroups}(A^*, \G[p^k]) & \cong \InternalHom{FormalGroups}(A^*, \Lambda_{k}^*) \\
& \cong A^{\times n}.
\end{align*}
That is, we would be able to give the simplest answer imaginable: we will have approximated \(E^0(BA)\) by the \(HRP\)-cohomology of a collection of points. The purpose of the next section is to construct a \(\Q \otimes E^0\)-algebra \(C_0\) with this property.


%Let \(\Lk = (\Z/p^k)^n\) so that there are canonical isomorphisms
%\[
%\Lk^* \cong (\Z^n)^*[p^k] \cong (S^1)^n[p^k] \cong (C_{p^k})^n.
%\]
%If we could find an \(R\) such that
%\[
%\Spec(R) \times_{\Spec(E^0)} \G[p^k] \cong \Lk^*
%\]
%as abelian group schemes, then we would win since
%\[
%\sheaf O_{\Lk^*} \cong \prod{\Lk^*} R \cong HR^*(\coprod{\Lk^*} \ast)
%\]
%and
%\[
%\Spec(R) \times_{\Spec(E^0)} \Hom(A^*, \G[p^k]) \cong \Hom(A^*, (C_{p^k})^n) \cong A^n.
%\]
%That is, we would be able to give the simplest answer imaginable - we will have approximated \(E^0(BA)\) by the \(HRP\)-cohomology of a collection of points. The purpose of the next section is to construct a \(\Q \otimes E^0\)-algebra \(C_0\) with this property.


\subsection*{The ring \(C_0\)}
%Essentially follow the construction in the current document.


The goal of this section is to construct a \((\Q \otimes E^0)\)--algebra \(C_0\) such that
\[
\Spec(C_0) \times_{\Spec(E^0)} \G[p^k] \cong \Lambda_{k}^*
\]
for each \(k>0\). That is, there should be an isomorphism of \(p\)--divisible groups
\[
\Spec(C_0) \times_{\Spec(E^0)} \G \cong \Lambda^*.
\]
This does not uniquely specify \(C_0\), but we can fix this deficiency by asking that \(C_0\) carry the universal isomorphism of \(p\)-divisible groups 
\[
u \co \Lambda^* \xrightarrow{\cong} \Spec(C_0) \times_{\Spec(E^0)} \G.
\]
This means that, given any \(E^0\)--algebra \(R\) and an isomorphism of \(p\)--divisible groups
\[
f \co \Lambda^* \xrightarrow{\cong} \Spec(R) \times_{\Spec(E^0)} \G,
\]
there is a unique map of \(E^0\)--algebras
\[
C_0 \xrightarrow{u_f} R
\]
such that \(f = \Spec(R) \times_{\Spec(C_0)} u\), where the pullback is along \(u_f\).

The idea of the construction of \(C_0\) is quite simple.  \Cref{app:abeliangroupdualhom} implies \[E^0(B\Lk) \cong \sheaf O(\InternalHom{FormalGroups}(\Lk^*,\G[p^k])).\]  To construct \(C_0\), it suffices to understand the open subscheme of the mapping scheme \(\InternalHom{FormalGroups}(\Lk^*,\G[p^k])\) which consists of the homomorphisms that are isomorphisms. 

Since \(E^0(B\Lk)\) corepresents \(\InternalHom{FormalGroups}(\Lk^*,\G[p^k])\), it carries the universal homomorphism of group schemes
\[
u \co \Lk^* \rightarrow \G[p^k].
\]
Thus the identity map \(E^0(B\Lk) \xrightarrow{1} E^0(B\Lk)\) corresponds to a homomorphism of abelian groups
\[
\Lk^* \rightarrow \G[p^k](E^0(B\Lk))
\]
or equivalently a map of commutative group schemes
\[
u \colon \Lk^* \rightarrow \Spec(E^0(B\Lk)) \times_{\Spec(E^0)} \G[p^k].
\]
This map is a consequence of the exponential adjunction for finite flat commutative group schemes: the identity map
\[
1 \co \InternalHom{FormalGroups}(\Lk^*,\G[p^k]) \to \InternalHom{FormalGroups}(\Lk^*,\G[p^k])
\]
is adjoint to the evaluation map
\[
\mathrm{ev} \co \Lk^* \times  \InternalHom{FormalGroups}(\Lk^*,\G[p^k]) \to \G[p^k].
\]
Given \(\alpha \in \Lk^*\), the naturality of Proposition \ref{app:abeliangroupdualhom} implies that behavior of
\[
\InternalHom{FormalGroups}(\Lk^*,\G[p^k]) \to \G[p^k]
\]
on rings of functions is precisely
\begin{equation}
E^0(B\alpha) \co E^0(BC_{p^k}) \to E^0(B\Lk).
\end{equation}
Using the fixed coordinate \(x\), this is equivalent to a map of \(E^0(B\Lk)\)-algebras
\begin{equation}
u^* \co E^0(B\Lk)\ps{x}/[p^k]_{\G}(x) \xrightarrow{} \prod_{\Lk^*} E^0(B\Lk).
\end{equation}
Fixing a set of generators for \(\Lk^*\), gives an isomorphism
\[
E^0(B\Lk) \cong E^0\ps{x_1, \ldots, x_d}/([p^k]_{\G}(x_1), \ldots, [p^k]_{\G}(x_d)),
\]
and the behavior of \(u^*\) on the factor corresponding to \((i_1, \ldots, i_d) \in \Lk^*\) is then
\[
u^* \colon x \mapsto [i_1]_{\G}(x_1)+_{\G} \ldots +_{\G} [i_k]_{\G}(x_d).
\]
By \Cref{app:ab}, these are free \(E^0(B\Lk)\)-modules of the same rank. To force the map to be an isomorphism, it suffices to invert the determinant. Let
\begin{align*}
S_k &= \im \big (\Lk^* \setminus 0 \rightarrow \G[p^k](E^0(B\Lk)) \big ) \\ &= \{ [i_1]_{\G}(x_1)+_{\G} \ldots +_{\G} [i_d]_{\G}(x_d)| 0 \neq (i_1, \ldots, i_d) \in \Lk^* \}.
\end{align*}

\begin{lemma} \label{det}
Inverting the determinant of \(u^*\) is equivalent to inverting \(S_k\).
\end{lemma}
\begin{proof}
For \(\bar{i} = (i_1, \ldots, i_d) \in \Lk^*\), let
\[
[\bar{i}](\bar{x}) = [i_1]_{\G}(x_1) +_{\G} \ldots +_{\G} [i_d]_{\G}(x_d).
\]

The matrix defining \(u^*\) is given by the Vandermonde matrix
\[
\left(
\begin{array}{ccccc}
1 & [\bar{j_1}](\bar{x}) & [\bar{j_1}](\bar{x})^2 & \ldots & [\bar{j_{1}}](\bar{x})^{p^k-1} \\
1 & [\bar{j_2}](\bar{x}) & [\bar{j_2}](\bar{x})^2 & \ldots & [\bar{j_{2}}](\bar{x})^{p^k-1}\\
\vdots & \vdots & \vdots & \ddots & \vdots \\
1 & [\bar{j_{p^k}}](\bar{x}) & [\bar{j_{p^k}}](\bar{x})^2 & \ldots & [\bar{j_{p^k}}](\bar{x})^{p^k-1}\\
\end{array} \right),
\]
where \(\{\bar{j_1},\ldots,\bar{j_{p^k}}\} = \Lk^*\).

The determinant of this matrix is
\[
\prod_{1 \leq s < t \leq p^k} ([\bar{j_t}](\bar{x}) - [\bar{j_s}](\bar{x})).
\]
It is a basic fact of subtraction for formal group laws that
\[
x -_{\G} y = v(x-y)
\]
for a unit \(v\). 

Thus the determinant is, up to multiplication by a unit, 
\begin{align*}
\prod_{1 \leq s < t \leq p^k} ([\bar{j_t}](\bar{x}) -_{\G} [\bar{j_s}](\bar{x})) & = \prod_{1 \leq s < t \leq p^k} ([\bar{j_t} - \bar{j_s}](\bar{x})) \\
& = \prod_{0 \neq \bar{j} \in \Lk^*} [\bar{j}](\bar{x})^{p^k-1}. \qedhere
\end{align*} 
\end{proof}

\begin{corollary}
Inverting \(S_k\) in \(E^0(B\Lk)\) inverts \(p\). \pushQED\qed \qedhere \popQED
\end{corollary}

Set \(C_{0,k} = S_{k}^{-1}E^0(B\Lk)\).  The set \(S_k\) consists largely of zero divisors. Because of this, we should be concerned that \(C_{0,k}\) is the zero ring. The following result of Hopkins--Kuhn--Ravenel saves the day:

\begin{proposition}[{\cite[Proposition 6.5]{HKR}}]\label{CZeroIsFaithfullyFlat}
The ring \(C_{0,k}\) is a faithfully flat \(\Q \otimes E^0\)-algebra. \pushQED\qed \qedhere \popQED
\end{proposition}

By construction, the \(E^0\)--algebra \(C_{0,k}\) carries the universal isomorphism
\[
\Lk^* \xrightarrow{\cong} \G[p^k].
\]
It follows that the colimit \[C_0 = \colim_{k} C_{0,k}\] carries the universal isomorphism of \(p\)--divisible groups
\[
\Lambda^* \xrightarrow{\cong} C_0 \otimes \G.
\]
This is because a map \(C_0 \rightarrow R\) is a compatible system of maps \(C_{0,k} \rightarrow R\) and the map \(C_{0,k} \rightarrow C_{0,k+1}\) restricts the universal isomorphism
\[
\Lambda_{k+1}^{*} \xrightarrow{\cong} \G[p^{k+1}]
\]
to the \(p^k\)--torsion.
%
%
% \(C_{0,k}\)-algebra for all \(k\). Thus the canonical map \(p^{-1}\E^0\otimes_{\E^0}\E^0(B\Lk) \xrightarrow{} C_0\) corresponds to an isomorphism
%\[
%\Lk^* \xrightarrow{\cong} C_0 \otimes \G_{et}[p^k].
%\]
%But more, the maps fit together for all \(k\) giving a map 
%\[
%\Colim{k} \text{ } p^{-1}\E^0\otimes_{\E^0}\E^0(B\Lk) \xrightarrow{} C_0
%\]
%and this implies the isomorphism.

%We can use \(C_0\) to construct a new cohomology theory by extension of coefficients:
%\[
%C_{0}^0(X) = C_0 \otimes_{p^{-1}\E^0} (p^{-1}\E)^0(X).
%\]

% \begin{example}
% Let us fix a coordinate \(x\) on \(\G_{\E}\). When \(k=1\) 
% \[
% C_{0,1}' \cong p^{-1}\E^0 \otimes_{\E^0} \E^0\ps{x_1, \ldots, x_n}/([p](x_1)). 
% \]

% \end{example}

\begin{example} \label{padicktheory}
Let us work out \(C_0\) in the case of \(E_{\G_m} = K_p\). In this case 
\[
\G = \G_m,
\]
the formal multiplicative group over \(\Z_p\).  Because \(\G_m\) is the formal completion of the multiplicative group scheme \(\mathbb G_m\), we have
\[
\G_m[p^k] \cong \mathbb{G}_m[p^k].
\]
With the standard coordinate, this isomorphism is given by
\begin{align*}
\Z_p[x]/([p^k]_{\G_m}(x)) & \cong \Z_p[x]/(x^{p^k}-1) \\
x & \mapsto x-1,
\end{align*}
from which we calculate 
\[
E^0(B\Lk) = K_{p}^0(B\Z/p^k) \cong \Z_p[x]/([p^k]_{\G_m}(x)) \cong \Z_p[x]/(x^{p^k}-1).
\]
We have
\[
S_k = \{x,[2]_{\G_m}(x),\ldots, [p^k-1]_{\G_m}(x)\},
\]
which is
\[
\{x-1,x^2-1, \ldots, x^{p^k-1}-1\}
\]
after applying the isomorphism.
There is the factorization
\[
x^{p^k}-1 = \prod_{i=1,\ldots,k}\Phi_{p^i}(x),
\]
where \(\Phi_{p^i}(x)\) is the \(p^i\)th cyclotomic polynomial. But
\[
x^{p^{k-1}}-1 = \prod_{i=1,\ldots,(k-1)}\Phi_{p^i}(x)
\]
and this is one of the elements in \(S_k\). Clearly it is a zero-divisor in \(K_{p}^0(B\Z/p^k)\), thus
\[
S_{k}^{-1}K_{p}^0(B\Z/p^k) \cong S_{k}^{-1}\Z_p[x]/(\Phi_{p^k}(x)).
\]
We can also see that inverting the elements of \(S_{k}\) inverts \(p\). It suffices to show that the ideal generated by \(p\) in \(S_{k}^{-1}\Z_p[x]/(\Phi_{p^k}(x))\) is the whole ring. Recall the identity of cyclotomic polynomials 
\[
\Phi_{p^k}(x) = \Phi_p(x^{p^{k-1}}).
\]
Thus we have
\[
\big ( S_{k}^{-1}\Z_p[x]/(\Phi_{p^k}(x)) \big )/(p) \cong S_{k}^{-1}\F_p[x]/(\Phi_{p}(x^{p^{k-1}})) \cong S_{k}^{-1}\F_p[x]/(\Phi_{p}(x)^{p^{k-1}}),
\]
but \(\Phi_{p}(x) \in S_k\), so this is the zero-ring.

Now that we have argued that inverting \(S_k\) inverts \(p\), we see that we have a canonical map
\[
\Q_p \otimes_{\Z_p} \Z_p[x]/(x^{p^k}-1) \xrightarrow{} S_{k}^{-1}\Z_p[x]/(x^{p^k}-1) 
\]
%(x^{p^{k-1}}-1)^{-1}\Q_p[x]/(x^{p^k}-1)
and this factors through the quotient map
\[
\Q_p[x]/(x^{p^k}-1) \xrightarrow{} \Q_p[x]/(\Phi_{p^k}(x)).
\]
But \(\Q_p[x]/(\Phi_{p^k}(x))\) is a field so the canonical map
\[
\Q_p[x]/(\Phi_{p^k}(x)) \xrightarrow{\cong} (x^{p^{k-1}}-1)^{-1}\Q_p[x]/(x^{p^k}-1)
\]
is an isomorphism. Thus \(C_{0,k}\) is just \(\Q_p\) adjoin a primitive \(p^k\)th root of unity and
\[
C_0 = \colim_k C_{0,k} \cong \colim_k \Q_p(\zeta_{p^k})
\]
is \(\Q_p\) with all \(p\)-power roots of unity adjoined---a totally ramified extension of \(\Q_p\) with Galois group \(\Z_{p}^{\times}\).
\end{example}

\subsection*{The inertia groupoid}
%Introduce the inertia groupoid and rational cohomology composed with the inertia groupoid.

Recall our goal of approximating \(E^*(EG \times_G X)\) by the \(HC_0P\)--cohomology of some space determined by \(EG \times_G X\). Stated more precisely, we'd like to find a space \(F(EG \times_G X)\) and a map of cohomology theories
\[
E^*(EG \times_G X) \to HC_0P^*(F(EG \times_G X))
\]
such that the induced map
\[
C_0 \otimes_{E^0} E^*(EG \times_G X) \to HC_0P^*(F(EG \times_G X))
\]
is an isomorphism. The purpose of this section is understand the operation \(F(-)\). 

One condition that we have already agreed on is that \(HC_0P^0(F(BC_{p^k}))\) must be canonically isomorphic to \(HC_0P^0(\Lk^*) \cong \prod_{\Lk^*}C_0\). Thus there is a natural first guess for what \(F(-)\) might be: we could set 
\[
F(EG \times_G X) = \InternalHom{Spaces}(B\Lambda, EG \times_G X).
\]
When \(G = C_{p^k}\) and \(X = *\), there is an equivalence
\[
\InternalHom{Spaces}(B\Lambda, BC_{p^k}) \simeq \coprod_{\Lk^*} BC_{p^k}.
\]
Since \(C_0\) is a rational algebra (so \(|G|\) is invertible in \(C_0\)), we have 
\[
HC_0P^0(BG) \cong C_0
\]
for any finite group \(G\). Thus
\[
HC_0P^0\left(\coprod_{\Lk^*} BC_{p^k}\right) \cong \prod_{\Lk^*}C_0,
\]
which is what we wanted.

However, there is a problem with this construction: the functor \(\InternalHom{Spaces}(B\Lambda,-)\) does not send finite complexes to finite complexes and does not preserve homotopy colimits, as can be seen by applying it to the homotopy pushout diagram
\begin{center}
\begin{tikzcd}
\ast \coprod \ast \arrow{r} \arrow{d} & \ast \arrow{d} \\
\ast \arrow{r} & S^1,
\end{tikzcd}
\end{center}
thus \(HC_0P^*(F(-))\) is not a cohomology theory.  It turns out that the \textit{inertia groupoid functor} fixes both of these problems at once, and we describe it now.

%
% We'd like to do this in a naturally in \(X\) so that when \(G = C_p\) and \(X = *\)
%The purpose of this section is to study a construction that plays a key role in the character map. The construction takes in a finite \(G\)-CW complex and produces a finite \(G\)-CW complex. The resulting \(G\)-CW complex can be understood in various ways, one of which uses the language of topological groupoids. 

%Recall that \(\Lambda = \Z_{p}^d\).

\begin{definition}
For a finite group \(G\), let
\[
G_{p}^{d} = \InternalHom{TopologicalGroups}(\Lambda,G)
\]
denote the set of continuous homomorphisms from \(\Lambda\) to \(G\).
\end{definition}

Choosing a basis for \(\Lambda\) gives an isomorphism to the set of \(d\)--tuples of commuting \(p\)--power order elements of \(G\):
\[
G_{p}^{d} \cong \{(g_1,\ldots, g_d)|[g_i,g_j] = e, \text{ } g_{i}^{p^k}=e \text{ for } k \text{ large enough}\}.
\]
The group \(G\) acts on this set by conjugation and we will write \(G_{p}^{d}/G\) for the quotient by this action.

\begin{definition} \label{fix}
For a finite \(G\)--CW complex \(X\), let
\[
\Fix_d(X) = \coprod_{\alpha \in G_{p}^{d}} X^{\im \alpha},
\]
where \(X^{\im \alpha}\) is the fixed points of \(X\) with respect to the image of \(\alpha\).
\end{definition}

This is a finite \(G\)--CW complex by intertwining the action of \(G\) on \(G_{p}^{d}\) by conjugation and the action of \(G\) on \(X\). To be precise, for \(x \in X^{\im \alpha}\), we let \(gx \in X^{\im g \alpha g^{-1}}\).

The \(G\)--space \(\Fix_d(X)\) is also known as the \(d\)--fold inertia groupoid of the \(G\)--space \(X\).\footnote{This is really a \(p\)--adic form of the inertia groupoid since \(\Lambda = \Z_{p}^{\times d}\) and not \(\Z^{\times d}\).} That is, the definition above can be made for all finite groups uniformly in the category of topological groupoids. Let \(X \mmod G\) be the action topological groupoid associated to the finite \(G\)--CW complex \(X\). This is the groupoid object in topological spaces with object space \(X\) and morphism space \(X \times G\). The source and target maps
\[
X \times G \rightrightarrows X,
\]
are the projection and action maps, respectively.

%where the maps are the source and target maps (of course there are also composition, inverse, and identity maps). 

The category of topological groupoids is easy to enrich in spaces, the mapping space between two topological groupoids is just the subspace of the product of the space of maps between the objects and the space of maps between the morphisms that are maps of topological groupoids. 

This construction can be upgraded further to an enrichment in topological groupoids so that the object space is the mapping space described above. Let \([\underline{1}] := (0 \rightarrow 1)\) be the free-standing isomorphism and let \([\underline{0}]\) be the category with one object and only the identity morphism. There are two functors 
\[
[\underline{0}] \rightarrow [\underline{1}]
\]
picking out \(0\) and \(1\). The internal mapping topological groupoid between two action groupoids \(W \mmod F\) and \(X\mmod G\), \(\InternalHom{TopologicalGroupoids}(W \mmod F, X \mmod G)\), can be described in the following way: The objects are the space of maps from \(W \mmod F \times [\underline{0}] \cong W \mmod F\) to \(X \mmod G\) and morphisms are the space of maps from \(W \mmod F \times [\underline{1}]\) to \(X \mmod G\). The structure maps are induced by the functors \([\underline{0}] \rightrightarrows [\underline{1}]\) and the cocomposition map
\[
[\underline{1}] \rightarrow [\underline{1}] \sqcup_{[\underline{0}]} [\underline{1}].
\]
% The space of functors between the two groupoids is given by the fiber product
%\[
%\xymatrix{\hom_{top.gpd}(W \mmod F, X \mmod G) \ar[r] \ar[d] & \hom_{top}(W \times F, X \times G) \ar[d]^{(s_*,t_*)} \\ \hom_{top}(W,X) \ar[r]^-{(s^*,t^*)} & \hom_{top}(W \times F, X) \times \hom_{top}(W \times F, X).}
%\]
%
%Let \([\underline{1}] := (0 \rightarrow 1)\) be the free-standing isomorphism. Now we have that
%\[
%\text{Hom}_{top.gpd}(W \mmod F, X \mmod G) = \big(\hom_{top.gpd}(W \mmod F \times [\underline{1}], X \mmod G) \rightrightarrows \hom_{top.gpd}(W \mmod F, X \mmod G)\big),
%\]
%where the maps are induced by the inclusion of \(0\) and \(1\) in \([\underline{1}]\).

\begin{lemma} \label{inertialemma}
Let \(\ast \mmod \Lambda\) be the topological groupoid with a single object and with \(\Lambda\) automorphisms. There is a natural isomorphism 
\[
\InternalHom{TopologicalGroupoids}(\ast \mmod \Lambda, X \mmod G) \cong \Fix_d(X) \mmod G.
\]
\end{lemma} 
\begin{proof}
We indicate the proof. We only need to do this for \(n=1\) as higher \(n\) follow by adjunction. It also suffices to replace \(\ast \mmod \Z_p\) by \(\ast \mmod \Z/p^k\) for \(k\) large enough. Now a functor
\[
\ast \mmod \Z/p^k \xrightarrow{} X \mmod G
\]
picks out an element of \(X\) and a \(p\)--power order element of \(G\) that fixes it. Thus the collection of functors are in bijective correspondence with
\[
\coprod_{\alpha \in G^{1}_{p}} X^{\im \alpha}.
\]
A natural transformation between two functors is a map
\begin{center}
\begin{tikzcd}
\Z/p^k \sqcup \Z/p^k \sqcup \Z/p^k \arrow{r} \arrow[shift left=0.5em]{d} \arrow[shift right=0.5em]{d} & G \times X \arrow[shift left=0.5em]{d} \arrow[shift right=0.5em]{d} \\
\ast \sqcup \ast \arrow{r} & X.
\end{tikzcd}
\end{center}
The domain here is \(\ast \mmod \Z/p^k \times [\underline{1}]\). The two \(\Z/p^k\)'s on the sides come from the identity morphisms in \([\underline{1}]\). The \(\Z/p^k\) in the middle is the important one. Given two functors picking out \(x_1 \in X^{\im \alpha_1}\) and \(x_2 \in X^{\im \alpha_2}\), a commuting diagram above implies that a morphism from \(x_1\) to \(x_2\) in the inertia groupoid is an element \(g \in G\) sending \(x_1\) to \(x_2\) in \(X\). However, the composition diagram implies that \(g\) must conjugate \(\alpha_1\) to \(\alpha_2\).
\end{proof}

We now compute \(EG \times_G \Fix_d(X)\) in several cases.

\begin{example} \label{xapoint2}
When \(X = \ast\),
\[
EG \times_G \Fix_d(\ast) \cong EG \times_{G}^{\text{conj}} G_{p}^{d}.
\]
Fixing an element \(\alpha\) in a conjugacy class \([\alpha]\) of \(G_{p}^{d}/G\), the stabilizer of the element is precisely the centralizer of the image of \(\alpha\) in \(G\). Thus there is an equivalence
\[
EG \times_{G}^{\text{conj}} G_{p}^{d} \simeq \coprod_{[\alpha] \in G_{p}^{d}/G} BC(\im \alpha).
\]
\end{example}

\begin{example}
When \(G\) is a \(p\)--group there is an isomorphism
\[
\CatOf{TopologicalGroups}(\Lambda,G) \cong \CatOf{TopologicalGroups}(\Z^{\times d}, G).
\]
In this case there is an equivalence
\[
EG \times_{G}^{\mathrm{conj}} G_{p}^{d} \simeq L^dBG,
\]
where \(L\) is the free loop space functor. 
\end{example}

The next example shows that the inertia groupoid construction has the property that we desire when applied to abelian groups.
\begin{example} \label{zpk}
When \(G = C_{p^k}\),
\[
EG \times_{G}^{\text{conj}} G_{p}^{d} \simeq \coprod_{\alpha \in \Lk^*} BC_{p^k}.
\]
\end{example}

The inertia groupoid construction has another important property: 
\begin{proposition}[{\cite[Section 6]{KuhnCharacterRings}}]
Let \(E\) be a cohomology theory, then
\[
E^*(EG \times_G \Fix_d(-))
\]
is a cohomology theory on finite \(G\)-CW complexes. \pushQED\qed \qedhere \popQED
\end{proposition}




\begin{example} \label{app:classfncs}
Let \(C_{0}^{*} = \pi_{-*}HC_0P \cong C_0[u,u^{-1}]\) where \(|u|=-2\). When \(X = *\), \Cref{xapoint2} gives an isomorphism
\begin{align*}
HC_{0}P^*(EG \times_G \Fix_d(*)) &\cong HC_{0}P^{*}\left(\coprod_{[\alpha] \in G_{p}^{d}/G} BC(\im \alpha)\right) \\
&\cong \prod_{[\alpha] \in G_{p}^{d}/G}C_{0}^{*}.
\end{align*}
The last isomorphism follows from the fact that \(HC_{0}P^{*}(BG) = C_{0}^{*}\). The codomain is the ring of generalized class functions on \(G_{p}^{d}\) taking values in \(C_{0}^{*}\):
\[
\mathrm{Cl}(G_{p}^{d}, C_{0}^{*}) = \{C_{0}^{*} \text{--valued functions on } G_{p}^{d}/G\}.
\]
One might want to concentrate on the degree \(0\) part, in which case we have an isomorphism
\[
HC_{0}P^0(EG \times_G \Fix_d(*)) \cong \mathrm{Cl}(G_{p}^{d}, C_{0}).
\]
\end{example}


\subsection*{Complex oriented descent}
%Definition of complex oriented descent for a cohomology theory. Prove that if two cohomology theories satisfy complex oriented descent and we have a map between them, then it suffices to show that the map is an isomorphism on finite abelian groups.

Given a pair of cohomology theories \(E\) and \(F\) as well as a map of cohomology theories \(g \colon E^*(-) \rightarrow F^*(-)\), the axioms of a cohomology theory (homotopy invariance, Mayer-Vietoris, \ldots) ensure that \(g\) is an isomorphism if and only if it is an isomorphism on a point. For equivariant cohomology theories only a small change is required, a map
\[
g \colon E_{G}^*(-) \rightarrow F_{G}^{*}(-)
\]
is an isomorphism if and only if it is an isomorphism on spaces of the form \(G/H\). These spaces are the generalization of the point to \(G\)--spaces. The purpose of complex oriented descent is to give conditions on Borel equivariant cohomology theories \(E\) and \(F\) so that a map \(g\) is an isomorphism if and only if it is an isomorphism on all spaces of the form \(G/A\), where \(A \subseteq G\) is abelian. The idea is that certain equivariant cohomology theories have the property that, for a \(G\)-CW complex \(X\), \(E_{G}^*(X)\) can be recovered from \(E_{G}^*(F)\) for a certain space \(F\) with the property that the cells of \(F\) are all of the form \(G/A \times D^n\).

In the cases that we are interested in, the space \(F\) will be the bundle of complete flags associated to a vector bundle on \(X\). Let \(f \colon V \rightarrow X\) be an \(m\)-dimensional complex vector bundle. Associated to \(V\) is a space \(\Flag(V)\) over \(X\) such that the fiber over \(x \in X\) is the space of sequences of inclusions
\[
\{0\} \subset V_1 \subset V_2 \subset \ldots \subset V_{m-1} \subset f^{-1}(x),
\]
where \(V_i \subset f^{-1}(x)\) is an \(i\)-dimensional subspace.

\begin{lemma}
Let \(V\) be an \(m\)--dimensional complex vector space, then there is an equivalence
\[
\Flag(V) \simeq U(m)/\mathbb{T},
\]
where \(\mathbb{T} \cong (S^1)^m\) is the maximal torus in \(U(m)\).
\end{lemma}
\begin{proof}
The idea is simply that \(U(m)\) acts transitively on \(\Flag(V)\) and the stabilizer is \(\mathbb{T}\). This can be seen by viewing a flag as a sequence of unit length normal vectors.
\end{proof}


\begin{proposition}[{\cite[Proposition 2.4(2)]{HKR}}] \label{finitefreeflags}
Let \(X\) be a space and let \(\Flag(V) \rightarrow X\) be the bundle of complete flags of a complex vector bundle \(V\) over \(X\), then \(E^*(\Flag(V))\) is a finitely-generated free \(E^*(X)\)--module. \pushQED\qed \qedhere \popQED
\end{proposition}

Since \(f \colon V \rightarrow X\) is an \(m\)--dimensional complex vector bundle, it follows (cf.\ \Cref{DefnAssociatedBundle}) that \(EG \times V \rightarrow EG \times X\) is an \(m\)--dimensional complex vector bundle (this is the external product with a \(0\)--dimensional complex vector bundle). Since \(G\) acts freely on \(EG \times X\), the quotient \(EG \times_G V \rightarrow EG \times_G X\) is also an \(m\)--dimensional complex vector bundle.

We proceed with two propositions, the proofs of which can both be found in Hopkins--Kuhn--Ravenel~\cite[Proposition 2.6]{HKR}. The first proposition explains the relationship between flag bundles and the Borel construction, the second proposition gives the relationship between the inertia groupoid construction and flag bundles.

\begin{proposition} \label{app:orbitflags}
Let \(X\) be a finite \(G\)--CW complex and let \(V \rightarrow X\) be a \(G\)--equivariant vector bundle, then there is an equivalence
\[
EG \times_G \Flag(V) \simeq \Flag(EG \times_G V)
\]
of spaces over \(EG \times_G X\).
\end{proposition}

\begin{proposition} \label{app:inertiaflags}
Let \(X\) be a finite \(G\)--CW complex and let \(V \rightarrow X\) be a \(G\)--equivariant vector bundle, then \(\Flag(V)^A\) is a disjoint union of fiber products of flag bundles over \(X^A\) when \(A \subseteq G\) is abelian.
\end{proposition}

Recall from \Cref{StrictCechDescentRemark} that a commutative ring map \(R \rightarrow S\) gives rise to a cosimplicial commutative ring
\[\left\{
\begin{tikzcd}[ampersand replacement=\&]
S \arrow[leftarrow]{r} \arrow[shift left=\baselineskip]{r} \arrow[shift right=\baselineskip]{r} \&
\begin{array}{c} S \\ \otimes_R \\ S \end{array} \arrow[shift left=(2*\baselineskip)]{r} \arrow[leftarrow, shift left=\baselineskip]{r} \arrow{r} \arrow[leftarrow, shift right=\baselineskip]{r} \arrow[shift right=(2*\baselineskip)]{r} \&
\begin{array}{c} S \\ \otimes_R \\ S \\ \otimes_R \\ S \end{array} \arrow[shift left=(3*\baselineskip)]{r} \arrow[leftarrow, shift left=(2*\baselineskip)]{r} \arrow[shift left=\baselineskip]{r} \arrow[leftarrow]{r} \arrow[shift right=\baselineskip]{r} \arrow[leftarrow, shift right=(2*\baselineskip)]{r} \arrow[shift right=(3*\baselineskip)]{r} \&
\cdots
\end{tikzcd}
\right\}.\]
Recall also from \Cref{OriginalFFDescent} that if \(S\) is faithfully flat as an \(R\)-algebra, then the above complex is exact except at the first spot and there the homology is \(R\).  This digression has important consequences for us. Since \(E^*(\Flag(V))\) is free as an \(E^*(X)\)-module, there is an isomorphism
\[
E^*(\Flag(V) \times_X \Flag(V)) \cong E^*(\Flag(V)) \otimes_{E^*(X)} E^*(\Flag(V))
\]
and applying \(E^*(-)\) to the simplicial space
\[\left\{
\begin{tikzcd}[ampersand replacement=\&]
\Flag(V) \arrow{r} \arrow[leftarrow,shift left=\baselineskip]{r} \arrow[leftarrow,shift right=\baselineskip]{r} \&
\begin{array}{c} \Flag(V) \\ \times_X \\ \Flag(V) \end{array} \arrow[leftarrow, shift left=(2*\baselineskip)]{r} \arrow[shift left=\baselineskip]{r} \arrow[leftarrow]{r} \arrow[shift right=\baselineskip]{r} \arrow[leftarrow, shift right=(2*\baselineskip)]{r} \&
\begin{array}{c} \Flag(V) \\ \times_X \\ \Flag(V) \\ \times_X \\ \Flag(V) \end{array} \arrow[leftarrow, shift left=(3*\baselineskip)]{r} \arrow[shift left=(2*\baselineskip)]{r} \arrow[leftarrow, shift left=\baselineskip]{r} \arrow{r} \arrow[leftarrow, shift right=\baselineskip]{r} \arrow[shift right=(2*\baselineskip)]{r} \arrow[leftarrow, shift right=(3*\baselineskip)]{r} \&
\cdots
\end{tikzcd}
\right\}\]
produces the descent object for the canonical map \(E^*(X) \rightarrow E^*(\Flag(V))\). Since a finitely-generated free extension is faithfully flat, \Cref{finitefreeflags} implies that
\begin{equation*}
E^*(X) \cong \ker \big (E^*(\Flag(V)) \rightarrow E^*(\Flag(V) \times_X \Flag(V)) \big ).
\end{equation*}

Since it suffices to take the trivial bundle over \(X\) in these constructions, we can now recover \(E^*(X)\) in a functorial way from \(E^*(\Flag(V))\) and from \(E^*(\Flag(V) \times_X \Flag(V))\).

\begin{proposition}
Let \(\rho \colon G \rightarrow U(n)\) be an injective map, then \(U(n)/\mathbb{T}\) with the induced \(G\)-action is a finite \(G\)--CW complex with abelian stabilizers.
\end{proposition}
\begin{proof}
By \Cref{app:CWapprox}, since \(U(n)/\mathbb{T}\) is compact it is equivalent to a finite \(G\)--CW complex. Let \(A \subset G\) be abelian, then \(\rho(A) \subset u\mathbb{T}u^{-1}\) for some \(u \in U(n)\). Thus for \(a \in A\), \(a = utu^{-1}\) for some \(t \in \mathbb{T}\), so \(A\) fixes the coset \(u\mathbb{T}\).

On the other hand, if \(H \subset G\) is nonabelian then it cannot be contained inside a maximal
torus because the representation \(\rho\) is faithful. Therefore it will not fix any coset of \(\mathbb{T} \subset U(n)\).
\end{proof}

Putting these results together, if \(X\) is a finite \(G\)--CW complex we can take \(V \cong X \times \C^n \rightarrow X\) to be the \(G\)-vector bundle where \(G\) acts on \(\C^n\) through a faithful representation \(\rho \colon G \hookrightarrow U(n)\). Then 
\[
EG \times_G (X \times \C^n) \rightarrow EG \times_G X
\]
is an \(n\)--dimensional vector bundle over \(EG \times_G X\) and
\[
\Flag(EG \times_G (X \times \C^n)) \simeq EG \times_G \Flag(X \times \C^n) \simeq EG \times_G (X \times U(n)/\mathbb{T}).
\]
Since \(E^{*}_{G}(X) = E^*(EG \times_G X)\) can be recovered from the flag bundle as above, these equivalences show that \(E^*(EG \times_G X)\) can be recovered from spaces with fixed points only for abelian subgroups. 

\begin{proposition} \label{app:codescent}
Let \(g \colon E_{G}^*(-) \rightarrow F_{G}^*(-)\) be a map of Borel equivariant cohomology theories on finite \(G\)-CW complexes satisfying complex oriented descent, then \(g\) is an isomorphism if and only if it is an isomorphism when applied to spaces of the form \(X = G/A\) for \(A \subset G\) abelian.
\end{proposition}

%We are interested in applying complex oriented descent to two cohomology theories, \(E = (E_{\Gamma})^*(EG \times_G -)\) and \(F = HC_0^*(EG \times_G \Fix_n(-))\). To apply complex oriented descent to \(HC_0^*(EG \times_G \Fix_n(-))\) we need a result regarding fixed points of bundles.


\subsection*{The character map}
%Essentially follow the construction in the current document. All of the ingredients have been introduced. Prove that the map induces an isomorphism on finite abelian groups. Brief discussion of equivariance.

%The goal of this section is to apply Proposition \ref{codescent} to \(E = C_0 \otimes_{E^0} (E_{\Gamma})^*(EG \times_G -)\) and \(F = HC_0^*(EG \times_G \Fix_n(-))\). We will construct a map of Borel equivariant cohomology theories 

The goal of this section is to construct a map of Borel equivariant cohomology theories 
\[
\chi_G \colon E^*(EG \times_G -) \rightarrow HC_{0}P^{*}(EG \times_G \Fix_d(-))
\]
called the character map and apply \Cref{app:codescent} to prove the following theorem.

\begin{theorem} \label{app:mainthm1}
For any finite \(G\)--CW complex \(X\), the map \(\chi_G\) induces an isomorphism
\[
C_0 \otimes_{E^0} E^*(EG \times_G X) \xrightarrow{\cong} HC_{0}P^{*}(EG \times_G \Fix_d(X)).
\]
\end{theorem}

The map \(\chi_G\) will be constructed as the composite of two maps: a ``topological map'' induced by a map of topological spaces and an ``algebraic map'' that is a consequence of the algebro-geometric description of \(C_0\).

Assume that \(k\) is large enough that any map \(\Lambda \rightarrow G\) factors through \(\Lk\).  \Cref{inertialemma} provides us with equivalences
\begin{align*}
\InternalHom{TopologicalGroupoids}(* \mmod \Lambda, X \mmod G) & \simeq \\
\InternalHom{TopologicalGroupoids}(* \mmod \Lk, X \mmod G) & \simeq \Fix_d(X)\mmod G.
\end{align*}
The topological part of the character map is the map
\[
E^*(EG \times_G X) \xrightarrow{} E^*(B\Lk \times EG\times_G \Fix_d(X))
\]
that we get by applying \(E^*(-)\) to the geometric realization of the evaluation map
\[
\mathrm{ev} \colon * \mmod \Lk \times \InternalHom{TopologicalGroupoid}(* \mmod \Lk, X \mmod G) \xrightarrow{} X \mmod G.
\]
Because \(E^*(B\Lk)\) is finitely generated and free over \(E^*\), there is a K\"unneth isomorphism
\[
E^*(B\Lk \times EG\times_G \Fix_d(X)) \cong E^*(B\Lk) \otimes_{E^*} E^*(EG \times_G \Fix_d(X)).
\]
Since \(E^*(B\Lk)\) is even periodic, this can be further identified with
\[
E^0(B\Lk) \otimes_{E^0} E^*(EG \times_G \Fix_d(X)).
\]
Recall that the definition of \(C_0\) furnishes us with a canonical map
\[
E^0(B\Lk) \xrightarrow{} C_0.
\]
%\cong C_0 \otimes_{\Q \otimes E^0} H(\Q \otimes E^0)P^*(EG \times_G \Fix_n(X))
The algebraic part of the character map
\[
E^0(B\Lk) \otimes_{E^0} E^*(EG \times_G \Fix_d(X)) \xrightarrow{} HC_{0}P^{*}(EG \times_G \Fix_d(X)) 
\]
is defined to be the tensor product of the canonical ring map
\[
E^0(B\Lk) \xrightarrow{} C_0
\]
with the ring map induced by the map of cohomology theories
\[
E \rightarrow H(\Q \otimes E^0)P \rightarrow HC_0P
\]
given by rationalization followed by the canonical map \(\Q \otimes E^0 \rightarrow C_0\).
%\[
%E^0(EG \times_G \Fix_n(X)) \rightarrow (\Q \otimes E)^0(EG \times_G \Fix_n(X))
%\]
%over the map \(E^0 \rightarrow \Q \otimes E^0\).

The topological and algebraic maps compose to give the character map
\[
\chi_G \colon E^*(EG \times_G X) \xrightarrow{} HC_{0}P^{*}(EG \times_G \Fix_d(X)).
\]


\begin{example} \label{app:charmapexample}
When \(X = \ast\), \Cref{app:classfncs} shows that the codomain of the character map is the ring of generalized class functions 
\[
\mathrm{Cl}(G_{p}^{d}, C_{0}^{*}) = \prod_{[\alpha] \in G_{p}^{d}/G} C_{0}^*. 
\]
Unwrapping the definition of \(\chi_G\) when \(X = \ast\) gives the following simple formula: Given \([\alpha] \in G_{p}^{d}/G\), the character map to the factor corresponding to \([\alpha]\) is just
\[
E^*(BG) \xrightarrow{E^*(B\alpha)} E^*(B\Lk) \xrightarrow{\text{can}} C_{0}^{*}.
\]
\end{example}


\begin{proposition}
The Borel equivariant cohomology theories \(E^*(EG \times_G -)\) and \(HC_{0}P^{*}(EG \times_G \Fix_d(-))\) both satisfy complex oriented descent.
\end{proposition}
\begin{proof}
For \(E^*(EG \times_G -)\), this follows immediately from the previous discussion.  The key ingredient is \Cref{finitefreeflags} along with \Cref{app:orbitflags}, which say that \(E^*(EG \times_G \Flag(V))\) is a finitely generated free \(E^*(EG \times_G X)\)--module. 

It is a bit harder to prove that \(HC_{0}P^{*}(EG \times_G \Fix_d(-))\) satisfies complex oriented descent. First recall that since \(G\) is finite and \(C_0\) is rational, for any \(G\)-space \(X\) there is an isomorphism
\[
HC_0P^*(EG \times_G X) \cong HC_0P^*(X)^G.
\]
By Proposition \ref{app:inertiaflags}, \(\Fix_d(\Flag(V))\) is component-wise a disjoint union of fiber products of flag bundles over the components of \(\Fix_d(X)\). Thus Proposition \ref{finitefreeflags} implies that
\[
HC_0P^*(\Fix_d(\Flag(V)))
\]
is finitely generated and free over \(HC_0P^*(\Fix_d(X))\). Applying fixed points \((-)^G\) to the exact sequence
\begin{align*}
0 \to HC_0P^*(\Fix_d(X)) & \to HC_0P^*(\Fix_d(\Flag(V))) \\
& \to HC_0P^*(\Fix_d(\Flag(V)) \times_{\Fix_d(X)} \Fix_d(\Flag(V)))
\end{align*}
gives an exact sequence
\begin{align*}
0 \to HC_0P^*(\Fix_d(X))^G & \to HC_0P^*(\Fix_d(\Flag(V)))^G \\
& \to HC_0P^*(\Fix_d(\Flag(V)) \times_{\Fix_d(X)} \Fix_d(\Flag(V)))^G
\end{align*}
since fixed points is left exact. This implies that \(HC_0P^*(EG \times_G \Fix_d(X))\) can be recovered from \(HC_0P^*(EG \times_G-)\) of finite \(G\)--CW complexes with abelian stabilizers.
\end{proof}


\begin{proof}[{Proof of \Cref{app:mainthm1}}]
By \ref{app:codescent}, it suffices to prove that the map is an isomorphism for \(X = G/A\). Since we have an equivalence
\[
(G/A) \mmod G \simeq * \mmod A,
\]
we are reduced to the case \(G = A\) and \(X = *\). Since both sides of the character map have Kunneth isomorphisms for products of finite abelian groups, we are reduced to cyclic groups. We want to show that 
\[
C_0 \otimes_{E^0} E^*(BC_{p^k}) \xrightarrow{\cong} \mathrm{Cl}((C_{p^k})_{p}^{d}, C_{0}^{*})
\]
is an isomorphism. Since we have isomorphisms
\[
C_0 \otimes_{E^0} E^*(BC_{p^k}) \cong C_{0} \otimes_{E_0} E^* \otimes_{E^0} E^0(BC_{p^k}) \cong C_{0}^* \otimes_{C_0} C_0 \otimes_{E^0} E^0(BC_{p^k})
\]
and
\[
\mathrm{Cl}((C_{p^k})_{p}^{d}, C_{0}^{*}) \cong C_{0}^* \otimes_{C_0} \mathrm{Cl}((C_{p^k})_{p}^{d}, C_{0})
\]
and since \(C_{0}^*\) is faithfully flat over \(C_0\), it suffices to prove the isomorphism in degree \(0\). Example \ref{app:charmapexample} that this map is given by 
\[
C_0 \otimes_{E^0} E^0(BC_{p^k}) \xrightarrow{\prod C_{0} \otimes_{E^0} E^0(B\alpha)} E^0(B\Lk) \xrightarrow{\mathrm{incl}} \prod_{\alpha \in \Lk^*}C_0.
\]
This is the same as the map \(C_0 \otimes_{E^0(B\Lk)} u^*\) built far above.


% But the description of the character map in this case from Example \ref{app:charmapexample} agrees with the ring of functions (Equation \ref{app:gfunctions}) on the algebro-geometric map of . 
%
%When \(G = C_{p^k}\) and \(X = \ast\), we see from Example \ref{zpk} that the character map reduces to
%\[
%E^0(BC_{p^k}) \xrightarrow{} \prod_{\Lk^*}C_{0}.
%\]
%From Example \ref{charmapexample} that the map is induced by the elements of the Pontyagin dual of \(\Lk\) in exactly the same way as the construction of the isomorphism in Lemma \ref{isolemma}. Thus this map is precisely the global sections of the canonical map
%\[
%\Lambda_{k}^* \xrightarrow{} \G_{E}[p^k]. \qedhere
%\]
\end{proof}



%\begin{example} \label{trivG}
%When \(X\) is a point, the character map takes the form
%\begin{align*}
%E^*(BG) & \to \prod_{[\alpha] \in G_{p}^{d}/G} HC_0P^*(BC(\im \alpha)) \cong \prod_{[\alpha] \in G_{p}^{d}/G} HC_0P^*.
%\end{align*}
%More generally, when \(G\) acts on \(X\) trivially, the character map takes the form
%\begin{align*}
%E^*(BG \times X) & \to \prod_{[\alpha] \in G_{p}^{d}/G} HC_{0}P^{*}(BC(\im \alpha) \times X) \cong \perod_{[\alpha] \in G_{p}^{d}/G} HC_{0}P^{*}(X).
%\end{align*}
%%This is a generalization of the previous example that is often useful. 
%\end{example}

%Because \(C_0\) is rational, the \(G\) action on \(\Fix_d(X)\) can be ``pulled out":
%\[
%HC_{0}P^*(EG \times_G \Fix_d(X)) \cong (HC_{0}P^*(\Fix_d(X)))^G.
%\]

%In this example we build on the previous example and Example \ref{topexample}.

%The following example explains how \(\Phi_G\) was built to reduce to the algebraic geometry of \pdiv groups. 

\begin{example}
When \(n=1\) and \(X = *\) the character map produces a \(p\)--complete version of the classical character map from representation theory that is due to Adams~\cite[Section 2]{AdamsClassifyingSpacesII}). The map takes the form
\[
K_{p}^{0}(BG) \xrightarrow{} \mathrm{Cl}(G_{p}^{1}, C_0),
\]
where \(C_0\) is the maximal ramified extension of \(\Q_p\) discussed in \Cref{padicktheory}. This map is injective. Above height \(1\), though, this is not generally true~\cite{Kriz}.
\end{example}

\begin{remark}
It follows from the proof of Theorem \ref{app:mainthm1} that when working with a fixed group \(G\), it suffices to take \(C_{0,k}\) as the coefficients of the codomain of the character map, where \(k \geq 0\) is large enough that any continuous map \(\Z_p \xrightarrow{} G\) factors through \(\Z/p^k\).
\end{remark}

\subsection{The action of \(\mathrm{GL}_d(\Z_p)\)}

There is a natural action of \(\mathrm{Aut}(\Lambda) \cong \mathrm{GL}_d(\Z_p)\) on the isomorphism
\[
C_0 \otimes \chi_G \colon C_0 \otimes_{E^0} E^*(EG \times_G X) \xrightarrow{\cong} HC_{0}P^{*}(EG \times_G \Fix_d(X)).
\]
Recall that \(\Spec C_0\) carries the universal isomorphism of \(p\)-divisible groups
\[
\Lambda^* \xrightarrow{\cong} \G.
\]
Precomposing with the Pontryagin dual of an element in \(\mathrm{Aut}(\Lambda)\) gives another isomorphism and thus induces an \(E^0\)-algebra automorphism of \(C_0\). The quotient map
\[
\mathrm{Aut}(\Lambda) \to \mathrm{Aut}(\Lk)
\]  
gives an action of \(\mathrm{Aut}(\Lambda)\) on \(E^0(B\Lk)\) and, by Proposition \ref{app:abeliangroupdualhom}, the inclusion \(E^0(B\Lk) \to C_0\) is equivariant for these actions.

There is also an action of \(\mathrm{Aut}(\Lambda)\) on \(\InternalHom{TopologicalGroupoids}(* \mmod \Lambda, X \mmod G)\) given by precomposition. These actions combine to give a conjugation action on \(HC_{0}P^{*}(EG \times_G \Fix_d(X))\). Explictly, for \(\phi \in \mathrm{Aut}(\Lambda)\) this action is induced by the action on 
\[
* \mmod \Lambda \times \InternalHom{TopologicalGroupoids}(* \mmod \Lambda, X \mmod G)
\]
given by \(\phi \times (\phi^{-1})^*\). There is an action of \(\mathrm{Aut}(\Lambda)\) on \(C_0 \otimes_{E^0} E^*(EG \times_G X)\) by just acting on \(C_0\).



\begin{proposition}
The character map is \(\mathrm{Aut}(\Lambda)\)-equivariant.
\end{proposition}
\begin{proof}
This follows immediately from the fact that the evaluation map
\[
\mathrm{ev} \colon * \mmod \Lambda \times \InternalHom{TopologicalGroupoids}(* \mmod \Lambda, X \mmod G) \to X \mmod G
\]
is \(\mathrm{Aut}(\Lambda)\)-equivariant for the trivial action on \(X \mmod G\) and the action described above on the domain.
\end{proof}

It turns out that \(C_0\) is an \(\mathrm{Aut}(\Lambda)\)-Galois extension of \(\Q \otimes E^0\). The following result is Corollary 6.8(iii) of \cite{HKR}:
\begin{proposition}
There is an isomorphism
\[
C_{0}^{\mathrm{Aut}(\Lambda)} \cong \Q \otimes E^0.
\]
\end{proposition}

By taking \(\mathrm{Aut}(\Lambda)\)-fixed points of \(C_0 \otimes \chi_G\) we get an isomorphism
\[
\Q \otimes \chi_G \colon \Q \otimes E^*(EG \times_G X) \xrightarrow{\cong} HC_{0}P^{*}(EG \times_G \Fix_d(X))^{\mathrm{Aut}(\Lambda)}.
\]
In particular, when \(X = *\), we find that
\[
\Q \otimes E^0(BG) \cong \mathrm{Cl}(G_{p}^{d}, C_0)^{\mathrm{Aut}(\Lambda)}.
\]
That is, the rationalization of the \(E\)-cohomology of \(BG\) is a subring of generalized class functions.
