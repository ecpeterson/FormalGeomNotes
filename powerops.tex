% -*- root: main.tex -*-



\chapter{Power operations}

Our goal in this Appendix is to give a tour of the interaction of the $\sigma$--orientation with a topic of modern research, the theory of $E_\infty$ ring spectra, in a manner consistent with the rest of the topics in this book.  Because the theory of $E_\infty$ ring spectra (in particular: their algebraic geometry) is still very much developing, we have no hope of stating results in their maximum strength or giving a completely clear picture---the maximum strength is unknown and the picture is still resolving.  Although $E_\infty$ ring spectra themselves were introduced decades ago, we will even avoid giving a proper definition of them here, instead referring to the original work of May and collaborators~\cite{EKMM} and the more recent work of Lurie~\cite[Chapter 7]{LurieHA} for a proper treatment.  In acknowledgement of this underwhelming level of rigor, we have downgraded our discussion from a Case Study to an Appendix.

As far as we are concerned, $E_\infty$ ring spectra appear in order to solve the following problem: given two ring spectra $R$ and $S$ in the homotopy category, the set of homotopy classes of ring maps $\CatOf{RingSpectra}(R, S)$ forms a subset of the set of all homotopy classes $[R, S] = \pi_0 \CatOf{Spectra}(R, S)$, selected by a homomorphism condition.  There is no meaningful way to enrich this to a \emph{space} of ring spectrum maps from $R$ to $S$, which inhibits us from understanding an obstruction theory for ring spectra, i.e., approximating $R$ by ``nearby'' ring spectra $R'$ in a way that relates $\CatOf{RingSpectra}(R, S)$ and $\CatOf{RingSpectra}(R', S)$ by a fiber sequence.

The extra data that accomplishes this mapping space feat turns out to be an explicit naming of the homotopies controlling the associativity and commutativity of the ring spectrum multiplication, which are subject to highly intricate compatibility conditions.\footnote{This is rather analogous to the extra data required on a \emph{space}, beyond just a multiplication, which allows one to use the bar construction to assemble a delooping.}  Again, rather than spell this out, it suffices for our purposes to say that there is such a notion of a structured ring spectrum that begets a mapping space between two such.  Additionally, we record the following omnibus theorem as indication that this program overlaps with the one we have been describing already:
\begin{theorem}
\citeme{this is a mega-theorem}
\todo{This is the first mention of $\tmf$.  Give a description of $\tmf$, $\TMF$, and $\Tmf$ in terms of $\moduli{ell}$.}
The classical $K$--theories $KU$ and $KO$, the Eilenberg--Mac Lane spectra $HR$, the Morava $E$--theories $E_\Gamma$, their fixed point spectra, the Thom spectra arising from the $J$--homomorphism (including $MO$, $MSO$, $M\Spin$, $M\String$, $MU$, $MSU$, and $MU[6, \infty)$), the spectra $\TMF$, $\Tmf$, and $\tmf$ are all $E_\infty$--ring spectra.\footnote{Notably, the Morava $K$--theories are \emph{not} $E_\infty$ rings at finite heights.} \qed
\end{theorem}

There is also a forgetful map down to ring spectra in the homotopy category, which come equipped with extra with factorizations expressing these associativity and commutativity relations.  Specifically, recall the following definition from the discussion in \Cref{QuillenPowerOpnsSection}:

\begin{definition}[{cf.\ \Cref{QuillenPowerOpnsSection}}]
\citeme{BMMS}
An $H_\infty$ ring spectrum is a ring spectrum $E$ equipped with factorizations $\mu_n$ as in
\begin{center}
\begin{tikzcd}
E^{\sm n} \arrow["\mu"]{r} \arrow{d} & E \\
E^{\sm n}_{h\Sigma_n} \arrow{ru}[description]{\mu_n},
\end{tikzcd}
\end{center}
which are subject to compatibilities induced by the inclusions $\Sigma_n \times \Sigma_m \subseteq \Sigma_{n+m}$ and the inclusions $\Sigma_n \wr \Sigma_m \subseteq \Sigma_{nm}$.
\end{definition}

\begin{lemma}
Each $E_\infty$ ring spectrum gives rise to an $H_\infty$ ring spectrum in the homotopy category. \qed
\end{lemma}

\noindent We care about this secondary definition because our results thus far have all concerned the cohomology of spaces, which is, at its core, a calculation at the level of \emph{homotopy classes}.  This is therefore as much of the $E_\infty$ structure as one could hope would interact with our analyses in the preceding Case Studies.

In \Cref{QuillenPowerOpnsSection}, \Cref{StabilizingTheMUSteenrodOps}, and \Cref{CalculationOfMUStarSection}, we introduced an $H_\infty$ ring structure on $MU$ and used it to make a calculation of the coefficient ring $MU_*$.  Our primary goal in this Appendix is to introduce an $H_\infty$ ring structure on certain chromatically interesting spectra, including Morava's theories $E_\Gamma$, and to describe the compatibility laws arising from intertwining these two $H_\infty$ structures.  The culminating result is as follows:

\begin{theorem}
\citeme{Put a forward reference to a section below, and another reference to the definition of norm-coherence}
An orientation $MU[6, \infty) \to E_\Gamma$ is $H_\infty$ if and only if the induced cubical structure is ``norm-coherent''. \qed
\end{theorem}

\noindent Before addressing this, we discuss in \Cref{CharacterTheorySection} an important phenomenon: after deleting certain forms of torsion, the Morava $E$--homology of a finite spectrum with respect to a formal group $\Gamma$ can be well--approximated by its Morava $E$--homology with respect to a formal group $\Gamma'$ satisfying $\height{\Gamma'} < \height{\Gamma}$.  This is interesting in its own right, and we will quickly see that the precise form of the approximation bears directly on the study of power operations.  Finally, with the homotopy category exposed, we give a summary of the known results about $E_\infty$ orientations themselves in \Cref{JuvitopTalkSection}.













\section{Rational phenomena: character theory for Lubin--Tate spectra}\label{CharacterTheorySection}

There's a sufficient amount of reliance on character theory in Matt's thesis that we should talk about it.  You should write that action and then backtrack here to see what you need for it.

See Morava's \textit{Local fields} paper

\begin{remark}
Theorem 2.6 of Greenlees--Strickland for a nice transchromatic perspective.  See also work of Stapleton and Schlank--Stapleton, of course.\todo{Flesh this out.}
\end{remark}


------

\begin{theorem}\citeme{Theorem A}
Let $E$ be any complex-oriented cohomology theory.  Take $G$ to be a finite group and let $\CatOf{Ab}_G$ be the full subcategory of the orbit category of $G$ built out of abelian subgroups of $G$.  Finally, let $X$ be a finite $G$--CW complex.  Then, each of the natural maps \[E^*(EG \times_G X) \to \lim_{A \in \CatOf{Ab}_G} E^*(EG \times_A X) \to \int_{A \in \CatOf{Ab}_G} E^*(BA \times X^A)\] becomes an isomorphism after inverting the order of $G$.  In particular, there is an isomorphism
\[\pushQED{\qed}
\frac{1}{|G|} E^* BG \to \lim_{A \in \CatOf{Ab}_G} \frac{1}{|G|} E^* BA. \qedhere
\popQED\]
\end{theorem}

This is an analogue of Artin's theorem:
\begin{theorem}
There is an isomorphism
\[\pushQED{\qed}
\frac{1}{|G|} R(G) \to \lim_{C \in \CatOf{Cyclic}_G} \frac{1}{|G|} R(C). \qedhere
\popQED\]
\end{theorem}


------

HKR intro material connecting Theorem A to character theory:

Recall that classical characters for finite groups are defined in the following situation: take $L = \Q^{\mathrm{ab}}$ to be the smallest characteristic $0$ field containing all roots of unity, and for a finite group $G$ let $Cl(G; L)$ be the ring of class functions on $G$ with values in $L$.  The units in the profinite integers $\widehat{\Z}$ act on $L$ as the Galois group over $\Q$, and since $G = \CatOf{Groups}(\widehat{\Z}, G)$ they also act naturally on $G$.  Together, this gives a conjugation action on $Cl(G; L)$: for $\phi \in \widehat{\Z}$, $g \in G$, and $\chi \in Cl(G; L)$, one sets \[(\phi \cdot \chi)(g) = \phi(\chi(\phi^{-1}(g))).\]  The character map is a ring homomorphism \[\chi: R(G) \to Cl(G; L)^{\widehat{\Z}},\] and this induces isomorphisms \[\chi: L \otimes R(G) \xrightarrow{\simeq} Cl(G; L)\] and even \[\chi: \Q \otimes R(G) \xrightarrow{\simeq} Cl(G; L)^{\widehat{\Z}}.\]

Now take $E = E_\Gamma$ to be a Morava $E$--theory of finite height $d = \height(\Gamma)$.  Take $E^*(B\Z_p^d)$ to be topologized by $B(\Z/p^j)^d$.  A character $\alpha: \Z_p^d \to S^1$ will induce a map $\alpha^*: E^* \CP^\infty \to E^* B\Z_p^d$.  We define $L(E^*) = S^{-1} E^*(B\Z_p^d)$, where $S$ is the set of images of a coordinate on $\CP^\infty_E$ under $\alpha^*$ for nonzero characters $\alpha$.  Note that this ring inherits an $\operatorname{Aut}(\Z_p^d)$ action by $E^*$--algebra maps.

The analogue of $Cl(G; L)$ will be $Cl_{d,p}(G; L(E^*))$, defined to be the ring of functions $\chi: G_{d, p} \to L(E^*)$ stable under $G$--orbits.  Noting that \[G_{d,p} = \operatorname{Hom}(\Z_p^d, G),\] one sees that $\operatorname{Aut}(\Z_p^d)$ acts on $G_{d,p}$ and thus on $Cl_{d,p}(G; L(E^*))$ as a ring of $E^*$--algebra maps: given $\phi \in \operatorname{Aut}(\Z_p^d)$, $\alpha \in G_{d,p}$, and $\chi \in Cl_{d,p}(G; L(E^*))$ one lets \[(\phi \cdot \chi)(\alpha) = \phi(\chi(\phi^{-1}(\alpha))).\]

Now we introduce a finite $G$--CW complex $X$.  Let \[\operatorname{Fix}_{d, p}(G, X) = \coprod_{\alpha \in \operatorname{Hom}(\Z_p^d, G)} X^{\operatorname{im} \alpha}.\]  This space has commuting actions of $G$ and $\operatorname{Aut}(\Z_p^d)$.  We set \[Cl_{d, p}(G, X; L(E^*)) = L(E^*) \otimes_{E^*} E^*(\operatorname{Fix}_{d,p}(G, X))^G,\] which is again an $E^*$--algebra acted on by $\operatorname{Aut}(Z_p^d)$.  We define the character map ``componentwise'': a homomorphism $\alpha \in \operatorname{Hom}(\Z_p^d, G)$ induces \[E^*(EG \times_G X) \to E^*(B\Z_p^d) \otimes_{E^*} E^*(X^{\operatorname{im} \alpha}) \to L(E^*) \otimes_{E^*} E^*(X^{\operatorname{im} \alpha}).\]  Taking the direct sum over $\alpha$, this assembles into a map \[\chi_{d,p}^G: E^*(EG \times_G X) \to Cl_{d,p}(G, X; L(E^*))^{\operatorname{Aut}(Z_p^d)}.\]
\todo{Nat taught you how to say all these things with $p$--adic tori, which was \emph{much} clearer.}
\begin{theorem}\citeme{Theorem C}
The invariant ring is $L(E^*)^{\operatorname{Aut}(\Z_p^d)} = p^{-1} E^*$, and $L(E^*)$ is faithfully flat over $p^{-1} E^*$.\todo{Checking this invariant ring claim is easiest done by comparing the functors the two things corepresent.}  The character map $\chi_{d,p}^G$ induces isomorphisms
\begin{align*}
\chi_{d,p}^G \co L(E^*) \otimes_{E^*} E^*(EG \times_G X) & \xrightarrow{\simeq} Cl_{d,p}(G, X; L(E^*)), \\
\chi_{d,p}^G \co p^{-1} E^*(EG \times_G X) & \xrightarrow{\simeq} Cl_{d,p}(G, X; L(E^*))^{\operatorname{Aut}(\Z_p^d)}.
\end{align*}
In particular, when $X = *$, there are isomorphisms
\begin{align*}
\chi_{d,p}^G \co L(E^*) \otimes_{E^*} E^*(BG) & \xrightarrow{\simeq} Cl_{d,p}(G; L(E^*)), \\
\chi_{d,p}^G \co p^{-1} E^*(BG) & \xrightarrow{\simeq} Cl_{d,p}(G; L(E^*))^{\operatorname{Aut}(\Z_p^d)}. \qed
\end{align*}
\end{theorem}

------

Jack gives an interpretation of this in terms of formal $\sheaf{O}_L$--modules.

------

I also have this summary of Nat's of the classical case:

It's not easy to decipher if you weren't there for the conversation, but here's my take on it. First, the map we wrote down today was the non-equivariant chern character: it mapped non-equivariant $KU \otimes \Q$ to non-equivariant $H\Q$, periodified. The first line on Nat's board points out that if you use this map on Borel-equivariant cohomology, you get nothing interesting: $K^0(BG)$ is interesting, but $H\Q^*(BG) = H\Q^*(*)$ collapses for finite $G$. So, you have to do something more impressive than just directly marry these two constructions to get something interesting.

That bottom row is Nat's suggestion of what ``more interesting'' could mean. (Not really his, of course, but I don't know who did this first. Chern, I suppose.) For an integer $n$, there's an evaluation map of (forgive me) topological stacks \[* \mmod (\Z/n) \times \operatorname{Hom}(* \mmod (\Z/n), * \mmod G) \xrightarrow{\mathrm{ev}} * \mmod G\] which upon applying a global-equivariant theory like $K_G$ gives \[K_{\Z/n}(*) \otimes K_G(\coprod_{\text{conjugacy classes of $g$ in $G$}} *) \xleftarrow{ev^*} K_G(*).\]

Now, apply the genuine $G$-equivariant Chern character to the $K_G$ factor to get \[K_{\Z/n}(*) \otimes H\Q_G(\coprod *) \from K_{\Z/n}(*) \otimes K_G(\coprod *),\] where the coproduct is again taken over conjugacy classes in G. Now, compute $K_{\Z/n}(*) = R(\Z/n) = \Z[x] / (x^n - 1)$, and insert this calculation to get \[K_{\Z/n}(*) \otimes H\Q_G(\coprod *) = \Q(\zeta_n) \otimes (\bigoplus_{\text{conjugacy classes}} \Q),\] where $\zeta_n$ is an $n${\th} root of unity.  As $n$ grows large, this selects sort of the part of the complex numbers $\C$ that the character theory of finite groups cares about, and so following all the composites we've built a map \[K_G(*) \to \C \otimes (\bigoplus_{\text{conjugacy classes}} \C).\]  The claim, finally, is that this map sends a $G$-representation (thought of as a point in $K_G(*)$) to its class function decomposition.














\section{Orientations and power operations}\label{PowerOpnsChapter}

\begin{center}
\textbf{\Large This is completely under construction.}
\end{center}



Our introduction of $E_\infty$ rings also automatically introduces a few interesting accompanying functors:
\begin{center}
\begin{tikzcd}
\CatOf{Spaces} \arrow["E^{(-)_+}", bend left]{rr} & \CatOf{Modules}_E \arrow["\P_E", shift left=0.4em]{r} \arrow[shift left=0.4em]{d} & \EinftyRings_E \arrow[shift left=0.4em]{l} \arrow[shift left=0.4em]{d} \\
& \CatOf{Spectra} \arrow[shift left=0.4em, "(-) \sm E"]{u} \arrow["\P", shift left=0.4em]{r} & \EinftyRings. \arrow[shift left=0.4em, "(-) \sm E"]{u} \arrow[shift left=0.4em]{l}
\end{tikzcd}
\end{center}
The first functor sends a space $X$ to its spectrum of $E$--cochains $E^{\Susp^\infty_+ X}$, and the other two functors form a free/forgetful monad resolving a mapping space in $\EinftyRings_E$ by a sequence of mapping spaces in $\CatOf{Modules}_E$.  These kinds of functors are familiar to us from the discussion of contexts in \Cref{StableContextLecture} and \Cref{UnstableContextsSection}, and the recipe applied in those situations gives an analogous story here.  First, there is a natural map \[\CatOf{Spaces}(*, X) \to \EinftyRings_E(E^{X_+}, E),\] which one hopes is an equivalence under (often very strong) hypotheses on $E$ and on $X$.\footnote{It is an unpublished theorem of Hopkins and Lurie that if $X$ is a space admitting a finite Postnikov system with at most $\height \Gamma$ stages and involving only finite groups, then the natural map $F(*, X) \to \EinftyRings_{E_\Gamma/}(E_\Gamma^{X_+}, E_\Gamma)$ is an equivalence.}  Second, the adjunction gives a mechanism for resolving $E^{X_+}$, which feeds into a spectral sequence computing this right-hand mapping space.  The functors $\P$ and $\P_E$ can be given by explicit formula:
\begin{align*}
\P(X) & = \bigvee_{j=0}^\infty X^{\sm j}_{h\Sigma_j}, &
\P_E(M) & = \bigvee_{j=0}^\infty M^{\sm_E j}_{h\Sigma_j}.
\end{align*}
Finally, we expect the homotopy groups of this resolution to form a quasicoherent sheaf over a suitable \emph{$E_\infty$ context}, which arises as the simplicial scheme associated to this resolution in the case where $X$ is a point.  In this case, we can explicitly name some of the terms in this resolution: the bottom two stages take the form
\begin{center}
\begin{tikzcd}
\EinftyRings_E(E, E) \\
\EinftyRings_E(\P_E(E), E) \arrow{u} \arrow{r} & \EinftyRings_E(\P_E^2(E), E) \arrow[shift left=0.4em]{l} \arrow[shift right=0.4em]{l} \arrow[shift left=0.4em] {r} \arrow[shift right=0.4em]{r} & \arrow[shift left=0.8em]{l} \arrow{l} \arrow[shift right=0.8em]{l} \cdots.
\end{tikzcd}
\end{center}
The available adjunctions give a more explicit presentation of these terms:
\begin{align*}
\EinftyRings_E(\P_E(E), E) & \simeq \CatOf{Modules}_E(E, E) \\
& \simeq \CatOf{Spectra}(\S, E),
\intertext{which on homotopy groups computes the coefficient ring of $E$, and}
\EinftyRings_E(\P_E^2(E), E) & \simeq \CatOf{Modules}_E(\P_E(E), E) \\
& \simeq \CatOf{Modules}_E\left(\bigvee_{j=0}^\infty E^{\sm_E j}_{h\Sigma_j}, E\right) \\
& \simeq \CatOf{Spectra}\left(\bigvee_{j=0}^\infty \S^{\sm j}_{h\Sigma_j}, E\right) \\
& \simeq \prod_{j=0}^\infty \CatOf{Spectra}\left(B\Sigma_j, E\right),
\end{align*}
which on homotopy groups is made up of a product of the cohomology rings $E^*(B\Sigma_j)$.  The higher terms track the compositional behavior of these summands.

\begin{remark}[{\cite{BousfieldUnstableLocalization,MahowaldThompson}}]
One of the first places these ideas appear in the literature is in work of Mahowald and Thompson.  Bousfield defined an unstable local homotopy type associated to a (simply-connected) space and a homology theory.  In the case of the space $S^{2n-1}$ and $p$--adic $K$--theory, Mahowald and Thompson calculated that $L_K S^{2n-1}$ appears as the homotopy fiber \[L_K S^{2n-1} \to L_K \Loops^\infty \Sigma^\infty S^{2n-1} \to L_K \Loops^\infty \Sigma^\infty ((S^{2n-1})^{\sm p}_{h\Sigma_p}),\] which is an abbreviated form of the monadic resolution described above.
\end{remark}

\begin{remark}
\citeme{This should be in...Charles's work?}
In general, if $E^* B\Sigma_j$ is sufficiently nice, then the the $E^2$--page of the monadic descent spectral sequence computes the derived functors of derivations, taken in a suitable category of monad-algebras for the monad specifying the behavior of power operations.
\end{remark}






\subsection*{Strickland's theorems}

We thus set out to understand the formal schemes constituting the $E_\infty$ context associated to Morava $E$--theory.  As described in \Cref{UnstableContextsSection} and \Cref{UnstableAlgebraicModelSection}, the correct language for these phenomena are schemes defined on Hopf rings, together with the adjunction between classical rings and Hopf rings consisting of the $\ast$--square--zero extension functor and the $\ast$--indecomposables functor.  The rings $E^0 B\Sigma_j$ assemble into a Hopf ring using the following structure:
\begin{itemize}
    \item The $\ast$--product comes from the stable transfer maps $B\Sigma_{i+j} \to B\Sigma_i \times B\Sigma_j$.
    \item The $\circ$--product comes from the diagonal maps $B\Sigma_j \to B\Sigma_j \times B\Sigma_j$.
    \item The diagonal comes from the block-inclusion maps $B\Sigma_i \times B\Sigma_j \to B\Sigma_{i+j}$.
\end{itemize}

\begin{definition}
Accordingly, we set the \textit{natural $E_\infty$ context} to be \[\Econtext{E_\Gamma} = \SpH E^0 B\Sigma_*.\]
\end{definition}

The effect of this functor on classical rings is given by \Cref{HopfRingsAndRingsAdjunction}: $\Econtext{E_\Gamma}(T) = \CatOf{Algebras}_{E_\Gamma^0}(Q^* E^0 B\Sigma_*, T)$, where \[Q^* E^0 B\Sigma_* = \frac{E^0 B\Sigma_*}{\im(\Tr_{\Sigma_{*_1} \times \Sigma_{*_2}}^{\Sigma_{*_1 + *_2}}\co E^0 B\Sigma_{*_1} \times E^0 B\Sigma_{*_2} \to E^0 B\Sigma_{*_1 + *_2})}.\]  The ideal appearing in this equation is called the \textit{transfer ideal}, written $I_{\Tr}$.

\begin{remark}
In terms of the descent spectral sequence described in the previous subsection, all of the $*$--decomposables are in the image of the $d^1$--differential, so do not contribute to even the $E^2$--page of the spectral sequence.
\end{remark}

We select a particular value of $j$ and set out to understand the formal scheme $\Spec (E^0 B\Sigma_j / I_{\Tr})$, which appears as the $j${\th} graded piece of $\Econtext{E_\Gamma}$ as restricted to classical rings.

\begin{example}
To gain a foothold, it is helpful to further specify to a particular case---say, $j = p$.  In light of the results of \Cref{CharacterTheorySection}, we might begin by analyzing the (maximal) abelian subgroups of $\Sigma_p$, of which there is only one: the transitive subgroup $C_p \subseteq \Sigma_p$.  In \Cref{KtheoryConvertsTorsionToTorsion}, we calculated $BS^1[p]_E = \CP^\infty_E[p]$, and we now make the further observation that the regular representation map $BS^1[p] \to BU(p)$ induces the following map on cohomological formal schemes:
\begin{center}
\begin{tikzcd}
BS^1[p]_E \arrow{r} & BU(p)_E \\
\InternalHom{FormalGroups}(\Z/p, \CP^\infty_E) \arrow{r} \arrow[equal]{u} & \Div_p^+ \CP^\infty_E \arrow[equal]{u},
\end{tikzcd}
\end{center}
where the bottom arrow sends such a homomorphism to its image divisor.  This map belongs to a larger diagram of schemes:
\begin{center}
\begin{tikzcd}
\Spf E^0 B\Sigma_p / I_{\Tr} \arrow{r} & (B\Sigma_p)_E \arrow{r} & BU(p)_E \\
\Spf E^0 BC_p / I_{\Tr} \arrow{r} \arrow{u} & (BC_p)_E \arrow{u} \arrow{ru}.
\end{tikzcd}
\end{center}
The effect of killing the transfer ideal in $(BC_p)_E$ is to force the image divisor to be a subdivisor of $\CP^\infty_E[p]$\todo{Explain why.} (i.e., the zero homomorphism is disallowed), and this subscheme of homomorphisms is written $\Level(\Z/p, \CP^\infty_E)$.  Finally, passing to $B\Sigma_p$ from $BC_p$ exactly destroys the choice of generator of $\Z/p$, i.e., it encodes passing from the homomorphism to the image divisor.  This winds up giving an isomorphism \[\Spf E^0 B\Sigma_p / I_{\Tr} \cong \Sub_p \CP^\infty_E,\] where $\Sub_p$ denotes the subscheme of $\Div_p^+$ consisting of those effective Weil divisors of rank $p$ which are subgroup divisors.
\end{example}

The broad features of this example hold for a general index $j$.
\begin{definition}
The \textit{abelian $E_\infty$ context} is formed by considering the inclusions \[\bigvee_{\substack{A \le \Sigma_j \\ \text{$A$ abelian}}} BA \to B\Sigma_j.\]
\end{definition}
\noindent A consequence of \Cref{CharacterTheorySection} is that this map is \emph{injective} on Morava $E$--cohomology, so that we can understand the natural $E_\infty$ context in terms of this larger object.  A benefit to this auxiliary context is that we can already predict its behavior in Morava $E$--theory: the cohomological formal scheme associated to an abelian group can be presented as an internal scheme of group homomorphisms, just as above.  Strickland has proven the following results about each of these contexts:

\begin{theorem}[{\cite[Theorem 1.1]{StricklandEthyOfBSigma}}]
There is an isomorphism \[\Spf E^0 B\Sigma_j / I_{\Tr} \cong \Sub_j \CP^\infty_E,\] where $\Sub_j$ denotes the subscheme of $\Div_j^+$ consisting of those effective Weil divisors of rank $j$ which are subgroup divisors.\footnote{If $j$ is not a power of $p$, this is the terminal scheme.} \qed
\end{theorem}

\begin{theorem}\citeme{Strickland somewhere}
For a finite abelian group $A$, there is an isomorphism \[\Spf E^0 BA / I_{\Tr} \cong \Level(A^*, \CP^\infty_E),\] where $\Level(A^*, \CP^\infty_E)$ denotes the subscheme of $\InternalHom{FormalGroups}(A^*, \CP^\infty_E)$ subject to the condition that $A^*[n]$ forms a subdivisor of $\CP^\infty_E[n]$.\footnote{If $A[p] \cong (\Z/p)^{\times k}$ has $k > \height{\Gamma}$, then this is the terminal scheme.}  \qed
\end{theorem}

\begin{remark}
An important piece of intuition about the schemes $\Level(A^*, \CP^\infty_E)$ is that they form a kind of replacement for the nonexistent ``scheme of monomorphisms''.  Specifically, the $p$--series for a Lubin--Tate universal deformation group is only once $x$--divisible, and hence the divisor $\G[p]$ only contains the divisor $[0]$ with multiplicity one.  This excludes noninjective morphisms in this case.  On the other hand, the only subgroups of the formal group restricted to the special fiber are of the form $p^m \cdot [0]$.  In particular, any level structure on $\G$ restricts to a morphism with this image divisor at the special fiber, and hence functoriality considerations force us to count these---which are \emph{not} images of monomorphisms---as level structures as well.\footnote{This kind of reasoning applies to domains of characteristic $0$.}
\end{remark}

\begin{remark}[{\cite{StricklandFiniteSubgps}}]
The schemes $\Sub_j \G$ and $\Level(A, \G)$ are known to possess many very pleasant algebraic properties: they are finite and free of predictable rank, they have Galois descent properties, the schemes $\Level(A, \G)$ are all reduced, there are important decompositions coming from presenting a subgroup scheme as a flag of smaller subgroups, \ldots.  Indeed, these algebraic results form important ingredients to the proof of the connection with homotopy theory~\cite[Section 9]{StricklandEthyOfBSigma}.
\end{remark}










\subsection*{Isogenies and the Lubin--Tate moduli}

In this subsection, we seek a comparison of the natural $E_\infty$ context and the unstable context considered in \Cref{UnstableContextsSection}.  Our model for the unstable context in \Cref{UnstableAlgebraicModelSection} focuses on the effect of unstable operations on the cohomology of $\CP^\infty$, as summarized in the following result:

\begin{lemma}[{mild extension of \Cref{UnstableCooperationsChapter}}]
There is an isomorphism \[\Spec Q^* \pi_* L_\Gamma(E \sm E) \cong \InternalHom{FormalGroups}(\CP^\infty_E, \CP^\infty_E). \qed\]
\end{lemma}

\noindent In order to form a comparison map between these two contexts, we will want algebraic constructions that trade a subgroup divisor (i.e., a point in the natural $E_\infty$ context) for a formal group homomorphism (i.e., a point in the unstable context).  It will be useful to phrase our ideas in the language of \emph{isogenies}.

\begin{definition}[{\cite[Definition 5.17]{StricklandFSFG}}]
Take $C$ and $D$ to be formal curves over $X$.  A map $f\co C \to C'$ is an \textit{isogeny} (of degree $d$) when the induced map $C \to C \times_X C'$ exhibits $C$ as a divisor (of rank $d$) on $C \times_X C'$ as $C'$--schemes.
\end{definition}

\begin{remark}[{cf.\ \Cref{DivHasPushforwards}}]
In this case, a divisor $D$ on $C'$ gives rise to a divisor $f^* D$ on $C$ by scheme-theoretic pullback:
\begin{center}
\begin{tikzcd}
f^* D \arrow{r} \arrow{d} & D \arrow{d} \\
C \arrow["f"]{r} & C',
\end{tikzcd}
\end{center}
altogether inducing a map \[f^*\co \Div_n^+ C' \to \Div_{nd}^+ C.\]  This map interacts with pushforward by $f_* f^* D = d \cdot D$, where $d$ is the degree of the isogeny.
\end{remark}

The usual source of examples of isogenies are polynomial maps between curves.  In fact, this is close to the general case, and the following result is the source of much intuition:
\begin{lemma}[Weierstrass preparation]
Let $R$ be a complete local ring.  Every degree $d$ isogeny $f\co \A^1_R \to \A^1_R$ admits a unique factorization as a coordinate change and a monic polynomial of degree $d$. \qed
\end{lemma}

\noindent In the case of formal groups over a perfect field of positive characteristic, this reduces to two familiar structural results:

\begin{corollary}
Every nonzero map of formal groups over a perfect field of positive charactersitic can be factored as an iterate of Frobenius and a coordinate change.\footnote{Incidentally, the Frobenius iterate appearing in the Weierstrass factorization of the multiplication-by-$p$ isogeny $p\co \G \to \G$ is another definition of the height of $\G$.} \qed
\end{corollary}

\begin{corollary}
A map of formal groups over a complete local ring with a perfect positive-characteristic residue field is an isogeny if and only if the kernel subscheme of the map is a divisor. \qed
\end{corollary}

This last result is the fountainhead of the connection we are seeking: isogenies are exactly the class of formal group homomorphisms which have subgroup divisors as kernels.  On the other hand, the unstable context consists of \emph{all} formal group \emph{endomorphisms}, so we would like a construction in the opposite direction which converts a subgroup to an \emph{endomorphism}.  As a first approximation to this goal, and in keeping with our exploration of the basic features of isogenies, we note that isogenies of formal groups are automatically fppf-surjective and that it is an accordingly good idea to address \emph{quotients} of formal groups as well as the basic isomorphism theorems.

\begin{definition}
For $K \subseteq \G$ be a subgroup divisor, we define the quotient group $\G/K$ to be the formal scheme whose ring of functions is the coequalizer
\begin{center}
\begin{tikzcd}
\sheaf O_{\G/K} \arrow{r} & \sheaf O_{\G} \arrow[shift left=0.4em, "\mu^*"]{r} \arrow[shift right=0.4em, "1 \otimes \eta^*"']{r} & \sheaf O_{\G} \otimes \sheaf O_K.
\end{tikzcd}
\end{center}
\end{definition}

\begin{lemma}[{\cite[Theorem 5.3.2-3]{StricklandFiniteSubgps}}]
The functor $\G/K$ is again a $1$--dimensional smooth commutative formal group. \qed
\end{lemma}

The inclusion of rings of functions determines an isogeny $q\co \G \to \G/K$ of degree $|K|$.  In this particular case, the induced pullback map $q^*$ of divisor schemes has an especially easy formulation:
\begin{lemma}
Pullback along the isogeny $q\co \G \to \G/K$ is computed by \[q^* \co D \mapsto D * K,\] where $*\co \Div_n^+ \G \times \Div_d^+ \G \to \Div_{nd}^+ \G$ is the convolution product of divisors on formal groups as described in \Cref{ProductMapOfDivisorSchemes}. \qed
\end{lemma}

In the case that $K$ is specified by a level structure, this admits a further refinement:
\begin{corollary}
Let $\ell\co A \to \G$ be a level structure parametrizing a subgroup divisor $K$.  The divisor pullback map can then be computed by the expansion \[q^* D = \sum_{a \in A} \tau_{\ell(a)}^* D,\] where $\tau_g\co \G \to \G$ is the translation by $g$ map. \qed
\end{corollary}

\begin{definition}
This second construction can be upgraded to an assignment from \emph{functions} on $\G$ to \emph{functions} on $\G/K$, rather than just the ideals that they generate.  Specifically, we define the \textit{norm} of $\phi \in \sheaf O_{\G}$ along a level structure $\ell\co A \to \G$ by the formula \[N_\ell \phi = \prod_{a \in A} \tau_{\ell(a)}^* \phi.\]  An often-useful property of this norm construction is that if $\phi$ is a coordinate on $\G$, then $N_\ell \phi$ is a coordinate on $\G/K$~\cite[Theorem 5.3.1]{StricklandFiniteSubgps}.
\end{definition}

\begin{remark}
In general, the pullback map $q^*$ admits the following description: an isogeny $q\co C \to C'$ gives a presentation of $\sheaf O_C$ as a finite free $\sheaf O_{C'}$--module.  A function $\phi \in \sheaf O_C$ therefore begets a linear endomorphism $\phi \cdot (-) \in \operatorname{GL}_{\sheaf O_{C'}}(\sheaf O_C)$, and the determinant of this map gives an element of $q^* \phi \in \sheaf O_{C'}$.  Letting $\phi$ range, this gives a multiplicative (but not typically additive) map $q^*\co \sheaf O_C \to \sheaf O_{C'}$.  If a divisor $D$ is specified as the zero-locus of a function $\phi_D$, the divisor $q^* D$ is specified as the zero-locus of $q^* \phi_D$.
\end{remark}

Our last technical remark is that this definition of quotient does, indeed, have the universal property of a quotient:

\begin{lemma}[Third isomorphism theorem for formal groups, {\cite[Theorem 5.3.4]{StricklandFiniteSubgps}}]
If $f\co \G \to \mathbb H$ is any isogeny with kernel divisor $K$, then there is a uniquely specified commuting triangle
\begin{center}
\begin{tikzcd}
& \G \arrow["q"']{ld} \arrow["f"]{rd} \\
\G/K \arrow["g", "\simeq"']{rr} & & \mathbb H. \qed
\end{tikzcd}
\end{center}
\end{lemma}

We now use this Lemma, along with properties of the Lubin--Tate moduli problem, to associate endo-isogenies to subgroup divisors.  To begin, consider the multiplication--by--$p$ endo-isogeny of a Lubin--Tate group $\G$.  Since $\G$ is finite height, this map is an isogeny and the Lemma above gives rise to an isomorphism
\begin{center}
\begin{tikzcd}
& \G \arrow["q"']{ld} \arrow["p"]{rd} \\
\G/\G[p] \arrow["g", "\simeq"']{rr} & & \G.
\end{tikzcd}
\end{center}
In particular, this diagram shows that the quotient map $\G / \G[p]$ is \emph{again} a universal deformation, as witnessed by a preferred isomorphism to $\G$.  Recall that the following diagram expresses the data of a deformation:
\begin{center}
\begin{tikzcd}
\Gamma \arrow{d} & i^* \Gamma \arrow{l} \arrow{rd} \arrow[equal, "\phi"]{r} & R/\m \otimes G \arrow{r} \arrow{d} & G \arrow{rd} \arrow[equal, "\psi"]{r} & R \otimes \G \arrow{d} \arrow{r} & \G \arrow{d} \\
\Spec k & & \Spec R/\m \arrow["i"]{ll} \arrow{rr} & & \Spf R \arrow{r} & (\mathcal M_{\mathbf{fg}})^\wedge_\Gamma.
\end{tikzcd}
\end{center}


\begin{center}
\begin{tikzcd}[column sep=0em]
 & \operatorname{Frob}^j \otimes \Gamma & & \arrow{ll} \operatorname{Frob}^j \otimes i^* \Gamma \arrow[equal, "\phi'", densely dotted]{rr} & & R/\m \otimes \G' \arrow{rr} & & \G' \\
\Gamma \arrow{ru} & & i^* \Gamma \arrow{ll} \arrow[equal, "\phi"]{rr} \arrow{ru} & & R/\m \otimes G \arrow{rr} \arrow{ru}[description]{R/\m \otimes g} & & \G \arrow{ru}[description]{g}
\end{tikzcd}
\end{center}








In turn, this observation shows that there is some automorphism of Lubin--Tate space\footnote{An obvious but common point of confusion is that the map $\psi_{\G[p]}$ is \emph{not} an $LT_\Gamma$--algebra map.} \[\psi_{\G[p]}\co (\moduli{fg})^\wedge_\Gamma \to (\moduli{fg})^\wedge_\Gamma\] over which $g\co \G/\G[p] \cong \psi_{\G[p]}^* \G$ covers the identity.

The main observation powering this is that the subgroup structure of $\Gamma$ is extremely simple: every subgroup finds expression as a $p^j$--multiple of the zero-point.  This also means that every quotient isogeny, not just the multiplication-by-$p$ endoisogeny, acquires a similar expression as a deformation.

\begin{lemma}
There is a unique sequence of maps \[q_{p^k}\co \G \times \Sub_{p^k} \G \to \G\] determined by the following properties:
\begin{enumerate}
    \item The kernel of $q_{p^k}$ is exactly the universal subgroup.
    \item (\textit{Deformation of Frobenius}:) At the special fiber, $q_{p^k}$ reduces to the $k${\th} Frobenius iterate.  (There is \emph{one} of these in each star-isomorphism class.)
    \qedhere
\end{enumerate}
\end{lemma}

\begin{corollary}
There is there a family of commuting triangles
\begin{center}
\begin{tikzcd}
& \G \arrow["N_K"']{ld} \arrow["q_K"]{rd} \\
\G / K \arrow["{g_K \quad \simeq}"]{rr} & & \psi_K^* \G
\end{tikzcd}
\end{center}
indexed on subgroups $K \subseteq \G$. \qed
\end{corollary}



The power operation for $E$--theory has this description.  In particular, this gives an algebraic model for the map from the $E_\infty$ contexts to the unstable context.



In all these constructions give \emph{two different} sections to the map assigning an isogeny to its kernel subgroup divisor:
\begin{enumerate}
    \item A subgroup divisor has an associated norm construction.  The targets of these norm constructions vary, but this requires no special assumption about the structure of the formal group.
    \item In the case of the Lubin--Tate group, a subgroup divisor also has an associated deformation of Frobenius for which it is the kernel.  These are all (up to change-of-base) endo-isogenies of the Lubin--Tate group.
\end{enumerate}




\begin{remark}[{\cite[1.4.2.3]{CCO}}]
Let $R$ be a (nice) complete local ring, and let $\G$ and $\G'$ be two $p$--divisible groups over $R$.  There is an injection \[\Isog_R(\G, \G') \into \Isog_{R/\m}(\G, \G').\]
\end{remark}











\subsection*{$H_\infty$ orientations}

\begin{theorem}[{\cite[Proposition VIII.7.2]{BMMS}}]
Let $E$ and $F$ be $H_\infty$ ring spectra with power operations $P_E$ and $P_F$.  Suppose that $F$ is $p$--local.  Let $f\co E \to F$ be a ring map such that the equation $f_* \circ P^{\Sigma_p}_E = P^{\Sigma_p}_F \circ f_*$ holds; then $f$ is an $H_\infty$ map. \qed
\end{theorem}

\begin{theorem}
Let $E$ be an $E_\infty$ ring spectrum and let $X$ be a space such that $E^{\Susp^\infty_+ X}$ is a wedge of copies of $E$.  The map \[\iota^* \otimes \Delta^*\co \widetilde E^* X^{\sm G}_{hG} \to \widetilde E^* X^{\sm G} \oplus \widetilde E^*(X \sm BG_+)\] is then injective.\footnote{The actual statement of McClure's result~\cite[Proposition VIII.7.3]{BMMS} has several additional hypotheses: $\pi_* E$ is taken to be even-concentrated and free over $\Z_{(p)}$, $X$ has homology free abelian in even dimensions and zero in odd dimensions, and $X$ and $E$ are both taken to be finite type.  Although this is the theorem cited in the source material~\cite[Section 4]{Ando}~\cite[Proof of Proposition 6.1]{AHSHinfty}, the version that has the weak hypotheses that we require only appeared in print much later.}\todo{Find a Lawson citation for this last claim.} \qed
\end{theorem}

We apply these theorems to the case of an orientation $f\co MUP \to E_\Gamma$, which together show that $f$ is $H_\infty$ if and only if it is compatible with the cyclic $C_p$--power operation.  This power operation was described from $MUP$ in \Cref{AjAndBjAreInTheFGLSubring}, where we found that it applies the norm construction to $f$ for the universal $\Z/p$--level structure on $\G$.  The cyclic power operation on Morava $E$--theory was determined in the previous section to act by pullback along the deformation of Frobenius isogeny associated to the same universal level structure.  This condition is important enough to warrant a name:

\begin{definition}
A coordinate $\phi\co \G \to \A^1$ (or, more generally, a section $s \in \Gamma(\L \downarrow \G)$ of a line bundle $\L$ on $\G$) on a Lubin--Tate group $\G$ is said to be \textit{norm-coherent} when for all subgroups $K \subseteq \G$, we have that $N_K \phi$ and $\psi_K^* \phi$ (or, more generally, $N_K s$ and $\phi_K^* s$) are related by the isomorphism $g_K$.
\end{definition}

\begin{corollary}
The subset of those orientations which are $H_\infty$ correspond exactly to those coordinates which are norm-coherent. \qed
\end{corollary}

Having now finally cast the desired result, our main result is that such coordinates are remarkably common:

\begin{theorem}
Let $\L$ be a line bundle on $\G$ a Lubin--Tate formal group, and let $s_0 \in \Gamma(\L_0 \downarrow \Gamma)$ be a section over the special fiber.  Then there exists exactly one section $s \in \Gamma(\L \downarrow \G)$ restricting to $s_0$ on the special fiber and satisfying the norm-coherence condition above.
\end{theorem}
\begin{proof}[Proof sketch]

\end{proof}

We injected the extra generality into this proof in order to lift our analysis to the other kinds of orientations discussed in \Cref{ChapterSigmaOrientation}.

\begin{theorem}
Orientations $MU[6, \infty) \to E_\Gamma$ which are $H_\infty$ correspond to norm-coherent cubical structures.  If $\height{\Gamma} \le 2$, this also holds for $M\String$--orientations.
\end{theorem}


\begin{example}[{\cite[Section 2.7]{Ando}, cf.\ \Cref{ArtinHasseExponential}}]
The norm-coherent coordinate on $\G_m$ is $x + y - \beta x y$.
\end{example}


\begin{example}
The $\sigma$--orientation is norm-coherent by the unicity clause in \Cref{Theta3IsTrivial}.
\end{example}



\begin{remark}
Power operations in Tate $K$--theory vs this story.
\end{remark}











---------------------------

This MO conversation looks interesting: http://chat.stackexchange.com/transcript/message/29465746\#29465746.

Bousfield's paper \textit{On $\lambda$--rings and the $K$--theory of infinite loopspaces} claims to have applications to Mahowald--Thompson type resolutions.

Don Davis has a \textit{Handbook of Algebraic Topology} chapter on unstable $v_1$--periodic homotopy of spheres.

Baker's POWER OPERATIONS AND COACTIONS IN HIGHLY COMMUTATIVE HOMOLOGY THEORIES seems like a nice place to learn about these things. He advertises some interaction with the traditional context story, which is appealing, and he mostly treats the case of ordinary homology, which we probably ought to spend a section on.


------------

FPFP Example 4.6 is a nice summary of a main result of HKR character theory.  Compare also Example 4.8.

------------


Jeremy pointed out that $\beta \in E_* QS^n$ comes up when considering a class $\alpha \in \pi_n R$ for a commutative $E$--algebra, which we promote to a class $E \sm QS^n \to R$, and then precompose with the homotopy class $\beta$.

This is also related to recent work of Behrens--Rezk on the Bousfield--Kuhn functor...






If you can figure out what Hood was doing with Dyer--Lashof operations in $H\F_2$ and the logarithm/exponential coordinates, that'd be pretty cool to include.  (Neil has a Steenrod Algebra note that brushes against this.)

Barry's description of the image of \[E_\infty(A, B) \to \CatOf{Spaces}(\Loops^\infty A, \Loops^\infty B)\] for $K(1)$--local $A$ and $B$ using $p$--adic moments is pretty digestable.  That might belong in here, or at least it could be referenced.  (I guess it didn't ever get published??)  (Maybe just cite it and give also a reference to the Appendix A.1 stuff.)













\subsection*{Subgroups and level structures}


\todo[color=red,inline]{Matt in and before Theorem 3.3.2 describes the ring $D_k$ as the \emph{image} of the localization map $E_n(B\Lambda_k) \to S^{-1} E_n(B\Lambda_k)$ rather than as the whole target.  Why??  He cites HKR for this, but the citation is meaningless because the theorem numbering scheme is so old.  Ah, comparing with Lemma 3.3.3 yields a clue: $D_k$ has a universal property as it sits under $E_n$, rather than under $E_n(B\Lambda_k)$...}


----

\begin{lemma}\citeme{Prop 6.2 of HKR}
The following conditions on a homomorphism \[\phi: \Lambda_r^* \to F[p^r](R)\] are equivalent:
\begin{enumerate}
\item For all $\alpha \ne 0$ in $\Lambda_r^*$, $\phi(\alpha)$ is a unit (resp., not a zero-divisor).
\item The Hopf algebra homomorphism \[R\ps{x} / [p^r](x) \to R^{\Lambda_r^*}\] is an isomorphism (resp., a monomorphism). \qed
\end{enumerate}
\end{lemma}

\begin{lemma}\citeme{Shortly after Prop 6.2 of HKR. Section 7?}
Let $\L_r(R)$ be the set of all group homomorphism \[\phi: \Lambda_r^* \to F[p^r](R)\] satisfying either of the conditions 1 or 2 above.  This functor is representable by a ring \[L_r(E^*) := S^{-1} E^*(B\Lambda_r)\] that is finite and faithfully flat over $p^{-1} E^*$.  (Here $S$ is generated by the $\phi(\alpha)$ with $\alpha \ne 0$, $\phi: \Lambda_r^* \to F[p^r](E^* B\Lambda_r)$ the canonical map.)
\end{lemma}

---

\begin{definition}
A \textit{finite subgroup} of $\G$ will mean a divisor $K$ on $\G$ which is also a subgroup scheme.  Let $\sheaf{O}_{\G/K}$ be the equalizer
\begin{center}
\begin{tikzcd}
\sheaf O_{\G/K} \arrow{r} & \sheaf O_{\G} \arrow[shift left=0.2cm]{r}{\mu^*} \arrow[shift right=0.2cm]{r}{\pi^*} & \sheaf O_K \otimes_{\sheaf O_X} \sheaf O_{\G}.
\end{tikzcd}
\end{center}
\end{definition}

\begin{lemma}\citeme{Theorem 5.3 of Finite Subgroups}
Write $y = N_\pi \mu^* x \in \sheaf O_{\G}$.\footnote{Remember that if $f: X \to Y$ is a finite flat map, then $N_f: \sheaf O_X \to \sheaf O_Y$ is the nonadditive map sending $u$ to the determinant of multiplication by $u$, considered as an $\sheaf O_Y$--linear endomorphism of $\sheaf O_X$.}  Then $y \equiv x^{p^m} \pmod{\m_X}$ and $\sheaf O_{\G/K} = \sheaf O_X\ps{y}$.  Moreover, the projection $\G \to \G/K$ is the categorical cokernel of $K \to \G$.  This all commutes with base change: given $f: Y \to X$ we have $f^* \G / f^* K = f^*(\G/K)$. \qed \todo{Expand this out in the case of a subgroup scheme given by a sum of point divisors.}
\end{lemma}

Section 7: level--$A$ structures: smooth, finite, flat

\todo{Be careful to distinguish the physical group $A$ from the associated \emph{constant group scheme}.}
As discussed long ago, for finite abelian $p$--groups there's a scheme \[\InternalHom{FormalGroups}(A, \G)(Y) = \InternalHom{Groups}(A, \G(Y)).\]  If $\G$ were a discrete group, we could decompose this as \[\text{``$\InternalHom{FormalGroups}(A, \G) = \coprod_{B \le A} \Mono(A/B, \G)$''}\] along the different kernel types of homomorphisms, but $\Mono$ does not exist as a scheme.\todo{Come up with a really compelling example.  You had one when you were talking to Danny and Jeremy.  Probably you got it \emph{from} Jeremy.}  Level structures approximate this as best one can be approximating $\G$ by something essentially discrete: an \'etale group scheme.

For a map $\phi: A \to \G(Y)$, we write $[\phi A] = \sum_{a \in A}[\phi(a)]$.  We also write $\Lambda = (\Q_p / \Z_p)^n$, so that $\Lambda[p^m] = (\Z/p^m)^{\times n}$.  Note \[|\CatOf{AbelianGroups}(A, \Lambda)| = |A|^n = \operatorname{rank} \left( \InternalHom{FormalGroups}(A, \G) \to X \right).\]

\begin{definition}
A \textit{level--$A$ structure} on $\G$ over an $X$--scheme $Y$ is a map $\phi: A \to \G(Y)$ such that $[\phi A[p]] \le G[p]$ as divisors.  A \textit{level--$m$ structure} means a level--$\Lambda[p^m]$ structure.
\end{definition}

\begin{lemma}\citeme{Prop 7.2-4 of Finite Subgroups}
The functor from schemes over $X$ to sets given by \[Y \mapsto \{\text{level--$A$ structures on $\G$ over Y}\}\] is represented by a finite flat scheme $\Level(A, \G)$ over $X$.  It is contravariantly functorial for monomorphisms of abelian groups.  Also, if $\phi: A \to \G$ is a level structure then $[\phi A]$ is a subgroup divisor and $[\phi A[p^k]] \le \G[p^k]$ for all $k$.  In fact, if $A = \Lambda[p^m]$ then $[\phi A] = \G[p^m]$.  \qed  \todo{I can't imagine proving this.  It's worth noting that it's proven by considering just the universal case, which we know to be smooth.}
\end{lemma}

In Section 26 of FPFP Neil says there's a decomposition into irreducible components \[\operatorname{Hom}(A, \G) = \operatorname{Hom}(A, \G_{\mathrm{red}}) = \bigcup_B \Level(A/B, \G)\] and this $\bigcup$ turns into a $\coprod$ after inverting $p$.  He also mentions this as motivation in Finite Subgroups, but he doesn't appear to prove it?

Section 8: maps among level--$A$ schemes, their Galois behavior

\begin{theorem}\citeme{Theorem 8.1 of Finite Subgroups}
Let $A$, $B$ be finite abelian $p$--groups of rank at most $n$, and let $u: A \to B$ be a monomorphism. Then:
\begin{enumerate}
\item \[\CatOf{FormalSchemes}_X(\Level(B, \G), \Level(A, \G)) = \operatorname{Mono}(A, B).\]
\item Such homomorphisms are detected by the behavior at the generic point.
\item The map $u^!: \Level(B, \G) \to \Level(A, \G)$ is finite and flat.
\item If $B \simeq \Lambda[p^m]$, then $u^!$ is a Galois covering.
\item The torsion subgroup of $\G(\Level(A, \G))$ is $A$. \qed
\end{enumerate}
\end{theorem}

Section 10: moduli of subgroup schemes

\begin{theorem}\citeme{Theorem 10.1 of Finite Subgroups}
The functor \[Y \mapsto \{\text{subgroups of $\G \times_X Y$ of degree $p^m$}\}\] is represented by a finite flat scheme $\Sub_{p^m}(\G)$ over $X$ of degree $|\Sub_{p^m}(\Lambda)|$.  The formation commutes with base change. \qed
\end{theorem}

We can at least give the construction: let $D$ be the universal divisor defined over $Y = \Div_{p^m}(\G)$ with equation $f_D(x) = \sum_{k=0}^{p^m} c_k x^k$.  There are unique elements $a_{ij} \in \sheaf O_Y$ such that \[f(x +_F y) = \sum_{i,j=0}^{p^m-1} a_{ij} x^i y^j \pmod{f(x), f(y)}.\]  Define \[\Sub_{p^m}(\G) = \Spf \sheaf O_Y / (c_0, a_{ij} \mid 0 \le i, j < p^m).\]  Finiteness, flatness, and rank counting are what take real work, starting with an arithmetic fracture square.

Section 13: deformation theory of isogenies

\begin{definition}
Suppose we have a morphism of formal groups
\begin{center}
\begin{tikzcd}
\G_0 \arrow{r}{q_0} \arrow{d} & \G'_0 \arrow{d} \\
X_0 \arrow{r}{f_0} & X'_0
\end{tikzcd}
\end{center}
such that the induced map $\G_0 \to f_0^* \G'_0$ is an isogeny of degree $p^m$.  By a deformation of $q_0$ we mean a prism
\begin{center}
\begin{tikzcd}
\mathbb H \arrow{rd}{q} \arrow{dd} & & \mathbb H_0 \arrow{ll} \arrow{rr} \arrow{rd} \arrow{dd} & & \G_0 \arrow{rd}{q_0} \arrow{dd} \\
& \mathbb H' & & \mathbb H'_0 \arrow[crossing over]{ll} \arrow[crossing over]{rr} & & \G'_0 \arrow{dd} \\
Y \arrow{rd}{1} & & Y_0 \arrow{ll} \arrow{rr} \arrow{rd}{1} & & X_0 \arrow{rd}{f_0} \\
& Y \arrow[crossing over,leftarrow]{uu} & & Y_0 \arrow{ll} \arrow{rr} \arrow[crossing over,leftarrow]{uu} & & X'_0,
\end{tikzcd}
\end{center}
where the middle face is the pullback of the left face, the back-right and front-right faces are pullbacks, so that $q$ is also an isogeny of degree $p^m$.
\end{definition}

Let $\G/X$ be the universal deformation of $\G_0$, let $a: \Sub_{p^m}(\G) \to X$ be the usual projection, and let $K < a^* \G$ be the universal example of a subgroup of degree $p^m$.  As $\Sub_{p^m}(\G)$ is a closed subscheme of $\Div_{p^m}(\G)$ and $\Div_{p^m}(\G)_0 = X_0$, we see that $\Sub_{p^m}(\G)_0 = X_0$.  There is a unique subgroup of order $p^m$ of $\G_0$ defined over $X_0$, viz.\ the divisor $p^m[0] = \Spf \sheaf O_{\G_0} / x^{p^m}$.  In particular, $K_0 = p^m[0] = \ker(q_0)$.  It follows that there is a pullback diagram as shown below:
\begin{center}
\begin{tikzcd}
(a^* \G/K)_0 \arrow{r}{\simeq} \arrow{d} & \G_0 / p^m[0] \arrow{r}{\overline q_0, \simeq} \arrow{d} & \G'_0 \arrow{d} \\
\Sub_{p^m}(\G)_0 \arrow{r}{a_0, \simeq} & X_0 \arrow{r}{f_0, \simeq} & X'_0.
\end{tikzcd}
\end{center}
We see that $a^* \G \to a^* \G/K$ is a deformation of $q_0$, and it is terminal in the category of such.

Now let $\G' / X'$ be the universal deformation of $\G'_0 / X'_0$.  The above construction also exhibits $a^* \G/K$ as a deformation of $\G'_0$, so it is classified by a map $b: \Sub_{p^m}(\G) \to X'$ extending the map $b_0 = f_0 \circ a_0: \Sub_{p^m}(\G)_0 \to X'_0$.

\begin{theorem}\citeme{Prop 13.1 of Finite Subgroups, \emph{hard}}
$b$ is finite and flat of degree $|\Sub_{p^m}(\Lambda)|$. \qed
\end{theorem}

\todo[inline]{Cf. Matt's thesis's Prop 2.5.1: $\Phi$ is a formal group over $\F_p$, $F$ a lift of $\Phi$ to $E_n$, $H$ a finite subgroup of $F(D_k)$, then $F/H$ is a lift of $\Phi$ to $D_k$.  (This is because the quotient map to $F/H$ reduces to $t \mapsto t^{p^r}$ for some $r$ over $\F_p$, which is an endomorphism of $\Phi$, so the quotient map over the residue field doesn't do anything!)  See also Prop 2.5.4, where he characterizes all isogenies of this sort as arising from this construction.}

Section 15: formulas for computation
Section 16: examples

------

Describe the action by $GL_n(\Z_p)$. (Hint at the action by $M_{n \times n}(\Z_p)$ with $\det \ne 0$.)

------

There are union maps \[B\Sigma_j \times B\Sigma_k \to B\Sigma_{j+k},\] stable transfer maps \[B\Sigma_{j+k} \to B\Sigma_j \times B\Sigma_k,\] and diagonal maps \[B\Sigma_j \to B\Sigma_j \times B\Sigma_j.\]  These induce a coproduct $\psi$ as well as products $\times$ and $\bullet$ on $E^0 \P \S^0$, where $\P\S^0 = \coprod_{j=0}^\infty B\Sigma_j$ is the free $E_\infty$--ring on $\S^0$.  This is a Hopf ring, and under $\times$ alone it is a formal power series ring.  The $\times$--indecomposables (which, I guess, are analogues of considering additive unstable cooperations) are \[Q^\times E^0 \P\S^0 = \prod_{k \ge 0} \left( E^0 B\Sigma_{p^k} / \operatorname{tr} E^0 B\Sigma_{p^{k-1}}^p \right),\] where the $k${\th} factor in the product is naturally isomorphic to $\sheaf{O}_{\Sub_{p^k}(\G)}$.  The primitives are also accessible as the kernel of the dual restriction map.

Theorem 3.2 shows that $E^0 B\Sigma_k$ is free over $E^0$, Noetherian, and of rank controlled by generalized binomial coefficients.  Prop 3.4 is the only place where work gets done, and it's all in terms of $K$--theory and HKR characters.

There's actually an extra coproduct, coming from applying $D$ to the fold map $S^0 \vee S^0 \to S^0$.

The main content of Prop 5.1 (due to Kashiwabara) is that $K_0 \P \S^0$ injects into $K_0 \OS{BP}{0}$.  Grading $K_0 \P \S^0$ using the $k$--index in $B\Sigma_k$, you can see that it's of graded finite type, so we need only know it has no nilpotent elements to see that $K_0 \P \S^0$ is $\ast$--polynomial.  This follows from our computation that $K_0 \OS{BP}{0}$ is a tensor of power series and Laurent series rings.  Corollary 5.2 is about $K_0 Q S^0$, which is the group completion of $K_0 \P \S^0$, so it's the tensor of $K_0 \P \S^0$ with a graded field.

Prop 5.6, using a double bar spectral sequence method, shows that $K^0 Q S^2$ is a formal power series algebra.  Tracking the spectral sequences through, you'll find that $Q^\times K^0 Q S^0$ agrees with $P K^0 Q S^2$.  (You'll also notice that $K^0 Q S^2$ only has one product on it, cf.\ Remark 5.4.)

Snaith's theorem says $\Sigma^\infty QX = \Sigma^\infty \P X$ for connected spaces $X$.  You can also see (just after Theorem 6.2) the nice equivalences \[\P_k S^2 \simeq B\Sigma_k^{V_k} \simeq \P_k(S^0)^{V_k},\] where superscript denotes Thom complex.  So, for a complex-orientable cohomology theory, you can learn about $\P_k S^0$ from $\P_k S^2$.  In particular, we finally learn that $E^0 \P S^0$ is a formal power series $\times$--algebra (once checking that the Thom isomorphism is a ring map).  (We already knew the homological version of this claim.)

Section 8 has a nice discussion about indecomposables and primitives, to help move back and forth between homology and cohomology.  It probably helps most with the dimension count argument below that we aren't going to get into.

Start again with $D_{p^k} S^2 \simeq B\Sigma_{p^k}^{V_{p^k}}$.  We can associate to this a divisor $\ThomDivisor(V_{p^k})$ on $(B\Sigma_{p^k})_E$, which we know little about, but it is classified by a map to $\Div_{p^k} \CP^\infty_E$.  This receives a closed inclusion from $\Sub_{p^k} \CP^\infty_E$, so their pullback $Z_k$ is the largest subscheme of $(B\Sigma_{p^k})_E$ over which $\ThomDivisor(V_{p^k})$ is a subgroup divisor.
\begin{center}
\begin{tikzcd}
H_k \arrow{rr} \arrow{dd} & & \ThomDivisor(V_{p^k}) \\
& Z_k \arrow{rr} \arrow{rd} & & \Sub_{p^k} \CP^\infty_E \arrow{rd} \\
\Spf E^0 B\Sigma_{p^k} / \mathrm{tr} \arrow{rr} \arrow[densely dotted]{ru} & & (B\Sigma_{p^k})_E \arrow{rr} \arrow[crossing over,leftarrow]{uu} & & \Div_{p^k} \CP^\infty_E
\end{tikzcd}
\end{center}
We will show the existence of the dashed map, implying that the restricted divisor $H_k$ is a subgroup divisor on $Y_k = \Spf E^0 B\Sigma_{p^k} / \mathrm{tr}$.

(Prop 9.1:) This proof falls into two parts: first we construct a family of maps to $(B\Sigma_{p^k})_E$ on whose image $\ThomDivisor(V_{p^k})$ restricts to a subgroup divisor, and then we show that the union of their images is exactly $Y_k$.  Let $A$ be an abelian $p$--subgroup of $\Sigma_{p^k}$ that acts transitively on $\{1, \ldots, p^k\}$ (i.e., it is not boosted from some transfer).  The restriction of $V_{p^k}$ to $A$ is the regular representation, which splits as a sum of characters $V_{p^k}|_A = \bigoplus_{\L \in A^*} \L$.  Identifying $BA_E = \InternalHom{FormalGroups}(A^*, \CP^\infty_E)$, $\ThomDivisor(V_{p^k})$ restricts all the way to $\sum_{\L \in A^*} [\phi(\L)]$, with $\phi: A^* \to $``$\Gamma(\operatorname{Hom}(A^*, \G), \G)$''.  In Finite Subgroups of Formal Groups (see Props 22 and 32), we learned that the restriction of $\ThomDivisor(V_{p^k})$ further to $\Level(A^*, \CP^\infty_E)$ is a subgroup divisor.  So, our collection of maps are those of the form \[\Level(A^*, \CP^\infty_E) \to \InternalHom{FormalGroups}(A^*, \CP^\infty_E) = BA_E \to (B\Sigma_{p^k})_E.\]  Here, finally, is where we have to do some real work involving Chern classes and commutative algebra, so I'm inclined to skip it in the lectures.  Finally, you do a dimension count to see that $Z_k$ and $\Spf E^0 B\Sigma_{p^k} / \mathrm{tr}$ have the same dimension (which requires checking enough commutative algebra to see that ``dimension'' even makes sense), and so you show the map is injective and you're done.


-----

Here's Neil's proof of the joint images claim.  It seems like a clear enough use of character theory that we should include it, if we can make character theory itself clear.

Recall from [18, Theorem 23] that $\Level(A^*,\G)$ is a smooth scheme, and thus that $D(A) = \sheaf O_{\Level(A^*,\G)}$ is an integral domain. Using [18, Proposition 26], we see that when $\L \in A^*$ is nontrivial, we have $\phi(\L) \ne 0$ as sections of $\G$ over $\Level(A^*, \G)$, and thus $e(\L) = x(\phi(\L)) \ne 0$ in $D(A)$. It follows that that $c_{p^k} = \prod_{\L \ne 1} e(\L)$ is not a zero-divisor in $D(A)$. On the other hand, if $A'$ is an Abelian $p$-subgroup of $\Sigma_{p^k}$ which does not act transitively on $\{1, \ldots, p^k\}$, then the restriction of $V_{p^k} − 1$ to $A'$ has a trivial summand, and thus $c_{p^k}$ maps to zero in $D(A')$. Next, we recall the version of generalised character theory described in [8, Appendix A].
\[p^{-1} E^0 BG = \left(\prod_A p^{-1} D(A)\right)^G\]
where $A$ runs over all Abelian $p$-subgroups of $G$. As $\overline R_k = E^0(B\Sigma_{p^k} )/ ann(c_{p^k} )$ and everything in sight is torsion-free, we see that $p^{−1} \overline R_k$ is the quotient of $p^{−1}E^0B\Sigma_{p^k}$ by the annihilator of the image of $c_{p^k}$ . Using our analysis of the images of $c_{p^k}$ in the rings $D(A)$, we conclude that
\[p^{-1} \overline R_k = \left(\prod_A p^{−1}D(A)\right)^{\Sigma_{p^k}},\]
where the product is now over all transitive Abelian $p$-subgroups. This implies that for such $A$, the map $E^0B\Sigma_{p^k} \to D(A)$ factors through $\overline R_k$, and that the resulting maps $\overline R_k \to D(A)$ are jointly injective. This means that $Y_k = \Spf \overline R_k$ is the union of the images of the corresponding schemes $\Level(A^*,\G)$, as required.




------Stuff cribbed directly from AHS------



\begin{enumerate}
\item If $V = (1 - \L)$ is the reduced canonical line bundle over $\CP^\infty$, then using all the above we have \[\ThomSheaf{V} \cong \pi^* 0^* \sheaf I (0) \otimes \sheaf I(0)^{-1} = \Theta^1(\sheaf I(0)),\] where $\pi: \G_E \to S_E$ is the structural map and $\Theta^1$ is the usual.
\item Let $A$ be a finite abelian group.  An element $a \in A$ can be regarded as a character of $A^*$, and we let $V_a$ denote the associated line bundle over $BA^*$.  This gives a group homomorphism $\chi\co A \to \G(BA^*_E)$.  The line bundle $\ThomSheaf{V_a \otimes V \otimes \L}$ over $BA^*_E \times X_E \times \G$ is \[\ThomSheaf{V_a \otimes V \otimes \L} \cong T_a^* \sheaf I(D^{-1}),\] and taking $V$ to be the trivial line bundle over a point gives \[\ThomSheaf{V_a \otimes \L} \cong T_a^* \sheaf I(0) = \sheaf I(a^{-1}).\]
\item Now let $V_{reg} = \bigoplus_{a \in A} V_a$ be the regular representation of $A^*$.  Over the scheme $(BA^*)_E \times \G$, the line bundle associated to the Thom complex of $V_{reg} \otimes V \otimes \L$ is \[\ThomSheaf{V_{reg} \otimes V \otimes \L} \cong \bigotimes_{a \in A} T_a^* \sheaf I(D^{-1}) \cong \sheaf I \left( \sum_{a \in A} T_a^* D^{-1} \right).\]  In particular, \[\ThomSheaf{V_{reg} \otimes \L} \cong \bigotimes_{a \in A} T_a \sheaf I(0) \cong \sheaf I(\chi).\]
\item Suppose that the map \[\widetilde \chi\co (BA^*)_E \to \InternalHom{FormalGroups}(A, \G)\] is an isomorphism.  Given a level structure and cokernel pair \[A_T \xrightarrow{\ell} i^* \G \xrightarrow{q} \G',\] changing base along $T \times \G \xrightarrow{\chi_\ell} \InternalHom{FormalGroups}(A, \G) \times \G$ gives \[\chi_\ell^* \ThomSheaf{V_{reg} \otimes \L} \cong q^* N_q \sheaf I_{\G}(0) \cong q^* \sheaf I_{\G'}(0) \cong \sheaf I_{\G}(\ell).\]
\item Restricting the above example to $BA^*$, we find \[\chi_\ell^* \ThomSheaf{V_{reg}} = 0_{\G}^* q^* \sheaf I_{\G'}(0) = 0_{\G'}^* \sheaf I_{\G'}(0) = \omega_{\G'}.\]
\end{enumerate}



\begin{lemma}\citeme{Lemma 3.19 of AHS $H_\infty$}
The map $\psi_\ell^V$ has the following properties:
\begin{enumerate}
\item If $m$ trivializes $\ThomSheaf{V}$ then $\psi_\ell^V(m)$ trivializes $\chi_\ell^* \ThomSheaf{V_{reg} \otimes V}$.
\item $\psi_\ell^{V_1 \oplus V_2} = \psi_\ell^{V_1} \otimes \psi_\ell^{V_2}$.
\item For $f\co Y \to X$ a map, $\psi_\ell^{f^* V} = f^* \psi_\ell^V$. \qed
\end{enumerate}
\end{lemma}

In particular, we can apply this to $X = \CP^\infty$ and $\ThomSheaf{\L - 1} = \sheaf I(0)$.  Then 8.11 gives \[\psi_\ell^{\L - 1} \co (\psi_\ell^F)^* \sheaf I_{\G}(0) \to \chi_\ell^* \ThomSheaf{V_{reg} \otimes (\L - 1)} = \sheaf I_{i^* \G}(\ell).\]


\begin{lemma}\citeme{Eqn 5.3, generalizes Quillen's splitting formula}
For $V$ a vector bundle on a space $X$ and $V_{reg}$ the (vector bundle over $BA^*$ induced from) the regular representation on $A$, there is an isomorphism of sheaves over $(BA^* \times X)_E$ \[\ThomSheaf{V_{reg} \otimes V} \cong \bigotimes_{a \in A} \widetilde T_a \ThomSheaf{V}.\]
\end{lemma}

\begin{lemma}\citeme{Prop 7.5}
Take $\pi_0 E$ to be a complete local ring and $\G_E$ to be of finite height.  If $B^* \subset A^*$ is a proper subgroup, then the following composite map of $\pi_0 E$--modules is zero: \[\pi_0 E^{BB^*_+} \xrightarrow{transfer} \pi_0 E^{BA^*_+} \xrightarrow{\chi_\ell} \sheaf O(T).\]
\end{lemma}
\begin{proof}
It suffices to consider the tautological level structure over $\Level(A, \G)$.  We may take $A$ to be a $p$--group, and indeed for now we set $A = \Z/p$, $B = 0$.  For $t \in \pi_0 E^{\CP^\infty_+}$ a coordinate with formal group law $F$, we have \[\pi_0 E^{BA^*_+} \cong \pi_0 E \ps{t} / [p]_F(t)\] and $\tau: \pi_0 E^{BB^*_+} = \pi_0 E \to \pi_0 E^{BA^*_+}$ is given by $\tau(1) = \<p\>_F(t)$, where $\<p\>_F(t) = [p]_F(t) / t$ is the ``reduced $p$--series''.  The result then follows from the isomorphism $\sheaf O(\Level(\Z/p, \G_E)) \cong \pi_0 E\ps{t} / \<p\>_F(t)$.  The result then follows in general by induction: $B^*$ can be taken to be a \emph{maximal} proper subgroup of $A^*$, with cokernel $\Z/p$.
\end{proof}

\begin{lemma}\citeme{Prop 9.24} \todo{One of the reduction steps in Prop 6.1 is handled by 9.24, which is in turn equivalent to a basic case of an HKR theorem, so should be stated on that day (or in the algebraic day).}
The natural map \[\sheaf O(\InternalHom{FormalGroups}(\Z/p, \G)) \to R \times \sheaf O(\Level(\Z/p, \G))\] is injective.
\end{lemma}
\begin{proof}
\todo{Fill this.}
\end{proof}

------ Descent along level structures, simplicially (Section 11) ------

\todo[inline]{Actually, this section appears \emph{not} to be about $\FGps$, and instead it's about the \emph{coarse moduli quotient} to the functor of formal groups, which is not locally representable.  I'm a little confused about this---I intend to ask Mike what's going on.}

\begin{lemma}\citeme{AHS Lemma 11.3}
For $\ell\co A \to \G$ a level structure and $B \subseteq A$ a subgroup, the induced map $\ell|_B\co B \to \G$ is a level structure and the quotient $\G / \ell|_B$ receives a level structure $\ell'\co A/B \to \G/\ell|_B$. \qed
\end{lemma}

This gives us enough compatibility among quotients to use the two maps above to assemble the $\Level_*$ schemes into a simplicial object.  Most face maps just omit a subgroup, except for the last face map, since the zero subgroup is not permitted to be omitted.  Instead, the last face map sends the string of subgroups $0 = A_n \subseteq A_{n-1} \subseteq \cdots \subseteq A_0$ and level structure $\ell\co A_0 \to \G$ to the quotient string $0 = A_{n-1} / A_{n-1} \subseteq \cdots \subseteq A_0 / A_{n-1}$ and quotient level structure $\ell\co A_0 / A_{n-1} \to \G/\ell|_{A_{n-1}}$.  The degeneracy maps come from lengthening one of these strings by an identity inclusion.

\begin{definition}\citeme{Definition 11.10, Remark 11.11}
Let $\G\co F \to \FGps$ be a functor over formal groups, and define schemes $\Level(A, F) = \Level(A) \times_{\G} F$ and $\Level_n(F) = \Level_n \times_{\G} F$.  Then, \textit{descent data for level structures on $F$} is the structure of a simplicial scheme on $\Level_*(F)$, together with a morphism of simplicial schemes $\Level_*(F) \to \Level_*$.  It is enough to specify a map $d_1\co \Level_1(F) \to F$, use that to build the simplicial scheme structure as in the above Lemma, and assert that the following square commutes:
\begin{center}
\begin{tikzcd}
\Level_1(F) \arrow{r} \arrow{d}{d_1} & \Level_1 \arrow{d}{d_1} \\
F \arrow{r} & \FGps.
\end{tikzcd}
\end{center}
\end{definition}

\begin{example}
Let $\G\co S \to \FGps$ be a formal group of finite height over a $p$--local formal scheme $S$.  The functor $\Level(A, \G)$ is exactly the functor defined in Section 9 (see above), and in particular it is represented by an $S$--scheme.  The maps $\psi_\ell$ and $f_\ell$ from Definition 3.1 amount to giving a map $d_1\co \Level_1(\G) \to S$ and an isogeny $q\co d_0^* \G \to d_1^* \G$ whose kernel on $\Level(A, \G)$ is $A$.  The other conditions on Definition 3.1 exactly ensure that $(\Level_*(\G), d_*, s_*)$ is a simplicial functor and over $\Level_2(\G)$ the relevant hexagonal diagram commutes:
\begin{center}
\begin{tikzcd}
& d_0^* d_0^* \G \arrow[equal]{ld} \arrow{rd}{d_0^* q} \\
d_1^* d_0^* \G \arrow{d}{d_1^* q} & & d_0^* d_1^* \G \arrow[equal]{d} \\
d_1^* d_1^* \G \arrow[equal]{rd} & & d_2^* d_0^* \G \arrow{ld}{d_2^* q} \\
& d_2^* d_1^* \G.
\end{tikzcd}
\end{center}
\end{example}

\begin{example}
We now further package this into a single object.  Let $\underline{\G}$ be the functor over $\FGps$ whose value on $R$ is the set of pullback diagrams
\begin{center}
\begin{tikzcd}
\G' \arrow{r}{f} \arrow{d} & \G \arrow{d} \\
\Spf R \arrow{r}{i} & S
\end{tikzcd}
\end{center}
such that the map $\G' \to i^* \G$ induced by $f$ is a homomorphism (hence isomorphism) of formal groups over $\Spf R$.  For a finite abelian group $A$, write $\Level(A, \underline{\G})(R)$ for the set of diagrams
\begin{center}
\begin{tikzcd}
A_{\Spf R} \arrow{r}{\ell} \arrow{rd} & \G' \arrow{r}{f} \arrow{d} & \G \arrow{d} \\
& \Spf R \arrow{r}{i} & S
\end{tikzcd}
\end{center}
where the square forms a point in $\underline{\G}(R)$ and $\ell$ is a level--$A$ structure.  Giving a map of functors $d_1\co \Level_1(\underline{\G}) \to \underline{\G}$ making the above square commute is to give a pullback diagram
\begin{center}
\begin{tikzcd}
\G / \ell \arrow{r} \arrow{d} & \G \arrow{d} \\
\Level_1(\G) \arrow{r} & S,
\end{tikzcd}
\end{center}
or equivalently a map of formal schemes $\Level_1(\G) \to S$ and an isogeny $q\co d_0^* \G d_1^* \G$ whose kernel on $\Level(A, \G)$ is $A$.  Therefore, descent data for level structures on the formal group $\G$ (in the sense of Section 3) are equivalent to descent data for level structures on the functor $\underline{\G}$.
\end{example}

------ Section 12: Descent for level structures on Lubin--Tate groups ------

Let $k$ be perfect of positive characteristic $p$, and let $\Gamma$ be a formal group of finite height over $k$.  Recall that this induces a relative Frobenius
\begin{center}
\begin{tikzcd}
\Gamma \arrow{r}{F} \arrow[bend left]{rr}{\phi_\Gamma} \arrow{rd} & \phi_k^* \Gamma \arrow{r} \arrow{d} & \Gamma \arrow{d} \\
& \Spec k \arrow{r}{\phi_k} & \Spec k.
\end{tikzcd}
\end{center}
The map $F$ is an isogeny of degree $p$, with kernel the divisor $p \cdot [0]$.  Recall also that a deformation $H$ of $\Gamma$ to $T$ induces a map $\underline{H} \to \Def(\Gamma)$, and there is a universal such $\G$ over the ground scheme $S \cong \Spf \W(k)\ps{u_1, \ldots, u_{d-1}}$ such that $\underline{\G} \to \Def(\Gamma)$ is an isomorphism of functors over $\FGps$.

Now consider a point in $\Level(A, \Def \Gamma)$:
\begin{center}
\begin{tikzcd}
A_T \arrow{r}{\ell} \arrow{rd} & H \arrow{d} & H_0 \arrow{l} \arrow{r}{f} \arrow{d} & \Gamma \arrow{d} \\
& T & T_0 \arrow{l} \arrow{r}{j} & \Spec k.
\end{tikzcd}
\end{center}
The level structure $\ell$ gives rise to a quotient isogeny $q\co H \to H'$.  Since $A$ is sent to $0$ in $\sheaf O_{T_0}$, there is a canonical map $\bar q$ fitting into the diagram
\begin{center}
\begin{tikzcd}
H \arrow{rr}{q} \arrow{rdd} & & H' \arrow{ldd} \\
& & & H_0 \arrow{rdd} \arrow[crossing over]{lllu} \arrow{r} & H_0' \arrow[crossing over]{llu} \arrow{dd} \arrow[densely dotted]{r}{\bar q} & (\phi^r)^* H_0 \arrow{ldd} \arrow{r} \arrow[leftarrow, bend right, crossing over]{ll} & H_0 \arrow{dd} \arrow{r}{f} & \Gamma \arrow{dd} \\
& T \\
& & & & T_0 \arrow{lllu} \arrow{rr}{\phi^r} & & T_0 \arrow{r}{j} & \Spec k.
\end{tikzcd}
\end{center}
The map $\bar q$ combines with the rest of the maps to exhibit $H'$ as a deformation of $\Gamma$, and hence we get a natural transformation \[d_1\co \Level_1(\Def(\Gamma)) \to \Def(\Gamma).\]  Since $\phi^r \phi^s = \phi^{r+s}$, this gives descent data for level structures on $\Def(\Gamma)$.  Identifying this functor with $\underline{\G}$ using Lubin--Tate theory, we equivalently have shown the existence of descent data for level structures on $\underline{\G}$.

Incidentally, the descent data constructed here is also the descent data that would come from the structure of an $E_\infty$--orientation on the Morava $E$--theory $E_d$, essentially because the divisor associated to the kernel of the relative Frobenius on the special fiber is forced to be $p[0]$, and everything is dictated by how the deformation theory \emph{has} to go (and the fact that the topological operations we're studying induce deformation-theoretic-describable operations on algebra).
















\section{Orientations by \texorpdfstring{$E_\infty$}{Eoo} maps}\label{JuvitopTalkSection}

A more modern take on the story of the $\sigma$--orientation passes directly through the algebra of $E_\infty$--ring spectra.  Though technically intensive, our reward for grappling with this will be the modularity of the $\String$--orientation, enriching \Cref{WittensTheoremForBU6} to the real setting.  Luckily, most of the basic ideas are classically familiar, centering on a particular functor \[\gl_1\co E_\infty\CatOf{RingSpectra} \to \CatOf{Spectra}.\]  This functor derives its name from two compatible sources: for one, its underlying infinite loopspace is the construction $GL_1$ described in \Cref{LectureThomSpectra}; and secondly, it participates in an adjunction
\begin{center}
\begin{tikzcd}[column sep=4em]
\CatOf{ConnectiveSpectra} \arrow[shift left=0.3\baselineskip, "\Susp^\infty_+ \Omega^\infty"]{r} & E_\infty\CatOf{RingSpectra} \arrow[shift left=0.3\baselineskip, "\gl_1"]{l}
\end{tikzcd}
\end{center}
analogous to the adjunction between the group of units and the group-ring constructions in classical algebra.  Its relevance to us is its participation in the theory of highly structured Thom spectra.  Let $j\co g \to \gl_1 \S$ be a map of connective spectra, begetting a map $J\co G \to \GL_1 \S$ of infinite loopspaces, where we have written $G = \Loops^\infty g$.
\begin{lemma}
\citeme{May? or ABGHR?}
The Thom spectrum of the map $BJ$ is presented by the pushout of $E_\infty$ rings\footnote{This is a kind of ``twisted group-ring'' construction.}
\begin{center}
\begin{tikzcd}
\Susp^\infty_+ \GL_1 \S \arrow["\Loops^\infty \Susp j"]{r} \arrow{d} & \Susp^\infty_+ \Loops^\infty \gl_1 \S / g \arrow{d} \\
\S \arrow{r} & MG. \qed
\end{tikzcd}
\end{center}
\end{lemma}

\begin{corollary}
\citeme{May, but also ABGHR}
There is a natural equivalence between the space of null-homotopies of the composite \[g \xrightarrow j \gl_1 \S \xrightarrow{\gl_1 \eta_R} \gl_1 R\] and the space of $E_\infty$ ring maps $MG \to R$, where $MG$ is the Thom spectrum of the stable spherical bundle classified by $J$.
\end{corollary}
\begin{proof}
Applying the mapping space functor $E_\infty(-, R)$ to the pushout diagram in the Lemma, we have a pullback diagram of mapping spaces:
\begin{center}
\begin{tikzcd}
E_\infty(\Susp^\infty_+ \GL_1 \S, R) & E_\infty(\Susp^\infty_+ \Loops^\infty \gl_1 \S / g, R) \arrow{l} \\
E_\infty(\S, R) \arrow{u} & E_\infty(MG, R) \arrow{l} \arrow{u}.
\end{tikzcd}
\end{center}
We can reidentify each of the three terms to get
\begin{center}
\begin{tikzcd}
\CatOf{Spectra}(\gl_1 \S, \gl_1 R) & \CatOf{Spectra}(\gl_1 \S / g, \gl_1 R) \arrow{l} \\
\{\gl_1 \eta_R\} \arrow{u} & E_\infty(MG, R) \arrow{l} \arrow{u},
\end{tikzcd}
\end{center}
hence $E_\infty(MG, R)$ appears as the fiber at $\gl_1 \eta_R$ of the restriction map, which coincides with the space of nullhomotopies as claimed.
\end{proof}

\begin{corollary}
\citeme{AHR}
The mapping set $E_\infty(Mj, R)$ is nonempty if and only if $\gl_1 \eta_R \circ j$ is null-homotopic.  If this is the case, then $E_\infty(Mj, R)$ is a torsor for $[\Susp g, \gl_1 R]$.
\end{corollary}

Ando, Hopkins, and Rezk have used this presentation to understand the mapping space $E_\infty(M\String, \tmf)$.  In this Appendix, we will use this same technology to understand the mapping space $E_\infty(M\Spin, KO_{(p)})$, which proceeds along entirely similar lines but is a \emph{considerably} simpler computation.\footnote{Ando, Hopkins, and Rezk also do $E_\infty(M\Spin, KO)$ as a warm-up computation~\cite[Section 7]{AHR}, and we are further $p$--localizing that result so as not to have to think about arithmetic fracture.  Working arithmetically globally should be an easy exercise for the reader.}  The approach to this computation is to mix the presentation above with chromatic fracture applied to the target:\todo{Put a pullback corner here.}
\begin{center}
\begin{tikzcd}
M\Spin \arrow{r} \arrow[bend left=15]{rr} \arrow{rrd} \arrow{rd} & KO_{(p)} \arrow{r} \arrow[crossing over]{d} & KO_p \arrow{d} \\
& \Q \otimes KO \arrow{r} & \Q \otimes KO_p.
\end{tikzcd}
\end{center}
So, we seek a pair of $E_\infty$ ring maps into the rationalization and the $p$--completion of $KO$ which agree on the $p$--local ad\`eles, which involves understanding not just the mapping spaces but also the pushforward maps between them.


\subsubsection{Rational orientations}

We begin with the two rational nodes in the pullback diagram.  As a first approximation to our goal, consider the problem of giving a complex orientation $MU \to \Q \otimes R$ of a rational ring spectrum $\Q \otimes R$.  There is an automatic such orientation granted by
\begin{center}
\begin{tikzcd}
MU \arrow[densely dotted, "D"]{r} \arrow{rd} & \Q \otimes R \\
\S \arrow{u} \arrow[crossing over]{ru} \arrow{r} & H\Q \arrow{u}
\end{tikzcd}
\end{center}
constructed out of the unit map $\S \to MU$, the unit map $\S \to \Q \otimes R$, the rationalization map $S \to \Q \otimes \S \cong H\Q$, and the standard additive orientation $MU \to H\Q$ of an Eilenberg--Mac Lane spectrum.  When $E_\infty(MU, T)$ is nonempty, it is a torsor for $[bu, \gl_1 T]$, and since we have a preferred orientation $D$ we thus have isomorphisms \[\pi_0 E_\infty(MU, \Q \otimes R) \xleftarrow{\cong} [bu, \gl_1 \Q \otimes R] \xleftarrow{\cong} [bu, \Q \otimes \gl_1 R] \xrightarrow{\cong} [\Q \otimes bu, \Q \otimes \gl_1 R],\] the last of which is specified by a sequence of rational numbers $(t_{2k})_{k \ge 1}$.  The role played by the sequence $(t_{2k})$ is to perturb the Thom class.

\begin{lemma}
Write $x$ for the Thom class of $\L$ on $\CP^\infty$ in $(\Q \otimes R)$--cohomology as furnished by the automatic orientation $D$.  The Thom class associated to some other orientation of $\Q \otimes R$ is tracked by a difference series $x / \exp_F(x)$, and the sequence $(t_k)$ above is expressed by $x / \exp_F(x) = \exp(\sum_k t_k/k! \cdot x^k)$.\todo{This is confused.}
\end{lemma}
\begin{proof}[Proof sketch]
Let $v^k\co S^{2k} \to BU$ be the $k${\th} power of the class $\L$, so that it comes from a restriction \[S^{2k} \to (\CP^\infty)^{\sm k} \xrightarrow{\L^{\boxtimes k}} BU.\]  The Thom class for this bundle comes from the top Chern class, which is the top coefficient in the product of total Chern classes applied to the individual bundles.  Following the usual formulas shows the map $v^k$ to behave on homotopy by multiplication by $(-1)^k t_k$.
\end{proof}

Now we move away from $MU$.  There are three directions for generalization: connective orientations, real orientations, and non-complex targets.
\begin{enumerate}
\item Rationally, the analysis of Ando--Hopkins--Strickland identifies $[BU\<2k\>, \Q \otimes R]$ with $k$--variate symmetric multiplicative $2$--cocycles over $R$, every one of which arises as $\delta^1$ repeatedly applied to a univariate series.  In homotopy theoretic terms, this means that every $MU\<2k\>$--orientation of a rational spectrum factors through an $MU$--orientation.
\item The cofiber sequence $kO \to kU \to \Susp^2 kO$ splits rationally, using the idempotents $\frac{1 \pm \chi}{2}$ on $kU$.  Accordingly, $MU$--orientations of rational spectra that factor through $MSO$--orientations have an invariance property under $\chi$: $-[-1](x) = x$, corresponding to the idempotent factor $+$.  This pattern continues for the characteristic series of connective orientations.
\item This same cofiber sequence and idempotent splitting also tells us that rational $KU$--cohomology classes in the image of $KO$--cohomology are $\chi$--invariant, i.e., they belong to the $-$ factor.
\end{enumerate}

Our main example is the usual orientation $MU \to KU$ that selects the formal group law $x + y - xy$.  This is associated to the difference Thom class $x / (e^x - 1) = x / \exp_{\G_m}(x)$.  To make this difference $[-1]$--invariant (and hence give a complex-orientation of $KO$), we use the averaged exponential class $(e^{x/2} - 1) - (e^{-x/2} - 1)$.\footnote{Incidentally, this is equal to $2\operatorname{sinh}(x/2)$.}  In turn, we use the Lemma to calculate the behavior on homotopy of the associated orientation:\footnote{This comes out of applying $d\log$ to the fraction.} \[\frac{x}{e^{x/2} - e^{-x/2}} = \exp\left(-\sum_{k=2}^\infty \frac{B_k}{k} \cdot \frac{x^k}{k!}\right).\]  Finally, we calculate the effect of the orientation on the second half of the factorization \[MSU \to M\Spin \to KO,\] again using the relevant idempotent, which has the effect of halving the coefficients in the characteristic series: $-\frac{B_k}{2k}$.\footnote{While we're here, you might want to observe that elements in $[bu, \gl_1 R]$ push forward to elements in $[bu, \gl_1 \Q \otimes R]$ which do not disturb the denominators of the elements $t_k$.  (On the other hand, the ``Miller invariant'' associated to a rational ring spectrum is \emph{zero}, because arbitrary elements in $[bu, \gl_1 \Q \otimes R]$ can completely destroy the denominators.)}

This discussion accounts for both $E_\infty(M\Spin, \Q \otimes KO)$ and $E_\infty(M\Spin, \Q \otimes KO_p)$: the set of rational characteristic series includes into the set of ad\`elic characteristic series as the subset with rational coefficients.




\subsubsection{Finite place orientations}\label{FinitePlaceOrientationsSubsection}
\newcommand{\spin}{\mathit{spin}}

We want now to understand $E_\infty(M\Spin, KO_p)$ and its map to $E_\infty(M\Spin, \Q \otimes KO_p)$.  Here's the initial set-up:
\begin{center}
\begin{tikzcd}
\spin \arrow{r}[description]{j} & \gl_1 \S \arrow{r} \arrow{rd}[description]{\gl_1 \eta_{KO_p}} & Cj \arrow[densely dotted, "A"']{d} \\
& & \gl_1 KO_p.
\end{tikzcd}
\end{center}
We are looking to understand the space of filler diagrams $A$ (i.e., vertical maps with choice of homotopy of the precomposite to $\gl_1 \eta_{KO_p}$).  Notice first that there is a natural cofiber sequence to be placed on the bottom row:
\todo{I don't like the placing of this $A$. I want it to indicate a choice of filler.}
\begin{center}
\begin{tikzcd}[column sep=1em]
\spin \arrow{r}[description]{j} \arrow[red]{rd} & \gl_1 \S \arrow{r} \arrow[red]{d} \arrow{rd}[description]{\gl_1 \eta_{KO_p}} & Cj \arrow[densely dotted, "A"']{d} \\
& \Susp^{-1} \Q/\Z \otimes \gl_1 KO_p \arrow{r} & \gl_1 KO_p \arrow{r} & \Q \otimes \gl_1 KO_p \arrow{r} & \Q/\Z \otimes \gl_1 KO_p.
\end{tikzcd}
\end{center}
There is a canonical red vertical lift of $\gl_1 \eta_{KO_p}$ since $\gl_1 \S$ is a torsion spectrum, and this precomposes with $j$ to give another vertical map.  Notice now that selecting a filler triangle $A$ gives a commuting square with choice of homotopy and that $[\gl_1 \S, \Q \otimes \gl_1 KO_p] = 0$, and hence we would get a natural map (and natural homotopy) off of the homotopy cofibers:
\begin{center}
\begin{tikzcd}[column sep=1em]
\spin \arrow{r}[description]{j} \arrow{rd} & \gl_1 \S \arrow{r} \arrow{d} \arrow{rd}[description]{\gl_1 \eta_{KO_p}} & Cj \arrow[densely dotted, "A"' near start]{d}[description]{B} \arrow{r} & b\spin \arrow{r} \arrow{rd} \arrow[densely dotted]{d}[description]{C}& b\gl_1 \S \arrow{d} \\
& \Susp^{-1} \Q/\Z \otimes \gl_1 KO_p \arrow{r} & \gl_1 KO_p \arrow{r} & \Q \otimes \gl_1 KO_p \arrow{r} & \Q/\Z \otimes \gl_1 KO_p,
\end{tikzcd}
\end{center}
where $C$ is a map making the triangle it belongs to commute.  This all gives a function assigning $A$ to $B$ and $A$ to $C$ (and, in fact, the latter assignment factors through the former).

In order to show nonconstructively that the set of $A$s is nonempty, we might try to discern that $\gl_1 \eta_{KO_p} \circ j \in [\spin, \gl_1 KO_p]$ is zero by demonstrating something about the mapping set $[\spin, \gl_1 KO_p]$ itself.  We proceed by a sequence of quite improbable steps, beginning with the following Theorem original to Ando--Hopkins--Rezk:
\begin{theorem}[{\cite[Theorem 4.11]{AHR}}]
Let $R$ be a $E(d)$--local $E_\infty$ ring spectrum, and set $F$ to be the fiber \[F \to \gl_1 R \to L_d \gl_1 R.\]  Then $\pi_* F$ is torsion and $F$ satisfies the coconnectivity condition $F \simeq F(-\infty, d]$. \qed
\end{theorem}

\noindent It follows that $\gl_1 KO_p \to L_1 \gl_1 KO_p$ is a $1$--connected map, and hence \[[\spin, \gl_1 KO_p] = [\spin, L_1 \gl_1 KO_p].\]  In fact, we can even pass to the $K(1)$--localization, if we digress for a moment to introduce Rezk's logarithmic cohomology operation.

\begin{lemma}[{\cite[Theorem 1.1]{Kuhn}}]
For each $d \ge 1$ there is a functor $\Phi_d\co \CatOf{Spaces}_{*/} \to \CatOf{Spectra}$ which commutes with finite limits, is insensitive to upward truncation, and which evaluates on infinite loopspaces to give $\Phi_d(\Loops^\infty X) = \widehat L_d X$.\footnote{Importantly, $\Phi_d$ does \emph{not} care about the actual infinite loopspace structure on $\Loops^\infty X$, just that it has \emph{some} lift to a spectrum $X$.}\footnote{There is also a version of this theorem for $d = 0$, but since rational localization has no periodic behavior the results as not nearly as striking.} \qed
\end{lemma}

\begin{definition}[{\cite[Section 3]{RezkLogarithm}}]
The natural equivalence $(\GL_1 R)[1, \infty) \to (\Loops^\infty R)[1, \infty)$ gives rise to a map $\ell$ as in the diagram
\begin{center}
\begin{tikzcd}
& \Phi_d (\GL_1 R)[1, \infty) \arrow["\simeq"]{r} & \Phi_d (\Loops^\infty R)[1, \infty) \\
\gl_1 R \arrow{r} \arrow[bend left=15, "\ell_d" near end]{rr} & \widehat L_d \gl_1 R \arrow["\simeq"]{r} \arrow[equal, crossing over]{u} & \widehat L_d R \arrow[equal, crossing over]{u} .
\end{tikzcd}
\end{center}
\end{definition}

\begin{remark}
Applying the logarithm to the corners in the height $1$ chromatic fracture square yields the following identification:
\begin{center}
\begin{tikzcd}
& L_1 \gl_1 R \arrow{rr} \arrow{dd} & & \widehat L_1 R \arrow{dd} \\
L_1 \gl_1 R \arrow[crossing over]{rr} \arrow{dd} \arrow[equal]{ru} & & \widehat L_1 \gl_1 R \arrow{ru}[description]{\ell_1} \\
& \widehat L_0 R \arrow{rr} & & \widehat L_0 \widehat L_1 R \\
\widehat L_0 \gl_1 R \arrow{rr} \arrow{ru}[description]{\ell_0} & & \widehat L_0 \widehat L_1 \gl_1 R \arrow{ru}[description]{\widehat L_0 \ell_1} \arrow[crossing over, leftarrow]{uu} .
\end{tikzcd}
\end{center}
The front and back faces are connected by logarithms of \emph{different} heights---or, equivalently, the bottom horizontal arrow of the back face is \emph{twisted} from the usual chromatic fracture presentation of $L_1 R$.  The identification of this map is the usual sticking point in this approach.
\end{remark}

\begin{theorem}[{\cite[Theorem 1.9]{RezkLogarithm}}]
For $R$ a $K(1)$--local $E_\infty$ ring with $\pi_0 R$ torsion--free, the map $\pi_0 \ell_1\co \pi_0 R^\times \to \pi_0 R$ is given by the formula\footnote{The analogue of this formula for $E_\Gamma$ (but not an arbitrary $K(d)$--local $E_\infty$ ring spectrum) is given in \cite[Subsection 1.10]{RezkLogarithm}.} \[\ell_1(x) = \frac{1}{p} \log\left(\frac{x^p}{\psi^p x}\right) = \sum_{k=1}^\infty \frac{p^{k-1}}{k} \left(\frac{\theta(x)}{x^p}\right)^k. \qed\]
\end{theorem}

\begin{corollary}
The natural map $L_1 \gl_1 KO_p \to \widehat L_1 \gl_1 KO_p$ is a connective equivalence.
\end{corollary}
\begin{proof}
We specialize the above square to $R = KO_p$:
\begin{center}
\begin{tikzcd}
& & & KO_p \arrow{dd} \\
L_1 \gl_1 KO_p \arrow{rr} \arrow{dd} & & \widehat L_1 \gl_1 KO_p \arrow{ru}[description]{\ell_1} \\
& L_0 KO_p[4, \infty) \arrow{rr} & & L_0 KO_p \\
L_0 \gl_1 KO_p \arrow{rr} \arrow{ru}[description]{\ell_0} & & L_0 \widehat L_1 \gl_1 KO_p. \arrow{ru}[description]{\ell_1} \arrow[crossing over, leftarrow]{uu}
\end{tikzcd}
\end{center}
The behavior of the back horizontal map is determined by Rezk's formula for the logarithm.  \textbf{It acts by some nonzero number in every positive degree,}\todo{Justify this.} hence the fiber has the form $\prod_{k=-\infty}^0 \Susp^{4k-1} H\Q$.  Since the front face is a fiber square, this is also a calculation of the fiber of the map in the Lemma statement.\footnote{As a corollary of this same method, the Rezk logarithm for $R = KU^\wedge_p$ gives an equivalence $\gl_1 KU^\wedge_p[3, \infty) \to KU^\wedge_p[3, \infty)$.  This was previously known by nonconstructive methods to Adams and Priddy~\cite[Corollary 1.4]{AdamsPriddy}.}
\end{proof}

As a consequence, we have identifications \[\gl_1 \eta_{KO_p} \circ j \in [\spin, \gl_1 KO_p] \cong [\spin, L_1 \gl_1 KO_p] \cong [\spin, \widehat L_1 \gl_1 KO_p].\]  A direct application of the Rezk logarithm replaces $\widehat L_1 \gl_1 KO_p$ with $KO_p$, and the $K(1)$--localization of $\spin$ recovers $\Susp^{-1} KO_p$.  Altogether, this identifies $\gl_1 \eta_{KO_p} \circ j$ with a point in the mapping set $[\Susp^{-1} KO_p, KO_p]$---and we mark this as a point where we would like to understand the space of $KO$--operations.

We claim also that the kernel of the assignment $A \mapsto C$ is easy to understand: two fillers $A$ are related by an element of $[b\spin, \gl_1 KO_p]$, and their corresponding $C$s are related by the corresponding element of $[b\spin, \Q \otimes \gl_1 KO_p]$.  This set is rational, hence factors through the rationalization of $[b\spin, \gl_1 KO_p]$ where it must already be null, and hence it is a torsion element of $[b\spin, \gl_1 KO_p]$.  Meanwhile, the same argument as above identifies \[[b\spin, \gl_1 KO_p] = [KO_p, KO_p],\] which we again mark as a point where we would like to understand the space of $KO$--operations.  In particular, if we were to find the group of degree-preserving $KO$--operations to be torsion-free, then the assignment $A \mapsto C$ would be \emph{injective}.

We would like to understand the behavior of $C$ on homotopy based on some data about $A$.  This serves two purposes: there is the necessary condition that the triangle formed by $C$ and the canonical map $b\spin \to \Q / \Z \otimes \gl_1 KO_p$ commute, and then also the composite \[b\spin \xrightarrow{C} \Q \otimes \gl_1 KO_p \to (\gl_1 (\Q \otimes KO_p))[1, \infty)\] describes the map into the ad\`elic component.  In order to gain access to $C$, first notice that we can postcompose $B$ with the localization map off of $\gl_1 KO_p$ as in \Cref{MainAHRDiagram}.\footnote{Importantly, and differently from what every source says, this isn't a map of cofiber sequences and so the back second vertical map does not have to exist.}  This gives a new map $B'\co KO_p \to KO_p$---another reason to understand $KO$--operations.

We are now in a position to compute the action of $C$ on a homotopy class in $\pi_* b\spin$ by chasing through the following steps:
\begin{enumerate}
    \item We push such a class forward to $\widehat L_1 b\spin \simeq KO_p$ along the localization map.
    \item We then pull it back to $\widehat L_1 Cj \simeq KO_p$ along $KO_p \xrightarrow{1 - \psi^c} KO_p$, which acts by multiplication by $(1 - c^k)$ on $\pi_{4k}$.
    \item We push it down along $B'$ to $\widehat L_1 \gl_1 KO_p \simeq KO_p$, which acts by an unknown factor.
    \item We include it into the rational component of $\Q \otimes \widehat L_1 gl_1 KO_p$, using the fact that $\pi_* \widehat L_1 \gl_1 KO_p$ is torsion--free.
    \item Finally, we pull it back to $\Q \otimes \gl_1 KO_p$ along the logarithm $\ell_1$, which acts by multiplication by $(1 - p^{k-1})$ using Rezk's $K(1)$--local formula.\footnote{The formula for the logarithm in nonzero degrees comes from thinking of the logarithm as a \emph{natural transformation} and applying it to the mapping set $\ell\co \gl_1 KO^0(S^{2n}) \to KO^0(S^{2n})$.}
\end{enumerate}
The effect of this sequence of steps is \[t_{4k} = (1 - c^k)^{-1} b_{4k} (1 - p^{k-1})^{-1},\] where $t_{4k}$ and $b_{4k}$ are the effects on $\pi_{4k}$ of the maps $C$ and $B'$ respectively.  In the course of this proof, we are using the fact that division in the ring $\Z_p$ is unique when it is possible---the more responsible-looking equation to write is \[b_{4k} = (1 - c^k) t_{4k} (1 - p^{k-1}).\]

\begin{sidewaysfigure}
\centering
\begin{tikzcd}[column sep=0em]
& & & \widehat L_1 \S \arrow{rr} \arrow[equal]{d} & & KO_p \arrow[equal]{d} \arrow["1 - \psi^c"]{rr} & & KO_p \arrow[equal]{d} \\
& & & \widehat L_1 \gl_1 \S \arrow{rr} & & \widehat L_1 Cj \arrow[densely dotted, "B'" near start]{dd} \arrow{rr} & & \widehat L_1 b\spin \\
\spin \arrow{rr}[description]{j} \arrow{rrdd} & & \gl_1 \S \arrow{rr} \arrow{dd} \arrow{ru} \arrow{rrdd}[description]{\gl_1 \eta_{KO_p}} & & Cj \arrow{ru} \arrow[densely dotted, "A"' near start]{dd}[description]{B} \arrow[crossing over]{rr} & & b\spin \arrow{ru} \arrow[crossing over]{rr} \arrow[bend left=20]{rrdd} & & b\gl_1 \S \arrow{dd} \\
& & & & & \widehat L_1 \gl_1 KO_p \arrow{rr} & & \Q \otimes \widehat L_1 \gl_1 KO_p \\
& & \Susp^{-1} \Q/\Z \otimes \gl_1 KO_p \arrow{rr} & & \gl_1 KO_p \arrow{ru} \arrow{rr} & & \Q \otimes \gl_1 KO_p \arrow{ru} \arrow{rr} \arrow[densely dotted, leftarrow, crossing over]{uu}[description]{C} & & \Q/\Z \otimes \gl_1 KO_p.
\end{tikzcd}
\caption{A diagram showing the interconnections among the main components of the $p$--primary part of the Ando--Hopkins--Rezk argument.}\label{MainAHRDiagram}
\end{sidewaysfigure}

Now, finally, the diagonal map $b\spin \to \Q/\Z \otimes \gl_1 KO_p$ becomes relevant.  To check the commutativity of the triangle with $C$, we need only compare the results of the composite on homotopy since the map $C$ targets a rational spectrum and hence is determined its effect on homotopy.  The following invariance property makes this map accessible:\footnote{It is also possible to compute the effect of this map on homotopy using the $S^1$--transfer.  This is the subject of a paper by Miller~\cite{MillerBernoulliNos}, after which the Miller invariant is named, and also the subject of further research by Baker and company~\cite{BCGHRW}.}

\begin{theorem}[{\cite[Proposition 3.15 and Corollary 3.16]{AHR}}]
For any $A_\infty$ orientation $\phi\co MU \to R$ of an $A_\infty$ ring spectrum $R$, the denominators of the characteristic series associated to $\Q \otimes \phi$ compute the behavior of the map $\pi_* BU \to \Q / \Z \otimes GL_1 R$. \qed
\end{theorem}

\begin{corollary}
The numbers $t_{4k}$ describing the effect of $C$ satisfy the congruences \[t_{4k} \equiv -\frac{B_k}{2k} \pmod{\Z}.\]
\end{corollary}
\begin{proof}[Proof sketch]
The Todd orientation $MU \to KU$ is known to be $A_\infty$~\cite[Theorem V.4.1]{EKMM}, and the characteristic series of the Todd orientation has coefficients $B_k$.  The extra division by $2$ is picked up by studying the map $\pi_* BSU \to \pi_* B\Spin$ and the map $\pi_* KO \to \pi_* KU$.
\end{proof}

We have thus identified the legal fillers $C$ as those sequences of rational numbers $t_{4k}$ satisfying conditions:
\begin{enumerate}
    \item $t_{4k}$ has the correct denominators: for $k \ge 1$, $t_{4k} \equiv -B_k/(2k) \pmod{\Z}$.
    \item $b_{4k}$ is the effect on homotopy of some map $B'\co KO_p \to KO_p$.
\end{enumerate}


\subsubsection{Stable $KO$ operations}
\newcommand{\cts}{\mathrm{cts}}

We have identified three points where we want to understand the collection of stable $KO$ operations.  Although much of the main text of this book has been concerned with this sort of subject, this does not appear to be so immediately accessible: we want operations rather than cooperations, and $KO$ is \emph{not} a complex-orientable ring spectrum.  It is close to one, though, and we gain access to it through familiar approximation.

The easy initial calculation is $K^\vee K = \cts(\Z_p^\times, \Z_p)$, the ring of $\Z_p$--valued functions\footnote{Not homomorphisms!} on $\Z_p^\times$ which are continuous for the adic topologies on the domain and the target.  This comes out of the stable cooperations of Landweber flat homology theories discussed in \Cref{DefnChromaticHomologyThys}, where we showed that $E_\Gamma$ has cooperations given by the ring of functions on the pro-\'etale group scheme $\Aut \Gamma$.  For $\Gamma = \G_m$, this group scheme $\Aut \G_m$ is constant at $\Z_p^\times$, so that $K^\vee K$ is the ring of $\Z_p$--valued functions on $\Z_p^\times$.  Turning to cohomology, it follows by the universal coefficient spectral sequence that $K^0 K = \Hom(\cts(\Z_p^\times, \Z_p), \Z_p)$ and that $K^1 K = 0$.  These correspondences behave as follows:
\begin{enumerate}
    \item The Kronecker pairing \[\S^0 \xrightarrow{c} K \sm K \xrightarrow{1 \sm f} K \sm K \xrightarrow{\mu} K\] is computed by the evaulation pairing \[(c \in K^\vee K, f \in K^0 K) \mapsto f(c).\]
    \item The stable operation $\psi^\lambda$ attached to $[\lambda] \in \Aut \G_m$ is evaluation at $\lambda$.
    \item The stable cooperation $v^{-k} \sm v^k \in \pi_0 K \sm K$ corresponds to the polynomial function $x \mapsto x^k$, as justified by the computation \[\operatorname{ev}_{\lambda}(v^{-k} \sm v^k) = \frac{\psi^\lambda v^k}{v^k} = \frac{\lambda^k v^k}{v^k} = \lambda^k.\]
\end{enumerate}

\noindent These last two facts mean that the behavior of a stable operation on homotopy is identical information to the values of a functional $f$ on the standard polynomial functions $x^k$.  We record this algebraic model as follows:
\begin{lemma}
For any $N \ge 0$, the assignment \[\Hom(\cts(\Z_p^\times, \Z_p), \Z_p) \xrightarrow{(f(x \mapsto x^k))_k} \prod_{k \ge N} \Z_p\] is injective.  A sequence $(x_k)$ is said to be a \emph{K\"ummer sequence} when it lies in this image.\footnote{A bit more explicitly: $(x_k)$ is K\"ummer when for all $h(x) = \sum_{k=N}^n a_k x^k \in \Q[x]$ we have $\sum_{k=N}^m a_k x_k \in \Z_p$.} \qed
\end{lemma}

\begin{remark}
An interesting feature of the Lemma is the auxiliary index $N$, which is \emph{not} part of the property of being K\"ummer.  In $p$--adic geometry, this is reflected by the $p$--adic convergence of the sequence \[d + (p-1)p^r \xrightarrow{r \to \infty} d,\] and hence the continuous reconstruction property \[x_d = \lim_{r \to \infty} x_{d + (p-1)p^r}.\]  In homotopy theory, this is reflected by the reconstruction property $K \sm K[2k, \infty) \simeq K \sm K$.
\end{remark}

\begin{remark}
With this computation in hand, the $p$--local operations $KU_{(p)} \sm KU_{(p)}$ can be recovered from arithmetic fracture, as can the global operations $KU \sm KU$.  The answer is quite similar: $\pi_0 KU \sm KU$ is populated by rational polynomials which evaluate to integers on all integer inputs, called \textit{numerical polynomials}.
\end{remark}

We now pass from $KU$ to $KO$.  To begin, use the Tate trick
\begin{align*}
K \sm KO & \simeq K \sm (K^{hC_2}) & \text{($KO$ is a homotopy fixed point spectrum)} \\
& \simeq K \sm (K_{hC_2}) & \text{(Tate objects vanish $K(1)$--locally)} \\
& \simeq (K \sm K)_{hC_2} & \text{(homotopy colimits pull past smash products)} \\
& \simeq (K \sm K)^{hC_2}, & \text{(Tate objects vanish $K(1)$--locally)}
\end{align*}
so that $\pi_0 K \sm KO = \cts(\Z_p^\times / C_2, \Z_p)$.  Taking fixed points again, we then also have $\pi_* KO \sm KO = \cts(\Z_p^\times / C_2, KO_*)$, and $KO^* KO$ is the $KO_*$--linear dual.  It follows that $[\Susp^{-1} KO, KO] = 0$ and that $[KO, KO] = \Hom(\cts(\Z_p^\times / C_2, \Z_p), \Z_p)$ is torsion-free, which account for our outstanding claims.


\subsubsection{Mazur's construction of Kubota--Leopoldt $p$--adic $L$--functions}

Having learned enough about $KO$--operations to justify the program enacted in the previous subsections, we now need to show that there exist sequences of $p$--adic integers satisfying those criteria.

\begin{theorem}[Mazur]
For any auxiliary $c \in \Z_p^\times$, there is a functional $f_c$ satisfying\footnote{\[\text{Explicitly,\;} f_c(h) = \int_{\Z_p^\times} h(x) d\mu_c = \lim_{r \to \infty} \frac{1}{p^r} \sum_{\substack{0 \le i < p^r \\ p \nmid i}} \int_i^{ci} \frac{h(t)}{t} dt.\]}\footnote{With considerable effort, this output can be halved~\cite[Section 10.3]{AHR}.}\footnote{It also satisfies the normalizing property $\int_{\Z_p^\times} d\mu_c = \frac{1}{p} \log(c^{p-1})$.} \[f_c(x^{k \ge 1}) = \frac{-B_k}{k}(1 - p^{k-1})(1 - c^k).\]
\end{theorem}

\noindent This Theorem is stated in exactly the generality it was originally proven, and so uou might wonder why Mazur had already proven \emph{exactly} what we needed.  To understand his program, recall these two facts about $\zeta$:
\begin{enumerate}
    \item Except for a real Euler factor, $\zeta$ is basically the Mellin transform of the measure $\frac{dx}{e^x - 1}$ (i.e., its sequence of moments): \[\zeta(s) = \frac{1}{\Gamma(s)} \int_0^\infty x^{s-1} \frac{dx}{e^x - 1}.\]
    \item For any $k \in \Z_{> 0}$, $\zeta(1 - k) = -B_k / k$, where $\frac{t}{e^t - 1} = \sum_{k=0}^\infty B_k \frac{t^k}{k!}$.
\end{enumerate}
Mazur's idea was to build a $p$--adic $\zeta$--function by investigating similar $p$--adic integrals, beginning with certain finitary approximations to this one.  To begin, a Bernoulli polynomial for $k \in \Z_{>0}$ is \[\sum_{k=0}^\infty B_k(x) \frac{t^k}{k!} = \frac{t e^{tx}}{e^t - 1}.\]  These polynomials beget Bernoulli distributions according to the rule
\begin{align*}
\Z/p^n\Z & \xrightarrow{E_k} \Q \subseteq \Q_p \\
x \in [0, p^n) & \mapsto k^{-1} p^{n(k-1)} B_k(x p^{-n}).
\end{align*}
A distribution in general is a function on $\Z_p$ such that its value at any node in the $p$--adic tree is equal to the sum of the values of its immediate children, and the $p$--adic integral of a locally constant function with respect to such a distribution is defined by their convolution.  For example, the constant function $1$ factors through $\Z/p$, hence \[\int_{\Z_p} dE_k = \overset{\text{non-obvious}}{\overbrace{\frac{1}{k} \sum_{a=0}^{p-1} B_k\left(\frac{a}{p}\right) = \frac{B_k(0)}{k}}} = \frac{B_k}{k}.\]

However, this distribution is not a \emph{measure}, in the sense that it is not bounded and hence does not extend to a functional on all continuous functions (rather than just locally constant ones).  The standard fix for this is called \emph{regularization}: pick $c \in \Z$ with $p \nmid c$, and set $E_{k,c}(x) = E_k(x) - c^kE_k(c^{-1}x)$.  This is a measure, and for $k \ge 1$ it has total volume given by \[\int_{\Z_p} dE_{k,c} = \int_{\Z_p} dE_k - c^k \int_{\Z_p} dE_k(c^{-1}x) = \frac{B_k}{k}(1 - c^k).\]

These measures interrelate: $E_{k, c} = x^{k-1} E_{1, c}$, and hence the single measure $E_{1, c}$ has all of these values as moments.  We would like to perform $p$--adic interpolation in $k$ to remove the restriction $k \ge 1$, but this is not naively possible: if $k = 0$, say, then we naively have $E_{0, c} = x^{-1} E_{1, c}$, which will not make sense whenever $x \in p\Z_p$.  This is most easily solved by restricting $x$ to lie in $\Z_p^\times$, which has a predictable effect for $k \in \Z_{> 0}$:
\begin{align*}
\int_{\Z_p^\times} x^{k-1} dE_{1,c} & = \int_{\Z_p} x^{k-1} dE_{1,c} - \int_{p\Z_p} x^{k-1} dE_{1, c} \\
& = \int_{\Z_p} x^{k-1} dE_{1,c} - p^{k-1} \int_{\Z_p} x^{k-1} dE_{1, c} \\
& = \frac{B_k}{k}(1 - c^k)(1 - p^{k-1}).
\end{align*}
Hence, the Mellin transform of the measure $dE_{1,c}$ on $\Z_p^\times$ gives a sort of $p$--adic interpolation of the $\zeta$--function.

It also has \emph{exactly} the properties we need to guarantee the existence of an $E_\infty$ orientation $M\Spin \to KO$.  It is remarkable that the three factors in \[\int_{\Z_p^\times} x^{k-1} d E_{1, c} = \frac{B_k}{k} (1 - c^k)(1 - p^{k-1})\] have discernable provenances in the two fields.  In stable homotopy theory these arise respectively in the characteristic series of the orientation $MU \to KU$, in the finite Adams resolution for the $K(1)$--local sphere, and in the Rezk logarithm.  In $p$--adic analytic number theory, they arise as the special values of the $\zeta$--function, the regularization to make it a measure, and the restriction to perform $p$--adic interpolation.  It is completely mysterious how or if these operations correspond.

\begin{remark}
These Bernoulli sequences are \emph{not} the only sequences satisfying these reconstruction properties---in fact, there are infinitely many, and an explicit presentation of them is available~\cite{SprangNaumann}.  Sprang leaves open whether there is a way to single out the Bernoulli solution among the rest, and it seems plausible that this is the only solution with a ``reasonable'' growth rate (as measured in $\R$).  It would also be great if this ``real place'' condition had something to do with a smooth cohomology theory like differential real $K$--theory.
\end{remark}


\subsubsection{Footnotes on the $\tmf$ case}

The case of the orientation $M\String \to \tmf$ has all of the same trappings, but its order of complexity is $(-)^{3/2}$ of the above case, essentially because the height $1$ chromatic fracture \emph{square} gets replaced by the height $2$ chromatic fracture \emph{cube}.  (There is also the issue of the more complicated coefficient ring $\tmf_*$ over $KO_*$.)  Many of the steps remain the same:
\begin{enumerate}
    \item Begin with a rational orientation, which is basically the Witten genus valued in holomorphic expansions of modular forms.
    \item Analyze the homotopy type of $\widehat L_1 \tmf$ and compare it to that of $KO$.  This lets us use another universal coefficient theorem to lift our description of $KO^* KO$ as $KO^*$--valued measures to $\widehat L_1 \tmf^* KO$ as $\widehat L_1 \tmf^*$--valued measures.
    \item The homotopy type of $\widehat L_2 \tmf$ is ``naively irrelevant'' in the chromatic fracture square: maps $b\Spin \to \widehat L_2 \tmf$ factor through $\widehat L_2 b\Spin = \widehat L_2 KO_p = 0$.
    \item However, the logarithm's presence in the chromatic fracture square $\widehat L_1 \tmf \to \widehat L_1 \widehat L_2 \tmf$ has a real effect that must be understood.  This is not easy: the height $2$ logarithm is not so accessible, so this requires a real understanding of power operations in $\tmf$.
    \item You also have to calculate the Miller invariant associated to $\tmf$.  In the case of $E_\infty(M\Spin, KO)$, one uses the $A_\infty$ orientation $MU \to KU$, as well as an understanding of the maps $\pi_* BU \to \pi_* BO$ and $\pi_* KO \to \pi_* KU$.  The case of $E_\infty(M\String, \tmf)$ is similar: one constructs an $A_\infty$ lift of the $\sigma$--orientation $MU[6, \infty) \to K^{\Tate}$, as well as an understanding of the maps $\pi_* BU[6, \infty) \to \pi_* B\String$ and $\pi_* \tmf \to \pi_* K^{\Tate}$.\todo{I think the conclusion here is that you have to be a $q$--expansion of a modular form (of a particular weight) with constant term a Bernoulli number and every other coefficient integral.  Mike told me that this (or something like this) fully determines these generalized Eisenstein series; that's nice.}
    \item Finally, you have to ramp up the algebraic part of the calculation by identifying the analogues of the Mazur moments in $\pi_* \widehat L_1 \tmf$.  These turn out to be normalized Eisenstein series.
\end{enumerate}

\begin{remark}
The presence of such interesting arithmetic invariants (Bernoulli numbers, Bernoulli polynomials, generalized Eisenstein series, \ldots) hiding in the Miller invariant and its analogues is very striking.  One wonders what the analogous values are (or perhaps the values stemming from the iterated $S^1$--transfer of Baker et al~\cite{BCGHRW}) associated to a Morava $E_\Gamma$.
\end{remark}

\begin{remark}
Some more open questions about this can be found in 
\todo[inline]{Mike has some open questions about the end of this analysis (and in particular about the fiber of the Atkin map that appears in the $K(1)$--local analysis of $\tmf$, an analogue of the chromatic splitting fiber) at the end of his talk notes \textit{The $\String$ orientation of $\tmf$}.  Some of that should be copied here.}
\end{remark}

\todo{$KU$ is known to have a unique $E_\infty$ structure by work of Baker--Richter.  Is this also true of $K^{\Tate}$?  If so, it lends a lot of credibility to this Miller invariant calculation and its relation to $\tmf$.}
\todo{I think it's possible to show, as a side-example, that the total exterior power operation $\lambda^q\co K \to K\ps{q}$ is an $E_\infty$ map where $K\ps{q}$ is $K^{\Tate}$ and \emph{not}, e.g., $K^{\CP^\infty}$.}

















